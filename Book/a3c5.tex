\documentclass[11pt,letterpaper,twoside]{book} 
\usepackage{amsmath}
\usepackage{amssymb}
\usepackage{color}
\usepackage[T1]{fontenc}
\usepackage{geometry}
\usepackage{graphicx}
\usepackage{hyperref}
\usepackage{listings}
\usepackage{makeidx}
\usepackage{paralist}
\usepackage{sectsty}
\usepackage{textcomp}
\geometry{left=2cm,right=2cm,top=2.5cm,bottom=2.5cm}

% Lowercase section headers
\let\MakeUppercase\relax

% No colored border on hyperlinks
\hypersetup{colorlinks=false,pdfborder={0 0 0}}

% For the horizontal rule under sections
\sectionfont{\sectionrule{0pt}{0pt}{-10pt}{0.8pt}}

% For displaying code
\definecolor{dkgreen}{rgb}{0,0.6,0}
\definecolor{gray}{rgb}{0.5,0.5,0.5}
\definecolor{mauve}{rgb}{0.58,0,0.82}
\lstset{
  aboveskip=5mm,
  basicstyle={\footnotesize\ttfamily},
  belowskip=5mm,
  breaklines=true,
  breakatwhitespace=true,
  frame=none,
  columns=flexible,
  commentstyle=\color{dkgreen},
  keepspaces=true,
  keywordstyle=\color{blue},
  language=C++,
  % literate={\ \ }{{\ \ \ \ }}1,  % Convert 2 spaces to 4 
  numbers=left,
  numbersep=10pt,
  numberstyle=\scriptsize\ttfamily\color{gray},
  rulecolor=\color{black},
  showstringspaces=false,
  stringstyle=\color{mauve},
  tabsize=2
}

\begin{document} 
 
\frontmatter
\pagenumbering{gobble}

\begin{titlepage}
	\centering
	\newgeometry{left=2.5cm,right=2.5cm,top=2.5cm,bottom=2.5cm}

	\vspace*{6 cm}

	\textbf{\LARGE Alex's Anthology of Algorithms}
	\\[1.0\baselineskip]
	\textbf{\Large Common Code for Contests in Concise C++}
	\\[1.0\baselineskip]
	{\large\textit{Draft v0.9, Feb 16, 2020}}

	\vspace*{11 cm}

	{\Large Alex Li}
	\\[0.5\baselineskip]
	{\small \textcircled{c} 2015---. All rights reserved.}
\end{titlepage}

\pagebreak
\vspace*{8 cm}
\centering

{\large This page is intentionally left blank.}

\vspace{\fill}
\makeatletter

\let\cleardoublepage\clearpage
\pagenumbering{roman}
\chapter{Preface}

\raggedright
\setlength{\parskip}{0.5em}

Visit \href{https://github.com/alxli/algorithm-anthology}{\texttt{github.com/alxli/algorithm-anthology}} for the most up-to-date digital version of this book.

\section*{Introduction}

Welcome to a comprehensive collection of common algorithms and data structures. The ultimate goal of this book is not to explain concepts from the ground up, but instead to explore the finer details behind their \textit{implementations}. There are many potential ways you can use this, for instance:
\begin{itemize}
	\item as a reference to help you better understand topics that you have only studied on a high level,
	\item as a printed codebook, which is a permitted resource for contests such as the ACM ICPC, or
	\item to cross-check existing code you have written for contest or coding interview questions.
\end{itemize}

Before diving into any section, it is strongly recommended that you have already studied the algorithms involved. Reading the code first is never an ideal approach to properly understand an algorithm. You should instead try proving (its correctness and time/space complexity) and implementing it from scratch.

Every topic to be explored is easily researchable online. Thus instead of including theoretical discussions, I document just enough to establish the problem being solved, notation being used, and any special trickery involved. I have also included small, non-rigorous examples to demonstrate usage of the code.

We mentioned that the implementation itself is the focus, but what makes an implementation good? The code is written with the following principles in mind:
\begin{itemize}
	\item \textit{Clarity:} A reader already familiar with an algorithm should have no problem understanding how its implementation works. Consistency in naming conventions should be emphasized, and any tricks or language-specific hacks should be documented.
	\item \textit{Concision:} To minimize the amount of scrolling and searching during the frenzy of time contests, it is helpful for code to be compact. Shorter code is also generally easier to understand, as long as it is not overly cryptic. Finally, each implementation should fit in a single source file as required by nearly all online judging systems.
	\item \textit{Efficiency:} The code here is designed to be performant on real contests, and should maintain a low constant overhead. This is often challenging in the face of clarity and tweakability, but we can hope for contest setters to be liberal with time and memory limits. If the code here times out, you can reasonably rule out insufficient constant optimization and assume that you are choosing an algorithm from a suboptimal complexity class.
	\item \textit{Genericness:} Implementations should be easy to adapt to achieve slightly different goals. One may want to tweak some core logic, parameters, data types, etc. In timed contests, we would certainly prefer this process to be as painless as possible. C++ templates are often used to increase tweakability at a slight cost to simplicity.
	\item \textit{Portability:} Different contest environments use different compiler builds. In order to maximize compatibility, non-standard and newer features are avoided. The decision to follow C++98 standards is due to many contest systems being stuck on an older version of the language. Moreover, minimizing newer C++ features will make the code more language-agnostic.
\end{itemize}

As these points and the title both suggest, there is a slight bias towards contests. Compiling a codebook for my personal reference during contests was indeed how this project got started. This work has become much more multipurpose now. Whatever your use case is, I hope you discover something enlightening.

\begin{flushright}
Cheers.\\
--- Alex
\end{flushright}

\section*{Portability Notes}

All programs were tested with GCC and compiled for a 32-bit target using the switches below:

\begin{verbatim}
   g++ -std=gnu++98 -pedantic -Wall -Wno-long-long -O2
\end{verbatim}

This means the following are assumed about data types:

\begin{compactitem}
	\item \texttt{bool} and \texttt{char} are 8-bit.
	\item \texttt{int} and \texttt{float} are 32-bit.
	\item \texttt{double} and \texttt{long long} are 64-bit.
	\item \texttt{long double} is 96-bit.
\end{compactitem}
\vspace{5pt}

Programs are highly portable (ISO C++ 1998 compliant), except in the following regards:

\begin{itemize}
	\item Usage of \texttt{long long} and dependent features, which are compliant in C99/C++0x or later. 64-bit integers are a must for many contest problems.
	\item Usage of variable sized arrays. While easily replaced by vectors, they are generally simpler and avoid dynamic memory (which some argue is a bad idea for contests).
	\item Usage of GCC's built-in functions like \texttt{\_\_builtin\_popcount()} and \texttt{\_\_builtin\_clz()}. These can be extremely convenient, but are straightforward to implement if unavailable. See here for a reference: \url{https://gcc.gnu.org/onlinedocs/gcc/Other-Builtins.html}
	\item Usage of compound-literals, e.g. \texttt{vec.push\_back((mystruct)\{a, b, c\})}. This adds a little more concision by not requiring a constructor definition.
	\item Hacks that may depend on the platform (e.g. endianness), such as getting the signbit with type-punned pointers. Be weary of portability for all bitwise/lower level code.
\end{itemize}

\tableofcontents

\mainmatter
\chapter{Elementary Algorithms}

\section{Array Transformations}
\setcounter{section}{1}
\setcounter{subsection}{0}
\subsection{Sorting Algorithms}
\begin{lstlisting}
/*

These functions are equivalent to std::sort(), taking random-access iterators
as a range [lo, hi) to be sorted. Elements between lo and hi (including the
element pointed to by lo but excluding the element pointed to by hi) will be
sorted into ascending order after the function call. Optionally, a comparison
function object specifying a strict weak ordering may be specified to replace
the default operator <.

These functions are not meant to compete with standard library implementations
in terms terms of speed. Instead, they are meant to demonstrate how common
sorting algorithms can be concisely implemented in C++.

*/

#include <algorithm>
#include <functional>
#include <iterator>
#include <vector>

/*

Quicksort repeatedly selects a pivot and partitions the range so that elements
comparing less than the pivot precede the pivot, and elements comparing greater
or equal follow it. Divide and conquer is then applied to both sides of the
pivot until the original range is sorted. Despite having a worst case of O(n^2),
quicksort is often faster in practice than merge sort and heapsort, which both
have a worst case time complexity of O(n log n).

The pivot chosen in this implementation is always a middle element of the range
to be sorted. To reduce the likelihood of encountering the worst case, the pivot
can be chosen in better ways (e.g. randomly, or using the "median of three"
technique).

Time Complexity (Average): O(n log n).
Time Complexity (Worst): O(n^2).
Space Complexity: O(log n) auxiliary stack space.
Stable?: No.

*/

template<class It, class Compare>
void quicksort(It lo, It hi, Compare comp) {
  if (hi - lo < 2) {
    return;
  }
  typedef typename std::iterator_traits<It>::value_type T;
  T pivot = *(lo + (hi - lo)/2);
  It i, j;
  for (i = lo, j = hi - 1; ; ++i, --j) {
    while (comp(*i, pivot)) {
      ++i;
    }
    while (comp(pivot, *j)) {
      --j;
    }
    if (i >= j) {
      break;
    }
    std::swap(*i, *j);
  }
  quicksort(lo, i, comp);
  quicksort(i, hi, comp);
}

template<class It>
void quicksort(It lo, It hi) {
  typedef typename std::iterator_traits<It>::value_type T;
  quicksort(lo, hi, std::less<T>());
}

/*

Merge sort first divides a list into n sublists of one element each, then
recursively merges the sublists into sorted order until only a single sorted
sublist remains. Merge sort is a stable sort, meaning that it preserves the
relative order of elements which compare equal by operator < or the custom
comparator given.

An analogous function in the C++ standard library is std::stable_sort(), except
that the implementation here requires sufficient memory to be available. When
O(n) auxiliary memory is not available, std::stable_sort() falls back to a time
complexity of O(n log^2 n) whereas the implementation here will simply fail.

Time Complexity (Average): O(n log n).
Time Complexity (Worst): O(n log n).
Space Complexity: O(log n) auxiliary stack space and O(n) auxiliary heap space.
Stable?: Yes.

*/

template<class It, class Compare>
void mergesort(It lo, It hi, Compare comp) {
  if (hi - lo < 2) {
    return;
  }
  It mid = lo + (hi - lo - 1)/2, a = lo, c = mid + 1;
  mergesort(lo, mid + 1, comp);
  mergesort(mid + 1, hi, comp);
  typedef typename std::iterator_traits<It>::value_type T;
  std::vector<T> merged;
  while (a <= mid && c < hi) {
    merged.push_back(comp(*c, *a) ? *c++ : *a++);
  }
  if (a > mid) {
    for (It it = c; it < hi; ++it) {
      merged.push_back(*it);
    }
  } else {
    for (It it = a; it <= mid; ++it) {
      merged.push_back(*it);
    }
  }
  for (int i = 0; i < hi - lo; i++) {
    *(lo + i) = merged[i];
  }
}

template<class It>
void mergesort(It lo, It hi) {
  typedef typename std::iterator_traits<It>::value_type T;
  mergesort(lo, hi, std::less<T>());
}

/*

Heapsort first rearranges an array to satisfy the max-heap property. Then, it
repeatedly pops the max element of the heap (the left, unsorted subrange),
moving it to the beginning of the right, sorted subrange until the entire range
is sorted. Heapsort has a better worst case time complexity than quicksort and
also a better space complexity than merge sort.

The C++ standard library equivalent is calling std::make_heap(lo, hi), followed
by std::sort_heap(lo, hi).

Time Complexity (Average): O(n log n).
Time Complexity (Worst): O(n log n).
Space Complexity: O(1) auxiliary.
Stable?: No.

*/

template<class It, class Compare>
void heapsort(It lo, It hi, Compare comp) {
  typename std::iterator_traits<It>::value_type tmp;
  It i = lo + (hi - lo)/2, j = hi, parent, child;
  for (;;) {
    if (i <= lo) {
      if (--j == lo) {
        return;
      }
      tmp = *j;
      *j = *lo;
    } else {
      tmp = *(--i);
    }
    parent = i;
    child = lo + 2*(i - lo) + 1;
    while (child < j) {
      if (child + 1 < j && comp(*child, *(child + 1))) {
        child++;
      }
      if (!comp(tmp, *child)) {
        break;
      }
      *parent = *child;
      parent = child;
      child = lo + 2*(parent - lo) + 1;
    }
    *(lo + (parent - lo)) = tmp;
  }
}

template<class It>
void heapsort(It lo, It hi) {
  typedef typename std::iterator_traits<It>::value_type T;
  heapsort(lo, hi, std::less<T>());
}

/*

Comb sort is an improved bubble sort. While bubble sort increments the gap
between swapped elements for every inner loop iteration, comb sort fixes the gap
size in the inner loop, decreasing it by a particular shrink factor in every
iteration of the outer loop. The shrink factor of 1.3 is empirically determined
to be the most effective.

Even though the worst case time complexity is O(n^2), a well chosen shrink
factor ensures that the gap sizes are co-prime, in turn requiring astronomically
large n to make the algorithm exceed O(n log n) steps. On random arrays, comb
sort is only 2-3 times slower than merge sort. Its short code length length
relative to its good performance makes it a worthwhile algorithm to remember.

Time Complexity (Worst): O(n^2).
Space Complexity: O(1) auxiliary.
Stable?: No.

*/

template<class It, class Compare>
void combsort(It lo, It hi, Compare comp) {
  int gap = hi - lo;
  bool swapped = true;
  while (gap > 1 || swapped) {
    if (gap > 1) {
      gap = (int)((double)gap / 1.3);
    }
    swapped = false;
    for (It it = lo; it + gap < hi; ++it) {
      if (comp(*(it + gap), *it)) {
        std::swap(*it, *(it + gap));
        swapped = true;
      }
    }
  }
}

template<class It>
void combsort(It lo, It hi) {
  typedef typename std::iterator_traits<It>::value_type T;
  combsort(lo, hi, std::less<T>());
}

/*

Radix sort is used to sort integer elements with a constant number of bits in
linear time. This implementation only works on ranges pointing to unsigned
integer primitives. The elements in the input range do not strictly have to be
unsigned types, as long as their values are nonnegative integers.

In this implementation, a power of two is chosen to be the base for the sort
so that bitwise operations can be easily used to extract digits. This avoids the
need to use modulo and exponentiation, which are much more expensive operations.
In practice, it's been demonstrated that 2^8 is the best choice for sorting
32-bit integers (approximately 5 times faster than std::sort(), and typically
2-4 faster than radix sort using any other power of two chosen as the base).

Time Complexity: O(n*w) for n integers of w bits each.
Space Complexity: O(n + w) auxiliary.

*/

template<class UnsignedIt>
void radix_sort(UnsignedIt lo, UnsignedIt hi) {
  if (hi - lo < 2) {
    return;
  }
  const int radix_bits = 8;
  const int radix_base = 1 << radix_bits;  // e.g. 2^8 = 256
  const int radix_mask = radix_base - 1;  // e.g. 2^8 - 1 = 0xFF
  int num_bits = 8*sizeof(*lo);  // 8 bits per byte
  typedef typename std::iterator_traits<UnsignedIt>::value_type T;
  T *buf = new T[hi - lo];
  for (int pos = 0; pos < num_bits; pos += radix_bits) {
    int count[radix_base] = {0};
    for (UnsignedIt it = lo; it != hi; ++it) {
      count[(*it >> pos) & radix_mask]++;
    }
    T *bucket[radix_base], *curr = buf;
    for (int i = 0; i < radix_base; curr += count[i++]) {
      bucket[i] = curr;
    }
    for (UnsignedIt it = lo; it != hi; ++it) {
      *bucket[(*it >> pos) & radix_mask]++ = *it;
    }
    std::copy(buf, buf + (hi - lo), lo);
  }
  delete[] buf;
}

/*** Example Usage and Output:

mergesort() with default comparisons: 1.32 1.41 1.62 1.73 2.58 2.72 3.14 4.67
mergesort() with 'compare_as_ints()': 1.41 1.73 1.32 1.62 2.72 2.58 3.14 4.67
------
Sorting five million integers...
std::sort():  0.355s
quicksort():  0.426s
mergesort():  1.263s
heapsort():   1.093s
combsort():   0.827s
radix_sort(): 0.076s

***/

#include <cassert>
#include <cstdlib>
#include <ctime>
#include <iomanip>
#include <iostream>
#include <vector>
using namespace std;

template<class It>
void print_range(It lo, It hi) {
  while (lo != hi) {
    cout << *lo++ << " ";
  }
  cout << endl;
}

template<class It>
bool sorted(It lo, It hi) {
  while (++lo != hi) {
    if (*lo < *(lo - 1)) {
      return false;
    }
  }
  return true;
}

bool compare_as_ints(double i, double j) {
  return (int)i < (int)j;
}

int main () {
  {  // Can be used to sort arrays like std::sort().
    int a[] = {32, 71, 12, 45, 26, 80, 53, 33};
    quicksort(a, a + 8);
    assert(sorted(a, a + 8));
  }
  {  // STL containers work too.
    int a[] = {32, 71, 12, 45, 26, 80, 53, 33};
    vector<int> v(a, a + 8);
    quicksort(v.begin(), v.end());
    assert(sorted(v.begin(), v.end()));
  }
  {  // Reverse iterators work as expected.
    int a[] = {32, 71, 12, 45, 26, 80, 53, 33};
    vector<int> v(a, a + 8);
    heapsort(v.rbegin(), v.rend());
    assert(sorted(v.rbegin(), v.rend()));
  }
  {  // We can sort doubles just as well.
    double a[] = {1.1, -5.0, 6.23, 4.123, 155.2};
    vector<double> v(a, a + 5);
    combsort(v.begin(), v.end());
    assert(sorted(v.begin(), v.end()));
  }
  {  // Must use radix_sort with unsigned values, but sorting in reverse works!
    int a[] = {32, 71, 12, 45, 26, 80, 53, 33};
    vector<int> v(a, a + 8);
    radix_sort(v.rbegin(), v.rend());
    assert(sorted(v.rbegin(), v.rend()));
  }

  // Example from: http://www.cplusplus.com/reference/algorithm/stable_sort
  double a[] = {3.14, 1.41, 2.72, 4.67, 1.73, 1.32, 1.62, 2.58};
  {
    vector<double> v(a, a + 8);
    cout << "mergesort() with default comparisons: ";
    mergesort(v.begin(), v.end());
    print_range(v.begin(), v.end());
  }
  {
    vector<double> v(a, a + 8);
    cout << "mergesort() with 'compare_as_ints()': ";
    mergesort(v.begin(), v.end(), compare_as_ints);
    print_range(v.begin(), v.end());
  }
  cout << "------" << endl;

  vector<int> v, v2;
  for (int i = 0; i < 5000000; i++) {
    v.push_back((rand() & 0x7fff) | ((rand() & 0x7fff) << 15));
  }
  v2 = v;
  cout << "Sorting five million integers..." << endl;
  cout.precision(3);

#define test(sort_function) {                            \
  clock_t start = clock();                               \
  sort_function(v.begin(), v.end());                     \
  double t = (double)(clock() - start) / CLOCKS_PER_SEC; \
  cout << setw(14) << left << #sort_function "(): ";     \
  cout << fixed << t << "s" << endl;                     \
  assert(sorted(v.begin(), v.end()));                    \
  v = v2;                                                \
}
  test(std::sort);
  test(quicksort);
  test(mergesort);
  test(heapsort);
  test(combsort);
  test(radix_sort);
  return 0;
}
\end{lstlisting}
\subsection{Array Rotation}
\begin{lstlisting}
/*

These functions are equivalent to std::rotate(), taking three iterators lo, mid,
and hi (lo <= mid <= hi) to perform a left rotation on the range [lo, hi). After
the function call, [lo, hi) will comprise of the concatenation of the elements
originally in [mid, hi) + [lo, mid). That is, the range [lo, hi) will be
rearranged in such a way that the element at mid becomes the first element of
the new range and the element at mid - 1 becomes the last element, all while
preserving the relative ordering of elements within the two rotated subarrays.

All three versions below achieve the same result using in-place algorithms.
Version 1 uses a straightforward swapping algorithm requiring ForwardIterators.
Version 2 requires BidirectionalIterators, employing a well-known trick with
three simple inversions. Version 3 requires random-access iterators, applying a
juggling algorithm which first divides the range into gcd(hi - lo, mid - lo)
sets and then rotates the corresponding elements in each set.

Time Complexity:
- O(n) per call to both functions, where n is the distance between lo and hi.

Space Complexity:
- O(1) auxiliary for all versions

*/

#include <algorithm>

template<class It>
void rotate1(It lo, It mid, It hi) {
  It next = mid;
  while (lo != next) {
    std::iter_swap(lo++, next++);
    if (next == hi) {
      next = mid;
    } else if (lo == mid) {
      mid = next;
    }
  }
}

template<class It>
void rotate2(It lo, It mid, It hi) {
  std::reverse(lo, mid);
  std::reverse(mid, hi);
  std::reverse(lo, hi);
}

int gcd(int a, int b) {
  return (b == 0) ? a : gcd(b, a % b);
}

template<class It>
void rotate3(It lo, It mid, It hi) {
  int n = hi - lo, jump = mid - lo;
  int g = gcd(jump, n), cycle = n / g;
  for (int i = 0; i < g; i++) {
    int curr = i, next;
    for (int j = 0; j < cycle - 1; j++) {
      next = curr + jump;
      if (next >= n) {
        next -= n;
      }
      std::iter_swap(lo + curr, lo + next);
      curr = next;
    }
  }
}

/*** Example Usage and Output:

before sort:  2 4 2 0 5 10 7 3 7 1
after sort:   0 1 2 2 3 4 5 7 7 10
rotate left:  1 2 2 3 4 5 7 7 10 0
rotate right: 0 1 2 2 3 4 5 7 7 10

***/

#include <algorithm>
#include <cassert>
#include <iostream>
#include <vector>
using namespace std;

int main() {
  vector<int> v0, v1, v2, v3;
  for (int i = 0; i < 10000; i++) {
    v0.push_back(i);
  }
  v1 = v2 = v3 = v0;
  int mid = 5678;
  std::rotate(v0.begin(), v0.begin() + mid, v0.end());
  rotate1(v1.begin(), v1.begin() + mid, v1.end());
  rotate2(v2.begin(), v2.begin() + mid, v2.end());
  rotate3(v3.begin(), v3.begin() + mid, v3.end());
  assert(v0 == v1 && v0 == v2 && v0 == v3);

  // Example from: http://en.cppreference.com/w/cpp/algorithm/rotate
  int a[] = {2, 4, 2, 0, 5, 10, 7, 3, 7, 1};
  vector<int> v(a, a + 10);
  cout << "before sort:  ";
  for (int i = 0; i < (int)v.size(); i++) {
    cout << v[i] << " ";
  }
  cout << endl;

  // Insertion sort.
  for (vector<int>::iterator i = v.begin(); i != v.end(); ++i) {
    rotate1(std::upper_bound(v.begin(), i, *i), i, i + 1);
  }
  cout << "after sort:   ";
  for (int i = 0; i < (int)v.size(); i++) {
    cout << v[i] << " ";
  }
  cout << endl;

  // Simple rotation to the left.
  rotate2(v.begin(), v.begin() + 1, v.end());
  cout << "rotate left:  ";
  for (int i = 0; i < (int)v.size(); i++) {
    cout << v[i] << " ";
  }
  cout << endl;

  // Simple rotation to the right.
  rotate3(v.rbegin(), v.rbegin() + 1, v.rend());
  cout << "rotate right: ";
  for (int i = 0; i < (int)v.size(); i++) {
    cout << v[i] << " ";
  }
  cout << endl;

  return 0;
}
\end{lstlisting}
\subsection{Counting Inversions}
\begin{lstlisting}
/*

The number of inversions for an array a[] is defined as the number of ordered
pairs (i, j) such that i < j and a[i] > a[j]. This is roughly how "close" an
array is to being sorted, but is *not* the minimum number of swaps required to
sort the array. If the array is sorted, then the inversion count is 0. If the
array is sorted in decreasing order, then the inversion count is maximal. The
following two functions are each techniques to efficiently count inversions.

- inversions(lo, hi) uses merge sort to return the number of inversions given
  two random-access iterators as a range [lo, hi). The input range will be
  sorted after the function call. This requires operator < to be defined on the
  iterators' value type.
- inversions(n, a[]) uses a power-of-two trick to return the number of
  inversions for an array a[] of n nonnegative integers. After calling the
  function, every value of a[] will be set to 0. The time and space complexity
  of this operation are functions of the magnitude of the maximum value in a[].
  To instead obtain a running time of O(n log n) on the number of elements,
  coordinate compression may be applied to a[] beforehand so that its maximum is
  strictly less than the length n itself.

Time Complexity:
- O(n log n) per call to inversion(lo, hi), where n is the distance between lo
  and hi.
- O(n log m) per call to inversions(n, a[]) where n is the distance between lo
  and hi and m is maximum value in a[].

Space Complexity:
- O(n) auxiliary space and O(log n) stack space for inversions(lo, hi).
- O(m) auxiliary heap space for inversions(n, a[]).

*/

#include <algorithm>
#include <iterator>
#include <vector>

template<class It>
long long inversions(It lo, It hi) {
  if (hi - lo < 2) {
    return 0;
  }
  It mid = lo + (hi - lo - 1)/2, a = lo, c = mid + 1;
  long long res = 0;
  res += inversions(lo, mid + 1);
  res += inversions(mid + 1, hi);
  typedef typename std::iterator_traits<It>::value_type T;
  std::vector<T> merged;
  while (a <= mid && c < hi) {
    if (*c < *a) {
      merged.push_back(*(c++));
      res += (mid - a) + 1;
    } else {
      merged.push_back(*(a++));
    }
  }
  if (a > mid) {
    for (It it = c; it != hi; ++it) {
      merged.push_back(*it);
    }
  } else {
    for (It it = a; it <= mid; ++it) {
      merged.push_back(*it);
    }
  }
  for (It it = lo; it != hi; ++it) {
    *it = merged[it - lo];
  }
  return res;
}

long long inversions(int n, int a[]) {
  int mx = 0;
  for (int i = 0; i < n; i++) {
    mx = std::max(mx, a[i]);
  }
  long long res = 0;
  std::vector<int> count(mx);
  while (mx > 0) {
    std::fill(count.begin(), count.end(), 0);
    for (int i = 0; i < n; i++) {
      if (a[i] % 2 == 0) {
        res += count[a[i] / 2];
      } else {
        count[a[i] / 2]++;
      }
    }
    mx = 0;
    for (int i = 0; i < n; i++) {
      mx = std::max(mx, a[i] /= 2);
    }
  }
  return res;
}

/*** Example Usage ***/

#include <cassert>

int main() {
  {
    int a[] = {6, 9, 1, 14, 8, 12, 3, 2};
    assert(inversions(a, a + 8) == 16);
  }
  {
    int a[] = {6, 9, 1, 14, 8, 12, 3, 2};
    assert(inversions(8, a) == 16);
  }
  return 0;
}
\end{lstlisting}
\subsection{Coordinate Compression}
\begin{lstlisting}
/*

Given two ForwardIterators as a range [lo, hi) of n numerical elements, reassign
each element in the range to an integer in [0, k), where k is the number of
distinct elements in the original range, while preserving the initial relative
ordering of elements. That is, if a[] is an array of the original values and b[]
is the compressed values, then every pair of indices i, j (0 <= i, j < n) shall
satisfy a[i] < a[j] if and only if b[i] < b[j].

Both implementations below require operator < to be defined on the iterator's
value type. Version 1 performs the compression by sorting the array, removing
duplicates, and binary searching for the position of each original value.
Version 2 achieves the same result by inserting all values in a balanced binary
search tree (std::map) which automatically removes duplicate values and supports
efficient lookups of the compressed values.

Time Complexity:
- O(n log n) per call to either function, where n is the distance between lo and
  hi.

Space Complexity:
- O(n) auxiliary heap space.

*/

#include <algorithm>
#include <iterator>
#include <map>
#include <vector>

template<class It> void compress1(It lo, It hi) {
  typedef typename std::iterator_traits<It>::value_type T;
  std::vector<T> v(lo, hi);
  std::sort(v.begin(), v.end());
  v.resize(std::unique(v.begin(), v.end()) - v.begin());
  for (It it = lo; it != hi; ++it) {
    *it = (int)(std::lower_bound(v.begin(), v.end(), *it) - v.begin());
  }
}

template<class It> void compress2(It lo, It hi) {
  typedef typename std::iterator_traits<It>::value_type T;
  std::map<T, int> m;
  for (It it = lo; it != hi; ++it) {
    m[*it] = 0;
  }
  typename std::map<T, int>::iterator it = m.begin();
  for (int id = 0; it != m.end(); it++) {
    it->second = id++;
  }
  for (It it = lo; it != hi; ++it) {
    *it = m[*it];
  }
}

/*** Example Usage and Output:

0 4 4 1 3 2 5 5
0 4 4 1 3 2 5 5
1 0 2 0 3 1

***/

#include <iostream>
using namespace std;

template<class It> void print_range(It lo, It hi) {
  while (lo != hi) {
    cout << *lo++ << " ";
  }
  cout << endl;
}

int main() {
  {
    int a[] = {1, 30, 30, 7, 9, 8, 99, 99};
    compress1(a, a + 8);
    print_range(a, a + 8);
  }
  {
    int a[] = {1, 30, 30, 7, 9, 8, 99, 99};
    compress2(a, a + 8);
    print_range(a, a + 8);
  }
  {  // Non-integral types work too, as long as ints can be assigned to them.
    double a[] = {0.5, -1.0, 3, -1.0, 20, 0.5};
    compress1(a, a + 6);
    print_range(a, a + 6);
  }
  return 0;
}
\end{lstlisting}
\subsection{Selection (Quickselect)}
\begin{lstlisting}
/*

nth_element2() is equivalent to std::nth_element(), taking random-access
iterators lo, nth, and hi as the range [lo, hi) to be partially sorted. The
values in [lo, hi) are rearranged such that the value pointed to by nth is the
element that would be there if the range were sorted. Furthermore, the range is
partitioned such that no value in [lo, nth) compares greater than the value
pointed to by nth and no value in (nth, hi) compares less. This implementation
requires operator < to be defined on the iterator's value type.

Time Complexity:
- O(n) on average per call to nth_element2(), where n is the distance between lo
  and hi.

Space Complexity:
- O(1) auxiliary.

*/

#include <algorithm>
#include <cstdlib>
#include <iterator>

int rand32() {
  return (rand() & 0x7fff) | ((rand() & 0x7fff) << 15);
}

template<class It>
void nth_element2(It lo, It nth, It hi) {
  for (;;) {
    std::iter_swap(lo + rand32() % (hi - lo), hi - 1);
    typename std::iterator_traits<It>::value_type mid = *(hi - 1);
    It k = lo - 1;
    for (It it = lo; it != hi; ++it) {
      if (!(mid < *it)) {
        std::iter_swap(++k, it);
      }
    }
    if (nth < k) {
      hi = k;
    } else if (k < nth) {
      lo = k + 1;
    } else {
      return;
    }
  }
}

/*** Example Usage and Output:

2 3 3 4 5 6 6 7 9

***/

#include <cassert>
#include <iostream>
using namespace std;

template<class It>
void print_range(It lo, It hi) {
  while (lo != hi) {
    cout << *lo++ << " ";
  }
  cout << endl;
}

int main () {
  int n = 9;
  int a[] = {5, 6, 4, 3, 2, 6, 7, 9, 3};
  nth_element2(a, a + n/2, a + n);
  assert(a[n/2] == 5);
  print_range(a, a + n);
  return 0;
}
\end{lstlisting}

\section{Array Queries}
\setcounter{section}{2}
\setcounter{subsection}{0}
\subsection{Longest Increasing Subsequence}
\begin{lstlisting}
/*

Given two random-access iterators lo and hi specifying a range [lo, hi),
determine a longest subsequence of the range such that all of its elements are
in strictly ascending order. This implementation requires operator < to be
defined on the iterator's value type. The subsequence is not necessarily
contiguous or unique, so only one such answer will be found. The answer is
computed using binary search and dynamic programming.

Time Complexity:
- O(n log n) per call to longest_increasing_subsequence(), where n is the
  distance between lo and hi.

Space Complexity:
- O(n) auxiliary heap space for longest_increasing_subsequence().

*/

#include <iterator>
#include <vector>

template<class It>
std::vector<typename std::iterator_traits<It>::value_type>
longest_increasing_subsequence(It lo, It hi) {
  int len = 0, n = hi - lo;
  std::vector<int> prev(n), tail(n);
  for (int i = 0; i < n; i++) {
    int l = -1, h = len;
    while (h - l > 1) {
      int mid = (l + h)/2;
      if (*(lo + tail[mid]) < *(lo + i)) {
        l = mid;
      } else {
        h = mid;
      }
    }
    if (len < h + 1) {
      len = h + 1;
    }
    prev[i] = h > 0 ? tail[h - 1] : -1;
    tail[h] = i;
  }
  std::vector<typename std::iterator_traits<It>::value_type> res(len);
  for (int i = tail[len - 1]; i != -1; i = prev[i]) {
    res[--len] = *(lo + i);
  }
  return res;
}

/*** Example Usage and Output:

-5 1 9 10 11 13

***/

#include <iostream>
using namespace std;

template<class It> void print_range(It lo, It hi) {
  while (lo != hi) {
    cout << *lo++ << " ";
  }
  cout << endl;
}

int main () {
  int a[] = {-2, -5, 1, 9, 10, 8, 11, 10, 13, 11};
  vector<int> res = longest_increasing_subsequence(a, a + 10);
  print_range(res.begin(), res.end());
  return 0;
}
\end{lstlisting}
\subsection{Maximal Subarray Sum (Kadane)}
\begin{lstlisting}
/*

Given an array of numbers (at least one of which must be positive), determine
the maximum possible sum of any contiguous subarray. Kadane's algorithm scans
through the array, at each index computing the maximum positive sum subarray
ending there. Either this subarray is empty (in which case its sum is zero) or
it consists of one more element than the maximum sequence ending at the previous
position. This can be adapted to compute the maximal submatrix sum as well.

- max_subarray_sum(lo, hi, &res_lo, &res_hi) returns the maximal subarray sum
  for the range [lo, hi), where lo and hi are random-access iterators to
  numeric types. This implementation requires operators + and < to be defined on
  the iterators' value type. Optionally, two int pointers may be passed to store
  the inclusive boundary indices [res_lo, res_hi] of the resulting subarray. By
  convention, an input range consisting of only negative values will yield a
  size 1 subarray consisting of the maximum value.
- max_submatrix_sum(matrix, &r1, &c1, &r2, &c2) returns the largest sum of any
  rectangular submatrix for a matrix of n rows by m columns. The matrix should
  be given as a 2-dimensional vector, where the outer vector must contain n
  vectors each of size m. This implementation requires operators + and < to be
  defined on the iterators' value type. Optionally, four int pointers may be
  passed to store the boundary indices of the resulting subarray, with (r1, c1)
  specifiying the top-left index and (r2, c2) specifying the bottom-right index.
  By convention, an input matrix consisting of only negative values will yield a
  size 1 submatrix consisting of the maximum value.

Time Complexity:
- O(n) per call to max_subarray_sum(), where n is the distance between lo and
  hi.
- O(n*m^2) per call to max_submatrix_sum(), where n is the number of rows and m
  is the number of columns in the matrix.

Space Complexity:
- O(1) auxiliary for max_subarray_sum().
- O(n) auxiliary heap space for max_submatrix_sum(), where n is the number of
  rows in the matrix.

*/

#include <algorithm>
#include <cstddef>
#include <iterator>
#include <limits>
#include <vector>

template<class It>
typename std::iterator_traits<It>::value_type
max_subarray_sum(It lo, It hi, int *res_lo = NULL, int *res_hi = NULL) {
  typedef typename std::iterator_traits<It>::value_type T;
  int curr_begin = 0, begin = 0, end = -1;
  T sum = 0, max_sum = std::numeric_limits<T>::min();
  for (It it = lo; it != hi; ++it) {
    sum += *it;
    if (sum < 0) {
      sum = 0;
      curr_begin = (it - lo) + 1;
    } else if (max_sum < sum) {
      max_sum = sum;
      begin = curr_begin;
      end = it - lo;
    }
  }
  if (end == -1) {  // All negative. By convention, return the maximum value.
    for (It it = lo; it != hi; ++it) {
      if (max_sum < *it) {
        max_sum = *it;
        begin = it - lo;
        end = begin;
      }
    }
  }
  if (res_lo != NULL && res_hi != NULL) {
    *res_lo = begin;
    *res_hi = end;
  }
  return max_sum;
}

template<class T>
T max_submatrix_sum(const std::vector<std::vector<T> > &matrix,
    int *r1 = NULL, int *c1 = NULL, int *r2 = NULL, int *c2 = NULL) {
  int n = matrix.size(), m = matrix[0].size();
  std::vector<T> sums(n);
  T sum, max_sum = std::numeric_limits<T>::min();
  for (int clo = 0; clo < m; clo++) {
    std::fill(sums.begin(), sums.end(), 0);
    for (int chi = clo; chi < m; chi++) {
      for (int i = 0; i < n; i++) {
        sums[i] += matrix[i][chi];
      }
      int rlo, rhi;
      sum = max_subarray_sum(sums.begin(), sums.end(), &rlo, &rhi);
      if (max_sum < sum) {
        max_sum = sum;
        if (r1 != NULL && c1 != NULL && r2 != NULL && c2 != NULL) {
          *r1 = rlo;
          *c1 = clo;
          *r2 = rhi;
          *c2 = chi;
        }
      }
    }
  }
  return max_sum;
}

/*** Example Usage and Output:

Maximal sum subarray:
4 -1 2 1

Maximal sum submatrix:
9 2
-4 1
-1 8

***/

#include <cassert>
#include <iostream>
using namespace std;

int main() {
  {
    int a[] = {-2, -1, -3, 4, -1, 2, 1, -5, 4};
    int lo = 0, hi = 0;
    assert(max_subarray_sum(a, a + 3) == -1);
    assert(max_subarray_sum(a, a + 9, &lo, &hi) == 6);
    cout << "Maximal sum subarray:" << endl;
    for (int i = lo; i <= hi; i++) {
      cout << a[i] << " ";
    }
    cout << endl;
  }
  {
    const int n = 4, m = 5;
    int a[n][m] = {{0, -2, -7, 0, 5},
                   {9, 2, -6, 2, -4},
                   {-4, 1, -4, 1, 0},
                   {-1, 8, 0, -2, 3}};
    vector<vector<int> > matrix(n);
    for (int i = 0; i < n; i++) {
      matrix[i] = vector<int>(a[i], a[i] + m);
    }
    int r1 = 0, c1 = 0, r2 = 0, c2 = 0;
    assert(max_submatrix_sum(matrix, &r1, &c1, &r2, &c2) == 15);
    cout << "\nMaximal sum submatrix:" << endl;
    for (int i = r1; i <= r2; i++) {
      for (int j = c1; j <= c2; j++) {
        cout << matrix[i][j] << " ";
      }
      cout << endl;
    }
  }
  return 0;
}
\end{lstlisting}
\subsection{Majority Element (Boyer-Moore)}
\begin{lstlisting}
/*

Given two ForwardIterators lo and hi specifying a range [lo, hi) of n elements,
return an iterator to the first occurrence of the majority element, or the
iterator hi if there is no majority element. The majority element is defined as
an element which occurs strictly more than floor(n/2) times in the range. This
implementation requires operator == to be defined on the iterator's value type.

Time Complexity:
- O(n) per call to majority(), where n is the size of the array.

Space Complexity:
- O(1) auxiliary.

*/

template<class It>
It majority(It lo, It hi) {
  int count = 0;
  It candidate = lo;
  for (It it = lo; it != hi; ++it) {
    if (count == 0) {
      candidate = it;
      count = 1;
    } else if (*it == *candidate) {
      count++;
    } else {
      count--;
    }
  }
  count = 0;
  for (It it = lo; it != hi; ++it) {
    if (*it == *candidate) {
      count++;
    }
  }
  if (count <= (hi - lo)/2) {
    return hi;
  }
  return candidate;
}

/*** Example Usage ***/

#include <cassert>

int main() {
  int a[] = {3, 2, 3, 1, 3};
  assert(*majority(a, a + 5) == 3);
  int b[] = {2, 3, 3, 3, 2, 1};
  assert(majority(b, b + 6) == b + 6);
  return 0;
}
\end{lstlisting}
\subsection{Subset Sum (Meet-in-the-Middle)}
\begin{lstlisting}
/*

Given random-access iterators lo and hi specifying a range [lo, hi) of integers,
return the minimum sum of any subset of the range that is greater than or equal
to a given integer v. This is a generalization of the NP-complete subset sum
problem, which asks whether a subset summing to 0 exists (equivalent in this
case to checking if v = 0 yields an answer of 0). This implementation uses a
meet-in-the-middle algorithm to precompute and search for a lower bound. Note
that 64-bit integers are used in intermediate calculations to avoid overflow.

Time Complexity:
- O(n*2^(n/2)) per call to sum_lower_bound(), where n is the distance between lo
  and hi.

Space Complexity:
- O(n) auxiliary heap space, where n is the number of array elements.

*/

#include <algorithm>
#include <limits>
#include <vector>

template<class It>
long long sum_lower_bound(It lo, It hi, long long v) {
  int n = hi - lo, llen = 1 << (n/2), hlen = 1 << (n - n/2);
  std::vector<long long> lsum(llen), hsum(hlen);
  for (int mask = 0; mask < llen; mask++) {
    for (int i = 0; i < n/2; i++) {
      if ((mask >> i) & 1) {
        lsum[mask] += *(lo + i);
      }
    }
  }
  for (int mask = 0; mask < hlen; mask++) {
    for (int i = 0; i < (n - n/2); i++) {
      if ((mask >> i) & 1) {
        hsum[mask] += *(lo + i + n/2);
      }
    }
  }
  std::sort(lsum.begin(), lsum.end());
  std::sort(hsum.begin(), hsum.end());
  int l = 0, h = hlen - 1;
  long long curr = std::numeric_limits<long long>::min();
  while (l < llen && h >= 0) {
    if (lsum[l] + hsum[h] <= v) {
      curr = std::max(curr, lsum[l] + hsum[h]);
      l++;
    } else {
      h--;
    }
  }
  return curr;
}

/*** Example Usage ***/

#include <cassert>

int main() {
  int a[] = {9, 1, 5, 0, 1, 11, 5};
  assert(sum_lower_bound(a, a + 7, 8) == 7);
  int b[] = {-7, -3, -2, 5, 8};
  assert(sum_lower_bound(b, b + 5, 0) == 0);
  return 0;
}
\end{lstlisting}
\subsection{Maximal Zero Submatrix}
\begin{lstlisting}
/*

Given a rectangular matrix with n rows and m columns consisting of only 0's and
1's as a two-dimensional vector of bool, return the area of the largest
rectangular submatrix consisting of only 0's. This solution uses a reduction to
the problem of finding the maximum rectangular area under a histogram, which is
efficiently solved using a stack algorithm.

Time Complexity:
- O(n*m) per call to max_zero_submatrix(), where n is the number of rows and m
  is the number of columns in the matrix.

Space Complexity:
- O(m) auxiliary heap space, where m is the number of columns in the matrix.

*/

#include <algorithm>
#include <stack>
#include <vector>

int max_zero_submatrix(const std::vector<std::vector<bool> > &matrix) {
  int n = matrix.size(), m = matrix[0].size(), res = 0;
  std::vector<int> d(m, -1), d1(m), d2(m);
  for (int r = 0; r < n; r++) {
    for (int c = 0; c < m; c++) {
      if (matrix[r][c]) {
        d[c] = r;
      }
    }
    std::stack<int> s;
    for (int c = 0; c < m; c++) {
      while (!s.empty() && d[s.top()] <= d[c]) {
        s.pop();
      }
      d1[c] = s.empty() ? -1 : s.top();
      s.push(c);
    }
    while (!s.empty()) {
      s.pop();
    }
    for (int c = m - 1; c >= 0; c--) {
      while (!s.empty() && d[s.top()] <= d[c]) {
        s.pop();
      }
      d2[c] = s.empty() ? m : s.top();
      s.push(c);
    }
    for (int j = 0; j < m; j++) {
      res = std::max(res, (r - d[j])*(d2[j] - d1[j] - 1));
    }
  }
  return res;
}

/*** Example Usage ***/

#include <cassert>
using namespace std;

int main() {
  const int n = 5, m = 6;
  bool a[n][m] = {{1, 0, 1, 1, 0, 0},
                  {1, 0, 0, 1, 0, 0},
                  {0, 0, 0, 0, 0, 1},
                  {1, 0, 0, 1, 0, 0},
                  {1, 0, 1, 0, 0, 1}};
  vector<vector<bool> > matrix(n);
  for (int i = 0; i < n; i++) {
    matrix[i] = vector<bool>(a[i], a[i] + m);
  }
  assert(max_zero_submatrix(matrix) == 6);
  return 0;
}
\end{lstlisting}

\section{Searching}
\setcounter{section}{3}
\setcounter{subsection}{0}
\subsection{Binary Search}
\begin{lstlisting}
/*

Binary search can be generally used to find the input value corresponding to any
output value of a monotonic (strictly non-increasing or strictly non-decreasing)
function in O(log n) time with respect to the domain size. This is a special
case of finding the exact point at which any given monotonic Boolean function
changes from true to false or vice versa. Unlike searching through an array,
discrete binary search is not restricted by available memory, making it useful
for handling infinitely large search spaces such as real number intervals.

- binary_search_first_true() takes two integers lo and hi as boundaries for the
  search space [lo, hi) (i.e. including lo, but excluding hi) and returns the
  smallest integer k in [lo, hi) for which the predicate pred(k) tests true. If
  pred(k) tests false for every k in [lo, hi), then hi is returned. This
  function must be used on a range in which there exists a constant k such that
  pred(x) tests false for every x in [lo, k) and true for every x in [k, hi).
- binary_search_last_true() takes two integers lo and hi as boundaries for the
  search space [lo, hi) (i.e. including lo, but excluding hi) and returns the
  largest integer k in [lo, hi) for which the predicate pred(k) tests true. If
  pred(k) tests false for every k in [lo, hi), then hi is returned. This
  function must be used on a range in which there exists a constant k such that
  pred(x) tests true for every x in [lo, k] and false for every x in (k, hi).
- fbinary_search() is the equivalent of binary_search_first_true() on floating
  point predicates. Since any interval of real numbers is dense, the exact
  target cannot be found due to floating point error. Instead, a value that is
  very close to the border between false and true is returned. The precision of
  the answer depends on the number of repetitions the function performs. Since
  each repetition bisects the search space, the absolute error of the answer is
  1/(2^n) times the distance between lo and hi after n repetitions. Although it
  is possible to control the error by looping while hi - lo is greater than an
  arbitrary epsilon, it is simpler to let the loop run for a desired number of
  iterations until floating point arithmetic break down. 100 iterations is
  usually sufficient, since the search space will be reduced to 2^-100 (roughly
  10^-30) times its original size. This implementation can be modified to find
  the "last true" point in the range by simply interchanging the assignments of
  lo and hi in the if-else statements.

Time Complexity:
- O(log n) calls will be made to pred() in binary_search_first_true() and
  binary_search_last_true(), where n is the distance between lo and hi.
- O(n) calls will be made to pred() in fbinary_search(), where n is the number
  of iterations performed.

Space Complexity:
- O(1) auxiliary for all operations.

*/

template<class Int, class IntPredicate>
Int binary_search_first_true(Int lo, Int hi, IntPredicate pred) {  // 000[1]11
  Int mid, _hi = hi;
  while (lo < hi) {
    mid = lo + (hi - lo)/2;
    if (pred(mid)) {
      hi = mid;
    } else {
      lo = mid + 1;
    }
  }
  if (!pred(lo)) {
    return _hi;  // All false.
  }
  return lo;
}

template<class Int, class IntPredicate>
Int binary_search_last_true(Int lo, Int hi, IntPredicate pred) {  // 11[1]000
  Int mid, _hi = hi--;
  while (lo < hi) {
    mid = lo + (hi - lo + 1)/2;
    if (pred(mid)) {
      lo = mid;
    } else {
      hi = mid - 1;
    }
  }
  if (!pred(lo)) {
    return _hi;  // All false.
  }
  return lo;
}

template<class DoublePredicate>
double fbinary_search(double lo, double hi, DoublePredicate pred) {  // 000[1]11
  double mid;
  for (int i = 0; i < 100; i++) {
    mid = (lo + hi)/2.0;
    if (pred(mid)) {
      hi = mid;
    } else {
      lo = mid;
    }
  }
  return lo;
}

/*** Example Usage ***/

#include <cassert>
#include <cmath>

// Simple predicate examples.
bool pred1(int x) { return x >= 3; }
bool pred2(int x) { return false; }
bool pred3(int x) { return x <= 5; }
bool pred4(int x) { return true; }
bool pred5(double x) { return x >= 1.2345; }

int main() {
  assert(binary_search_first_true(0, 7, pred1) == 3);
  assert(binary_search_first_true(0, 7, pred2) == 7);
  assert(binary_search_last_true(0, 7, pred3)  == 5);
  assert(binary_search_last_true(0, 7, pred4)  == 6);
  assert(fabs(fbinary_search(-10.0, 10.0, pred5) - 1.2345) < 1e-15);
  return 0;
}
\end{lstlisting}
\subsection{Ternary Search}
\begin{lstlisting}
/*

Given a unimodal function f(x) taking a single double argument, find its global
maximum or minimum point to a specified absolute error.

ternary_search_min() takes the domain [lo, hi] of a continuous function f(x) and
returns a number x such that f is strictly decreasing on the interval [lo, x]
and strictly increasing on the interval [x, hi]. For the function to be correct
and deterministic, such an x must exist and be unique.

ternary_search_max() takes the domain [lo, hi] of a continuous function f(x) and
returns a number x such that f is strictly increasing on the interval [lo, x]
and strictly decreasing on the interval [x, hi]. For the function to be correct
and deterministic, such an x must exist and be unique.

Time Complexity:
- O(log n) calls will be made to f(), where n is the distance between lo and hi
  divided by the specified absolute error (epsilon).

Space Complexity:
- O(1) auxiliary for both operations.

*/

template<class UnimodalFunction>
double ternary_search_min(double lo, double hi, UnimodalFunction f,
                          const double EPS = 1e-12) {
  double lthird, hthird;
  while (hi - lo > EPS) {
    lthird = lo + (hi - lo)/3;
    hthird = hi - (hi - lo)/3;
    if (f(lthird) < f(hthird)) {
      hi = hthird;
    } else {
      lo = lthird;
    }
  }
  return lo;
}

template<class UnimodalFunction>
double ternary_search_max(double lo, double hi, UnimodalFunction f,
                          const double EPS = 1e-12) {
  double lthird, hthird;
  while (hi - lo > EPS) {
    lthird = lo + (hi - lo)/3;
    hthird = hi - (hi - lo)/3;
    if (f(lthird) < f(hthird)) {
      lo = lthird;
    } else {
      hi = hthird;
    }
  }
  return hi;
}

/*** Example Usage ***/

#include <cassert>
#include <cmath>

bool equal(double a, double b) {
  return fabs(a - b) < 1e-7;
}

// Parabola opening up with vertex at (-2, -24).
double f1(double x) {
  return 3*x*x + 12*x - 12;
}

// Parabola opening down with vertex at (2/19, 8366/95) ~ (0.105, 88.063).
double f2(double x) {
  return -5.7*x*x + 1.2*x + 88;
}

// Absolute value function shifted to the right by 30 units.
double f3(double x) {
  return fabs(x - 30);
}

int main() {
  assert(equal(ternary_search_min(-1000, 1000, f1), -2));
  assert(equal(ternary_search_max(-1000, 1000, f2), 2.0/19));
  assert(equal(ternary_search_min(-1000, 1000, f3), 30));
  return 0;
}
\end{lstlisting}
\subsection{Hill Climbing}
\begin{lstlisting}
/*

Given a continuous function f(x, y) to double and a (possibly arbitrary) initial
guess (x0, y0), return a potential global minimum found through hill-climbing.
Optionally, two double pointers critical_x and critical_y may be passed to store
the input points to f at which the returned minimum value is attained.

Hill-climbing is a heuristic which starts at the guess, then considers taking
a single step in each of a fixed number of directions. The direction with the
best (in this case, minimum) value is chosen, and further steps are taken in it
until the answer no longer improves. When this happens, the step size is reduced
and the same process repeats until a desired absolute error is reached. The
technique's success heavily depends on the behavior of f and the initial guess.
Therefore, the result is not guaranteed to be the global minimum.

Time Complexity:
- O(d log n) call will be made to f, where d is the number of directions
  considered at each position and n is the search space that is approximately
  proportional to the maximum possible step size divided by the minimum possible
  step size.

Space Complexity:
- O(1) auxiliary.

*/

#include <cstddef>
#include <cmath>

template<class ContinuousFunction>
double find_min(ContinuousFunction f, double x0, double y0,
                double *critical_x = NULL, double *critical_y = NULL,
                const double STEP_MIN = 1e-9, const double STEP_MAX = 1e6,
                const int NUM_DIRECTIONS = 6) {
  static const double PI = acos(-1.0);
  double x = x0, y = y0, res = f(x0, y0);
  for (double step = STEP_MAX; step > STEP_MIN; ) {
    double best = res, best_x = x, best_y = y;
    bool found = false;
    for (int i = 0; i < NUM_DIRECTIONS; i++) {
      double a = 2.0*PI*i / NUM_DIRECTIONS;
      double x2 = x + step*cos(a), y2 = y + step*sin(a);
      double value = f(x2, y2);
      if (best > value) {
        best_x = x2;
        best_y = y2;
        best = value;
        found = true;
      }
    }
    if (!found) {
      step /= 2.0;
    } else {
      x = best_x;
      y = best_y;
      res = best;
    }
  }
  if (critical_x != NULL && critical_y != NULL) {
    *critical_x = x;
    *critical_y = y;
  }
  return res;
}

/*** Example Usage ***/

#include <cassert>
#include <cmath>

bool eq(double a, double b) {
  return fabs(a - b) < 1e-8;
}

// Paraboloid with global minimum at f(2, 3) = 0.
double f(double x, double y) {
  return (x - 2)*(x - 2) + (y - 3)*(y - 3);
}

int main() {
  double x, y;
  assert(eq(find_min(f, 0, 0, &x, &y), 0));
  assert(eq(x, 2) && eq(y, 3));
  return 0;
}
\end{lstlisting}
\subsection{Convex Hull Trick (Semi-Dynamic)}
\begin{lstlisting}
/*

Given a set of pairs (m, b) specifying lines of the form y = mx + b, process a
set of x-coordinate queries each asking to find the minimum y-value when any of
the given lines are evaluated at the specified x. For each add_line(m, b) call,
m must be the minimum m of all lines added so far. For each query(x) call, x
must be the maximum x of all queries made so far.

The following implementation is a concise, semi-dynamic version of the convex
hull optimization technique. It supports an an interlaced sequence of add_line()
and query() calls, as long as the preconditions of descending m and ascending x
are satisfied. As a result, it may be necessary to sort the lines and queries
before calling the functions. In that case, the overall time complexity will be
dominated by the sorting step.

Time Complexity:
- O(n) for any interlaced sequence of add_line() and query() calls, where n is
  the number of lines added. This is because the overall number of steps taken
  by add_line() and query() are respectively bounded by the number of lines.
  Thus a single call to either add_line() or query() will have an amortized O(1)
  running time.

Space Complexity:
- O(n) for storage of the lines.
- O(1) auxiliary for add_line() and query().

*/

#include <vector>

std::vector<long long> M, B;
int ptr = 0;

void add_line(long long m, long long b) {
  int len = M.size();
  while (len > 1 && (B[len - 2] - B[len - 1])*(m - M[len - 1]) >=
                    (B[len - 1] - b)*(M[len - 1] - M[len - 2])) {
    len--;
  }
  M.resize(len);
  B.resize(len);
  M.push_back(m);
  B.push_back(b);
}

long long query(long long x) {
  if (ptr >= (int)M.size()) {
    ptr = (int)M.size() - 1;
  }
  while (ptr + 1 < (int)M.size() &&
         M[ptr + 1]*x + B[ptr + 1] <= M[ptr]*x + B[ptr]) {
    ptr++;
  }
  return M[ptr]*x + B[ptr];
}

/*** Example Usage ***/

#include <cassert>

int main() {
  add_line(3, 0);
  add_line(2, 1);
  add_line(1, 2);
  add_line(0, 6);
  assert(query(0) == 0);
  assert(query(1) == 3);
  assert(query(2) == 4);
  assert(query(3) == 5);
  return 0;
}
\end{lstlisting}
\subsection{Convex Hull Trick (Fully-Dynamic)}
\begin{lstlisting}
/*

Given a set of pairs (m, b) specifying lines of the form y = mx + b, process a
set of x-coordinate queries each asking to find the minimum y-value when any of
the given lines are evaluated at the specified x. To instead have the queries
optimize for maximum y-value, call the constructor with query_max=true.

The following implementation is a fully dynamic variant of the convex hull
optimization technique, using a self-balancing binary search tree (std::set) to
support the ability to call add_line() and query() in any desired order.

Time Complexity:
- O(n) for any interlaced sequence of add_line() and query() calls, where n
  is the number of lines added. This is because the overall number of steps
  taken by add_line() and query() are respectively bounded by the number of
  lines. Thus a single call to either add_line() or query() will have an O(1)
  amortized running time.

Space Complexity:
- O(n) for storage of the lines.
- O(1) auxiliary for add_line() and query().

*/

#include <limits>
#include <set>

class hull_optimizer {
  struct line {
    long long m, b, value;
    double xlo;
    bool is_query, query_max;

    line(long long m, long long b, long long v, bool is_query, bool query_max)
        : m(m), b(b), value(v), xlo(-std::numeric_limits<double>::max()),
          is_query(is_query), query_max(query_max) {}

    double intersect(const line &l) const {
      if (m == l.m) {
        return std::numeric_limits<double>::max();
      }
      return (double)(l.b - b)/(m - l.m);
    }

    bool operator<(const line &l) const {
      if (l.is_query) {
        return query_max ? (xlo < l.value) : (l.value < xlo);
      }
      return m < l.m;
    }
  };

  std::set<line> hull;
  bool query_max;

  typedef std::set<line>::iterator hulliter;

  bool has_prev(hulliter it) const {
    return it != hull.begin();
  }

  bool has_next(hulliter it) const {
    return (it != hull.end()) && (++it != hull.end());
  }

  bool irrelevant(hulliter it) const {
    if (!has_prev(it) || !has_next(it)) {
      return false;
    }
    hulliter prev = it, next = it;
    --prev;
    ++next;
    return query_max ? (prev->intersect(*next) <= prev->intersect(*it))
                     : (next->intersect(*prev) <= next->intersect(*it));
  }

  hulliter update_left_border(hulliter it) {
    if ((query_max && !has_prev(it)) || (!query_max && !has_next(it))) {
      return it;
    }
    hulliter it2 = it;
    double value = it->intersect(query_max ? *--it2 : *++it2);
    line l(*it);
    l.xlo = value;
    hull.erase(it++);
    return hull.insert(it, l);
  }

 public:
  hull_optimizer(bool query_max = false) : query_max(query_max) {}

  void add_line(long long m, long long b) {
    line l(m, b, 0, false, query_max);
    hulliter it = hull.lower_bound(l);
    if (it != hull.end() && it->m == l.m) {
      if ((query_max && it->b < b) || (!query_max && b < it->b)) {
        hull.erase(it++);
      } else {
        return;
      }
    }
    it = hull.insert(it, l);
    if (irrelevant(it)) {
      hull.erase(it);
      return;
    }
    while (has_prev(it) && irrelevant(--it)) {
      hull.erase(it++);
    }
    while (has_next(it) && irrelevant(++it)) {
      hull.erase(it--);
    }
    it = update_left_border(it);
    if (has_prev(it)) {
      update_left_border(--it);
    }
    if (has_next(++it)) {
      update_left_border(++it);
    }
  }

  long long query(long long x) const {
    line q(0, 0, x, true, query_max);
    hulliter it = hull.lower_bound(q);
    if (query_max) {
      --it;
    }
    return it->m*x + it->b;
  }
};

/*** Example Usage ***/

#include <cassert>

int main() {
  hull_optimizer h;
  h.add_line(3, 0);
  h.add_line(0, 6);
  h.add_line(1, 2);
  h.add_line(2, 1);
  assert(h.query(0) == 0);
  assert(h.query(2) == 4);
  assert(h.query(1) == 3);
  assert(h.query(3) == 5);
  return 0;
}
\end{lstlisting}

\section{Cycle Detection}
\setcounter{section}{4}
\setcounter{subsection}{0}
\subsection{Cycle Detection (Floyd)}
\begin{lstlisting}
/*

Given a function f mapping a set of integers to itself and an x-coordinate in
the set, return a pair containing the (position, length) of a cycle in the
sequence of numbers obtained from repeatedly composing f with itself starting
with the initial x. Formally, since f maps a finite set S to itself, some value
is guaranteed to eventually repeat in the sequence:
  x[0], x[1]=f(x[0]), x[2]=f(x[1]), ..., x[n]=f(x[n - 1]), ...

There must exist a pair of indices i and j (i < j) such that x[i] = x[j]. When
this happens, the rest of the sequence will consist of the subsequence from x[i]
to x[j - 1] repeating indefinitely. The cycle detection problem asks to find
such an i, along with the length of the repeating subsequence. A well-known
special case is the problem of cycle-detection in a degenerate linked list.

Floyd's cycle-finding algorithm, a.k.a. the "tortoise and the hare algorithm",
is a space-efficient algorithm that moves two pointers through the sequence at
different speeds. Each step in the algorithm moves the "tortoise" one step
forward and the "hare" two steps forward in the sequence, comparing the sequence
values at each step. The first value which is simultaneously pointed to by both
pointers is the start of the sequence.

Time Complexity:
- O(m + n) per call to find_cycle_floyd(), where m is the smallest index of the
  sequence which is the beginning of a cycle, and n is the cycle's length.

Space Complexity:
- O(1) auxiliary.

*/

#include <utility>

template<class IntFunction>
std::pair<int, int> find_cycle_floyd(IntFunction f, int x0) {
  int tortoise = f(x0), hare = f(f(x0));
  while (tortoise != hare) {
    tortoise = f(tortoise);
    hare = f(f(hare));
  }
  int start = 0;
  tortoise = x0;
  while (tortoise != hare) {
    tortoise = f(tortoise);
    hare = f(hare);
    start++;
  }
  int length = 1;
  hare = f(tortoise);
  while (tortoise != hare) {
    hare = f(hare);
    length++;
  }
  return std::make_pair(start, length);
}

/*** Example Usage ***/

#include <cassert>
#include <set>
using namespace std;

int f(int x) {
  return (123*x*x + 4567890) % 1337;
}

void verify(int x0, int start, int length) {
  set<int> s;
  int x = x0;
  for (int i = 0; i < start; i++) {
    assert(!s.count(x));
    s.insert(x);
    x = f(x);
  }
  int startx = x;
  s.clear();
  for (int i = 0; i < length; i++) {
    assert(!s.count(x));
    s.insert(x);
    x = f(x);
  }
  assert(startx == x);
}

int main () {
  int x0 = 0;
  pair<int, int> res = find_cycle_floyd(f, x0);
  assert(res == make_pair(4, 2));
  verify(x0, res.first, res.second);
  return 0;
}
\end{lstlisting}
\subsection{Cycle Detection (Brent)}
\begin{lstlisting}
/*

Given a function f mapping a set of integers to itself and an x-coordinate in
the set, return a pair containing the (position, length) of a cycle in the
sequence of numbers obtained from repeatedly composing f with itself starting
with the initial x. Formally, since f maps a finite set S to itself, some value
is guaranteed to eventually repeat in the sequence:
  x[0], x[1]=f(x[0]), x[2]=f(x[1]), ..., x[n]=f(x[n - 1]), ...

There must exist a pair of indices i and j (i < j) such that x[i] = x[j]. When
this happens, the rest of the sequence will consist of the subsequence from x[i]
to x[j - 1] repeating indefinitely. The cycle detection problem asks to find
such an i, along with the length of the repeating subsequence. A well-known
special case is the problem of cycle-detection in a degenerate linked list.

While Floyd's cycle-finding algorithm finds cycles by simultaneously moving two
pointers at different speeds, Brent's algorithm keeps the tortoise pointer
stationary in every iteration, only teleporting it to the hare pointer at every
power of two. The smallest power of two at which they meet is the start of the
first cycle. This improves upon the constant factor of Floyd's algorithm by
reducing the number of calls made to f.

Time Complexity:
- O(m + n) per call to find_cycle_brent(), where m is the smallest index of the
  sequence which is the beginning of a cycle, and n is the cycle's length.

Space Complexity:
- O(1) auxiliary.

*/

#include <utility>

template<class IntFunction>
std::pair<int, int> find_cycle_brent(IntFunction f, int x0) {
  int power = 1, length = 1, tortoise = x0, hare = f(x0);
  while (tortoise != hare) {
    if (power == length) {
      tortoise = hare;
      power *= 2;
      length = 0;
    }
    hare = f(hare);
    length++;
  }
  hare = x0;
  for (int i = 0; i < length; i++) {
    hare = f(hare);
  }
  int start = 0;
  tortoise = x0;
  while (tortoise != hare) {
    tortoise = f(tortoise);
    hare = f(hare);
    start++;
  }
  return std::make_pair(start, length);
}

/*** Example Usage ***/

#include <cassert>
#include <set>
using namespace std;

int f(int x) {
  return (123*x*x + 4567890) % 1337;
}

void verify(int x0, int start, int length) {
  set<int> s;
  int x = x0;
  for (int i = 0; i < start; i++) {
    assert(!s.count(x));
    s.insert(x);
    x = f(x);
  }
  int startx = x;
  s.clear();
  for (int i = 0; i < length; i++) {
    assert(!s.count(x));
    s.insert(x);
    x = f(x);
  }
  assert(startx == x);
}

int main () {
  int x0 = 0;
  pair<int, int> res = find_cycle_brent(f, x0);
  assert(res == make_pair(4, 2));
  verify(x0, res.first, res.second);
  return 0;
}
\end{lstlisting}

\chapter{Data Structures}

\section{Heaps}
\setcounter{section}{1}
\setcounter{subsection}{0}
\subsection{Binary Heap}
\begin{lstlisting}
/*

Maintain a min-priority queue, that is, a collection of elements with support
for querying and extraction of the minimum. This implementation requires an
ordering on the set of possible elements defined by the < operator. A binary
min-heap implements a priority queue by inserting and deleting nodes into a
binary tree such that the parent of any node is always less than its children.

- binary_heap() constructs an empty priority queue.
- binary_heap(lo, hi) constructs a priority queue from two ForwardIterators,
  consisting of elements in the range [lo, hi).
- size() returns the size of the priority queue.
- empty() returns whether the priority queue is empty.
- push(v) inserts the value v into the priority queue.
- pop() removes the minimum element from the priority queue.
- top() returns the minimum element in the priority queue.

Time Complexity:
- O(1) per call to the first constructor, size(), empty(), and top().
- O(log n) per call to push() and pop(), where n is the number of elements
  in the priority queue.
- O(n) per call to the second constructor on the distance between lo and hi.

Space Complexity:
- O(n) for storage of the priority queue elements.
- O(1) auxiliary for all operations.

*/

#include <algorithm>
#include <stdexcept>
#include <vector>

template<class T>
class binary_heap {
  std::vector<T> heap;

 public:
  binary_heap() {}

  template<class It>
  binary_heap(It lo, It hi) {
    while (lo != hi) {
      push(*(lo++));
    }
  }

  int size() const {
    return heap.size();
  }

  bool empty() const {
    return heap.empty();
  }

  void push(const T &v) {
    heap.push_back(v);
    int i = heap.size() - 1;
    while (i > 0) {
      int parent = (i - 1)/2;
      if (!(heap[i] < heap[parent])) {
        break;
      }
      std::swap(heap[i], heap[parent]);
      i = parent;
    }
  }

  void pop() {
    if (heap.empty()) {
      throw std::runtime_error("Cannot pop from empty heap.");
    }
    heap[0] = heap.back();
    heap.pop_back();
    int i = 0;
    for (;;) {
      int child = 2*i + 1;
      if (child >= (int)heap.size()) {
        break;
      }
      if (child + 1 < (int)heap.size() && heap[child + 1] < heap[child]) {
        child++;
      }
      if (heap[child] < heap[i]) {
        std::swap(heap[i], heap[child]);
        i = child;
      } else {
        break;
      }
    }
  }

  T top() const {
    if (heap.empty()) {
      throw std::runtime_error("Cannot get top of empty heap.");
    }
    return heap[0];
  }
};

/*** Example Usage and Output:

-1
0
5
10
12

***/

#include <iostream>
using namespace std;

int main() {
  int a[] = {0, 5, -1, 12};
  binary_heap<int> h(a, a + 4);
  h.push(10);
  while (!h.empty()) {
    cout << h.top() << endl;
    h.pop();
  }
  return 0;
}
\end{lstlisting}
\subsection{Randomized Mergeable Heap}
\begin{lstlisting}
/*

Maintain a mergeable min-priority queue, that is, a collection of elements with
support for querying and extraction of the minimum as well as efficient merging
with other instances. This implementation requires an ordering on the set of
possible elements defined by the < operator. A randomized mergeable heap uses a
coin-toss to recursively combine nodes of two binary trees.

- randomized_heap() constructs an empty priority queue.
- randomized_heap(lo, hi) constructs a priority queue from two ForwardIterators,
  consisting of elements in the range [lo, hi).
- size() returns the size of the priority queue.
- empty() returns whether the priority queue is empty.
- push(v) inserts the value v into the priority queue.
- pop() removes the minimum element from the priority queue.
- top() returns the minimum element in the priority queue.
- absorb(h) inserts every value from h and sets h to the empty priority queue.

Time Complexity:
- O(1) per call to the first constructor, size(), empty(), and top().
- O(log n) expected worst case per call to push(), pop(), and absorb(), where n
  is the number of elements in the priority queue.
- O(n) per call to the second constructor on the distance between lo and hi.

Space Complexity:
- O(n) for storage of the priority queue elements.
- O(log n) auxiliary stack space for push(), pop(), and absorb().
- O(1) auxiliary for all other operations.

*/

#include <algorithm>
#include <cstddef>
#include <stdexcept>

template<class T>
class randomized_heap {
  struct node_t {
    T value;
    node_t *left, *right;

    node_t(const T &v) : value(v), left(NULL), right(NULL) {}
  } *root;

  int num_nodes;

  static node_t* merge(node_t *a, node_t *b) {
    if (a == NULL) {
      return b;
    }
    if (b == NULL) {
      return a;
    }
    if (b->value < a->value) {
      std::swap(a, b);
    }
    if (rand() % 2 == 0) {
      std::swap(a->left, a->right);
    }
    a->left = merge(a->left, b);
    return a;
  }

  static void clean_up(node_t *n) {
    if (n != NULL) {
      clean_up(n->left);
      clean_up(n->right);
      delete n;
    }
  }

 public:
  randomized_heap() : root(NULL), num_nodes(0) {}

  template<class It>
  randomized_heap(It lo, It hi) : root(NULL), num_nodes(0) {
    while (lo != hi) {
      push(*(lo++));
    }
  }

  ~randomized_heap() {
    clean_up(root);
  }

  int size() const {
    return num_nodes;
  }

  bool empty() const {
    return root == NULL;
  }

  void push(const T &v) {
    root = merge(root, new node_t(v));
    num_nodes++;
  }

  void pop() {
    if (empty()) {
      throw std::runtime_error("Cannot pop from empty heap.");
    }
    node_t *tmp = root;
    root = merge(root->left, root->right);
    delete tmp;
    num_nodes--;
  }

  T top() const {
    if (empty()) {
      throw std::runtime_error("Cannot get top of empty heap.");
    }
    return root->value;
  }

  void absorb(randomized_heap &h) {
    root = merge(root, h.root);
    h.root = NULL;
  }
};

/*** Example Usage and Output:

-1
0
5
10
12

***/

#include <iostream>
using namespace std;

int main() {
  randomized_heap<int> h, h2;
  h.push(12);
  h.push(10);
  h2.push(5);
  h2.push(-1);
  h2.push(0);
  h.absorb(h2);
  while (!h.empty()) {
    cout << h.top() << endl;
    h.pop();
  }
  return 0;
}
\end{lstlisting}
\subsection{Skew Heap}
\begin{lstlisting}
/*

Maintain a mergeable min-priority queue, that is, a collection of elements with
support for querying and extraction of the minimum as well as efficient merging
with other instances. This implementation requires an ordering on the set of
possible elements defined by the < operator. A skew heap attempts to maintain
balance by unconditionally swapping all nodes in the merge path when merging.

- skew_heap() constructs an empty priority queue.
- skew_heap(lo, hi) constructs a priority queue from two ForwardIterators,
  consisting of elements in the range [lo, hi).
- size() returns the size of the priority queue.
- empty() returns whether the priority queue is empty.
- push(v) inserts the value v into the priority queue.
- pop() removes the minimum element from the priority queue.
- top() returns the minimum element in the priority queue.
- absorb(h) inserts every value from h and sets h to the empty priority queue.

Time Complexity:
- O(1) per call to the first constructor, size(), empty(), and top().
- O(log n) amortized auxiliary per call to push(), pop(), and absorb(), where n
  is the number of elements in the priority queue.
- O(n) per call to the second constructor, where n is the distance between lo
  and hi.

Space Complexity:
- O(n) for storage of the priority queue elements.
- O(log n) amortized auxiliary stack space for push(), pop(), and absorb().
- O(1) auxiliary for all other operations.

*/

#include <algorithm>
#include <cstddef>
#include <stdexcept>

template<class T>
class skew_heap {
  struct node_t {
    T value;
    node_t *left, *right;

    node_t(const T &v) : value(v), left(NULL), right(NULL) {}
  } *root;

  int num_nodes;

  static node_t* merge(node_t *a, node_t *b) {
    if (a == NULL) {
      return b;
    }
    if (b == NULL) {
      return a;
    }
    if (b->value < a->value) {
      std::swap(a, b);
    }
    std::swap(a->left, a->right);
    a->left = merge(b, a->left);
    return a;
  }

  static void clean_up(node_t *n) {
    if (n != NULL) {
      clean_up(n->left);
      clean_up(n->right);
      delete n;
    }
  }

 public:
  skew_heap() : root(NULL), num_nodes(0) {}

  template<class It>
  skew_heap(It lo, It hi) : root(NULL), num_nodes(0) {
    while (lo != hi) {
      push(*(lo++));
    }
  }

  ~skew_heap() {
    clean_up(root);
  }

  int size() const {
    return num_nodes;
  }

  bool empty() const {
    return root == NULL;
  }

  void push(const T &v) {
    root = merge(root, new node_t(v));
    num_nodes++;
  }

  void pop() {
    if (empty()) {
      throw std::runtime_error("Cannot pop from empty heap.");
    }
    node_t *tmp = root;
    root = merge(root->left, root->right);
    delete tmp;
    num_nodes--;
  }

  T top() const {
    if (empty()) {
      throw std::runtime_error("Cannot get top of empty heap.");
    }
    return root->value;
  }

  void absorb(skew_heap &h) {
    root = merge(root, h.root);
    h.root = NULL;
  }
};

/*** Example Usage and Output:

-1
0
5
10
12

***/

#include <iostream>
using namespace std;

int main() {
  skew_heap<int> h, h2;
  h.push(12);
  h.push(10);
  h2.push(5);
  h2.push(-1);
  h2.push(0);
  h.absorb(h2);
  while (!h.empty()) {
    cout << h.top() << endl;
    h.pop();
  }
  return 0;
}
\end{lstlisting}
\subsection{Pairing Heap}
\begin{lstlisting}
/*

Maintain a mergeable min-priority queue, that is, a collection of elements with
support for querying and extraction of the minimum as well as efficient merging
with other instances. This implementation requires an ordering on the set of
possible elements defined by the < operator. A pairing heap is a heap-ordered
multi-way tree, using a two-pass merge to self-adjust during each deletion.

- pairing_heap() constructs an empty priority queue.
- pairing_heap(lo, hi) constructs a priority queue from two ForwardIterators,
  consisting of elements in the range [lo, hi).
- size() returns the size of the priority queue.
- empty() returns whether the priority queue is empty.
- push(v) inserts the value v into the priority queue.
- pop() removes the minimum element from the priority queue.
- top() returns the minimum element in the priority queue.
- absorb(h) inserts every value from h and sets h to the empty priority queue.

Time Complexity:
- O(1) per call to the first constructor, size(), empty(), top(), push(), and
  absorb().
- O(log n) amortized per call to pop().
- O(n) per call to the second constructor on the distance between lo and hi.

Space Complexity:
- O(n) for storage of the priority queue elements.
- O(log n) amortized auxiliary stack space for pop().
- O(1) auxiliary for all other operations.

*/

#include <cstddef>
#include <stdexcept>

template<class T>
class pairing_heap {
  struct node_t {
    T value;
    node_t *left, *next;

    node_t(const T &v) : value(v), left(NULL), next(NULL) {}

    void add_child(node_t *n) {
      if (left == NULL) {
        left = n;
      } else {
        n->next = left;
        left = n;
      }
    }
  } *root;

  int num_nodes;

  static node_t* merge(node_t *a, node_t *b) {
    if (a == NULL) {
      return b;
    }
    if (b == NULL) {
      return a;
    }
    if (a->value < b->value) {
      a->add_child(b);
      return a;
    }
    b->add_child(a);
    return b;
  }

  static node_t* merge_pairs(node_t *n) {
    if (n == NULL || n->next == NULL) {
      return n;
    }
    node_t *a = n, *b = n->next, *c = n->next->next;
    a->next = b->next = NULL;
    return merge(merge(a, b), merge_pairs(c));
  }

  static void clean_up(node_t *n) {
    if (n != NULL) {
      clean_up(n->left);
      clean_up(n->next);
      delete n;
    }
  }

 public:
  pairing_heap() : root(NULL), num_nodes(0) {}

  template<class It>
  pairing_heap(It lo, It hi) : root(NULL), num_nodes(0) {
    while (lo != hi) {
      push(*(lo++));
    }
  }

  ~pairing_heap() {
    clean_up(root);
  }

  int size() const {
    return num_nodes;
  }

  bool empty() const {
    return root == NULL;
  }

  void push(const T &v) {
    root = merge(root, new node_t(v));
    num_nodes++;
  }

  void pop() {
    if (empty()) {
      throw std::runtime_error("Cannot pop from empty heap.");
    }
    node_t *tmp = root;
    root = merge_pairs(root->left);
    delete tmp;
    num_nodes--;
  }

  T top() const {
    if (empty()) {
      throw std::runtime_error("Cannot get top of empty heap.");
    }
    return root->value;
  }

  void absorb(pairing_heap &h) {
    root = merge(root, h.root);
    h.root = NULL;
  }
};

/*** Example Usage and Output:

-1
0
5
10
12

***/

#include <iostream>
using namespace std;

int main() {
  pairing_heap<int> h, h2;
  h.push(12);
  h.push(10);
  h2.push(5);
  h2.push(-1);
  h2.push(0);
  h.absorb(h2);
  while (!h.empty()) {
    cout << h.top() << endl;
    h.pop();
  }
  return 0;
}
\end{lstlisting}

\section{Dictionaries}
\setcounter{section}{2}
\setcounter{subsection}{0}
\subsection{Binary Search Tree}
\begin{lstlisting}
/*

Maintain a map, that is, a collection of key-value pairs such that each possible
key appears at most once in the collection. This implementations requires an
ordering on the set of possible keys defined by the < operator on the key type.
A binary search tree implements this map by inserting and deleting keys into a
binary tree such that every node's left child has a lesser key and every node's
right child has a greater key.

- binary_search_tree() constructs an empty map.
- size() returns the size of the map.
- empty() returns whether the map is empty.
- insert(k, v) adds an entry with key k and value v to the map, returning true
  if an new entry was added or false if the key already exists (in which case
  the map is unchanged and the old value associated with the key is preserved).
- erase(k) removes the entry with key k from the map, returning true if the
  removal was successful or false if the key to be removed was not found.
- find(k) returns a pointer to a const value associated with key k, or NULL if
  the key was not found.
- walk(f) calls the function f(k, v) on each entry of the map, in ascending
  order of keys.

Time Complexity:
- O(1) per call to the constructor, size(), and empty().
- O(n) per call to insert(), erase(), find(), and walk(), where n is the number
  of nodes currently in the map.

Space Complexity:
- O(n) for storage of the map elements.
- O(n) auxiliary stack space for insert(), erase(), and walk().
- O(1) auxiliary for all other operations.

*/

#include <cstddef>

template<class K, class V>
class binary_search_tree {
  struct node_t {
    K key;
    V value;
    node_t *left, *right;

    node_t(const K &k, const V &v)
        : key(k), value(v), left(NULL), right(NULL) {}
  } *root;

  int num_nodes;

  static bool insert(node_t *&n, const K &k, const V &v) {
    if (n == NULL) {
      n = new node_t(k, v);
      return true;
    }
    if (k < n->key) {
      return insert(n->left, k, v);
    } else if (n->key < k) {
      return insert(n->right, k, v);
    }
    return false;
  }

  static bool erase(node_t *&n, const K &k) {
    if (n == NULL) {
      return false;
    }
    if (k < n->key) {
      return erase(n->left, k);
    } else if (n->key < k) {
      return erase(n->right, k);
    }
    if (n->left != NULL && n->right != NULL) {
      node_t *tmp = n->right, *parent = NULL;
      while (tmp->left != NULL) {
        parent = tmp;
        tmp = tmp->left;
      }
      n->key = tmp->key;
      n->value = tmp->value;
      if (parent != NULL) {
        return erase(parent->left, parent->left->key);
      }
      return erase(n->right, n->right->key);
    }
    node_t *tmp = (n->left != NULL) ? n->left : n->right;
    delete n;
    n = tmp;
    return true;
  }

  template<class KVFunction>
  static void walk(node_t *n, KVFunction f) {
    if (n != NULL) {
      walk(n->left, f);
      f(n->key, n->value);
      walk(n->right, f);
    }
  }

  static void clean_up(node_t *n) {
    if (n != NULL) {
      clean_up(n->left);
      clean_up(n->right);
      delete n;
    }
  }

 public:
  binary_search_tree() : root(NULL), num_nodes(0) {}

  ~binary_search_tree() {
    clean_up(root);
  }

  int size() const {
    return num_nodes;
  }

  bool empty() const {
    return root == NULL;
  }

  bool insert(const K &k, const V &v) {
    if (insert(root, k, v)) {
      num_nodes++;
      return true;
    }
    return false;
  }

  bool erase(const K &k) {
    if (erase(root, k)) {
      num_nodes--;
      return true;
    }
    return false;
  }

  const V* find(const K &k) const {
    node_t *n = root;
    while (n != NULL) {
      if (k < n->key) {
        n = n->left;
      } else if (n->key < k) {
        n = n->right;
      } else {
        return &(n->value);
      }
    }
    return NULL;
  }

  template<class KVFunction>
  void walk(KVFunction f) const {
    walk(root, f);
  }
};

/*** Example Usage and Output:

abcde
bcde

***/

#include <cassert>
#include <iostream>
using namespace std;

void printch(int k, char v) {
  cout << v;
}

int main() {
  binary_search_tree<int, char> t;
  t.insert(2, 'b');
  t.insert(1, 'a');
  t.insert(3, 'c');
  t.insert(5, 'e');
  assert(t.insert(4, 'd'));
  assert(*t.find(4) == 'd');
  assert(!t.insert(4, 'd'));
  t.walk(printch);
  cout << endl;
  assert(t.erase(1));
  assert(!t.erase(1));
  assert(t.find(1) == NULL);
  t.walk(printch);
  cout << endl;
  return 0;
}
\end{lstlisting}
\subsection{Treap}
\begin{lstlisting}
/*

Maintain a map, that is, a collection of key-value pairs such that each possible
key appears at most once in the collection. This implementations requires an
ordering on the set of possible keys defined by the < operator on the key type.
A treap is a binary search tree that is balanced by preserving a heap property
on the randomly generated priority value assigned to every node, thereby making
insertions and deletions run in O(log n) with high probability.

- treap() constructs an empty map.
- size() returns the size of the map.
- empty() returns whether the map is empty.
- insert(k, v) adds an entry with key k and value v to the map, returning true
  if an new entry was added or false if the key already exists (in which case
  the map is unchanged and the old value associated with the key is preserved).
- erase(k) removes the entry with key k from the map, returning true if the
  removal was successful or false if the key to be removed was not found.
- find(k) returns a pointer to a const value associated with key k, or NULL if
  the key was not found.
- walk(f) calls the function f(k, v) on each entry of the map, in ascending
  order of keys.

Time Complexity:
- O(1) per call to the constructor, size(), and empty().
- O(log n) on average per call to insert(), erase(), and find(), where n is the
  number of entries currently in the map.
- O(n) per call to walk().

Space Complexity:
- O(n) for storage of the map elements.
- O(log n) auxiliary stack space on average for insert(), erase(), and walk().
- O(1) auxiliary for all other operations.

*/

#include <cstdlib>

template<class K, class V>
class treap {
  struct node_t {
    static inline int rand32() {
      return (rand() & 0x7fff) | ((rand() & 0x7fff) << 15);
    }

    K key;
    V value;
    int priority;
    node_t *left, *right;

    node_t(const K &k, const V &v)
        : key(k), value(v), priority(rand32()), left(NULL), right(NULL) {}
  } *root;

  int num_nodes;

  static void rotate_left(node_t *&n) {
    node_t *tmp = n;
    n = n->right;
    tmp->right = n->left;
    n->left = tmp;
  }

  static void rotate_right(node_t *&n) {
    node_t *tmp = n;
    n = n->left;
    tmp->left = n->right;
    n->right = tmp;
  }

  static bool insert(node_t *&n, const K &k, const V &v) {
    if (n == NULL) {
      n = new node_t(k, v);
      return true;
    }
    if (k < n->key && insert(n->left, k, v)) {
      if (n->left->priority < n->priority) {
        rotate_right(n);
      }
      return true;
    }
    if (n->key < k && insert(n->right, k, v)) {
      if (n->right->priority < n->priority) {
        rotate_left(n);
      }
      return true;
    }
    return false;
  }

  static bool erase(node_t *&n, const K &k) {
    if (n == NULL) {
      return false;
    }
    if (k < n->key) {
      return erase(n->left, k);
    } else if (n->key < k) {
      return erase(n->right, k);
    }
    if (n->left != NULL && n->right != NULL) {
      if (n->left->priority < n->right->priority) {
        rotate_right(n);
        return erase(n->right, k);
      }
      rotate_left(n);
      return erase(n->left, k);
    }
    node_t *tmp = (n->left != NULL) ? n->left : n->right;
    delete n;
    n = tmp;
    return true;
  }

  template<class KVFunction>
  static void walk(node_t *n, KVFunction f) {
    if (n != NULL) {
      walk(n->left, f);
      f(n->key, n->value);
      walk(n->right, f);
    }
  }

  static void clean_up(node_t *n) {
    if (n != NULL) {
      clean_up(n->left);
      clean_up(n->right);
      delete n;
    }
  }

 public:
  treap() : root(NULL), num_nodes(0) {}

  ~treap() {
    clean_up(root);
  }

  int size() const {
    return num_nodes;
  }

  bool empty() const {
    return root == NULL;
  }

  bool insert(const K &k, const V &v) {
    if (insert(root, k, v)) {
      num_nodes++;
      return true;
    }
    return false;
  }

  bool erase(const K &k) {
    if (erase(root, k)) {
      num_nodes--;
      return true;
    }
    return false;
  }

  const V* find(const K &k) const {
    node_t *n = root;
    while (n != NULL) {
      if (k < n->key) {
        n = n->left;
      } else if (n->key < k) {
        n = n->right;
      } else {
        return &(n->value);
      }
    }
    return NULL;
  }

  template<class KVFunction>
  void walk(KVFunction f) const {
    walk(root, f);
  }
};

/*** Example Usage and Output:

abcde
bcde

***/

#include <cassert>
#include <iostream>
using namespace std;

void printch(int k, char v) {
  cout << v;
}

int main() {
  treap<int, char> t;
  t.insert(2, 'b');
  t.insert(1, 'a');
  t.insert(3, 'c');
  t.insert(5, 'e');
  assert(t.insert(4, 'd'));
  assert(*t.find(4) == 'd');
  assert(!t.insert(4, 'd'));
  t.walk(printch);
  cout << endl;
  assert(t.erase(1));
  assert(!t.erase(1));
  assert(t.find(1) == NULL);
  t.walk(printch);
  cout << endl;
  return 0;
}
\end{lstlisting}
\subsection{AVL Tree}
\begin{lstlisting}
/*

Maintain a map, that is, a collection of key-value pairs such that each possible
key appears at most once in the collection. This implementation requires an
ordering on the set of possible keys defined by the < operator on the key type.
An AVL tree is a binary search tree balanced by height, guaranteeing O(log n)
worst-case running time in insertions and deletions by making sure that the
heights of the left and right subtrees at every node differ by at most 1.

- avl_tree() constructs an empty map.
- size() returns the size of the map.
- empty() returns whether the map is empty.
- insert(k, v) adds an entry with key k and value v to the map, returning true
  if an new entry was added or false if the key already exists (in which case
  the map is unchanged and the old value associated with the key is preserved).
- erase(k) removes the entry with key k from the map, returning true if the
  removal was successful or false if the key to be removed was not found.
- find(k) returns a pointer to a const value associated with key k, or NULL if
  the key was not found.
- walk(f) calls the function f(k, v) on each entry of the map, in ascending
  order of keys.

Time Complexity:
- O(1) per call to the constructor, size(), and empty().
- O(log n) per call to insert(), erase(), and find(), where n is the number of
  entries currently in the map.
- O(n) per call to walk().

Space Complexity:
- O(n) for storage of the map elements.
- O(log n) auxiliary stack space for insert(), erase(), and walk().
- O(1) auxiliary for all other operations.

*/

#include <algorithm>
#include <cstddef>

template<class K, class V>
class avl_tree {
  struct node_t {
    K key;
    V value;
    int height;
    node_t *left, *right;

    node_t(const K &k, const V &v)
        : key(k), value(v), height(1), left(NULL), right(NULL) {}
  } *root;

  int num_nodes;

  static int height(node_t *n) {
    return (n != NULL) ? n->height : 0;
  }

  static void update_height(node_t *n) {
    if (n != NULL) {
      n->height = 1 + std::max(height(n->left), height(n->right));
    }
  }

  static void rotate_left(node_t *&n) {
    node_t *tmp = n;
    n = n->right;
    tmp->right = n->left;
    n->left = tmp;
    update_height(tmp);
    update_height(n);
  }

  static void rotate_right(node_t *&n) {
    node_t *tmp = n;
    n = n->left;
    tmp->left = n->right;
    n->right = tmp;
    update_height(tmp);
    update_height(n);
  }

  static int balance_factor(node_t *n) {
    return (n != NULL) ? (height(n->left) - height(n->right)) : 0;
  }

  static void rebalance(node_t *&n) {
    if (n == NULL) {
      return;
    }
    update_height(n);
    int bf = balance_factor(n);
    if (bf > 1 && balance_factor(n->left) >= 0) {
      rotate_right(n);
    } else if (bf > 1 && balance_factor(n->left) < 0) {
      rotate_left(n->left);
      rotate_right(n);
    } else if (bf < -1 && balance_factor(n->right) <= 0) {
      rotate_left(n);
    } else if (bf < -1 && balance_factor(n->right) > 0) {
      rotate_right(n->right);
      rotate_left(n);
    }
  }

  static bool insert(node_t *&n, const K &k, const V &v) {
    if (n == NULL) {
      n = new node_t(k, v);
      return true;
    }
    if ((k < n->key && insert(n->left, k, v)) ||
        (n->key < k && insert(n->right, k, v))) {
      rebalance(n);
      return true;
    }
    return false;
  }

  static bool erase(node_t *&n, const K &k) {
    if (n == NULL) {
      return false;
    }
    if (!(k < n->key || n->key < k)) {
      if (n->left != NULL && n->right != NULL) {
        node_t *tmp = n->right, *parent = NULL;
        while (tmp->left != NULL) {
          parent = tmp;
          tmp = tmp->left;
        }
        n->key = tmp->key;
        n->value = tmp->value;
        if (parent != NULL) {
          if (!erase(parent->left, parent->left->key)) {
            return false;
          }
        } else if (!erase(n->right, n->right->key)) {
          return false;
        }
      } else {
        node_t *tmp = (n->left != NULL) ? n->left : n->right;
        delete n;
        n = tmp;
      }
      rebalance(n);
      return true;
    }
    if ((k < n->key && erase(n->left, k)) ||
        (n->key < k && erase(n->right, k))) {
      rebalance(n);
      return true;
    }
    return false;
  }

  template<class KVFunction>
  static void walk(node_t *n, KVFunction f) {
    if (n != NULL) {
      walk(n->left, f);
      f(n->key, n->value);
      walk(n->right, f);
    }
  }

  static void clean_up(node_t *n) {
    if (n != NULL) {
      clean_up(n->left);
      clean_up(n->right);
      delete n;
    }
  }

 public:
  avl_tree() : root(NULL), num_nodes(0) {}

  ~avl_tree() {
    clean_up(root);
  }

  int size() const {
    return num_nodes;
  }

  bool empty() const {
    return root == NULL;
  }

  bool insert(const K &k, const V &v) {
    if (insert(root, k, v)) {
      num_nodes++;
      return true;
    }
    return false;
  }

  bool erase(const K &k) {
    if (erase(root, k)) {
      num_nodes--;
      return true;
    }
    return false;
  }

  const V* find(const K &k) const {
    node_t *n = root;
    while (n != NULL) {
      if (k < n->key) {
        n = n->left;
      } else if (n->key < k) {
        n = n->right;
      } else {
        return &(n->value);
      }
    }
    return NULL;
  }

  template<class KVFunction>
  void walk(KVFunction f) const {
    walk(root, f);
  }
};

/*** Example Usage and Output:

abcde
bcde

***/

#include <cassert>
#include <iostream>
using namespace std;

void printch(int k, char v) {
  cout << v;
}

int main() {
  avl_tree<int, char> t;
  t.insert(2, 'b');
  t.insert(1, 'a');
  t.insert(3, 'c');
  t.insert(5, 'e');
  assert(t.insert(4, 'd'));
  assert(*t.find(4) == 'd');
  assert(!t.insert(4, 'd'));
  t.walk(printch);
  cout << endl;
  assert(t.erase(1));
  assert(!t.erase(1));
  assert(t.find(1) == NULL);
  t.walk(printch);
  cout << endl;
  return 0;
}
\end{lstlisting}
\subsection{Red-Black Tree}
\begin{lstlisting}
/*

Maintain a map, that is, a collection of key-value pairs such that each possible
key appears at most once in the collection. This implementation requires an
ordering on the set of possible keys defined by the < operator on the key type.
A red black tree is a binary search tree balanced by coloring its nodes red or
black, then constraining node colors on any simple path from the root to a leaf.

- red_black_tree() constructs an empty map.
- size() returns the size of the map.
- empty() returns whether the map is empty.
- insert(k, v) adds an entry with key k and value v to the map, returning true
  if an new entry was added or false if the key already exists (in which case
  the map is unchanged and the old value associated with the key is preserved).
- erase(k) removes the entry with key k from the map, returning true if the
  removal was successful or false if the key to be removed was not found.
- find(k) returns a pointer to a const value associated with key k, or NULL if
  the key was not found.
- walk(f) calls the function f(k, v) on each entry of the map, in ascending
  order of keys.

Time Complexity:
- O(1) per call to the constructor, size(), and empty().
- O(log n) per call to insert(), erase(), and find(), where n is the number of
  entries currently in the map.
- O(n) per call to walk().

Space Complexity:
- O(n) for storage of the map elements.
- O(log n) auxiliary stack space for walk().
- O(1) auxiliary for all other operations.

*/

#include <algorithm>
#include <cstddef>

template<class K, class V>
class red_black_tree {
  enum color_t { RED, BLACK };
  struct node_t {
    K key;
    V value;
    color_t color;
    node_t *left, *right, *parent;

    node_t(const K &k, const V &v, color_t c)
        : key(k), value(v), color(c), left(NULL), right(NULL), parent(NULL) {}
  } *root, *LEAF_NIL;

  int num_nodes;

  void rotate_left(node_t *n) {
    node_t *tmp = n->right;
    if ((n->right = tmp->left) != LEAF_NIL) {
      n->right->parent = n;
    }
    if ((tmp->parent = n->parent) == LEAF_NIL) {
      root = tmp;
    } else if (n->parent->left == n) {
      n->parent->left = tmp;
    } else {
      n->parent->right = tmp;
    }
    tmp->left = n;
    n->parent = tmp;
  }

  void rotate_right(node_t *n) {
    node_t *tmp = n->left;
    if ((n->left = tmp->right) != LEAF_NIL) {
      n->left->parent = n;
    }
    if ((tmp->parent = n->parent) == LEAF_NIL) {
      root = tmp;
    } else if (n->parent->right == n) {
      n->parent->right = tmp;
    } else {
      n->parent->left = tmp;
    }
    tmp->right = n;
    n->parent = tmp;
  }

  void insert_fix(node_t *n) {
    while (n->parent->color == RED) {
      node_t *parent = n->parent;
      node_t *grandparent = n->parent->parent;
      if (parent == grandparent->left) {
        node_t *uncle = grandparent->right;
        if (uncle->color == RED) {
          grandparent->color = RED;
          parent->color = BLACK;
          uncle->color = BLACK;
          n = grandparent;
        } else {
          if (n == parent->right) {
            rotate_left(parent);
            n = parent;
            parent = n->parent;
          }
          rotate_right(grandparent);
          std::swap(parent->color, grandparent->color);
          n = parent;
        }
      } else if (parent == grandparent->right) {
        node_t *uncle = grandparent->left;
        if (uncle->color == RED) {
          grandparent->color = RED;
          parent->color = BLACK;
          uncle->color = BLACK;
          n = grandparent;
        } else {
          if (n == parent->left) {
            rotate_right(parent);
            n = parent;
            parent = n->parent;
          }
          rotate_left(grandparent);
          std::swap(parent->color, grandparent->color);
          n = parent;
        }
      }
    }
    root->color = BLACK;
  }

  void replace(node_t *n, node_t *replacement) {
    if (n->parent == LEAF_NIL) {
      root = replacement;
    } else if (n == n->parent->left) {
      n->parent->left = replacement;
    } else {
      n->parent->right = replacement;
    }
    replacement->parent = n->parent;
  }

  void erase_fix(node_t *n) {
    while (n != root && n->color == BLACK) {
      node_t *parent = n->parent;
      if (n == parent->left) {
        node_t *sibling = parent->right;
        if (sibling->color == RED) {
          sibling->color = BLACK;
          parent->color = RED;
          rotate_left(parent);
          sibling = parent->right;
        }
        if (sibling->left->color == BLACK && sibling->right->color == BLACK) {
          sibling->color = RED;
          n = parent;
        } else {
          if (sibling->right->color == BLACK) {
            sibling->left->color = BLACK;
            sibling->color = RED;
            rotate_right(sibling);
            sibling = parent->right;
          }
          sibling->color = parent->color;
          parent->color = BLACK;
          sibling->right->color = BLACK;
          rotate_left(parent);
          n = root;
        }
      } else {
        node_t *sibling = parent->left;
        if (sibling->color == RED) {
          sibling->color = BLACK;
          parent->color = RED;
          rotate_right(parent);
          sibling = parent->left;
        }
        if (sibling->left->color == BLACK && sibling->right->color == BLACK) {
          sibling->color = RED;
          n = parent;
        } else {
          if (sibling->left->color == BLACK) {
            sibling->right->color = BLACK;
            sibling->color = RED;
            rotate_left(sibling);
            sibling = parent->left;
          }
          sibling->color = parent->color;
          parent->color = BLACK;
          sibling->left->color = BLACK;
          rotate_right(parent);
          n = root;
        }
      }
    }
    n->color = BLACK;
  }

  template<class KVFunction>
  void walk(node_t *n, KVFunction f) const {
    if (n != LEAF_NIL) {
      walk(n->left, f);
      f(n->key, n->value);
      walk(n->right, f);
    }
  }

  void clean_up(node_t *n) {
    if (n != LEAF_NIL) {
      clean_up(n->left);
      clean_up(n->right);
      delete n;
    }
  }

 public:
  red_black_tree() : num_nodes(0) {
    root = LEAF_NIL = new node_t(K(), V(), BLACK);
  }

  ~red_black_tree() {
    clean_up(root);
    delete LEAF_NIL;
  }

  int size() const {
    return num_nodes;
  }

  bool empty() const {
    return num_nodes == 0;
  }

  bool insert(const K &k, const V &v) {
    node_t *curr = root, *prev = LEAF_NIL;
    while (curr != LEAF_NIL) {
      prev = curr;
      if (k < curr->key) {
        curr = curr->left;
      } else if (curr->key < k) {
        curr = curr->right;
      } else {
        return false;
      }
    }
    node_t *n = new node_t(k, v, RED);
    n->parent = prev;
    if (prev == LEAF_NIL) {
      root = n;
    } else if (k < prev->key) {
      prev->left = n;
    } else {
      prev->right = n;
    }
    n->left = n->right = LEAF_NIL;
    insert_fix(n);
    num_nodes++;
    return true;
  }

  bool erase(const K &k) {
    node_t *n = root;
    while (n != LEAF_NIL) {
      if (k < n->key) {
        n = n->left;
      } else if (n->key < k) {
        n = n->right;
      } else {
        break;
      }
    }
    if (n == LEAF_NIL) {
      return false;
    }
    color_t color = n->color;
    node_t *replacement;
    if (n->left == LEAF_NIL) {
      replacement = n->right;
      replace(n, n->right);
    } else if (n->right == LEAF_NIL) {
      replacement = n->left;
      replace(n, n->left);
    } else {
      node_t *tmp = n->right;
      while (tmp->left != LEAF_NIL) {
        tmp = tmp->left;
      }
      color = tmp->color;
      replacement = tmp->right;
      if (tmp->parent == n) {
        replacement->parent = tmp;
      } else {
        replace(tmp, tmp->right);
        tmp->right = n->right;
        tmp->right->parent = tmp;
      }
      replace(n, tmp);
      tmp->left = n->left;
      tmp->left->parent = tmp;
      tmp->color = n->color;
    }
    delete n;
    if (color == BLACK) {
      erase_fix(replacement);
    }
    return true;
  }

  const V* find(const K &k) const {
    node_t *n = root;
    while (n != LEAF_NIL) {
      if (k < n->key) {
        n = n->left;
      } else if (n->key < k) {
        n = n->right;
      } else {
        return &(n->value);
      }
    }
    return NULL;
  }

  template<class KVFunction>
  void walk(KVFunction f) const {
    walk(root, f);
  }
};

/*** Example Usage and Output:

abcde
bcde

***/

#include <cassert>
#include <iostream>
using namespace std;

void printch(int k, char v) {
  cout << v;
}

int main() {
  red_black_tree<int, char> t;
  t.insert(2, 'b');
  t.insert(1, 'a');
  t.insert(3, 'c');
  t.insert(5, 'e');
  assert(t.insert(4, 'd'));
  assert(*t.find(4) == 'd');
  assert(!t.insert(4, 'd'));
  t.walk(printch);
  cout << endl;
  assert(t.erase(1));
  assert(!t.erase(1));
  assert(t.find(1) == NULL);
  t.walk(printch);
  cout << endl;
  return 0;
}
\end{lstlisting}
\subsection{Splay Tree}
\begin{lstlisting}
/*

Maintain a map, that is, a collection of key-value pairs such that each possible
key appears at most once in the collection. This implementation requires an
ordering on the set of possible keys defined by the < operator on the key type.
A splay tree is a balanced binary search tree with the additional property that
recently accessed elements are quick to access again.

- splay_tree() constructs an empty map.
- size() returns the size of the map.
- empty() returns whether the map is empty.
- insert(k, v) adds an entry with key k and value v to the map, returning true
  if an new entry was added or false if the key already exists (in which case
  the map is unchanged and the old value associated with the key is preserved).
- erase(k) removes the entry with key k from the map, returning true if the
  removal was successful or false if the key to be removed was not found.
- find(k) returns a pointer to a const value associated with key k, or NULL if
  the key was not found.
- walk(f) calls the function f(k, v) on each entry of the map, in ascending
  order of keys.

Time Complexity:
- O(1) per call to the constructor, size(), and empty().
- O(log n) per call to insert(), erase(), and find(), where n is the number of
  entries currently in the map.
- O(n) per call to walk().

Space Complexity:
- O(n) for storage of the map elements.
- O(log n) auxiliary stack space for insert(), erase(), and walk().
- O(1) auxiliary for all other operations.

*/

#include <cstddef>

template<class K, class V>
class splay_tree {
  struct node_t {
    K key;
    V value;
    node_t *left, *right;

    node_t(const K &k, const V &v)
        : key(k), value(v), left(NULL), right(NULL) {}
  } *root;

  int num_nodes;

  static void rotate_left(node_t *&n) {
    node_t *tmp = n;
    n = n->right;
    tmp->right = n->left;
    n->left = tmp;
  }

  static void rotate_right(node_t *&n) {
    node_t *tmp = n;
    n = n->left;
    tmp->left = n->right;
    n->right = tmp;
  }

  static void splay(node_t *&n, const K &k) {
    if (n == NULL) {
      return;
    }
    if (k < n->key && n->left != NULL) {
      if (k < n->left->key) {
        splay(n->left->left, k);
        rotate_right(n);
      } else if (n->left->key < k) {
        splay(n->left->right, k);
        if (n->left->right != NULL) {
          rotate_left(n->left);
        }
      }
      if (n->left != NULL) {
        rotate_right(n);
      }
    } else if (n->key < k && n->right != NULL) {
      if (k < n->right->key) {
        splay(n->right->left, k);
        if (n->right->left != NULL) {
          rotate_right(n->right);
        }
      } else if (n->right->key < k) {
        splay(n->right->right, k);
        rotate_left(n);
      }
      if (n->right != NULL) {
        rotate_left(n);
      }
    }
  }

  static bool insert(node_t *&n, const K &k, const V &v) {
    if (n == NULL) {
      n = new node_t(k, v);
      return true;
    }
    splay(n, k);
    if (k < n->key) {
      node_t *tmp = new node_t(k, v);
      tmp->left = n->left;
      tmp->right = n;
      n->left = NULL;
      n = tmp;
    } else if (n->key < k) {
      node_t *tmp = new node_t(k, v);
      tmp->left = n;
      tmp->right = n->right;
      n->right = NULL;
      n = tmp;
    } else {
      return false;
    }
    return true;
  }

  static bool erase(node_t *&n, const K &k) {
    if (n == NULL) {
      return false;
    }
    splay(n, k);
    if (k < n->key || n->key < k) {
      return false;
    }
    node_t *tmp = n;
    if (n->left == NULL) {
      n = n->right;
    } else {
      splay(n->left, k);
      n = n->left;
      n->right = tmp->right;
    }
    delete tmp;
    return true;
  }

  template<class KVFunction>
  static void walk(node_t *n, KVFunction f) {
    if (n != NULL) {
      walk(n->left, f);
      f(n->key, n->value);
      walk(n->right, f);
    }
  }

  static void clean_up(node_t *n) {
    if (n != NULL) {
      clean_up(n->left);
      clean_up(n->right);
      delete n;
    }
  }

 public:
  splay_tree() : root(NULL), num_nodes(0) {}

  ~splay_tree() {
    clean_up(root);
  }

  int size() const {
    return num_nodes;
  }

  bool empty() const {
    return root == NULL;
  }

  bool insert(const K &k, const V &v) {
    if (insert(root, k, v)) {
      num_nodes++;
      return true;
    }
    return false;
  }

  bool erase(const K &k) {
    if (erase(root, k)) {
      num_nodes--;
      return true;
    }
    return false;
  }

  const V* find(const K &k) {
    splay(root, k);
    return (k < root->key || root->key < k) ? NULL : &(root->value);
  }

  template<class KVFunction>
  void walk(KVFunction f) const {
    walk(root, f);
  }
};

/*** Example Usage and Output:

abcde
bcde

***/

#include <cassert>
#include <iostream>
using namespace std;

void printch(int k, char v) {
  cout << v;
}

int main() {
  splay_tree<int, char> t;
  t.insert(2, 'b');
  t.insert(1, 'a');
  t.insert(3, 'c');
  t.insert(5, 'e');
  assert(t.insert(4, 'd'));
  assert(*t.find(4) == 'd');
  assert(!t.insert(4, 'd'));
  t.walk(printch);
  cout << endl;
  assert(t.erase(1));
  assert(!t.erase(1));
  assert(t.find(1) == NULL);
  t.walk(printch);
  cout << endl;
  return 0;
}
\end{lstlisting}
\subsection{Size Balanced Tree}
\begin{lstlisting}
/*

Maintain a map, that is, a collection of key-value pairs such that each possible
key appears at most once in the collection. In addition, support queries for
keys given their ranks as well as queries for the ranks of given keys. This
implementation requires an ordering on the set of possible keys defined by the
< operator on the key type. A size balanced tree augments each nodes with the
size of its subtree, using it to maintain balance and compute order statistics.

- size_balanced_tree() constructs an empty map.
- size() returns the size of the map.
- empty() returns whether the map is empty.
- insert(k, v) adds an entry with key k and value v to the map, returning true
  if an new entry was added or false if the key already exists (in which case
  the map is unchanged and the old value associated with the key is preserved).
- erase(k) removes the entry with key k from the map, returning true if the
  removal was successful or false if the key to be removed was not found.
- find(k) returns a pointer to a const value associated with key k, or NULL if
  the key was not found.
- select(r) returns a key-value pair of the node with a key of zero-based rank r
  in the map, throwing an exception if the rank is not between 0 and size() - 1.
- rank(k) returns the zero-based rank of key k in the map, throwing an
  exception if the key was not found in the map.
- walk(f) calls the function f(k, v) on each entry of the map, in ascending
  order of keys.

Time Complexity:
- O(1) per call to the constructor, size(), and empty().
- O(log n) per call to insert(), erase(), find(), select(), and rank(), where n
  is the number of entries currently in the map.
- O(n) per call to walk().

Space Complexity:
- O(n) for storage of the map elements.
- O(log n) auxiliary stack space for insert(), erase(), and walk().
- O(1) auxiliary for all other operations.

*/

#include <cstddef>
#include <stdexcept>
#include <utility>

template<class K, class V>
class size_balanced_tree {
  struct node_t {
    K key;
    V value;
    int size;
    node_t *left, *right;

    node_t(const K &k, const V &v)
        : key(k), value(v), size(1), left(NULL), right(NULL) {}

    inline node_t*& child(int c) {
      return (c == 0) ? left : right;
    }

    void update() {
      size = 1;
      if (left != NULL) {
        size += left->size;
      }
      if (right != NULL) {
        size += right->size;
      }
    }
  } *root;

  static inline int size(node_t *n) {
    return (n == NULL) ? 0 : n->size;
  }

  static void rotate(node_t *&n, int c) {
    node_t *tmp = n->child(c);
    n->child(c) = tmp->child(!c);
    tmp->child(!c) = n;
    n->update();
    tmp->update();
    n = tmp;
  }

  static void maintain(node_t *&n, int c) {
    if (n == NULL || n->child(c) == NULL) {
      return;
    }
    node_t *&tmp = n->child(c);
    if (size(tmp->child(c)) > size(n->child(!c))) {
      rotate(n, c);
    } else if (size(tmp->child(!c)) > size(n->child(!c))) {
      rotate(tmp, !c);
      rotate(n, c);
    } else {
      return;
    }
    maintain(n->left, 0);
    maintain(n->right, 1);
    maintain(n, 0);
    maintain(n, 1);
  }

  static bool insert(node_t *&n, const K &k, const V &v) {
    if (n == NULL) {
      n = new node_t(k, v);
      return true;
    }
    bool result;
    if (k < n->key) {
      result = insert(n->left, k, v);
      maintain(n, 0);
    } else if (n->key < k) {
      result = insert(n->right, k, v);
      maintain(n, 1);
    } else {
      return false;
    }
    n->update();
    return result;
  }

  static bool erase(node_t *&n, const K &k) {
    if (n == NULL) {
      return false;
    }
    bool result;
    int c = (k < n->key);
    if (k < n->key) {
      result = erase(n->left, k);
    } else if (n->key < k) {
      result = erase(n->right, k);
    } else {
      if (n->right == NULL || n->left == NULL) {
        node_t *tmp = n;
        n = (n->right == NULL) ? n->left : n->right;
        delete tmp;
        return true;
      }
      node_t *p = n->right;
      while (p->left != NULL) {
        p = p->left;
      }
      n->key = p->key;
      result = erase(n->right, p->key);
    }
    maintain(n, c);
    n->update();
    return result;
  }

  static std::pair<K, V> select(node_t *n, int r) {
    int rank = size(n->left);
    if (r < rank) {
      return select(n->left, r);
    } else if (r > rank) {
      return select(n->right, r - rank - 1);
    }
    return std::make_pair(n->key, n->value);
  }

  static int rank(node_t *n, const K &k) {
    if (n == NULL) {
      throw std::runtime_error("Cannot rank key that's not in tree.");
    }
    int r = size(n->left);
    if (k < n->key) {
      return rank(n->left, k);
    } else if (n->key < k) {
      return rank(n->right, k) + r + 1;
    }
    return r;
  }

  template<class KVFunction>
  static void walk(node_t *n, KVFunction f) {
    if (n != NULL) {
      walk(n->left, f);
      f(n->key, n->value);
      walk(n->right, f);
    }
  }

  static void clean_up(node_t *n) {
    if (n != NULL) {
      clean_up(n->left);
      clean_up(n->right);
      delete n;
    }
  }

 public:
  size_balanced_tree() : root(NULL) {}

  ~size_balanced_tree() {
    clean_up(root);
  }

  int size() const {
    return size(root);
  }

  bool empty() const {
    return root == NULL;
  }

  bool insert(const K &k, const V &v) {
    return insert(root, k, v);
  }

  bool erase(const K &k) {
    return erase(root, k);
  }

  const V* find(const K &k) const {
    node_t *n = root;
    while (n != NULL) {
      if (k < n->key) {
        n = n->left;
      } else if (n->key < k) {
        n = n->right;
      } else {
        return &(n->value);
      }
    }
    return NULL;
  }

  std::pair<K, V> select(int r) const {
    if (r < 0 || r >= size(root)) {
      throw std::runtime_error("Select rank must be between 0 and size() - 1.");
    }
    return select(root, r);
  }

  int rank(const K &k) const {
    return rank(root, k);
  }

  template<class KVFunction>
  void walk(KVFunction f) const {
    walk(root, f);
  }
};

/*** Example Usage and Output:

abcde
bcde

***/

#include <cassert>
#include <iostream>
using namespace std;

void printch(int k, char v) {
  cout << v;
}

int main() {
  size_balanced_tree<int, char> t;
  t.insert(2, 'b');
  t.insert(1, 'a');
  t.insert(3, 'c');
  t.insert(5, 'e');
  assert(t.insert(4, 'd'));
  assert(*t.find(4) == 'd');
  assert(!t.insert(4, 'd'));
  t.walk(printch);
  cout << endl;
  assert(t.erase(1));
  assert(!t.erase(1));
  assert(t.find(1) == NULL);
  t.walk(printch);
  cout << endl;
  assert(t.rank(2) == 0);
  assert(t.rank(3) == 1);
  assert(t.rank(5) == 3);
  assert(t.select(0).first == 2);
  assert(t.select(1).first == 3);
  assert(t.select(2).first == 4);
  return 0;
}
\end{lstlisting}
\subsection{Interval Treap}
\begin{lstlisting}
/*

Maintain a map from closed, one-dimensional intervals to values while supporting
efficient reporting of any or all entries that intersect with a given query
interval. This implementation uses std::pair to represent intervals, requiring
operators < and == to be defined on the numeric key type. A treap is used to
process the entries, where keys are compared lexicographically as pairs.

- interval_treap() constructs an empty map.
- size() returns the size of the map.
- empty() returns whether the map is empty.
- insert(lo, hi, v) adds an entry with key [lo, hi] and value v to the map,
  returning true if a new interval was added or false if the interval already
  exists (in which case the map is unchanged and the old value associated with
  the key is preserved).
- erase(lo, hi) removes the entry with key [lo, hi] from the map, returning true
  if the removal was successful or false if the interval was not found.
- find_key(lo, hi) returns a pointer to a const std::pair representing the key
  of some interval in the map which intersects with [lo, hi], or NULL if no such
  entry was found.
- find_value(lo, hi) returns a pointer to a const value of some entry in the map
  with a key that intersects with [lo, hi], or NULL if no such entry was found.
- find_all(lo, hi, f) calls the function f(lo, hi, v) on each entry in the map
  that overlaps with [lo, hi], in lexicographically ascending order of intervals.
- walk(f) calls the function f(lo, hi, v) on each interval in the map, in
  lexicographically ascending order of intervals.

Time Complexity:
- O(1) per call to the constructor, size(), and empty().
- O(log n) on average per call to insert(), erase(), and find_any(), where n is
  the number of intervals currently in the set.
- O(log n + m) on average per call to find_all(), where m is the number of
  intersecting intervals that are reported.
- O(n) per call to walk().

Space Complexity:
- O(n) for storage of the map elements.
- O(1) auxiliary for size() and empty().
- O(log n) auxiliary stack space on average for all other operations.

*/

#include <cstdlib>
#include <utility>

template<class K, class V>
class interval_treap {
  typedef std::pair<K, K> interval_t;

  struct node_t {
    static inline int rand32() {
      return (rand() & 0x7fff) | ((rand() & 0x7fff) << 15);
    }

    interval_t interval;
    V value;
    K max;
    int priority;
    node_t *left, *right;

    node_t(const interval_t &i, const V &v)
        : interval(i), value(v), max(i.second), priority(rand32()), left(NULL),
          right(NULL) {}

    void update() {
      max = interval.second;
      if (left != NULL && left->max > max) {
        max = left->max;
      }
      if (right != NULL && right->max > max) {
        max = right->max;
      }
    }
  } *root;

  int num_nodes;

  static void rotate_left(node_t *&n) {
    node_t *tmp = n;
    n = n->right;
    tmp->right = n->left;
    n->left = tmp;
    tmp->update();
  }

  static void rotate_right(node_t *&n) {
    node_t *tmp = n;
    n = n->left;
    tmp->left = n->right;
    n->right = tmp;
    tmp->update();
  }

  static bool insert(node_t *&n, const interval_t &i, const V &v) {
    if (n == NULL) {
      n = new node_t(i, v);
      return true;
    }
    if (i < n->interval && insert(n->left, i, v)) {
      if (n->left->priority < n->priority) {
        rotate_right(n);
      }
      n->update();
      return true;
    }
    if (i > n->interval && insert(n->right, i, v)) {
      if (n->right->priority < n->priority) {
        rotate_left(n);
      }
      n->update();
      return true;
    }
    return false;
  }

  static bool erase(node_t *&n, const interval_t &i) {
    if (n == NULL) {
      return false;
    }
    if (i < n->interval) {
      return erase(n->left, i);
    }
    if (i > n->interval) {
      return erase(n->right, i);
    }
    if (n->left != NULL && n->right != NULL) {
      bool res;
      if (n->left->priority < n->right->priority) {
        rotate_right(n);
        res = erase(n->right, i);
      } else {
        rotate_left(n);
        res = erase(n->left, i);
      }
      n->update();
      return res;
    }
    node_t *tmp = (n->left != NULL) ? n->left : n->right;
    delete n;
    n = tmp;
    return true;
  }

  static node_t* find_any(node_t *n, const interval_t &i) {
    if (n == NULL) {
      return NULL;
    }
    if (n->interval.first <= i.second && i.first <= n->interval.second) {
      return n;
    }
    if (n->left != NULL && i.first <= n->left->max) {
      return find_any(n->left, i);
    }
    return find_any(n->right, i);
  }

  template<class KVFunction>
  static void find_all(node_t *n, const interval_t &i, KVFunction f) {
    if (n == NULL || n->max < i.first) {
      return;
    }
    if (n->interval.first <= i.second && i.first <= n->interval.second) {
      f(n->interval.first, n->interval.second, n->value);
    }
    find_all(n->left, i, f);
    find_all(n->right, i, f);
  }

  template<class KVFunction>
  static void walk(node_t *n, KVFunction f) {
    if (n != NULL) {
      walk(n->left, f);
      f(n->interval.first, n->interval.second, n->value);
      walk(n->right, f);
    }
  }

  static void clean_up(node_t *n) {
    if (n != NULL) {
      clean_up(n->left);
      clean_up(n->right);
      delete n;
    }
  }

 public:
  interval_treap() : root(NULL), num_nodes(0) {}

  ~interval_treap() {
    clean_up(root);
  }

  int size() const {
    return num_nodes;
  }

  bool empty() const {
    return root == NULL;
  }

  bool insert(const K &lo, const K &hi, const V &v) {
    if (insert(root, std::make_pair(lo, hi), v)) {
      num_nodes++;
      return true;
    }
    return false;
  }

  bool erase(const K &lo, const K &hi) {
    if (erase(root, std::make_pair(lo, hi))) {
      num_nodes--;
      return true;
    }
    return false;
  }

  const interval_t* find_key(const K &lo, const K &hi) const {
    node_t *n = find_any(root, std::make_pair(lo, hi));
    return (n == NULL) ? NULL : &(n->interval);
  }

  const V* find_value(const K &lo, const K &hi) const {
    node_t *n = find_any(root, std::make_pair(lo, hi));
    return (n == NULL) ? NULL : &(n->value);
  }

  template<class KVFunction>
  void find_all(const K &lo, const K &hi, KVFunction f) const {
    find_all(root, std::make_pair(lo, hi), f);
  }

  template<class KVFunction>
  void walk(KVFunction f) const {
    walk(root, f);
  }
};

/*** Example Usage and Output:

Intervals intersecting [16, 20]: [15, 20] [10, 30] [5, 20] [10, 40]
All intervals: [5, 20] [10, 30] [10, 40] [12, 15] [15, 20]

***/

#include <cassert>
#include <iostream>
using namespace std;

void print(int lo, int hi, char v) {
  cout << " [" << lo << ", " << hi << "]";
}

int main() {
  interval_treap<int, char> t;
  t.insert(15, 20, 'a');
  t.insert(10, 30, 'b');
  t.insert(17, 19, 'c');
  t.insert(5, 20, 'd');
  t.insert(12, 15, 'e');
  t.insert(10, 40, 'f');
  assert(t.size() == 6);
  assert(!t.insert(5, 20, 'x'));
  t.erase(17, 19);
  assert(t.size() == 5);
  assert(*t.find_key(3, 9) == make_pair(5, 20));
  assert(*t.find_value(3, 9) == 'd');
  cout << "Intervals intersecting [16, 20]:";
  t.find_all(16, 20, print);
  cout << "\nAll intervals:";
  t.walk(print);
  cout << endl;
  return 0;
}
\end{lstlisting}
\subsection{Hash Map}
\begin{lstlisting}
/*

Maintain a map, that is, a collection of key-value pairs such that each possible
key appears at most once in the collection. This implementation requires the ==
operator to be defined on the key type. A hash map implements a map by hashing
keys into buckets using a hash function. This implementation resolves collisions
by chaining entries hashed to the same bucket into a linked list.

- hash_map() constructs an empty map.
- size() returns the size of the map.
- empty() returns whether the map is empty.
- insert(k, v) adds an entry with key k and value v to the map, returning true
  if an new entry was added or false if the key already exists (in which case
  the map is unchanged and the old value associated with the key is preserved).
- erase(k) removes the entry with key k from the map, returning true if the
  removal was successful or false if the key to be removed was not found.
- find(k) returns a pointer to a const value associated with key k, or NULL if
  the key was not found.
- operator[k] returns a reference to key k's associated value (which may be
  modified), or if necessary, inserts and returns a new entry with the default
  constructed value if key k was not originally found.
- walk(f) calls the function f(k, v) on each entry of the map, in no guaranteed
  order.

Time Complexity:
- O(1) per call to the constructor, size(), and empty().
- O(1) amortized per call to insert(), erase(), find(), and operator[].
- O(n) per call to walk(), where n is the number of entries in the map.

Space Complexity:
- O(n) for storage of the map elements.
- O(n) auxiliary heap space for insert().
- O(1) auxiliary for all other operations.

*/

#include <cstddef>
#include <list>

template<class K, class V, class Hash>
class hash_map {
  struct entry_t {
    K key;
    V value;

    entry_t(const K &k, const V &v) : key(k), value(v) {}
  };

  std::list<entry_t> *table;
  int table_size, num_entries;

  void double_capacity_and_rehash() {
    std::list<entry_t> *old = table;
    int old_size = table_size;
    table_size = 2*table_size;
    table = new std::list<entry_t>[table_size];
    num_entries = 0;
    typename std::list<entry_t>::iterator it;
    for (int i = 0; i < old_size; i++) {
      for (it = old[i].begin(); it != old[i].end(); ++it) {
        insert(it->key, it->value);
      }
    }
    delete[] old;
  }

 public:
  hash_map(int size = 128) : table_size(size), num_entries(0) {
    table = new std::list<entry_t>[table_size];
  }

  ~hash_map() {
    delete[] table;
  }

  int size() const {
    return num_entries;
  }

  bool empty() const {
    return num_entries == 0;
  }

  bool insert(const K &k, const V &v) {
    if (find(k) != NULL) {
      return false;
    }
    if (num_entries >= table_size) {
      double_capacity_and_rehash();
    }
    unsigned int i = Hash()(k) % table_size;
    table[i].push_back(entry_t(k, v));
    num_entries++;
    return true;
  }

  bool erase(const K &k) {
    unsigned int i = Hash()(k) % table_size;
    typename std::list<entry_t>::iterator it = table[i].begin();
    while (it != table[i].end() && !(it->key == k)) {
      ++it;
    }
    if (it == table[i].end()) {
      return false;
    }
    table[i].erase(it);
    num_entries--;
    return true;
  }

  V* find(const K &k) const {
    unsigned int i = Hash()(k) % table_size;
    typename std::list<entry_t>::iterator it = table[i].begin();
    while (it != table[i].end() && !(it->key == k)) {
      ++it;
    }
    if (it == table[i].end()) {
      return NULL;
    }
    return &(it->value);
  }

  V& operator[](const K &k) {
    V *ret = find(k);
    if (ret != NULL) {
      return *ret;
    }
    insert(k, V());
    return *find(k);
  }

  template<class KVFunction>
  void walk(KVFunction f) const {
    for (int i = 0; i < table_size; i++) {
      typename std::list<entry_t>::iterator it;
      for (it = table[i].begin(); it != table[i].end(); ++it) {
        f(it->key, it->value);
      }
    }
  }
};

/*** Example Usage and Output:

cab

***/

#include <cassert>
#include <iostream>
using namespace std;

struct class_hash {
  unsigned int operator()(int k) {
    return class_hash()((unsigned int)k);
  }

  unsigned int operator()(long long k) {
    return class_hash()((unsigned long long)k);
  }

  // Knuth's one-to-one multiplicative method.
  unsigned int operator()(unsigned int k) {
    return k * 2654435761u;  // Or just return k.
  }

  // Jenkins's 64-bit hash.
  unsigned int operator()(unsigned long long k) {
    k += ~(k << 32);
    k ^=  (k >> 22);
    k += ~(k << 13);
    k ^=  (k >>  8);
    k +=  (k <<  3);
    k ^=  (k >> 15);
    k += ~(k << 27);
    k ^=  (k >> 31);
    return k;
  }

  // Jenkins's one-at-a-time hash.
  unsigned int operator()(const std::string &k) {
    unsigned int hash = 0;
    for (unsigned int i = 0; i < k.size(); i++) {
      hash += ((hash + k[i]) << 10);
      hash ^= (hash >> 6);
    }
    hash += (hash << 3);
    hash ^= (hash >> 11);
    return hash + (hash << 15);
  }
};

void printch(const string &k, char v) {
  cout << v;
}

int main() {
  hash_map<string, char, class_hash> m;
  m["foo"] = 'a';
  m.insert("bar", 'b');
  assert(m["foo"] == 'a');
  assert(m["bar"] == 'b');
  assert(m["baz"] == '\0');
  m["baz"] = 'c';
  m.walk(printch);
  cout << endl;
  assert(m.erase("foo"));
  assert(m.size() == 2);
  assert(m["foo"] == '\0');
  assert(m.size() == 3);
  return 0;
}
\end{lstlisting}
\subsection{Skip List}
\begin{lstlisting}
/*

Maintain a map, that is, a collection of key-value pairs such that each possible
key appears at most once in the collection. This implementation requires both
the < and the == operators to be defined on the key type. A skip list maintains
a linked hierarchy of sorted subsequences with each successive subsequence
skipping over fewer elements than the previous one.

- skip_list() constructs an empty map.
- size() returns the size of the map.
- empty() returns whether the map is empty.
- insert(k, v) adds an entry with key k and value v to the map, returning true
  if an new entry was added or false if the key already exists (in which case
  the map is unchanged and the old value associated with the key is preserved).
- erase(k) removes the entry with key k from the map, returning true if the
  removal was successful or false if the key to be removed was not found.
- find(k) returns a pointer to a const value associated with key k, or NULL if
  the key was not found.
- operator[k] returns a reference to key k's associated value (which may be
  modified), or if necessary, inserts and returns a new entry with the default
  constructed value if key k was not originally found.
- walk(f) calls the function f(k, v) on each entry of the map, in ascending
  order of keys.

Time Complexity:
- O(1) per call to the constructor, size(), and empty().
- O(log n) on average per call to insert(), erase(), find(), and operator[],
  where n is the number of entries currently in the map.
- O(n) per call to walk().

Space Complexity:
- O(n) on average for storage of the map elements.
- O(n) auxiliary heap space for insert() and erase().
- O(1) auxiliary for all other operations.

*/

#include <cmath>
#include <cstdlib>
#include <vector>

template<class K, class V>
class skip_list {
  static const int MAX_LEVELS = 32;  // log2(max possible keys)

  struct node_t {
    K key;
    V value;
    std::vector<node_t*> next;

    node_t(const K &k, const V &v, int levels)
        : key(k), value(v), next(levels, (node_t*)NULL) {}
  } *head;

  int num_nodes;

  static int random_level() {
    static const double p = 0.5;
    int level = 1;
    while (((double)rand() / RAND_MAX) < p && std::abs(level) < MAX_LEVELS) {
      level++;
    }
    return std::abs(level);
  }

  static int node_level(const std::vector<node_t*> &v) {
    int i = 0;
    while (i < (int)v.size() && v[i] != NULL) {
      i++;
    }
    return i + 1;
  }

 public:
  skip_list() : head(new node_t(K(), V(), MAX_LEVELS)), num_nodes(0) {
    for (int i = 0; i < (int)head->next.size(); i++) {
      head->next[i] = NULL;
    }
  }

  ~skip_list() {
    delete head;
  }

  int size() const {
    return num_nodes;
  }

  bool empty() const {
    return num_nodes == 0;
  }

  bool insert(const K &k, const V &v) {
    std::vector<node_t*> update(head->next);
    int curr_level = node_level(update);
    node_t *n = head;
    for (int i = curr_level; i-- > 0; ) {
      while (n->next[i] != NULL && n->next[i]->key < k) {
        n = n->next[i];
      }
      update[i] = n;
    }
    n = n->next[0];
    if (n != NULL && n->key == k) {
      return false;
    }
    int new_level = random_level();
    if (new_level > curr_level) {
      for (int i = curr_level; i < new_level; i++) {
        update[i] = head;
      }
    }
    n = new node_t(k, v, new_level);
    for (int i = 0; i < new_level; i++) {
      n->next[i] = update[i]->next[i];
      update[i]->next[i] = n;
    }
    num_nodes++;
    return true;
  }

  bool erase(const K &k) {
    std::vector<node_t*> update(head->next);
    node_t *n = head;
    for (int i = node_level(update); i-- > 0; ) {
      while (n->next[i] != NULL && n->next[i]->key < k) {
        n = n->next[i];
      }
      update[i] = n;
    }
    n = n->next[0];
    if (n != NULL && n->key == k) {
      for (int i = 0; i < (int)update.size(); i++) {
        if (update[i]->next[i] != n) {
          break;
        }
        update[i]->next[i] = n->next[i];
      }
      delete n;
      num_nodes--;
      return true;
    }
    return false;
  }

  V* find(const K &k) const {
    node_t *n = head;
    for (int i = node_level(n->next); i-- > 0; ) {
      while (n->next[i] != NULL && n->next[i]->key < k) {
        n = n->next[i];
      }
    }
    n = n->next[0];
    return (n != NULL && n->key == k) ? &(n->value) : NULL;
  }

  V& operator[](const K &k) {
    V *ret = find(k);
    if (ret != NULL) {
      return *ret;
    }
    insert(k, V());
    return *find(k);
  }

  template<class KVFunction>
  void walk(KVFunction f) const {
    node_t *n = head->next[0];
    while (n != NULL) {
      f(n->key, n->value);
      n = n->next[0];
    }
  }
};

/*** Example Usage and Output:

abcde
bcde

***/

#include <cassert>
#include <iostream>
using namespace std;

void printch(int k, char v) {
  cout << v;
}

int main() {
  skip_list<int, char> l;
  l.insert(2, 'b');
  l.insert(1, 'a');
  l.insert(3, 'c');
  l.insert(5, 'e');
  assert(l.insert(4, 'd'));
  assert(*l.find(4) == 'd');
  assert(!l.insert(4, 'd'));
  l.walk(printch);
  cout << endl;
  assert(l.erase(1));
  assert(!l.erase(1));
  assert(l.find(1) == NULL);
  l.walk(printch);
  cout << endl;
  return 0;
}
\end{lstlisting}

\section{Range Queries in One Dimension}
\setcounter{section}{3}
\setcounter{subsection}{0}
\subsection{Sparse Table (Range Minimum Query)}
\begin{lstlisting}
/*

Given a static array with indices from 0 to n - 1, precompute a table that may
later be used perform range minimum queries on the array in constant time. This
version is simplified to only work on integer arrays.

The dynamic programming state dp[i][j] holds the index of the minimum value in
the sub-array starting at i and having length 2^j. Each dp[i][j] will always be
equal to either dp[i][j - 1] or dp[i + 2^(j - 1) - 1)][j - 1], whichever of the
indices corresponds to the smaller value in the array.

Time Complexity:
- O(n log n) per call to build(), where n is the size of the array.
- O(1) per call to query_min().

Space Complexity:
- O(n log n) for storage of the sparse table, where n is the size of the array.
- O(1) auxiliary for query().

*/

#include <vector>

const int MAXN = 1000;
std::vector<int> table, dp[MAXN];

void build(int n, int a[]) {
  table.resize(n + 1);
  for (int i = 2; i <= n; i++) {
    table[i] = table[i >> 1] + 1;
  }
  for (int i = 0; i < n; i++) {
    dp[i].resize(table[n] + 1);
    dp[i][0] = i;
  }
  for (int j = 1; (1 << j) < n; j++) {
    for (int i = 0; i + (1 << j) <= n; i++) {
      int x = dp[i][j - 1];
      int y = dp[i + (1 << (j - 1))][j - 1];
      dp[i][j] = (a[x] < a[y]) ? x : y;
    }
  }
}

int query(int a[], int lo, int hi) {
  int j = table[hi - lo];
  int x = dp[lo][j];
  int y = dp[hi - (1 << j) + 1][j];
  return (a[x] < a[y]) ? a[x] : a[y];
}

/*** Example Usage ***/

#include <cassert>

int main() {
  int arr[5] = {6, -2, 1, 8, 10};
  build(5, arr);
  assert(query(arr, 0, 3) == -2);
  return 0;
}
\end{lstlisting}
\subsection{Square Root Decomposition}
\begin{lstlisting}
/*

Maintain a fixed-size array (from 0 to size() - 1) while supporting dynamic
queries of contiguous sub-arrays and dynamic updates of individual indices.

The query operation is defined by an associative join_values() function which
satisfies join_values(x, join_values(y, z)) = join_values(join_values(x, y), z)
for all values x, y, and z in the array. The default definition below assumes a
numerical array type, supporting queries for the "min" of the target range.
Another possible query operation is "sum", in which the join_values() function
should be defined to return "a + b".

The update operation is defined by the join_value_with_delta() function which
determines the change made to array values. The default definition below
supports updates that "set" the chosen array index to a new value. Another
possible update operation is "increment", in which join_value_with_delta(v, d)
should be defined to return "v + d".

The operations supported by this data structure are identical to those of the
point update segment tree found in this section.

Time Complexity:
- O(n) per call to both constructors, where n is the size of the array.
- O(1) per call to size().
- O(sqrt n) per call to at(), update(), and query().

Space Complexity:
- O(n) for storage of the array elements.
- O(1) auxiliary for all operations.

*/

#include <algorithm>
#include <cmath>
#include <vector>

template<class T>
class sqrt_decomposition {
  static T join_values(const T &a, const T &b) {
    return std::min(a, b);
  }

  static T join_value_with_delta(const T &v, const T &d) {
    return d;
  }

  int len, blocklen;
  std::vector<T> value, block;

  void init() {
    blocklen = (int)sqrt(len);
    int nblocks = (len + blocklen - 1)/blocklen;
    for (int i = 0; i < nblocks; i++) {
      T blockval = value[i*blocklen];
      int blockhi = std::min(len, (i + 1)*blocklen);
      for (int j = i*blocklen + 1; j < blockhi; j++) {
        blockval = join_values(blockval, value[j]);
      }
      block.push_back(blockval);
    }
  }

 public:
  sqrt_decomposition(int n, const T &v = T()) : len(n), value(n, v) {
    init();
  }

  template<class It>
  sqrt_decomposition(It lo, It hi) : len(hi - lo), value(lo, hi) {
    init();
  }

  int size() const {
    return len;
  }

  T at(int i) const {
    return query(i, i);
  }

  T query(int lo, int hi) const {
    T res;
    int blocklo = ceil((double)lo/blocklen), blockhi = (hi + 1)/blocklen - 1;
    if (blocklo > blockhi) {
      res = value[lo];
      for (int i = lo + 1; i <= hi; i++) {
        res = join_values(res, value[i]);
      }
    } else {
      res = block[blocklo];
      for (int i = blocklo + 1; i <= blockhi; i++) {
        res = join_values(res, block[i]);
      }
      for (int i = lo; i < blocklo*blocklen; i++) {
        res = join_values(res, value[i]);
      }
      for (int i = (blockhi + 1)*blocklen; i <= hi; i++) {
        res = join_values(res, value[i]);
      }
    }
    return res;
  }

  void update(int i, const T &d) {
    value[i] = join_value_with_delta(value[i], d);
    int b = i/blocklen;
    int blockhi = std::min(len, (b + 1)*blocklen);
    block[b] = value[b*blocklen];
    for (int i = b*blocklen + 1; i < blockhi; i++) {
      block[b] = join_values(block[b], value[i]);
    }
  }
};

/*** Example Usage and Output:

Values: 6 -2 4 8 10

***/

#include <cassert>
#include <iostream>
using namespace std;

int main() {
  int arr[5] = {6, -2, 1, 8, 10};
  sqrt_decomposition<int> sd(arr, arr + 5);
  sd.update(2, 4);
  cout << "Values:";
  for (int i = 0; i < sd.size(); i++) {
    cout << " " << sd.at(i);
  }
  cout << endl;
  assert(sd.query(0, 3) == -2);
  return 0;
}
\end{lstlisting}
\subsection{Segment Tree (Point Update)}
\begin{lstlisting}
/*

Maintain a fixed-size array while supporting dynamic queries of contiguous
sub-arrays and dynamic updates of individual indices.

The query operation is defined by an associative join_values() function which
satisfies join_values(x, join_values(y, z)) = join_values(join_values(x, y), z)
for all values x, y, and z in the array. The default code below assumes a
numerical array type, defining queries for the "min" of the target range.
Another possible query operation is "sum", in which case the join_values()
function should be defined to return "a + b".

The update operation is defined by the join_value_with_delta() function, which
determines the change made to array values. The default definition below
supports updates that "set" the chosen array index to a new value. Another
possible update operation is "increment", in which join_value_with_delta(v, d)
should be defined to return "v + d".

- segment_tree(n, v) constructs an array of size n with indices from 0 to n - 1,
  inclusive, and all values initialized to v.
- segment_tree(lo, hi) constructs an array from two random-access iterators as a
  range [lo, hi), initialized to the elements of the range in the same order.
- size() returns the size of the array.
- at(i) returns the value at index i.
- query(lo, hi) returns the result of join_values() applied to all indices from
  lo to hi, inclusive. If the distance between lo and hi is 1, then the single
  specified value is returned.
- update(i, d) assigns the value v at index i to join_value_with_delta(v, d).

Time Complexity:
- O(n) per call to both constructors, where n is the size of the array.
- O(1) per call to size().
- O(log n) per call to at(), update(), and query().

Space Complexity:
- O(n) for storage of the array elements.
- O(log n) auxiliary stack space for update() and query().
- O(1) auxiliary for size().

*/

#include <algorithm>
#include <vector>

template<class T>
class segment_tree {
  static T join_values(const T &a, const T &b) {
    return std::min(a, b);
  }

  static T join_value_with_delta(const T &v, const T &d) {
    return d;
  }

  int len;
  std::vector<T> value;

  void build(int i, int lo, int hi, const T &v) {
    if (lo == hi) {
      value[i] = v;
      return;
    }
    int mid = lo + (hi - lo)/2;
    build(i*2 + 1, lo, mid, v);
    build(i*2 + 2, mid + 1, hi, v);
    value[i] = join_values(value[i*2 + 1], value[i*2 + 2]);
  }

  template<class It>
  void build(int i, int lo, int hi, It arr) {
    if (lo == hi) {
      value[i] = *(arr + lo);
      return;
    }
    int mid = lo + (hi - lo)/2;
    build(i*2 + 1, lo, mid, arr);
    build(i*2 + 2, mid + 1, hi, arr);
    value[i] = join_values(value[i*2 + 1], value[i*2 + 2]);
  }

  T query(int i, int lo, int hi, int tgt_lo, int tgt_hi) const {
    if (lo == tgt_lo && hi == tgt_hi) {
      return value[i];
    }
    int mid = lo + (hi - lo)/2;
    if (tgt_lo <= mid && mid < tgt_hi) {
      return join_values(
                query(i*2 + 1, lo, mid, tgt_lo, std::min(tgt_hi, mid)),
                query(i*2 + 2, mid + 1, hi, std::max(tgt_lo, mid + 1), tgt_hi));
    }
    if (tgt_lo <= mid) {
      return query(i*2 + 1, lo, mid, tgt_lo, std::min(tgt_hi, mid));
    }
    return query(i*2 + 2, mid + 1, hi, std::max(tgt_lo, mid + 1), tgt_hi);
  }

  void update(int i, int lo, int hi, int target, const T &d) {
    if (target < lo || target > hi) {
      return;
    }
    if (lo == hi) {
      value[i] = join_value_with_delta(value[i], d);
      return;
    }
    int mid = lo + (hi - lo)/2;
    update(i*2 + 1, lo, mid, target, d);
    update(i*2 + 2, mid + 1, hi, target, d);
    value[i] = join_values(value[i*2 + 1], value[i*2 + 2]);
  }

 public:
  segment_tree(int n, const T &v = T()) : len(n), value(4*len) {
    build(0, 0, len - 1, false, 0, v);
  }

  template<class It>
  segment_tree(It lo, It hi) : len(hi - lo), value(4*len) {
    build(0, 0, len - 1, lo);
  }

  int size() const {
    return len;
  }

  T at(int i) const {
    return query(i, i);
  }

  T query(int lo, int hi) const {
    return query(0, 0, len - 1, lo, hi);
  }

  void update(int i, const T &d) {
    update(0, 0, len - 1, i, d);
  }
};

/*** Example Usage and Output:

Values: 6 -2 4 8 10
The minimum value in the range [0, 3] is -2.

***/

#include <cassert>
#include <iostream>
using namespace std;

int main() {
  int arr[5] = {6, -2, 1, 8, 10};
  segment_tree<int> t(arr, arr + 5);
  t.update(2, 4);
  cout << "Values:";
  for (int i = 0; i < t.size(); i++) {
    cout << " " << t.at(i);
  }
  cout << endl;
  assert(t.query(0, 3) == -2);
  return 0;
}
\end{lstlisting}
\subsection{Segment Tree (Range Update)}
\begin{lstlisting}
/*

Maintain a fixed-size array while supporting both dynamic queries and updates of
contiguous subarrays via the lazy propagation technique.

The query operation is defined by an associative join_values() function which
satisfies join_values(x, join_values(y, z)) = join_values(join_values(x, y), z)
for all values x, y, and z in the array. The default code below assumes a
numerical array type, defining queries for the "min" of the target range.
Another possible query operation is "sum", in which case the join_values()
function should be defined to return "a + b".

The update operation is defined by the join_value_with_delta() and join_deltas()
functions, which determines the change made to array values. These must satisfy:
- join_deltas(d1, join_deltas(d2, d3)) = join_deltas(join_deltas(d1, d2), d3).
- join_value_with_delta(join_values(v, ...(m times)..., v), d, m)) should be
  equal to join_values(join_value_with_delta(v, d, 1), ...(m times)).
- if a sequence d_1, ..., d_m of deltas is used to update a value v, then
  join_value_with_delta(v, join_deltas(d_1, ..., d_m), 1) should be equivalent
  to m sequential calls to join_value_with_delta(v, d_i, 1) for i = 1..m.
The default code below defines updates that "set" the chosen array index to a
new value. Another possible update operation is "increment", in which case
join_value_with_delta(v, d, len) should be defined to return "v + d*len" and
join_deltas(d1, d2) should be defined to return "d1 + d2".

- segment_tree(n, v) constructs an array of size n with indices from 0 to n - 1,
  inclusive, and all values initialized to v.
- segment_tree(lo, hi) constructs an array from two random-access iterators as a
  range [lo, hi), initialized to the elements of the range in the same order.
- size() returns the size of the array.
- at(i) returns the value at index i, where i is between 0 and size() - 1.
- query(lo, hi) returns the result of join_values() applied to all indices from
  lo to hi, inclusive. If the distance between lo and hi is 1, then the single
  specified value is returned.
- update(i, d) assigns the value v at index i to join_value_with_delta(v, d).
- update(lo, hi, d) modifies the value at each array index from lo to hi,
  inclusive, by respectively joining them with d using join_value_with_delta().

Time Complexity:
- O(n) per call to both constructors, where n is the size of the array.
- O(1) per call to size().
- O(log n) per call to at(), update(), and query().

Space Complexity:
- O(n) for storage of the array elements.
- O(log n) auxiliary stack space for update() and query().
- O(1) auxiliary for size().

*/

#include <algorithm>
#include <vector>

template<class T>
class segment_tree {
  static T join_values(const T &a, const T &b) {
    return std::min(a, b);
  }

  static T join_value_with_delta(const T &v, const T &d, int len) {
    return d;
  }

  static T join_deltas(const T &d1, const T &d2) {
    return d2;  // For "set" updates, the more recent delta prevails.
  }

  int len;
  std::vector<T> value, delta;
  std::vector<bool> pending;

  void build(int i, int lo, int hi, const T &v) {
    if (lo == hi) {
      value[i] = v;
      return;
    }
    int mid = lo + (hi - lo)/2;
    build(i*2 + 1, lo, mid, v);
    build(i*2 + 2, mid + 1, hi, v);
    value[i] = join_values(value[i*2 + 1], value[i*2 + 2]);
  }

  template<class It>
  void build(int i, int lo, int hi, It arr) {
    if (lo == hi) {
      value[i] = *(arr + lo);
      return;
    }
    int mid = lo + (hi - lo)/2;
    build(i*2 + 1, lo, mid, arr);
    build(i*2 + 2, mid + 1, hi, arr);
    value[i] = join_values(value[i*2 + 1], value[i*2 + 2]);
  }

  void push_delta(int i, int lo, int hi) {
    if (pending[i]) {
      value[i] = join_value_with_delta(value[i], delta[i], hi - lo + 1);
      if (lo != hi) {
        int l = 2*i + 1, r = 2*i + 2;
        delta[l] = pending[l] ? join_deltas(delta[l], delta[i]) : delta[i];
        delta[r] = pending[r] ? join_deltas(delta[r], delta[i]) : delta[i];
        pending[l] = pending[r] = true;
      }
      pending[i] = false;
    }
  }

  T query(int i, int lo, int hi, int tgt_lo, int tgt_hi) {
    push_delta(i, lo, hi);
    if (lo == tgt_lo && hi == tgt_hi) {
      return value[i];
    }
    int mid = lo + (hi - lo)/2;
    if (tgt_lo <= mid && mid < tgt_hi) {
      return join_values(
                query(i*2 + 1, lo, mid, tgt_lo, std::min(tgt_hi, mid)),
                query(i*2 + 2, mid + 1, hi, std::max(tgt_lo, mid + 1), tgt_hi));
    }
    if (tgt_lo <= mid) {
      return query(i*2 + 1, lo, mid, tgt_lo, std::min(tgt_hi, mid));
    }
    return query(i*2 + 2, mid + 1, hi, std::max(tgt_lo, mid + 1), tgt_hi);
  }

  void update(int i, int lo, int hi, int tgt_lo, int tgt_hi, const T &d) {
    push_delta(i, lo, hi);
    if (hi < tgt_lo || lo > tgt_hi) {
      return;
    }
    if (tgt_lo <= lo && hi <= tgt_hi) {
      delta[i] = d;
      pending[i] = true;
      push_delta(i, lo, hi);
      return;
    }
    update(2*i + 1, lo, (lo + hi)/2, tgt_lo, tgt_hi, d);
    update(2*i + 2, (lo + hi)/2 + 1, hi, tgt_lo, tgt_hi, d);
    value[i] = join_values(value[2*i + 1], value[2*i + 2]);
  }

 public:
  segment_tree(int n, const T &v = T())
      : len(n), value(4*len), delta(4*len), pending(4*len, false) {
    build(0, 0, len - 1, v);
  }

  template<class It>
  segment_tree(It lo, It hi)
      : len(hi - lo), value(4*len), delta(4*len), pending(4*len, false) {
    build(0, 0, len - 1, lo);
  }

  int size() const {
    return len;
  }

  T at(int i) {
    return query(i, i);
  }

  T query(int lo, int hi) {
    return query(0, 0, len - 1, lo, hi);
  }

  void update(int i, const T &d) {
    update(0, 0, len - 1, i, i, d);
  }

  void update(int lo, int hi, const T &d) {
    update(0, 0, len - 1, lo, hi, d);
  }
};

/*** Example Usage and Output:

Values: 6 -2 4 8 10
Values: 5 5 5 1 5

***/

#include <cassert>
#include <iostream>
using namespace std;

int main() {
  int arr[5] = {6, -2, 1, 8, 10};
  segment_tree<int> t(arr, arr + 5);
  t.update(2, 4);
  cout << "Values:";
  for (int i = 0; i < t.size(); i++) {
    cout << " " << t.at(i);
  }
  cout << endl;
  assert(t.query(0, 3) == -2);
  t.update(0, 4, 5);
  t.update(3, 2);
  t.update(3, 1);
  cout << "Values:";
  for (int i = 0; i < t.size(); i++) {
    cout << " " << t.at(i);
  }
  cout << endl;
  assert(t.query(0, 3) == 1);
  return 0;
}
\end{lstlisting}
\subsection{Segment Tree (Compressed)}
\begin{lstlisting}
/*

Maintain a fixed-size array while supporting both dynamic queries and updates of
contiguous subarrays via the lazy propagation technique. This implementation
uses lazy initialization of nodes to conserve memory while supporting large
indices.

The query operation is defined by an associative join_values() function which
satisfies join_values(x, join_values(y, z)) = join_values(join_values(x, y), z)
for all values x, y, and z in the array. The default code below assumes a
numerical array type, defining queries for the "min" of the target range.
Another possible query operation is "sum", in which case the join_values()
function should be defined to return "a + b".

The update operation is defined by the join_value_with_delta() and join_deltas()
functions, which determines the change made to array values. These must satisfy:
- join_deltas(d1, join_deltas(d2, d3)) = join_deltas(join_deltas(d1, d2), d3).
- join_value_with_delta(join_values(v, ...(m times)..., v), d, m)) should be
  equal to join_values(join_value_with_delta(v, d, 1), ...(m times)).
- if a sequence d_1, ..., d_m of deltas is used to update a value v, then
  join_value_with_delta(v, join_deltas(d_1, ..., d_m), 1) should be equivalent
  to m sequential calls to join_value_with_delta(v, d_i, 1) for i = 1..m.
The default code below defines updates that "set" the chosen array index to a
new value. Another possible update operation is "increment", in which case
join_value_with_delta(v, d, len) should be defined to return "v + d*len" and
join_deltas(d1, d2) should be defined to return "d1 + d2".

- segment_tree(n, v) constructs an array of size n with indices from 0 to n - 1,
  inclusive, and all values initialized to v.
- segment_tree(lo, hi) constructs an array from two random-access iterators as a
  range [lo, hi), initialized to the elements of the range in the same order.
- size() returns the size of the array.
- at(i) returns the value at index i, where i is between 0 and size() - 1.
- query(lo, hi) returns the result of join_values() applied to all indices from
  lo to hi, inclusive. If the distance between lo and hi is 1, then the single
  specified value is returned.
- update(i, d) assigns the value v at index i to join_value_with_delta(v, d).
- update(lo, hi, d) modifies the value at each array index from lo to hi,
  inclusive, by respectively joining them with d using join_value_with_delta().

Time Complexity:
- O(n) per call to both constructors, where n is the size of the array.
- O(1) per call to size().
- O(log n) per call to at(), update(), and query().

Space Complexity:
- O(n) for storage of the array elements.
- O(log n) auxiliary stack space for update() and query().
- O(1) auxiliary for size().

*/

#include <algorithm>
#include <cstddef>

template<class T>
class segment_tree {
  static const int MAXN = 1000000000;

  static T join_values(const T &a, const T &b) {
    return std::min(a, b);
  }

  static T join_segment(const T &v, int len) {
    return v;
  }

  static T join_value_with_delta(const T &v, const T &d, int len) {
    return d;
  }

  static T join_deltas(const T &d1, const T &d2) {
    return d2;  // For "set" updates, the more recent delta prevails.
  }

  struct node_t {
    T value, delta;
    bool pending;
    node_t *left, *right;

    node_t(const T &v) : value(v), pending(false), left(NULL), right(NULL) {}
  } *root;

  T init;

  void update_delta(node_t *&n, const T &d, int len) {
    if (n == NULL) {
      n = new node_t(join_segment(init, len));
    }
    n->delta = n->pending ? join_deltas(n->delta, d) : d;
    n->pending = true;
  }

  void push_delta(node_t *n, int lo, int hi) {
    if (n->pending) {
      n->value = join_value_with_delta(n->value, n->delta, hi - lo + 1);
      if (lo != hi) {
        int mid = lo + (hi - lo)/2;
        update_delta(n->left, n->delta, mid - lo + 1);
        update_delta(n->right, n->delta, hi - mid);
      }
    }
    n->pending = false;
  }

  T query(node_t *n, int lo, int hi, int tgt_lo, int tgt_hi) {
    push_delta(n, lo, hi);
    if (lo == tgt_lo && hi == tgt_hi) {
      return (n == NULL) ? join_segment(init, hi - lo + 1) : n->value;
    }
    int mid = lo + (hi - lo)/2;
    if (tgt_lo <= mid && mid < tgt_hi) {
      return join_values(
          query(n->left, lo, mid, tgt_lo, std::min(tgt_hi, mid)),
          query(n->right, mid + 1, hi, std::max(tgt_lo, mid + 1), tgt_hi));
    }
    if (tgt_lo <= mid) {
      return query(n->left, lo, mid, tgt_lo, std::min(tgt_hi, mid));
    }
    return query(n->right, mid + 1, hi, std::max(tgt_lo, mid + 1), tgt_hi);
  }

  void update(node_t *&n, int lo, int hi, int tgt_lo, int tgt_hi, const T &d) {
    if (n == NULL) {
      n = new node_t(join_segment(init, hi - lo + 1));
    }
    push_delta(n, lo, hi);
    if (hi < tgt_lo || lo > tgt_hi) {
      return;
    }
    if (tgt_lo <= lo && hi <= tgt_hi) {
      n->delta = d;
      n->pending = true;
      push_delta(n, lo, hi);
      return;
    }
    int mid = lo + (hi - lo)/2;
    update(n->left, lo, mid, tgt_lo, tgt_hi, d);
    update(n->right, mid + 1, hi, tgt_lo, tgt_hi, d);
    n->value = join_values(n->left->value, n->right->value);
  }

  void clean_up(node_t *n) {
    if (n != NULL) {
      clean_up(n->left);
      clean_up(n->right);
      delete n;
    }
  }

 public:
  segment_tree(const T &v = T()) : root(NULL), init(v) {}

  ~segment_tree() {
    clean_up(root);
  }

  T at(int i) {
    return query(i, i);
  }

  T query(int lo, int hi) {
    return query(root, 0, MAXN, lo, hi);
  }

  void update(int i, const T &d) {
    return update(i, i, d);
  }

  void update(int lo, int hi, const T &d) {
    return update(root, 0, MAXN, lo, hi, d);
  }
};

/*** Example Usage and Output:

Values: 6 -2 4 8 10
Values: 5 5 5 1 5

***/

#include <cassert>
#include <iostream>
using namespace std;

int main() {
  segment_tree<int> t(0);
  t.update(0, 6);
  t.update(1, -2);
  t.update(2, 4);
  t.update(3, 8);
  t.update(4, 10);
  cout << "Values:";
  for (int i = 0; i < 5; i++) {
    cout << " " << t.at(i);
  }
  cout << endl;
  assert(t.query(0, 3) == -2);
  t.update(0, 4, 5);
  t.update(3, 2);
  t.update(3, 1);
  cout << "Values:";
  for (int i = 0; i < 5; i++) {
    cout << " " << t.at(i);
  }
  cout << endl;
  assert(t.query(0, 3) == 1);
  return 0;
}
\end{lstlisting}
\subsection{Implicit Treap}
\begin{lstlisting}
/*

Maintain a dynamically-sized array using a balanced binary search tree while
supporting both dynamic queries and updates of contiguous subarrays via the lazy
propagation technique. A treap maintains a balanced binary tree structure by
preserving the heap property on the randomly generated priority values of nodes,
thereby making insertions and deletions run in O(log n) with high probability.

The query operation is defined by an associative join_values() function which
satisfies join_values(x, join_values(y, z)) = join_values(join_values(x, y), z)
for all values x, y, and z in the array. The default code below assumes a
numerical array type, defining queries for the "min" of the target range.
Another possible query operation is "sum", in which case the join_values()
function should be defined to return "a + b".

The update operation is defined by the join_value_with_delta() and join_deltas()
functions, which determines the change made to array values. These must satisfy:
- join_deltas(d1, join_deltas(d2, d3)) = join_deltas(join_deltas(d1, d2), d3).
- join_value_with_delta(join_values(v, ...(m times)..., v), d, m)) should be
  equal to join_values(join_value_with_delta(v, d, 1), ...(m times)).
- if a sequence d_1, ..., d_m of deltas is used to update a value v, then
  join_value_with_delta(v, join_deltas(d_1, ..., d_m), 1) should be equivalent
  to m sequential calls to join_value_with_delta(v, d_i, 1) for i = 1..m.
The default code below defines updates that "set" the chosen array index to a
new value. Another possible update operation is "increment", in which case
join_value_with_delta(v, d, len) should be defined to return "v + d*len" and
join_deltas(d1, d2) should be defined to return "d1 + d2".

This data structure shares every operation of one-dimensional segment trees in
this section, with the additional operations empty(), insert(), erase(),
push_back(), and pop_back() analogous to those of std::vector (here, insert()
and erase() both take an index instead of an iterator).

Time Complexity:
- O(n) per call to both constructors, where n is the size of the array.
- O(1) per call to size() and empty().
- O(log n) on average per call to all other operations.

Space Complexity:
- O(n) for storage of the array elements.
- O(1) auxiliary for size() and empty().
- O(log n) auxiliary stack space for all other operations.

*/

#include <cstdlib>

template<class T>
class implicit_treap {
  static T join_values(const T &a, const T &b) {
    return a < b ? a : b;
  }

  static T join_value_with_delta(const T &v, const T &d, int len) {
    return d;
  }

  static T join_deltas(const T &d1, const T &d2) {
    return d2;
  }

  struct node_t {
    static inline int rand32() {
      return (rand() & 0x7fff) | ((rand() & 0x7fff) << 15);
    }

    T value, subtree_value, delta;
    bool pending;
    int size, priority;
    node_t *left, *right;

    node_t(const T &v)
        : value(v), subtree_value(v), pending(false), size(1),
          priority(rand32()), left(NULL), right(NULL) {}
  } *root;

  static int size(node_t *n) {
    return (n == NULL) ? 0 : n->size;
  }

  static void update_value(node_t *n) {
    if (n == NULL) {
      return;
    }
    n->subtree_value = n->value;
    if (n->left != NULL) {
      n->subtree_value = join_values(n->subtree_value, n->left->subtree_value);
    }
    if (n->right != NULL) {
      n->subtree_value = join_values(n->subtree_value, n->right->subtree_value);
    }
    n->size = 1 + size(n->left) + size(n->right);
  }

  static void update_delta(node_t *n, const T &d) {
    if (n != NULL) {
      n->delta = n->pending ? join_deltas(n->delta, d) : d;
      n->pending = true;
    }
  }

  static void push_delta(node_t *n) {
    if (n == NULL || !n->pending) {
      return;
    }
    n->value = join_value_with_delta(n->value, n->delta, 1);
    n->subtree_value = join_value_with_delta(n->subtree_value, n->delta,
                                             n->size);
    if (n->size > 1) {
      update_delta(n->left, n->delta);
      update_delta(n->right, n->delta);
    }
    n->pending = false;
  }

  static void merge(node_t *&n, node_t *left, node_t *right) {
    push_delta(left);
    push_delta(right);
    if (left == NULL) {
      n = right;
    } else if (right == NULL) {
      n = left;
    } else if (left->priority < right->priority) {
      merge(left->right, left->right, right);
      n = left;
    } else {
      merge(right->left, left, right->left);
      n = right;
    }
    update_value(n);
  }

  static void split(node_t *n, node_t *&left, node_t *&right, int i) {
    push_delta(n);
    if (n == NULL) {
      left = right = NULL;
    } else if (i <= size(n->left)) {
      split(n->left, left, n->left, i);
      right = n;
    } else {
      split(n->right, n->right, right, i - size(n->left) - 1);
      left = n;
    }
    update_value(n);
  }

  static void insert(node_t *&n, node_t *new_node, int i) {
    push_delta(n);
    if (n == NULL) {
      n = new_node;
    } else if (new_node->priority < n->priority) {
      split(n, new_node->left, new_node->right, i);
      n = new_node;
    } else if (i <= size(n->left)) {
      insert(n->left, new_node, i);
    } else {
      insert(n->right, new_node, i - size(n->left) - 1);
    }
    update_value(n);
  }

  static void erase(node_t *&n, int i) {
    push_delta(n);
    if (i == size(n->left)) {
      delete n;
      merge(n, n->left, n->right);
    } else if (i < size(n->left)) {
      erase(n->left, i);
    } else {
      erase(n->right, i - size(n->left) - 1);
    }
    update_value(n);
  }

  static node_t* select(node_t *n, int i) {
    push_delta(n);
    if (i < size(n->left)) {
      return select(n->left, i);
    }
    if (i > size(n->left)) {
      return select(n->right, i - size(n->left) - 1);
    }
    return n;
  }

  void clean_up(node_t *&n) {
    if (n != NULL) {
      clean_up(n->left);
      clean_up(n->right);
      delete n;
    }
  }

 public:
  implicit_treap(int n = 0, const T &v = T()) : root(NULL) {
    for (int i = 0; i < n; i++) {
      push_back(v);
    }
  }

  template<class It>
  implicit_treap(It lo, It hi) : root(NULL) {
    for (; lo != hi; ++lo) {
      push_back(*lo);
    }
  }

  ~implicit_treap() {
    clean_up(root);
  }

  int size() const {
    return size(root);
  }

  bool empty() const {
    return root == NULL;
  }

  void insert(int i, const T &v) {
    insert(root, new node_t(v), i);
  }

  void erase(int i) {
    erase(root, i);
  }

  void push_back(const T &v) {
    insert(size(), v);
  }

  void pop_back() {
    erase(size() - 1);
  }

  T at(int i) const {
    return select(root, i)->value;
  }

  T query(int lo, int hi) {
    node_t *l1, *r1, *l2, *r2, *t;
    split(root, l1, r1, hi + 1);
    split(l1, l2, r2, lo);
    T res = r2->subtree_value;
    merge(t, l2, r2);
    merge(root, t, r1);
    return res;
  }

  void update(int i, const T &d) {
    update(i, i, d);
  }

  void update(int lo, int hi, const T &d) {
    node_t *l1, *r1, *l2, *r2, *t;
    split(root, l1, r1, hi + 1);
    split(l1, l2, r2, lo);
    update_delta(r2, d);
    merge(t, l2, r2);
    merge(root, t, r1);
  }
};

/*** Example Usage and Output:

Values: -99 -2 1 8 10 11 (min: -99)
Values: -90 -2 1 8 10 11 (min: -90)
Values: 2 2 1 8 10 11 (min: 1)

***/

#include <iostream>
using namespace std;

void print(implicit_treap<int> &t) {
  cout << "Values:";
  for (int i = 0; i < t.size(); i++) {
    cout << " " << t.at(i);
  }
  cout << " (min: " << t.query(0, t.size() - 1) << ")" << endl;
}

int main() {
  int arr[5] = {99, -2, 1, 8, 10};
  implicit_treap<int> t(arr, arr + 5);
  t.push_back(11);
  t.push_back(12);
  t.pop_back();
  print(t);
  t.insert(0, 90);
  t.erase(1);
  print(t);
  t.update(0, 1, 2);
  print(t);
  return 0;
}
\end{lstlisting}

\section{Range Queries in Two Dimensions}
\setcounter{section}{4}
\setcounter{subsection}{0}
\subsection{Quadtree (Point Update)}
\begin{lstlisting}
/*

Maintain a two-dimensional array while supporting dynamic queries of rectangular
sub-arrays and dynamic updates of individual indices. This implementation uses
lazy initialization of nodes to conserve memory while supporting large indices.

The query operation is defined by the join_values() and join_region() functions
where join_values(x, join_values(y, z)) = join_values(join_values(x, y), z) for
all values x, y, and z in the array. The join_region(v, area) function must be
defined in conjunction to efficiently return the result of join_values() applied
to a rectangular sub-array of area elements. The default code below assumes a
numerical array type, defining queries for the "min" of the target range.
Another possible query operation is "sum", in which case join_values(a, b)
should return "a + b" and join_region(v, area) should return "v*area".

The update operation is defined by the join_value_with_delta() function, which
determines the change made to array values. The default code below defines
updates that "set" the chosen array index to a new value. Another possible
update operation is "increment", in which join_value_with_delta(v, d) should be
defined to return "v + d".

- quadtree(v) constructs a two-dimensional array with rows from 0 to MAXR and
  columns from 0 to MAXC, inclusive. All values are implicitly initialized to v.
- at(r, c) returns the value at row r, column c.
- query(r1, c1, r2, c2) returns the result of join_values() applied to every
  value in the rectangular region consisting of rows from r1 to r2, inclusive,
  and columns from c1 to c2, inclusive.
- update(r, c, d) assigns the value v at (r, c) to join_value_with_delta(v, d).

Time Complexity:
- O(1) per call to the constructor.
- O(max(MAXR, MAXC)) per call to at(), update(), and query().

Space Complexity:
- O(n) for storage of the array elements, where n is the number of updated
  entries in the array.
- O(sqrt(max(MAXR, MAXC))) auxiliary stack space for update(), query(), and
  at().

*/

#include <algorithm>
#include <cstddef>

template<class T>
class quadtree {
  static const int MAXR = 1000000000;
  static const int MAXC = 1000000000;

  static T join_values(const T &a, const T &b) {
    return std::min(a, b);
  }

  static T join_region(const T &v, int area) {
    return v;
  }

  static T join_value_with_delta(const T &v, const T &d) {
    return d;
  }

  struct node_t {
    T value;
    node_t *child[4];

    node_t(const T &v) : value(v) {
      for (int i = 0; i < 4; i++) {
        child[i] = NULL;
      }
    }
  };

  node_t *root;
  T init;

  // Helper variables for query().
  int tgt_r1, tgt_c1, tgt_r2, tgt_c2;
  T res;
  bool found;

  void query(node_t *n, int r1, int c1, int r2, int c2) {
    if (tgt_r2 < r1 || tgt_r1 > r2 || tgt_c2 < c1 || tgt_c1 > c2) {
      return;
    }
    if (n == NULL) {
      int rlen = std::min(r2, tgt_r2) - std::max(r1, tgt_r1) + 1;
      int clen = std::min(c2, tgt_c2) - std::max(c1, tgt_c1) + 1;
      T v = join_region(init, rlen*clen);
      res = found ? join_values(res, v) : v;
      found = true;
      return;
    }
    if (tgt_r1 <= r1 && r2 <= tgt_r2 && tgt_c1 <= c1 && c2 <= tgt_c2) {
      res = found ? join_values(res, n->value) : n->value;
      found = true;
      return;
    }
    int rmid = r1 + (r2 - r1)/2, cmid = c1 + (c2 - c1)/2;
    query(n->child[0], r1, c1, rmid, cmid);
    query(n->child[1], rmid + 1, c1, r2, cmid);
    query(n->child[2], r1, cmid + 1, rmid, c2);
    query(n->child[3], rmid + 1, cmid + 1, r2, c2);
  }

  // Helper variables for update().
  int tgt_r, tgt_c;
  T delta;

  void update(node_t *&n, int r1, int c1, int r2, int c2) {
    if (n == NULL) {
      n = new node_t(join_region(init, (r2 - r1 + 1)*(c2 - r1 + 1)));
    }
    if (tgt_r < r1 || tgt_r > r2 || tgt_c < c1 || tgt_c > c2) {
      return;
    }
    if (r1 == r2 && c1 == c2) {
      n->value = join_value_with_delta(n->value, delta);
      return;
    }
    int rmid = r1 + (r2 - r1)/2, cmid = c1 + (c2 - c1)/2;
    update(n->child[0], r1, c1, rmid, cmid);
    update(n->child[1], rmid + 1, c1, r2, cmid);
    update(n->child[2], r1, cmid + 1, rmid, c2);
    update(n->child[3], rmid + 1, cmid + 1, r2, c2);
    bool found = false;
    for (int i = 0; i < 4; i++) {
      n->value = found ? join_values(n->value, n->child[i]->value)
                       : n->child[i]->value;
      found = true;
    }
  }

  static void clean_up(node_t *n) {
    if (n != NULL) {
      for (int i = 0; i < 4; i++) {
        clean_up(n->child[i]);
      }
      delete n;
    }
  }

 public:
  quadtree(const T &v = T()) : root(NULL), init(v) {}

  ~quadtree() {
    clean_up(root);
  }

  T at(int r, int c) {
    return query(r, c, r, c);
  }

  T query(int r1, int c1, int r2, int c2) {
    tgt_r1 = r1;
    tgt_c1 = c1;
    tgt_r2 = r2;
    tgt_c2 = c2;
    found = false;
    query(root, 0, 0, MAXR, MAXC);
    return found ? res : join_region(init, (r2 - r1 + 1)*(c2 - c1 + 1));
  }

  void update(int r, int c, const T &d) {
    tgt_r = r;
    tgt_c = c;
    delta = d;
    update(root, 0, 0, MAXR, MAXC);
  }
};

/*** Example Usage and Output:

Values:
7 6 0
5 4 0
0 1 9

***/

#include <cassert>
#include <iostream>
using namespace std;

int main() {
  quadtree<int> t(0);
  t.update(0, 0, 7);
  t.update(0, 1, 6);
  t.update(1, 0, 5);
  t.update(1, 1, 4);
  t.update(2, 1, 1);
  t.update(2, 2, 9);
  cout << "Values:" << endl;
  for (int i = 0; i < 3; i++) {
    for (int j = 0; j < 3; j++) {
      cout << t.at(i, j) << " ";
    }
    cout << endl;
  }
  assert(t.query(0, 0, 0, 1) == 6);
  assert(t.query(0, 0, 1, 0) == 5);
  assert(t.query(1, 1, 2, 2) == 0);
  assert(t.query(0, 0, 1000000000, 1000000000) == 0);
  t.update(500000000, 500000000, -100);
  assert(t.query(0, 0, 1000000000, 1000000000) == -100);
  return 0;
}
\end{lstlisting}
\subsection{Quadtree (Range Update)}
\begin{lstlisting}
/*

Maintain a two-dimensional array while supporting both dynamic queries and
updates of rectangular sub-arrays via the lazy propagation technique. This
implementation uses lazy initialization of nodes to conserve memory while
supporting large indices.

The query operation is defined by the join_values() and join_region() functions
where join_values(x, join_values(y, z)) = join_values(join_values(x, y), z) for
all values x, y, and z in the array. The join_region(v, area) function must be
defined in conjunction to efficiently return the result of join_values() applied
to a rectangular sub-array of area elements. The default code below assumes a
numerical array type, defining queries for the "min" of the target range.
Another possible query operation is "sum", in which case join_values(a, b)
should return "a + b" and join_region(v, area) should return "v*area".

The update operation is defined by the join_value_with_delta() and join_deltas()
functions, which determines the change made to array values. These must satisfy:
- join_deltas(d1, join_deltas(d2, d3)) = join_deltas(join_deltas(d1, d2), d3).
- join_value_with_delta(join_values(v, ...(m times)..., v), d, m)) should be
  equal to join_values(join_value_with_delta(v, d, 1), ...(m times)).
- if a sequence d_1, ..., d_m of deltas is used to update a value v, then
  join_value_with_delta(v, join_deltas(d_1, ..., d_m), 1) should be equivalent
  to m sequential calls to join_value_with_delta(v, d_i, 1) for i = 1..m.
The default code below defines updates that "set" the chosen array index to a
new value. Another possible update operation is "increment", in which case
join_value_with_delta(v, d, area) should be defined to return "v + d*area" and
join_deltas(d1, d2) should be defined to return "d1 + d2".

- quadtree(v) constructs a two-dimensional array with rows from 0 to MAXR and
  columns from 0 to MAXC, inclusive. All values are implicitly initialized to v.
- at(r, c) returns the value at row r, column c.
- query(r1, c1, r2, c2) returns the result of join_values() applied to every
  value in the rectangular region consisting of rows from r1 to r2 and columns
  from c1 to c2, inclusive.
- update(r, c, d) assigns the value v at (r, c) to join_value_with_delta(v, d).
- update(r1, c1, r2, c2) modifies the value at each index of the rectangular
  region consisting of rows from r1 to r2 and columns from c1 to c2, inclusive,
  by respectively joining them with d using join_value_with_delta().

Time Complexity:
- O(1) per call to the constructor.
- O(max(MAXR, MAXC)) per call to at(), update(), and query().

Space Complexity:
- O(n) for storage of the array elements, where n is the number of updated
  entries in the array.
- O(sqrt(max(MAXR, MAXC))) auxiliary stack space for update(), query(), and
  at().

*/

#include <algorithm>
#include <cstddef>

template<class T>
class quadtree {
  static const int MAXR = 1000000000;
  static const int MAXC = 1000000000;

  static T join_values(const T &a, const T &b) {
    return std::min(a, b);
  }

  static T join_region(const T &v, int area) {
    return v;
  }

  static T join_value_with_delta(const T &v, const T &d, int area) {
    return d;
  }

  static T join_deltas(const T &d1, const T &d2) {
    return d2;  // For "set" updates, the more recent delta prevails.
  }

  struct node_t {
    T value, delta;
    bool pending;
    node_t *child[4];

    node_t(const T &v) : value(v), pending(false) {
      for (int i = 0; i < 4; i++) {
        child[i] = NULL;
      }
    }
  } *root;

  T init;

  void update_delta(node_t *&n, const T &d, int area) {
    if (n == NULL) {
      n = new node_t(join_region(init, area));
    }
    n->delta = n->pending ? join_deltas(n->delta, d) : d;
    n->pending = true;
  }

  void push_delta(node_t *n, int r1, int c1, int r2, int c2) {
    if (n->pending) {
      int rmid = r1 + (r2 - r1)/2, cmid = c1 + (c2 - c1)/2;
      int rlen = r2 - r1 + 1, clen = c2 - c1 + 1;
      n->value = join_value_with_delta(n->value, n->delta, rlen*clen);
      if (rlen*clen > 1) {
        int rlen1 = rmid - r1 + 1, rlen2 = rlen - rlen1;
        int clen1 = cmid - c1 + 1, clen2 = clen - clen1;
        update_delta(n->child[0], n->delta, rlen1*clen1);
        update_delta(n->child[1], n->delta, rlen2*clen1);
        update_delta(n->child[2], n->delta, rlen1*clen2);
        update_delta(n->child[3], n->delta, rlen2*clen2);
      }
      n->pending = false;
    }
  }

  // Helper variables for query() and update().
  int tgt_r1, tgt_c1, tgt_r2, tgt_c2;
  T res, delta;
  bool found;

  void query(node_t *n, int r1, int c1, int r2, int c2) {
    if (tgt_r2 < r1 || tgt_r1 > r2 || tgt_c2 < c1 || tgt_c1 > c2) {
      return;
    }
    if (n == NULL) {
      int rlen = std::min(r2, tgt_r2) - std::max(r1, tgt_r1) + 1;
      int clen = std::min(c2, tgt_c2) - std::max(c1, tgt_c1) + 1;
      T v = join_region(init, rlen*clen);
      res = found ? join_values(res, v) : v;
      found = true;
      return;
    }
    push_delta(n, r1, c1, r2, c2);
    if (tgt_r1 <= r1 && r2 <= tgt_r2 && tgt_c1 <= c1 && c2 <= tgt_c2) {
      res = found ? join_values(res, n->value) : n->value;
      found = true;
      return;
    }
    int rmid = r1 + (r2 - r1)/2, cmid = c1 + (c2 - c1)/2;
    query(n->child[0], r1, c1, rmid, cmid);
    query(n->child[1], rmid + 1, c1, r2, cmid);
    query(n->child[2], r1, cmid + 1, rmid, c2);
    query(n->child[3], rmid + 1, cmid + 1, r2, c2);
  }

  void update(node_t *&n, int r1, int c1, int r2, int c2) {
    if (n == NULL) {
      n = new node_t(join_region(init, (r2 - r1 + 1)*(c2 - r1 + 1)));
    }
    if (tgt_r2 < r1 || tgt_r1 > r2 || tgt_c2 < c1 || tgt_c1 > c2) {
      return;
    }
    push_delta(n, r1, c1, r2, c2);
    if (tgt_r1 <= r1 && r2 <= tgt_r2 && tgt_c1 <= c1 && c2 <= tgt_c2) {
      n->delta = delta;
      n->pending = true;
      push_delta(n, r1, c1, r2, c2);
      return;
    }
    int rmid = r1 + (r2 - r1)/2, cmid = c1 + (c2 - c1)/2;
    update(n->child[0], r1, c1, rmid, cmid);
    update(n->child[1], rmid + 1, c1, r2, cmid);
    update(n->child[2], r1, cmid + 1, rmid, c2);
    update(n->child[3], rmid + 1, cmid + 1, r2, c2);
    n->value = n->child[0]->value;
    for (int i = 1; i < 4; i++) {
      n->value = join_values(n->value, n->child[i]->value);
    }
  }

  static void clean_up(node_t *n) {
    if (n != NULL) {
      for (int i = 0; i < 4; i++) {
        clean_up(n->child[i]);
      }
      delete n;
    }
  }

public:
  quadtree(const T &v = T()) : root(NULL), init(v) {}

  ~quadtree() {
    clean_up(root);
  }

  T at(int r, int c) {
    return query(r, c, r, c);
  }

  T query(int r1, int c1, int r2, int c2) {
    tgt_r1 = r1;
    tgt_c1 = c1;
    tgt_r2 = r2;
    tgt_c2 = c2;
    found = false;
    query(root, 0, 0, MAXR, MAXC);
    return found ? res : join_region(init, (r2 - r1 + 1)*(c2 - c1 + 1));
  }

  void update(int r, int c, const T &d) {
    update(r, c, r, c, d);
  }

  void update(int r1, int c1, int r2, int c2, const T &d) {
    tgt_r1 = r1;
    tgt_c1 = c1;
    tgt_r2 = r2;
    tgt_c2 = c2;
    delta = d;
    update(root, 0, 0, MAXR, MAXC);
  }
};

/*** Example Usage and Output:

Values:
7 6 0
5 4 0
0 1 9

***/

#include <cassert>
#include <iostream>
using namespace std;

int main() {
  quadtree<int> t(0);
  t.update(0, 0, 7);
  t.update(0, 1, 6);
  t.update(1, 0, 5);
  t.update(1, 1, 4);
  t.update(2, 1, 1);
  t.update(2, 2, 9);
  cout << "Values:" << endl;
  for (int i = 0; i < 3; i++) {
    for (int j = 0; j < 3; j++) {
      cout << t.at(i, j) << " ";
    }
    cout << endl;
  }
  assert(t.query(0, 0, 0, 1) == 6);
  assert(t.query(0, 0, 1, 0) == 5);
  assert(t.query(1, 1, 2, 2) == 0);
  assert(t.query(0, 0, 1000000000, 1000000000) == 0);
  t.update(500000000, 500000000, -100);
  assert(t.query(0, 0, 1000000000, 1000000000) == -100);
  return 0;
}
\end{lstlisting}
\subsection{2D Segment Tree}
\begin{lstlisting}
/*

Maintain a two-dimensional array while supporting dynamic queries of rectangular
sub-arrays and dynamic updates of individual indices. This implementation uses
lazy initialization of nodes to conserve memory while supporting large indices.

The query operation is defined by the join_values() and join_region() functions
where join_values(x, join_values(y, z)) = join_values(join_values(x, y), z) for
all values x, y, and z in the array. The join_region(v, area) function must be
defined in conjunction to efficiently return the result of join_values() applied
to a rectangular sub-array of area elements. The default code below assumes a
numerical array type, defining queries for the "min" of the target range.
Another possible query operation is "sum", in which case join_values(a, b)
should return "a + b" and join_region(v, area) should return "v*area".

The update operation is defined by the join_value_with_delta() function, which
determines the change made to array values. The default code below defines
updates that "set" the chosen array index to a new value. Another possible
update operation is "increment", in which join_value_with_delta(v, d) should be
defined to return "v + d".

- segment_tree_2d(v) constructs a two-dimensional array with rows from 0 to
  MAXR and columns from 0 to MAXC, inclusive. All values are implicitly
  initialized to v.
- at(r, c) returns the value at row r, column c.
- query(r1, c1, r2, c2) returns the result of join_values() applied to every
  value in the rectangular region consisting of rows from r1 to r2, inclusive,
  and columns from c1 to c2, inclusive.
- update(r, c, d) assigns the value v at (r, c) to join_value_with_delta(v, d).

Time Complexity:
- O(1) per call to the constructor.
- O(log(MAXR)*log(MAXC)) per call to at(), update(), and query().

Space Complexity:
- O(n) for storage of the array elements, where n is the number of updated
  entries in the array.
- O(log(MAXR) + log(MAXC)) auxiliary stack space for update(), query(), and
  at().

*/

#include <algorithm>
#include <cstddef>

template<class T>
class segment_tree_2d {
  static const int MAXR = 1000000000;
  static const int MAXC = 1000000000;

  static T join_values(const T &a, const T &b) {
    return std::min(a, b);
  }

  static T join_region(const T &v, int area) {
    return v;
  }

  static T join_value_with_delta(const T &v, const T &d) {
    return d;
  }

  struct inner_node_t {
    T value;
    int low, high;
    inner_node_t *left, *right;

    inner_node_t(int lo, int hi, const T &v)
        : value(v), low(lo), high(hi), left(NULL), right(NULL) {}
  };

  struct outer_node_t {
    inner_node_t root;
    int low, high;
    outer_node_t *left, *right;

    outer_node_t(int lo, int hi, const T &v)
        : root(0, MAXC, v), low(lo), high(hi), left(NULL), right(NULL) {}
  } *root;

  T init;

  // Helper variables for query() and update().
  int tgt_r1, tgt_c1, tgt_r2, tgt_c2, width;

  template<class node_t>
  inline T call_query(node_t *n, int area) {
    return (n != NULL) ? query(n) : join_region(init, area);
  }

  T query(inner_node_t *n) {
    int lo = n->low, hi = n->high, mid = lo + (hi - lo)/2;
    if (tgt_c1 <= lo && hi <= tgt_c2) {
      T res = n->value;
      if (tgt_c1 < lo) {
        res = join_values(res, join_region(init, lo - tgt_c1 + 1));
      }
      if (hi < tgt_c2) {
        res = join_values(res, join_region(init, tgt_c2 - hi + 1));
      }
      return res;
    } else if (tgt_c2 <= mid) {
      return call_query(n->left, tgt_c2 - tgt_c1 + 1);
    } else if (mid < tgt_c1) {
      return call_query(n->right, tgt_c2 - tgt_c1 + 1);
    }
    return join_values(
        call_query(n->left, std::min(tgt_c2, mid) - tgt_c1 + 1),
        call_query(n->right, tgt_c2 - std::max(tgt_c1, mid + 1) + 1));
  }

  T query(outer_node_t *n) {
    int lo = n->low, hi = n->high, mid = lo + (hi - lo)/2;
    if (tgt_r1 <= lo && hi <= tgt_r2) {
      T res = query(&(n->root));
      if (tgt_r1 < lo) {
        res = join_values(res, join_region(init, width*(lo - tgt_r1 + 1)));
      }
      if (hi < tgt_r2) {
        res = join_values(res, join_region(init, width*(tgt_r2 - hi + 1)));
      }
      return res;
    } else if (tgt_r2 <= mid) {
      return call_query(n->left, tgt_r2 - tgt_r1 + 1);
    } else if (mid < tgt_r1) {
      return call_query(n->right, tgt_r2 - tgt_r1 + 1);
    }
    return join_values(
        call_query(n->left, width*(std::min(tgt_r2, mid) - tgt_r1 + 1)),
        call_query(n->right, width*(tgt_r2 - std::max(tgt_r1, mid + 1) + 1)));
  }

  void update(inner_node_t *n, int c, const T &d, bool leaf_row) {
    int lo = n->low, hi = n->high, mid = lo + (hi - lo)/2;
    if (lo == hi) {
      if (leaf_row) {
        n->value = join_value_with_delta(n->value, d);
      } else {
        n->value = d;
      }
      return;
    }
    inner_node_t *&target = (c <= mid) ? n->left : n->right;
    if (target == NULL) {
      target = new inner_node_t(c, c, init);
    }
    if (target->low <= c && c <= target->high) {
      update(target, c, d, leaf_row);
    } else {
      do {
        if (c <= mid) {
          hi = mid;
        } else {
          lo = mid + 1;
        }
        mid = lo + (hi - lo)/2;
      } while ((c <= mid) == (target->low <= mid));
      inner_node_t *tmp = new inner_node_t(lo, hi, init);
      if (target->low <= mid) {
        tmp->left = target;
      } else {
        tmp->right = target;
      }
      target = tmp;
      update(tmp, c, d, leaf_row);
    }
    T left_value = (n->left != NULL) ? n->left->value
                                     : join_region(init, mid - lo + 1);
    T right_value = (n->right != NULL) ? n->right->value
                                       : join_region(init, hi - mid);
    n->value = join_values(left_value, right_value);
  }

  void update(outer_node_t *n, int r, int c, const T &d) {
    int lo = n->low, hi = n->high, mid = lo + (hi - lo)/2;
    if (lo == hi) {
      update(&(n->root), c, d, true);
      return;
    }
    if (r <= mid) {
      if (n->left == NULL) {
        n->left = new outer_node_t(lo, mid, init);
      }
      update(n->left, r, c, d);
    } else {
      if (n->right == NULL) {
        n->right = new outer_node_t(mid + 1, hi, init);
      }
      update(n->right, r, c, d);
    }
    T value = join_region(init, hi - lo + 1);
    if (n->left != NULL || n->right != NULL) {
      tgt_c1 = tgt_c2 = c;
      T left_value = (n->left != NULL) ? query(&(n->left->root))
                                       : join_region(init, mid - lo + 1);
      T right_value = (n->right != NULL) ? query(&(n->right->root))
                                         : join_region(init, hi - mid);
      value = join_values(left_value, right_value);
    }
    update(&(n->root), c, value, false);
  }

  static void clean_up(inner_node_t *n) {
    if (n != NULL) {
      clean_up(n->left);
      clean_up(n->right);
      delete n;
    }
  }

  static void clean_up(outer_node_t *n) {
    if (n != NULL) {
      clean_up(n->root.left);
      clean_up(n->root.right);
      clean_up(n->left);
      clean_up(n->right);
      delete n;
    }
  }

 public:
  segment_tree_2d(const T &v = T())
      : root(new outer_node_t(0, MAXR, v)), init(v) {}

  ~segment_tree_2d() {
    clean_up(root);
  }

  T at(int r, int c) {
    return query(r, c, r, c);
  }

  T query(int r1, int c1, int r2, int c2) {
    tgt_r1 = r1;
    tgt_c1 = c1;
    tgt_r2 = r2;
    tgt_c2 = c2;
    width = c2 - c1;
    return query(root);
  }

  void update(int r, int c, const T &d) {
    update(root, r, c, d);
  }
};

/*** Example Usage and Output:

Values:
7 6 0
5 4 0
0 1 9

***/

#include <cassert>
#include <iostream>
using namespace std;

int main() {
  segment_tree_2d<int> t(0);
  t.update(0, 0, 7);
  t.update(0, 1, 6);
  t.update(1, 0, 5);
  t.update(1, 1, 4);
  t.update(2, 1, 1);
  t.update(2, 2, 9);
  cout << "Values:" << endl;
  for (int i = 0; i < 3; i++) {
    for (int j = 0; j < 3; j++) {
      cout << t.at(i, j) << " ";
    }
    cout << endl;
  }
  assert(t.query(0, 0, 0, 1) == 6);
  assert(t.query(0, 0, 1, 0) == 5);
  assert(t.query(1, 1, 2, 2) == 0);
  assert(t.query(0, 0, 1000000000, 1000000000) == 0);
  t.update(500000000, 500000000, -100);
  assert(t.query(0, 0, 1000000000, 1000000000) == -100);
  return 0;
}
\end{lstlisting}
\subsection{2D Range Tree}
\begin{lstlisting}
/*

Maintain a set of two-dimensional points while supporting queries for all points
that fall inside given rectangular regions. This implementation uses std::pair
to represent points, requiring operators < and == to be defined on the numeric
template type.

- range_tree(lo, hi) constructs a set from two random-access iterators to
  std::pair as a range [lo, hi) of points.
- query(x1, y1, x2, y2, f) calls the function f(i, p) on each point in the set
  that falls into the rectangular region consisting of rows from x1 to x2,
  inclusive, and columns from y1 to y2, inclusive. The first argument to f is
  the zero-based index of the point in the original range given to the
  constructor. The second argument is the point itself as an std::pair.

Time Complexity:
- O(n log n) per call to the constructor, where n is the number of points.
- O(log^2(n) + m) per call to query(), where m is the number of points that are
  reported by the query.

Space Complexity:
- O(n log n) for storage of the points.
- O(log^2(n)) auxiliary stack space for query().

*/

#include <algorithm>
#include <iterator>
#include <utility>
#include <vector>

template<class T>
class range_tree {
  typedef std::pair<T, T> point;
  typedef std::pair<int, T> colindex;

  std::vector<point> points;
  std::vector<std::vector<colindex> > columns;

  static inline bool comp1(const colindex &a, const colindex &b) {
    return a.second < b.second;
  }

  static inline bool comp2(const colindex &a, const T &v) {
    return a.second < v;
  }

  void build(int n, int lo, int hi) {
    if (points[lo].first == points[hi].first) {
      for (int i = lo; i <= hi; i++) {
        columns[n].push_back(point(i, points[i].second));
      }
      return;
    }
    int l = n*2 + 1, r = n*2 + 2, mid = lo + (hi - lo)/2;
    build(l, lo, mid);
    build(r, mid + 1, hi);
    columns[n].resize(columns[l].size() + columns[r].size());
    std::merge(columns[l].begin(), columns[l].end(),
               columns[r].begin(), columns[r].end(),
               columns[n].begin(), comp1);
  }

  // Helper variables for query().
  T x1, y1, x2, y2;

  template<class ReportFunction>
  void query(int n, int lo, int hi, ReportFunction f) {
    if (points[hi].first < x1 || x2 < points[lo].first) {
      return;
    }
    if (!(points[lo].first < x1 || x2 < points[hi].first)) {
      if (!columns[n].empty() && !(y2 < y1)) {
        typename std::vector<point>::iterator it;
        it = std::lower_bound(columns[n].begin(), columns[n].end(), y1, comp2);
        for (; it != columns[n].end() && it->second <= y2; ++it) {
          f(it->first, points[it->first]);
        }
      }
    } else if (lo != hi) {
      int mid = lo + (hi - lo)/2;
      query(n*2 + 1, lo, mid, f);
      query(n*2 + 2, mid + 1, hi, f);
    }
  }

 public:
  template<class It>
  range_tree(It lo, It hi) : points(lo, hi) {
    int n = std::distance(lo, hi);
    columns.resize(4*n + 1);
    std::sort(points.begin(), points.end());
    build(0, 0, n - 1);
  }

  template<class ReportFunction>
  void query(const T &x1, const T &y1, const T &x2, const T &y2,
             ReportFunction f) {
    this->x1 = x1;
    this->y1 = y1;
    this->x2 = x2;
    this->y2 = y2;
    query(0, 0, points.size() - 1, f);
  }
};

/*** Example Usage and Output:

(-1, -1) (2, -1) (2, 2) (1, 4)
(1, 4) (2, 2) (3, 1)

***/

#include <iostream>
using namespace std;

void print(int i, const pair<int, int> &p) {
  cout << "(" << p.first << ", " << p.second << ") ";
}

int main() {
  const int n = 10;
  int points[n][2] = {{1, 4}, {5, 4}, {2, 2}, {3, 1}, {6, -5}, {5, -1},
                      {3, -3}, {-1, -2}, {-1, -1}, {2, -1}};
  vector<pair<int, int> > v;
  for (int i = 0; i < n; i++) {
    v.push_back(make_pair(points[i][0], points[i][1]));
  }
  range_tree<int> t(v.begin(), v.end());
  t.query(-1, -1, 2, 5, print);
  cout << endl;
  t.query(1, 1, 4, 8, print);
  cout << endl;
  return 0;
}
\end{lstlisting}
\subsection{K-d Tree (2D Range Query)}
\begin{lstlisting}
/*

Maintain a set of two-dimensional points while supporting queries for all points
that fall inside given rectangular regions. This implementation uses std::pair
to represent points, requiring operators < and == to be defined on the numeric
template type.

- kd_tree(lo, hi) constructs a set from two random-access iterators to std::pair
  as a range [lo, hi) of points.
- query(x1, y1, x2, y2, f) calls the function f(i, p) on each point in the set
  that falls into the rectangular region consisting of rows from x1 to x2,
  inclusive, and columns from y1 to y2, inclusive. The first argument to f is
  the zero-based index of the point in the original range given to the
  constructor. The second argument is the point itself as an std::pair.

Time Complexity:
- O(n log n) per call to the constructor, where n is the number of points.
- O(log(n) + m) on average per call to query(), where m is the number of points
  that are reported by the query.

Space Complexity:
- O(n) for storage of the points.
- O(log n) auxiliary stack space for query().

*/

#include <algorithm>
#include <utility>
#include <vector>

template<class T>
class kd_tree {
  typedef std::pair<T, T> point;

  static inline bool comp1(const point &a, const point &b) {
    return a.first < b.first;
  }

  static inline bool comp2(const point &a, const point &b) {
    return a.second < b.second;
  }

  std::vector<point> tree, minp, maxp;
  std::vector<int> l_index, h_index;

  void build(int lo, int hi, bool div_x) {
    if (lo >= hi) {
      return;
    }
    int mid = lo + (hi - lo)/2;
    std::nth_element(tree.begin() + lo, tree.begin() + mid, tree.begin() + hi,
                     div_x ? comp1 : comp2);
    l_index[mid] = lo;
    h_index[mid] = hi;
    minp[mid].first = maxp[mid].first = tree[lo].first;
    minp[mid].second = maxp[mid].second = tree[lo].second;
    for (int i = lo + 1; i < hi; i++) {
      minp[mid].first = std::min(minp[mid].first, tree[i].first);
      minp[mid].second = std::min(minp[mid].second, tree[i].second);
      maxp[mid].first = std::max(maxp[mid].first, tree[i].first);
      maxp[mid].second = std::max(maxp[mid].second, tree[i].second);
    }
    build(lo, mid, !div_x);
    build(mid + 1, hi, !div_x);
  }

  // Helper variables for query().
  T x1, y1, x2, y2;

  template<class ReportFunction>
  void query(int lo, int hi, ReportFunction f) {
    if (lo >= hi) {
      return;
    }
    int mid = lo + (hi - lo)/2;
    T ax = minp[mid].first, ay = minp[mid].second;
    T bx = maxp[mid].first, by = maxp[mid].second;
    if (x2 < ax || bx < x1 || y2 < ay || by < y1) {
      return;
    }
    if (!(ax < x1 || x2 < bx || ay < y1 || y2 < by)) {
      for (int i = l_index[mid]; i < h_index[mid]; i++) {
        f(tree[i]);
      }
      return;
    }
    query(lo, mid, f);
    query(mid + 1, hi, f);
    if (tree[mid].first < x1 || x2 < tree[mid].first ||
        tree[mid].second < y1 || y2 < tree[mid].second) {
      return;
    }
    f(tree[mid]);
  }

 public:
  template<class It>
  kd_tree(It lo, It hi) : tree(lo, hi) {
    int n = std::distance(lo, hi);
    l_index.resize(n);
    h_index.resize(n);
    minp.resize(n);
    maxp.resize(n);
    build(0, n, true);
  }

  template<class ReportFunction>
  void query(const T &x1, const T &y1, const T &x2, const T &y2,
             ReportFunction f) {
    this->x1 = x1;
    this->y1 = y1;
    this->x2 = x2;
    this->y2 = y2;
    query(0, tree.size(), f);
  }
};

/*** Example Usage and Output:

(2, -1) (1, 4) (2, 2) (-1, -1)
(1, 4) (2, 2) (3, 1)

***/

#include <iostream>
using namespace std;

void print(const pair<int, int> &p) {
  cout << "(" << p.first << ", " << p.second << ") ";
}

int main() {
  const int n = 10;
  int points[n][2] = {{1, 4}, {5, 4}, {2, 2}, {3, 1}, {6, -5}, {5, -1},
                      {3, -3}, {-1, -2}, {-1, -1}, {2, -1}};
  vector<pair<int, int> > v;
  for (int i = 0; i < n; i++) {
    v.push_back(make_pair(points[i][0], points[i][1]));
  }
  kd_tree<int> t(v.begin(), v.end());
  t.query(-1, -1, 2, 5, print);
  cout << endl;
  t.query(1, 1, 4, 8, print);
  cout << endl;
  return 0;
}
\end{lstlisting}
\subsection{K-d Tree (Nearest Neighbor)}
\begin{lstlisting}
/*

Maintain a set of two-dimensional points while supporting queries for the
closest point in the set to a given query point. This implementation uses
std::pair to represent points, requiring operators <, ==, -, and long double
casting to be defined on the numeric template type.

- kd_tree(lo, hi) constructs a set from two random-access iterators to std::pair
  as a range [lo, hi) of points.
- nearest(x, y, can_equal) returns a point in the set that is closest to (x, y)
  by Euclidean distance. This may be equal to (x, y) only if can_equal is true.

Time Complexity:
- O(n log n) per call to the constructor, where n is the number of points.
- O(log n) on average per call to nearest().

Space Complexity:
- O(n) for storage of the points.
- O(log n) auxiliary stack space for nearest().

*/

#include <algorithm>
#include <limits>
#include <stdexcept>
#include <utility>
#include <vector>

template<class T>
class kd_tree {
  typedef std::pair<T, T> point;

  static inline bool comp1(const point &a, const point &b) {
    return a.first < b.first;
  }

  static inline bool comp2(const point &a, const point &b) {
    return a.second < b.second;
  }

  std::vector<point> tree;
  std::vector<bool> div_x;

  void build(int lo, int hi) {
    if (lo >= hi) {
      return;
    }
    int mid = lo + (hi - lo)/2;
    T minx, maxx, miny, maxy;
    minx = maxx = tree[lo].first;
    miny = maxy = tree[lo].second;
    for (int i = lo + 1; i < hi; i++) {
      minx = std::min(minx, tree[i].first);
      miny = std::min(miny, tree[i].second);
      maxx = std::max(maxx, tree[i].first);
      maxy = std::max(maxy, tree[i].second);
    }
    div_x[mid] = !((maxx - minx) < (maxy - miny));
    std::nth_element(tree.begin() + lo, tree.begin() + mid, tree.begin() + hi,
                     div_x[mid] ? comp1 : comp2);
    if (lo + 1 == hi) {
      return;
    }
    build(lo, mid);
    build(mid + 1, hi);
  }

  // Helper variables for nearest().
  long double min_dist;
  int id;

  void nearest(int lo, int hi, const T &x, const T &y, bool can_equal) {
    if (lo >= hi) {
      return;
    }
    int mid = lo + (hi - lo)/2;
    T dx = x - tree[mid].first, dy = y - tree[mid].second;
    long double d = dx*(long double)dx + dy*(long double)dy;
    if (d < min_dist && (can_equal || d != 0)) {
      min_dist = d;
      id = mid;
    }
    if (lo + 1 == hi) {
      return;
    }
    d = (long double)(div_x[mid] ? dx : dy);
    int l1 = lo, r1 = mid, l2 = mid + 1, r2 = hi;
    if (d > 0) {
      std::swap(l1, l2);
      std::swap(r1, r2);
    }
    nearest(l1, r1, x, y, can_equal);
    if (d*(long double)d < min_dist) {
      nearest(l2, r2, x, y, can_equal);
    }
  }

 public:
  template<class It>
  kd_tree(It lo, It hi) : tree(lo, hi) {
    int n = std::distance(lo, hi);
    if (n <= 1) {
      throw std::runtime_error("K-d tree must be have at least 2 points.");
    }
    div_x.resize(n);
    build(0, n);
  }

  point nearest(const T &x, const T &y, bool can_equal = true) {
    min_dist = std::numeric_limits<long double>::max();
    nearest(0, tree.size(), x, y, can_equal);
    return tree[id];
  }
};

/*** Example Usage ***/

#include <cassert>
using namespace std;

int main() {
  pair<int, int> p[3];
  p[0] = make_pair(0, 2);
  p[1] = make_pair(0, 3);
  p[2] = make_pair(-1, 0);
  kd_tree<int> t(p, p + 3);
  assert(t.nearest(0, 2, true) == make_pair(0, 2));
  assert(t.nearest(0, 2, false) == make_pair(0, 3));
  assert(t.nearest(0, 0) == make_pair(-1, 0));
  assert(t.nearest(-10000, 0) == make_pair(-1, 0));
  return 0;
}
\end{lstlisting}
\subsection{R-Tree (Nearest Segment)}
\begin{lstlisting}
/*

Maintain a set of two-dimensional line segments while supporting queries for the
closest segment in the set to a given query point. This implementation uses
integer points and long doubles for intermediate calculations.

- r_tree(lo, hi) constructs a set from two random-access iterators as a range
  [lo, hi) of segments.
- nearest(x, y) returns a segment in the set that contains some point which is
  as close or closer to (x, y) by Euclidean distance than any point on any
  other segment in the set.

Time Complexity:
- O(n log n) per call to the constructor, where n is the number of segments.
- O(log n) on average per call to nearest().

Space Complexity:
- O(n) for storage of the segments.
- O(log n) auxiliary stack space for nearest().

*/

#include <algorithm>
#include <limits>
#include <stdexcept>
#include <vector>

struct segment {
  int x1, y1, x2, y2;

  segment() : x1(0), y1(0), x2(0), y2(0) {}
  segment(int x1, int y1, int x2, int y2) : x1(x1), y1(y1), x2(x2), y2(y2) {}

  bool operator==(const segment &s) const {
    return (x1 == s.x1) && (y1 == s.y1) && (x2 == s.x2) && (y2 == s.y2);
  }
};

class r_tree {
  static inline bool cmp_x(const segment &a, const segment &b) {
    return a.x1 + a.x2 < b.x1 + b.x2;
  }

  static inline bool cmp_y(const segment &a, const segment &b) {
    return a.y1 + a.y2 < b.y1 + b.y2;
  }

  static inline int seg_dist(int v, int lo, int hi) {
    return (v <= lo) ? (lo - v) : (v >= hi ? v - hi : 0);
  }

  static long double point_to_segment_squared(int x, int y, const segment &s) {
    long long dx = s.x2 - s.x1, dy = s.y2 - s.y1;
    long long px = x - s.x1, py = y - s.y1;
    long long sqdist = dx*dx + dy*dy;
    long long dot = dx*px + dy*py;
    if (dot <= 0 || sqdist == 0) {
      return px*px + py*py;
    }
    if (dot >= sqdist) {
      return (px - dx)*(px - dx) + (py - dy)*(py - dy);
    }
    double q = (double)dot / sqdist;
    return (px - q*dx)*(px - q*dx) + (py - q*dy)*(py - q*dy);
  }

  std::vector<segment> tree;
  std::vector<int> minx, maxx, miny, maxy;

  void build(int lo, int hi, bool div_x) {
    if (lo >= hi) {
      return;
    }
    int mid = lo + (hi - lo)/2;
    std::nth_element(tree.begin() + lo, tree.begin() + mid, tree.begin() + hi,
                     div_x ? cmp_x : cmp_y);
    for (int i = lo; i < hi; i++) {
      minx[mid] = std::min(minx[mid], std::min(tree[i].x1, tree[i].x2));
      miny[mid] = std::min(miny[mid], std::min(tree[i].y1, tree[i].y2));
      maxx[mid] = std::max(maxx[mid], std::max(tree[i].x1, tree[i].x2));
      maxy[mid] = std::max(maxy[mid], std::max(tree[i].y1, tree[i].y2));
    }
    build(lo, mid, !div_x);
    build(mid + 1, hi, !div_x);
  }

  // Helper variables for nearest().
  double min_dist;
  int id;

  void nearest(int lo, int hi, int x, int y, bool div_x) {
    if (lo >= hi) {
      return;
    }
    int mid = lo + (hi - lo)/2;
    long double d = point_to_segment_squared(x, y, tree[mid]);
    if (min_dist > d) {
      min_dist = d;
      id = mid;
    }
    long long delta = div_x ? (2*x - tree[mid].x1 - tree[mid].x2)
                            : (2*y - tree[mid].y1 - tree[mid].y2);
    if (delta <= 0) {
      nearest(lo, mid, x, y, !div_x);
      if (mid + 1 < hi) {
        int mid1 = (mid + hi + 1)/2;
        long long dist = div_x ? seg_dist(x, minx[mid1], maxx[mid1])
                               : seg_dist(y, miny[mid1], maxy[mid1]);
        if (dist*dist < min_dist) {
          nearest(mid + 1, hi, x, y, !div_x);
        }
      }
    } else {
      nearest(mid + 1, hi, x, y, !div_x);
      if (lo < mid) {
        int mid1 = lo + (mid - lo)/2;
        long long dist = div_x ? seg_dist(x, minx[mid1], maxx[mid1]) :
                                 seg_dist(y, miny[mid1], maxy[mid1]);
        if (dist*dist < min_dist) {
          nearest(lo, mid, x, y, !div_x);
        }
      }
    }
  }

 public:
  template<class It>
  r_tree(It lo, It hi) : tree(lo, hi) {
    int n = std::distance(lo, hi);
    if (n <= 1) {
      throw std::runtime_error("R-tree must be have at least 2 segments.");
    }
    minx.assign(n, std::numeric_limits<int>::max());
    maxx.assign(n, std::numeric_limits<int>::min());
    miny.assign(n, std::numeric_limits<int>::max());
    maxy.assign(n, std::numeric_limits<int>::min());
    build(0, n, true);
  }

  segment nearest(int x, int y) {
    min_dist = std::numeric_limits<long double>::max();
    nearest(0, tree.size(), x, y, true);
    return tree[id];
  }
};

/*** Example Usage ***/

#include <cassert>
using namespace std;

int main() {
  segment s[4];
  s[0] = segment(0, 0, 0, 4);
  s[1] = segment(0, 4, 4, 4);
  s[2] = segment(4, 4, 4, 0);
  s[3] = segment(4, 0, 0, 0);
  r_tree t(s, s + 4);
  assert(t.nearest(-1, 2) == segment(0, 0, 0, 4));
  assert(t.nearest(100, 100) == segment(4, 4, 4, 0));
  return 0;
}
\end{lstlisting}

\section{Fenwick Trees}
\setcounter{section}{5}
\setcounter{subsection}{0}
\subsection{Fenwick Tree (Simple)}
\begin{lstlisting}
/*

Maintain an array of numerical type, allowing for updates of individual indices
(point update) and queries for the sum of contiguous sub-arrays (range queries).
This implementation assumes that the array is 1-based (i.e. has valid indices
from 1 to MAXN - 1, inclusive).

- initialize() resets the data structure.
- a[i] stores the value at index i.
- add(i, x) adds x to the value at index i.
- set(i, x) assigns the value at index i to x.
- sum(hi) returns the sum of all values at indices from 1 to hi, inclusive.
- sum(lo, hi) returns the sum of all values at indices from lo to hi, inclusive.

Time Complexity:
- O(n) per call to initialize(), where n is the size of the array.
- O(log n) per call to all other operations.

Space Complexity:
- O(n) for storage of the array elements.
- O(1) auxiliary for all operations.

*/

const int MAXN = 1000;
int a[MAXN + 1], t[MAXN + 1];

void initialize() {
  for (int i = 0; i <= MAXN; i++) {
    a[i] = t[i] = 0;
  }
}

void add(int i, int x) {
  a[i] += x;
  for (; i <= MAXN; i += i & -i) {
    t[i] += x;
  }
}

void set(int i, int x) {
  add(i, x - a[i]);
}

int sum(int hi) {
  int res = 0;
  for (; hi > 0; hi -= hi & -hi) {
    res += t[hi];
  }
  return res;
}

int sum(int lo, int hi) {
  return sum(hi) - sum(lo - 1);
}

/*** Example Usage and Output:

Values: 5 1 2 3 4

***/

#include <cassert>
#include <iostream>
using namespace std;

int main() {
  int v[] = {10, 1, 2, 3, 4};
  initialize();
  for (int i = 1; i <= 5; i++) {
    set(i, v[i - 1]);
  }
  add(1, -5);
  cout << "Values: ";
  for (int i = 1; i <= 5; i++) {
    cout << a[i] << " ";
  }
  cout << endl;
  assert(sum(2, 4) == 6);
  return 0;
}
\end{lstlisting}
\subsection{Fenwick Tree (Range Update, Point Query)}
\begin{lstlisting}
/*

Maintain an array of numerical type, allowing for contiguous sub-arrays to be
simultaneously incremented by arbitrary values (range update) and values at
individual indices to be queried (point query). This implementation assumes that
the array is 0-based (i.e. has valid indices from 0 to size() - 1, inclusive).

- size() returns the size of the array.
- at(i) returns the value at index i.
- add(i, x) adds x to the value at index i.
- add(lo, hi, x) adds x to the values at all indices from lo to hi, inclusive.

Time Complexity:
- O(n) per call to the constructor, where n is the size of the array.
- O(1) per call to size().
- O(log n) per call to at() and both add() functions.

Space Complexity:
- O(n) for storage of the array elements.
- O(1) auxiliary for all operations.

*/

#include <vector>

template<class T>
class fenwick_tree {
  int len;
  std::vector<int> t;

 public:
  fenwick_tree(int n) : len(n), t(n + 2) {}

  int size() const {
    return len;
  }

  T at(int i) const {
    T res = 0;
    for (i++; i > 0; i -= i & -i) {
      res += t[i];
    }
    return res;
  }

  void add(int i, const T &x) {
    for (i++; i <= len + 1; i += i & -i) {
      t[i] += x;
    }
  }

  void add(int lo, int hi, const T &x) {
    add(lo, x);
    add(hi + 1, -x);
  }
};

/*** Example Usage and Output:

Values: 5 10 15 10 10

***/

#include <iostream>
using namespace std;

int main() {
  fenwick_tree<int> t(5);
  t.add(0, 1, 5);
  t.add(1, 2, 5);
  t.add(2, 4, 10);
  cout << "Values: ";
  for (int i = 0; i < t.size(); i++) {
    cout << t.at(i) << " ";
  }
  cout << endl;
  return 0;
}
\end{lstlisting}
\subsection{Fenwick Tree (Point Update, Range Query)}
\begin{lstlisting}
/*

Maintain an array of numerical type, allowing for updates of individual indices
(point update) and queries for the sum of contiguous sub-arrays (range queries).
This implementation assumes that the array is 0-based (i.e. has valid indices
from 0 to size() - 1, inclusive).

- size() returns the size of the array.
- at(i) returns the value at index i.
- add(i, x) adds x to the value at index i.
- set(i, x) assigns the value at index i to x.
- sum(hi) returns the sum of all values at indices from 0 to hi, inclusive.
- sum(lo, hi) returns the sum of all values at indices from lo to hi, inclusive.

Time Complexity:
- O(n) per call to the constructor, where n is the size of the array.
- O(1) per call to size() and at().
- O(log n) per call to add(), set(), and both sum() functions.

Space Complexity:
- O(n) for storage of the array elements.
- O(1) auxiliary for all operations.

*/

#include <vector>

template<class T>
class fenwick_tree {
  int len;
  std::vector<int> a, t;

 public:
  fenwick_tree(int n) : len(n), a(n + 1), t(n + 1) {}

  int size() const {
    return len;
  }

  T at(int i) const {
    return a[i + 1];
  }

  void add(int i, const T &x) {
    a[++i] += x;
    for (; i <= len; i += i & -i) {
      t[i] += x;
    }
  }

  void set(int i, const T &x) {
    T inc = x - a[i + 1];
    add(i, inc);
  }

  T sum(int hi) {
    T res = 0;
    for (hi++; hi > 0; hi -= hi & -hi) {
      res += t[hi];
    }
    return res;
  }

  T sum(int lo, int hi) {
    return sum(hi) - sum(lo - 1);
  }
};

/*** Example Usage and Output:

Values: 5 1 2 3 4

***/

#include <cassert>
#include <iostream>
using namespace std;

int main() {
  int a[] = {10, 1, 2, 3, 4};
  fenwick_tree<int> t(5);
  for (int i = 0; i < 5; i++) {
    t.set(i, a[i]);
  }
  t.add(0, -5);
  cout << "Values: ";
  for (int i = 0; i < t.size(); i++) {
    cout << t.at(i) << " ";
  }
  cout << endl;
  assert(t.sum(1, 3) == 6);
  return 0;
}
\end{lstlisting}
\subsection{Fenwick Tree (Range Update, Range Query)}
\begin{lstlisting}
/*

Maintain an array of numerical type, allowing for contiguous sub-arrays to be
simultaneously incremented by arbitrary values (range update) and queries for
the sum of contiguous sub-arrays (range query). This implementation assumes that
the array is 0-based (i.e. has valid indices from 0 to size() - 1, inclusive).

- size() returns the size of the array.
- at(i) returns the value at index i.
- add(i, x) increments the value at index i by x.
- add(lo, hi, x) adds x to the values at all indices from lo to hi, inclusive.
- set(i, x) assigns the value at index i to x.
- sum(hi) returns the sum of all values at indices from 0 to hi, inclusive.
- sum(lo, hi) returns the sum of all values at indices from lo to hi, inclusive.

Time Complexity:
- O(n) per call to the constructor, where n is the size of the array.
- O(1) per call to size().
- O(log n) per call to all other operations.

Space Complexity:
- O(n) for storage of the array elements.
- O(1) auxiliary for all operations.

*/

#include <vector>

template<class T>
class fenwick_tree {
  int len;
  std::vector<T> t1, t2;

  T sum(const std::vector<T> &t, int i) {
    T res = 0;
    for (; i != 0; i -= i & -i) {
      res += t[i];
    }
    return res;
  }

  void add(std::vector<T> &t, int i, const T &x) {
    for (; i <= len + 1; i += i & -i) {
      t[i] += x;
    }
  }

 public:
  fenwick_tree(int n) : len(n), t1(n + 2), t2(n + 2) {}

  int size() const {
    return len;
  }

  void add(int lo, int hi, const T &x) {
    lo++;
    hi++;
    add(t1, lo, x);
    add(t1, hi + 1, -x);
    add(t2, lo, x*(lo - 1));
    add(t2, hi + 1, -x*hi);
  }

  void add(int i, const T &x) {
    return add(i, i, x);
  }

  void set(int i, const T &x) {
    add(i, x - at(i));
  }

  T sum(int hi) {
    hi++;
    return hi*sum(t1, hi) - sum(t2, hi);
  }

  T sum(int lo, int hi) {
    return sum(hi) - sum(lo - 1);
  }

  T at(int i) {
    return sum(i, i);
  }
};

/*** Example Usage and Output:

Values: 15 6 7 -5 4

***/

#include <cassert>
#include <iostream>
using namespace std;

int main() {
  int a[] = {10, 1, 2, 3, 4};
  fenwick_tree<int> t(5);
  for (int i = 0; i < t.size(); i++) {
    t.set(i, a[i]);
  }
  t.add(0, 2, 5);
  t.set(3, -5);
  cout << "Values: ";
  for (int i = 0; i < t.size(); i++) {
    cout << t.at(i) << " ";
  }
  cout << endl;
  assert(t.sum(0, 4) == 27);
  return 0;
}
\end{lstlisting}
\subsection{Fenwick Tree (Compressed)}
\begin{lstlisting}
/*

Maintain an array of numerical type, allowing for contiguous sub-arrays to be
simultaneously incremented by arbitrary values (range update) and queries for
the sum of contiguous sub-arrays (range query). This implementation uses
std::map for coordinate compression, allowing for large indices to be accessed
with efficient space complexity. That is, all array indices from 0 to MAXN,
inclusive, are accessible.

- at(i) returns the value at index i.
- add(i, x) adds x to the value at index i.
- add(lo, hi, x) adds x to the values at all indices from lo to hi, inclusive.
- set(i, x) assigns the value at index i to x.
- sum(hi) returns the sum of all values at indices from 0 to hi, inclusive.
- sum(lo, hi) returns the sum of all values at indices from lo to hi, inclusive.

Time Complexity:
- O(log^2 MAXN) per call to all member functions. If std::map is replaced with
  std::unordered_map, then the amortized running time will become O(log MAXN).

Space Complexity:
- O(n log MAXN) for storage of the array elements, where n is the number of
  distinct indices that have been accessed across all of the operations so far.
- O(1) auxiliary for all operations.

*/

#include <map>

template<class T>
class fenwick_tree {
  static const int MAXN = 1000000001;
  std::map<int, T> tmul, tadd;

  void add_helper(int at, int mul, T add) {
    for (int i = at; i <= MAXN; i |= i + 1) {
      tmul[i] += mul;
      tadd[i] += add;
    }
  }

 public:
  void add(int lo, int hi, const T &x) {
    add_helper(lo, x, -x*(lo - 1));
    add_helper(hi, -x, x*hi);
  }

  void add(int i, const T &x) {
    return add(i, i, x);
  }

  void set(int i, const T &x) {
    add(i, x - at(i));
  }

  T sum(int hi) {
    T mul = 0, add = 0;
    for (int i = hi; i >= 0; i = (i & (i + 1)) - 1) {
      if (tmul.find(i) != tmul.end()) {
        mul += tmul[i];
      }
      if (tadd.find(i) != tadd.end()) {
        add += tadd[i];
      }
    }
    return mul*hi + add;
  }

  T sum(int lo, int hi) {
    return sum(hi) - sum(lo - 1);
  }

  T at(int i) {
    return sum(i, i);
  }
};

/*** Example Usage and Output:

Values: 15 6 7 -5 4

***/

#include <cassert>
#include <iostream>
using namespace std;

int main() {
  int a[] = {10, 1, 2, 3, 4};
  fenwick_tree<int> t;
  for (int i = 0; i < 5; i++) {
    t.set(i, a[i]);
  }
  t.add(0, 2, 5);
  t.set(3, -5);
  cout << "Values: ";
  for (int i = 0; i < 5; i++) {
    cout << t.at(i) << " ";
  }
  cout << endl;
  assert(t.sum(0, 4) == 27);
  t.add(500000001, 500000010, 3);
  t.add(500000011, 500000015, 5);
  t.set(500000000, 10);
  assert(t.sum(0, 1000000000) == 92);
  return 0;
}
\end{lstlisting}
\subsection{2D Fenwick Tree (Simple)}
\begin{lstlisting}
/*

Maintain a 2D array of numerical type, allowing for updates of individual cells
in the matrix (point update) and queries for the sum of rectangular sub-matrices
(range query). This implementation assumes that array dimensions are 1-based
(i.e. rows have valid indices from 1 to MAXR, inclusive, and columns have valid
indices from 1 to MAXC, inclusive).

- initialize() resets the data structure.
- a[r][c] stores the value at index (r, c).
- add(r, c, x) adds x to the value at index (r, c).
- set(r, c, x) assigns x to the value at index (r, c).
- sum(r, c) returns the sum of the rectangle with upper-left corner (1, 1) and
  lower-right corner (r, c).
- sum(r1, c1, r2, c2) returns the sum of the rectangle with upper-left corner
  (r1, c1) and lower-right corner (r2, c2).

Time Complexity:
- O(n*m) per call to initialize(), where n is the number of rows and m is the
  number of columns.
- O(log(n)*log(m)) per call to all other operations.

Space Complexity:
- O(n*m) for storage of the array elements.
- O(1) auxiliary for all operations.

*/

const int MAXR = 100, MAXC = 100;
int a[MAXR + 1][MAXC + 1];
int bits[MAXR + 1][MAXC + 1];

void initialize() {
  for (int i = 0; i <= MAXR; i++) {
    for (int j = 0; j <= MAXC; j++) {
      a[i][j] = bits[i][j] = 0;
    }
  }
}

void add(int r, int c, int x) {
  a[r][c] += x;
  for (int i = r; i <= MAXR; i += i & -i) {
    for (int j = c; j <= MAXC; j += j & -j) {
      bits[i][j] += x;
    }
  }
}

void set(int r, int c, int x) {
  add(r, c, x - a[r][c]);
}

int sum(int r, int c) {
  int res = 0;
  for (int i = r; i > 0; i -= i & -i) {
    for (int j = c; j > 0; j -= j & -j) {
      res += bits[i][j];
    }
  }
  return res;
}

int sum(int r1, int c1, int r2, int c2) {
  return sum(r2, c2) + sum(r1 - 1, c1 - 1) -
         sum(r1 - 1, c2) - sum(r2, c1 - 1);
}

/*** Example Usage and Output:

Values:
5 6 0
3 0 0
0 0 9

***/

#include <cassert>
#include <iostream>
using namespace std;

int main() {
  initialize();
  set(1, 1, 5);
  set(1, 2, 6);
  set(2, 1, 7);
  add(3, 3, 9);
  add(2, 1, -4);
  cout << "Values:" << endl;
  for (int i = 1; i <= 3; i++) {
    for (int j = 1; j <= 3; j++) {
      cout << a[i][j] << " ";
    }
    cout << endl;
  }
  assert(sum(1, 1, 1, 2) == 11);
  assert(sum(1, 1, 2, 1) == 8);
  assert(sum(1, 1, 3, 3) == 23);
  return 0;
}
\end{lstlisting}
\subsection{2D Fenwick Tree (Compressed)}
\begin{lstlisting}
/*

Maintain a 2D array of numerical type, allowing for rectangular sub-matrices to
be simultaneously incremented by arbitrary values (range update) and queries for
the sum of rectangular sub-matrices (range query). This implementation uses
std::map for coordinate compression, allowing for large indices to be accessed
with efficient space complexity. That is, rows have valid indices from 0 to
MAXR, inclusive, and columns have valid indices from 0 to MAXC, inclusive.

- add(r, c, x) adds x to the value at index (r, c).
- add(r1, c1, r2, c2, x) adds x to all indices in the rectangle with upper-left
  corner (r1, c1) and lower-right corner (r2, c2).
- set(r, c, x) assigns x to the value at index (r, c).
- sum(r, c) returns the sum of the rectangle with upper-left corner (0, 0) and
  lower-right corner (r, c).
- sum(r1, c1, r2, c2) returns the sum of the rectangle with upper-left corner
  (r1, c1) and lower-right corner (r2, c2).
- at(r, c) returns the value at index (r, c).

Time Complexity:
- O(log^2(MAXR)*log^2(MAXC)) per call to all member functions. If std::map is
  replaced with std::unordered_map, then the amortized running time will become
  O(log(MAXR)*log(MAXC)).

Space Complexity:
- O(n*log(MAXR)*log(MAXC)) for storage of the array elements, where n is the
  number of distinct indices that have been accessed across all of the
  operations so far.
- O(1) auxiliary for all operations.

*/

#include <map>
#include <utility>

template<class T>
class fenwick_tree_2d {
  static const int MAXR = 1000000001;
  static const int MAXC = 1000000001;
  std::map<std::pair<int, int>, T> t1, t2, t3, t4;

  template<class Map>
  void add(Map &tree, int r, int c, const T &x) {
    for (int i = r + 1; i <= MAXR; i += i & -i) {
      for (int j = c + 1; j <= MAXC; j += j & -j) {
        tree[std::make_pair(i, j)] += x;
      }
    }
  }

  void add_helper(int r, int c, const T &x) {
    add(t1, 0, 0, x);
    add(t1, 0, c, -x);
    add(t2, 0, c, x*c);
    add(t1, r, 0, -x);
    add(t3, r, 0, x*r);
    add(t1, r, c, x);
    add(t2, r, c, -x*c);
    add(t3, r, c, -x*r);
    add(t4, r, c, x*r*c);
  }

 public:
  void add(int r1, int c1, int r2, int c2, const T &x) {
    add_helper(r2 + 1, c2 + 1, x);
    add_helper(r1, c2 + 1, -x);
    add_helper(r2 + 1, c1, -x);
    add_helper(r1, c1, x);
  }

  void add(int r, int c, const T &x) {
    add(r, c, r, c, x);
  }

  void set(int r, int c, const T &x) {
    add(r, c, x - at(r, c));
  }

  T sum(int r, int c) {
    r++;
    c++;
    T s1 = 0, s2 = 0, s3 = 0, s4 = 0;
    for (int i = r; i > 0; i -= i & -i) {
      for (int j = c; j > 0; j -= j & -j) {
        const std::pair<int, int> ij(i, j);
        s1 += t1[ij];
        s2 += t2[ij];
        s3 += t3[ij];
        s4 += t4[ij];
      }
    }
    return s1*r*c + s2*r + s3*c + s4;
  }

  T sum(int r1, int c1, int r2, int c2) {
    return sum(r2, c2) + sum(r1 - 1, c1 - 1) -
           sum(r1 - 1, c2) - sum(r2, c1 - 1);
  }

  T at(int r, int c) {
    return sum(r, c, r, c);
  }
};

/*** Example Usage and Output:

Values:
5 6 0
3 5 5
0 5 14

***/

#include <cassert>
#include <iostream>
using namespace std;

int main() {
  fenwick_tree_2d<int> t;
  t.set(0, 0, 5);
  t.set(0, 1, 6);
  t.set(1, 0, 7);
  t.add(2, 2, 9);
  t.add(1, 0, -4);
  t.add(1, 1, 2, 2, 5);
  cout << "Values:" << endl;
  for (int i = 0; i < 3; i++) {
    for (int j = 0; j < 3; j++) {
      cout << t.at(i, j) << " ";
    }
    cout << endl;
  }
  assert(t.sum(0, 0, 0, 1) == 11);
  assert(t.sum(0, 0, 1, 0) == 8);
  assert(t.sum(1, 1, 2, 2) == 29);
  t.set(500000000, 500000000, 100);
  assert(t.sum(0, 0, 1000000000, 1000000000) == 143);
  return 0;
}
\end{lstlisting}

\section{Tree Data Structures}
\setcounter{section}{6}
\setcounter{subsection}{0}
\subsection{Disjoint Set Forest (Simple)}
\begin{lstlisting}
/*

Maintain a set of elements partitioned into non-overlapping subsets. Each
partition is assigned a unique representative known as the parent, or root. The
following implements two well-known optimizations known as union-by-rank and
path compression. This version is simplified to only work on integer elements.

- initialize() resets the data structure.
- make_set(u) creates a new partition consisting of the single element u, which
  must not have been previously added to the data structure.
- find_root(u) returns the unique representative of the partition containing u.
- is_united(u, v) returns whether elements u and v belong to the same partition.
- unite(u, v) replaces the partitions containing u and v with a single new
  partition consisting of the union of elements in the original partitions.

A precondition to the last three operations is that make_set() must have been
previously called on their arguments.

Time Complexity:
- O(1) per call to initialize() and make_set().
- O(a(n)) per call to find_root(), is_united(), and unite(), where n is the
  number of elements that has been added via make_set() so far, and a(n) is the
  extremely slow growing inverse of the Ackermann function (effectively a very
  small constant for all practical values of n).

Space Complexity:
- O(n) for storage of the disjoint set forest elements.
- O(1) auxiliary for all operations.

*/

const int MAXN = 1000;
int num_sets, root[MAXN], rank[MAXN];

void initialize() {
  num_sets = 0;
}

void make_set(int u) {
  root[u] = u;
  rank[u] = 0;
  num_sets++;
}

int find_root(int u) {
  if (root[u] != u) {
    root[u] = find_root(root[u]);
  }
  return root[u];
}

bool is_united(int u, int v) {
  return find_root(u) == find_root(v);
}

void unite(int u, int v) {
  int ru = find_root(u), rv = find_root(v);
  if (ru == rv) {
    return;
  }
  num_sets--;
  if (rank[ru] < rank[rv]) {
    root[ru] = rv;
  } else {
    root[rv] = ru;
    if (rank[ru] == rank[rv]) {
      rank[ru]++;
    }
  }
}

/*** Example Usage ***/

#include <cassert>

int main() {
  initialize();
  for (char c = 'a'; c <= 'g'; c++) {
    make_set(c);
  }
  unite('a', 'b');
  unite('b', 'f');
  unite('d', 'e');
  unite('d', 'g');
  assert(num_sets == 3);
  assert(is_united('a', 'b'));
  assert(!is_united('a', 'c'));
  assert(!is_united('b', 'g'));
  assert(is_united('e', 'g'));
  return 0;
}
\end{lstlisting}
\subsection{Disjoint Set Forest (Compressed)}
\begin{lstlisting}
/*

Maintain a set of elements partitioned into non-overlapping subsets using a
collection of trees. Each partition is assigned a unique representative known as
the parent, or root. The following implements two well-known optimizations known
as union-by-rank and path compression. This version uses an std::map for storage
and coordinate compression (thus, element types must meet the requirements of
key types for std::map).

- make_set(u) creates a new partition consisting of the single element u, which
  must not have been previously added to the data structure.
- is_united(u, v) returns whether elements u and v belong to the same partition.
- unite(u, v) replaces the partitions containing u and v with a single new
  partition consisting of the union of elements in the original partitions.
- get_all_sets() returns all current partitions as a vector of vectors.

A precondition to the last three operations is that make_set() must have been
previously called on their arguments.

Time Complexity:
- O(1) per call to the constructor.
- O(log n) per call to make_set(), where n is the number of elements that have
  been added via make_set() so far.
- O(a(n) log n) per call to is_united() and unite(), where n is the number of
  elements that have been added via make_set() so far, and a(n) is the extremely
  slow growing inverse of the Ackermann function (effectively a very small
  constant for all practical values of n).
- O(n) per call to get_all_sets().

Space Complexity:
- O(n) for storage of the disjoint set forest elements.
- O(n) auxiliary heap space for get_all_sets().
- O(1) auxiliary for all other operations.

*/

#include <map>
#include <vector>

template<class T>
class disjoint_set_forest {
  int num_elements, num_sets;
  std::map<T, int> id;
  std::vector<int> root, rank;

  int find_root(int u) {
    if (root[u] != u) {
      root[u] = find_root(root[u]);
    }
    return root[u];
  }

 public:
  disjoint_set_forest() : num_elements(0), num_sets(0) {}

  int size() const {
    return num_elements;
  }

  int sets() const {
    return num_sets;
  }

  void make_set(const T &u) {
    if (id.find(u) != id.end()) {
      return;
    }
    id[u] = num_elements;
    root.push_back(num_elements++);
    rank.push_back(0);
    num_sets++;
  }

  bool is_united(const T &u, const T &v) {
    return find_root(id[u]) == find_root(id[v]);
  }

  void unite(const T &u, const T &v) {
    int ru = find_root(id[u]), rv = find_root(id[v]);
    if (ru == rv) {
      return;
    }
    num_sets--;
    if (rank[ru] < rank[rv]) {
      root[ru] = rv;
    } else {
      root[rv] = ru;
      if (rank[ru] == rank[rv]) {
        rank[ru]++;
      }
    }
  }

  std::vector<std::vector<T> > get_all_sets() {
    std::map<int, std::vector<T> > tmp;
    for (typename std::map<T, int>::iterator it = id.begin(); it != id.end();
         ++it) {
      tmp[find_root(it->second)].push_back(it->first);
    }
    std::vector<std::vector<T> > res;
    for (typename std::map<int, std::vector<T> >::iterator it = tmp.begin();
         it != tmp.end(); ++it) {
      res.push_back(it->second);
    }
    return res;
  }
};

/*** Example Usage and Output:

[a, b, f], [c], [d, e, g]

***/

#include <cassert>
#include <iostream>
using namespace std;

int main() {
  disjoint_set_forest<char> dsf;
  for (char c = 'a'; c <= 'g'; c++) {
    dsf.make_set(c);
  }
  dsf.unite('a', 'b');
  dsf.unite('b', 'f');
  dsf.unite('d', 'e');
  dsf.unite('d', 'g');
  assert(dsf.size() == 7);
  assert(dsf.sets() == 3);
  vector< vector<char> > s = dsf.get_all_sets();
  for (int i = 0; i < (int)s.size(); i++) {
    cout << (i > 0 ? ", [" : "[");
    for (int j = 0; j < (int)s[i].size(); j++) {
      cout << (j > 0 ? ", " : "") << s[i][j];
    }
    cout << "]";
  }
  cout << endl;
  return 0;
}
\end{lstlisting}
\subsection{Lowest Common Ancestor (Sparse Table)}
\begin{lstlisting}
/*

Given a tree, determine the lowest common ancestor of any two nodes in the tree.
The lowest common ancestor of two nodes u and v is the node that has the longest
distance from the root while having both u and v as its descendant. A nodes is
considered to be a descendant of itself. build() applies to a global,
pre-populated adjacency list adj[] which must only consist of nodes numbered
with integers between 0 (inclusive) and the total number of nodes (exclusive),
as passed in the function argument.

Time Complexity:
- O(n log n) per call to build(), where n is the number of nodes.
- O(log n) per call to lca().

Space Complexity:
- O(n log n) to store the sparse table, where n is the number of nodes.
- O(n) auxiliary stack space for build().
- O(1) auxiliary for lca().

*/

#include <vector>

const int MAXN = 1000;
std::vector<int> adj[MAXN], dp[MAXN];
int len, timer, tin[MAXN], tout[MAXN];

void dfs(int u, int p) {
  tin[u] = timer++;
  dp[u][0] = p;
  for (int i = 1; i < len; i++) {
    dp[u][i] = dp[dp[u][i - 1]][i - 1];
  }
  for (int j = 0; j < (int)adj[u].size(); j++) {
    int v = adj[u][j];
    if (v != p) {
      dfs(v, u);
    }
  }
  tout[u] = timer++;
}

void build(int nodes, int root = 0) {
  len = 1;
  while ((1 << len) <= nodes) {
    len++;
  }
  for (int i = 0; i < nodes; i++) {
    dp[i].resize(len);
  }
  timer = 0;
  dfs(root, root);
}

bool is_ancestor(int parent, int child) {
  return (tin[parent] <= tin[child]) && (tout[child] <= tout[parent]);
}

int lca(int u, int v) {
  if (is_ancestor(u, v)) {
    return u;
  }
  if (is_ancestor(v, u)) {
    return v;
  }
  for (int i = len - 1; i >= 0; i--) {
    if (!is_ancestor(dp[u][i], v)) {
      u = dp[u][i];
    }
  }
  return dp[u][0];
}

/*** Example Usage ***/

#include <cassert>
using namespace std;

int main() {
  adj[0].push_back(1);
  adj[1].push_back(0);
  adj[1].push_back(2);
  adj[2].push_back(1);
  adj[3].push_back(1);
  adj[1].push_back(3);
  adj[0].push_back(4);
  adj[4].push_back(0);
  build(5, 0);
  assert(lca(3, 2) == 1);
  assert(lca(2, 4) == 0);
  return 0;
}
\end{lstlisting}
\subsection{Lowest Common Ancestor (Segment Tree)}
\begin{lstlisting}
/*

Given a tree, determine the lowest common ancestor of any two nodes in the tree.
The lowest common ancestor of two nodes u and v is the node that has the longest
distance from the root while having both u and v as its descendant. A nodes is
considered to be a descendant of itself. build() applies to a global,
pre-populated adjacency list adj[] which must only consist of nodes numbered
with integers between 0 (inclusive) and the total number of nodes (exclusive),
as passed in the function argument.

Time Complexity:
- O(n log n) per call to build(), where n is the number of nodes.
- O(log n) per call to lca().

Space Complexity:
- O(n) for storage of the segment tree, where n is the number of nodes.
- O(n log n) auxiliary stack space for build().
- O(log n) auxiliary stack space for lca().

*/

#include <algorithm>
#include <vector>

const int MAXN = 1000;
std::vector<int> adj[MAXN];
int len, counter, depth[MAXN], dfs_order[2*MAXN], first[MAXN], minpos[8*MAXN];

void dfs(int u, int d) {
  depth[u] = d;
  dfs_order[counter++] = u;
  for (int j = 0; j < (int)adj[u].size(); j++) {
    int v = adj[u][j];
    if (depth[v] == -1) {
      dfs(v, d + 1);
      dfs_order[counter++] = u;
    }
  }
}

void build(int n, int lo, int hi) {
  if (lo == hi) {
    minpos[n] = dfs_order[lo];
    return;
  }
  int lchild = 2*n + 1, rchild = 2*n + 2, mid = lo + (hi - lo)/2;
  build(lchild, lo, mid);
  build(rchild, mid + 1, hi);
  minpos[n] = depth[minpos[lchild]] < depth[minpos[rchild]] ? minpos[lchild]
                                                            : minpos[rchild];
}

void build(int nodes, int root) {
  std::fill(depth, depth + nodes, -1);
  std::fill(first, first + nodes, -1);
  len = 2*nodes - 1;
  counter = 0;
  dfs(root, 0);
  build(0, 0, len - 1);
  for (int i = 0; i < len; i++) {
    if (first[dfs_order[i]] == -1) {
      first[dfs_order[i]] = i;
    }
  }
}

int get_minpos(int a, int b, int n, int lo, int hi) {
  if (a == lo && b == hi) {
    return minpos[n];
  }
  int mid = lo + (hi - lo)/2;
  if (a <= mid && mid < b) {
    int p1 = get_minpos(a, std::min(b, mid), 2*n + 1, lo, mid);
    int p2 = get_minpos(std::max(a, mid + 1), b, 2*n + 2, mid + 1, hi);
    return depth[p1] < depth[p2] ? p1 : p2;
  }
  if (a <= mid) {
    return get_minpos(a, std::min(b, mid), 2*n + 1, lo, mid);
  }
  return get_minpos(std::max(a, mid + 1), b, 2*n + 2, mid + 1, hi);
}

int lca(int u, int v) {
  return get_minpos(std::min(first[u], first[v]), std::max(first[u], first[v]),
                    0, 0, len - 1);
}

/*** Example Usage ***/

#include <cassert>
using namespace std;

int main() {
  adj[0].push_back(1);
  adj[1].push_back(0);
  adj[1].push_back(2);
  adj[2].push_back(1);
  adj[3].push_back(1);
  adj[1].push_back(3);
  adj[0].push_back(4);
  adj[4].push_back(0);
  build(5, 0);
  assert(lca(3, 2) == 1);
  assert(lca(2, 4) == 0);
  return 0;
}
\end{lstlisting}
\subsection{Heavy Light Decomposition}
\begin{lstlisting}
/*

Maintain a tree with values associated with either its edges or nodes, while
supporting both dynamic queries and dynamic updates of all values on a given
path between two nodes in the tree. Heavy-light decomposition partitions the
nodes of the tree into disjoint paths where all nodes have degree two, except
the endpoints of a path which has degree one.

The query operation is defined by an associative join_values() function which
satisfies join_values(x, join_values(y, z)) = join_values(join_values(x, y), z)
for all values x, y, and z in the tree. The default code below assumes a
numerical tree type, defining queries for the "min" of the target range.
Another possible query operation is "sum", in which case the join_values()
function should be defined to return "a + b".

The update operation is defined by the join_value_with_delta() and join_deltas()
functions, which determines the change made to values. These must satisfy:
- join_deltas(d1, join_deltas(d2, d3)) = join_deltas(join_deltas(d1, d2), d3).
- join_value_with_delta(join_values(v, ...(m times)..., v), d, m)) should be
  equal to join_values(join_value_with_delta(v, d, 1), ...(m times)).
- if a sequence d_1, ..., d_m of deltas is used to update a value v, then
  join_value_with_delta(v, join_deltas(d_1, ..., d_m), 1) should be equivalent
  to m sequential calls to join_value_with_delta(v, d_i, 1) for i = 1..m.
The default code below defines updates that "set" a path's edges or nodes to a
new value. Another possible update operation is "increment", in which case
join_value_with_delta(v, d, len) should be defined to return "v + d*len" and
join_deltas(d1, d2) should be defined to return "d1 + d2".

- heavy_light(n, adj[], v) constructs a new heavy light decomposition on a tree
  with n nodes defined by the adjacency list adj[], with all values initialized
  to v. The adjacency list must be a size n array of vectors consisting of only
  the integers from 0 to n - 1, inclusive. No duplicate edges should exist, and
  the graph must be connected.
- query(u, v) returns the result of join_values() applied to all values on the
  path from node u to node v.
- update(u, v, d) modifies all values on the path from node u to node v by
  respectively joining them with d using join_value_with_delta().

Time Complexity:
- O(n) per call to the constructor, where n is the number of nodes.
- O(log n) per call to query() and update();

Space Complexity:
- O(n) for storage of the decomposition.
- O(n) auxiliary stack space for the constructor.
- O(1) auxiliary for query() and update().

*/

#include <algorithm>
#include <stdexcept>
#include <vector>

template<class T>
class heavy_light {
  // Set this to true to store values on edges, false to store values on nodes.
  static const bool VALUES_ON_EDGES = true;

  static T join_values(const T &a, const T &b) {
    return std::min(a, b);
  }

  static T join_value_with_delta(const T &v, const T &d, int len) {
    return d;
  }

  static T join_deltas(const T &d1, const T &d2) {
    return d2;  // For "set" updates, the more recent delta prevails.
  }

  int counter, paths;
  std::vector<std::vector<T> > value, delta;
  std::vector<std::vector<bool> > pending;
  std::vector<std::vector<int> > len;
  std::vector<int> size, parent, tin, tout, path, pathlen, pathpos, pathroot;
  std::vector<int> *adj;

  void dfs(int u, int p) {
    tin[u] = counter++;
    parent[u] = p;
    size[u] = 1;
    for (int j = 0; j < (int)adj[u].size(); j++) {
      int v = adj[u][j];
      if (v != p) {
        dfs(v, u);
        size[u] += size[v];
      }
    }
    tout[u] = counter++;
  }

  int new_path(int u) {
    pathroot[paths] = u;
    return paths++;
  }

  void build_paths(int u, int path) {
    this->path[u] = path;
    pathpos[u] = pathlen[path]++;
    for (int j = 0; j < (int)adj[u].size(); j++) {
      int v = adj[u][j];
      if (v != parent[u]) {
        build_paths(v, (2*size[v] >= size[u]) ? path : new_path(v));
      }
    }
  }

  inline T join_value_with_delta(int path, int i) {
    return pending[path][i]
        ? join_value_with_delta(value[path][i], delta[path][i], len[path][i])
        : value[path][i];
  }

  void push_delta(int path, int i) {
    int d = 0;
    while ((i >> d) > 0) {
      d++;
    }
    for (d -= 2; d >= 0; d--) {
      int l = (i >> d), r = (l ^ 1), n = l/2;
      if (pending[path][n]) {
        value[path][n] = join_value_with_delta(path, n);
        delta[path][l] =
            pending[path][l] ? join_deltas(delta[path][l], delta[path][n])
                             : delta[path][n];
        delta[path][r] =
            pending[path][r] ? join_deltas(delta[path][r], delta[path][n])
                             : delta[path][n];
        pending[path][l] = pending[path][r] = true;
        pending[path][n] = false;
      }
    }
  }

  bool query(int path, int u, int v, T *res) {
    push_delta(path, u += value[path].size()/2);
    push_delta(path, v += value[path].size()/2);
    bool found = false;
    for (; u <= v; u = (u + 1)/2, v = (v - 1)/2) {
      if ((u & 1) != 0) {
        T value = join_value_with_delta(path, u);
        *res = found ? join_values(*res, value) : value;
        found = true;
      }
      if ((v & 1) == 0) {
        T value = join_value_with_delta(path, v);
        *res = found ? join_values(*res, value) : value;
        found = true;
      }
    }
    return found;
  }

  void update(int path, int u, int v, const T &d) {
    push_delta(path, u += value[path].size()/2);
    push_delta(path, v += value[path].size()/2);
    int tu = -1, tv = -1;
    for (; u <= v; u = (u + 1)/2, v = (v - 1)/2) {
      if ((u & 1) != 0) {
        delta[path][u] = pending[path][u] ? join_deltas(delta[path][u], d) : d;
        pending[path][u] = true;
        if (tu == -1) {
          tu = u;
        }
      }
      if ((v & 1) == 0) {
        delta[path][v] = pending[path][v] ? join_deltas(delta[path][v], d) : d;
        pending[path][v] = true;
        if (tv == -1) {
          tv = v;
        }
      }
    }
    for (int i = tu; i > 1; i /= 2) {
      value[path][i/2] = join_values(join_value_with_delta(path, i),
                                     join_value_with_delta(path, i ^ 1));
    }
    for (int i = tv; i > 1; i /= 2) {
      value[path][i/2] = join_values(join_value_with_delta(path, i),
                                     join_value_with_delta(path, i ^ 1));
    }
  }

  inline bool is_ancestor(int parent, int child) {
    return (tin[parent] <= tin[child]) && (tout[child] <= tout[parent]);
  }

 public:
  heavy_light(int n, std::vector<int> adj[], const T &v = T())
      : counter(0), paths(0), size(n), parent(n), tin(n), tout(n), path(n),
        pathlen(n), pathpos(n), pathroot(n), adj(adj) {
    dfs(0, -1);
    build_paths(0, new_path(0));
    value.resize(paths);
    delta.resize(paths);
    pending.resize(paths);
    len.resize(paths);
    for (int i = 0; i < paths; i++) {
      int m = pathlen[i];
      value[i].assign(2*m, v);
      delta[i].resize(2*m);
      pending[i].assign(2*m, false);
      len[i].assign(2*m, 1);
      for (int j = 2*m - 1; j > 1; j -= 2) {
        value[i][j/2] = join_values(value[i][j], value[i][j ^ 1]);
        len[i][j/2] = len[i][j] + len[i][j ^ 1];
      }
    }
  }

  T query(int u, int v) {
    if (VALUES_ON_EDGES && u == v) {
      throw std::runtime_error("No edge between u and v to be queried.");
    }
    bool found = false;
    T res = T(), value;
    int root;
    while (!is_ancestor(root = pathroot[path[u]], v)) {
      if (query(path[u], 0, pathpos[u], &value)) {
        res = found ? join_values(res, value) : value;
        found = true;
      }
      u = parent[root];
    }
    while (!is_ancestor(root = pathroot[path[v]], u)) {
      if (query(path[v], 0, pathpos[v], &value)) {
        res = found ? join_values(res, value) : value;
        found = true;
      }
      v = parent[root];
    }
    if (query(path[u], std::min(pathpos[u], pathpos[v]) + (int)VALUES_ON_EDGES,
              std::max(pathpos[u], pathpos[v]), &value)) {
      res = found ? join_values(res, value) : value;
      found = true;
    }
    if (!found) {
      throw std::runtime_error("Unexpected error: No values found.");
    }
    return res;
  }

  void update(int u, int v, const T &d) {
    if (VALUES_ON_EDGES && u == v) {
      return;
    }
    int root;
    while (!is_ancestor(root = pathroot[path[u]], v)) {
      update(path[u], 0, pathpos[u], d);
      u = parent[root];
    }
    while (!is_ancestor(root = pathroot[path[v]], u)) {
      update(path[v], 0, pathpos[v], d);
      v = parent[root];
    }
    update(path[u], std::min(pathpos[u], pathpos[v]) + (int)VALUES_ON_EDGES,
           std::max(pathpos[u], pathpos[v]), d);
  }
};

/*** Example Usage ***/

#include <cassert>
using namespace std;

int main() {
  //     w=40      w=20      w=10
  // 0---------1---------2---------3
  //                     |
  //                     ----------4
  //                         w=30
  vector<int> adj[5];
  adj[0].push_back(1);
  adj[1].push_back(0);
  adj[1].push_back(2);
  adj[2].push_back(1);
  adj[2].push_back(3);
  adj[3].push_back(2);
  adj[2].push_back(4);
  adj[4].push_back(2);
  heavy_light<int> hld(5, adj, 0);
  hld.update(0, 1, 40);
  hld.update(1, 2, 20);
  hld.update(2, 3, 10);
  hld.update(2, 4, 30);
  assert(hld.query(0, 3) == 10);
  assert(hld.query(2, 4) == 30);
  hld.update(3, 4, 5);
  assert(hld.query(1, 4) == 5);
  return 0;
}
\end{lstlisting}
\subsection{Link-Cut Tree}
\begin{lstlisting}
/*

Maintain a forest of trees with values associated with its nodes, while
supporting both dynamic queries and dynamic updates of all values on any path
between two nodes in a given tree. In addition, support testing of whether two
nodes are connected in the forest, as well as the merging and spliting of trees
by adding or removing specific edges. Link/cut forests divide each of its trees
into vertex-disjoint paths, each represented by a splay tree.

The query operation is defined by an associative join_values() function which
satisfies join_values(x, join_values(y, z)) = join_values(join_values(x, y), z)
for all values x, y, and z in the forest. The default code below assumes a
numerical forest type, defining queries for the "min" of the target range.
Another possible query operation is "sum", in which case the join_values()
function should be defined to return "a + b".

The update operation is defined by the join_value_with_delta() and join_deltas()
functions, which determines the change made to values. These must satisfy:
- join_deltas(d1, join_deltas(d2, d3)) = join_deltas(join_deltas(d1, d2), d3).
- join_value_with_delta(join_values(v, ...(m times)..., v), d, m)) should be
  equal to join_values(join_value_with_delta(v, d, 1), ...(m times)).
- if a sequence d_1, ..., d_m of deltas is used to update a value v, then
  join_value_with_delta(v, join_deltas(d_1, ..., d_m), 1) should be equivalent
  to m sequential calls to join_value_with_delta(v, d_i, 1) for i = 1..m.
The default code below defines updates that "set" a path's nodes to a new value.
Another possible update operation is "increment", in which case
join_value_with_delta(v, d, len) should be defined to return "v + d*len" and
join_deltas(d1, d2) should be defined to return "d1 + d2".

- link_cut_forest() constructs an empty forest with no trees.
- size() returns the number of nodes in the forest.
- trees() returns the number of trees in the forest.
- make_root(i, v) creates a new tree in the forest consisting of a single node
  labeled with the integer i and value initialized to v.
- is_connected(a, b) returns whether nodes a and b are connected.
- link(a, b) adds an edge between the nodes a and b, both of which must exist
  and not be connected.
- cut(a, b) removes the edge between the nodes a and b, both of which must
  exist and be connected.
- query(a, b) returns the result of join_values() applied to all values on the
  path from the node a to node b.
- update(a, b, d) modifies all the values on the path from node a to node b by
  respectively joining them with d using join_value_with_delta().

Time Complexity:
- O(1) per call to the constructor, size(), and trees().
- O(log n) amortized per call to all other operations, where n is the number of
  nodes.

Space Complexity:
- O(n) for storage of the forest, where n is the number of nodes.
- O(1) auxiliary for all operations.

*/

#include <algorithm>
#include <cstddef>
#include <map>
#include <stdexcept>

template<class T>
class link_cut_forest {
  static T join_values(const T &a, const T &b) {
    return std::min(a, b);
  }

  static T join_value_with_delta(const T &v, const T &d, int len) {
    return d;
  }

  static T join_deltas(const T &d1, const T &d2) {
    return d2;  // For "set" updates, the more recent delta prevails.
  }

  struct node_t {
    T value, subtree_value, delta;
    int size;
    bool rev, pending;
    node_t *left, *right, *parent;

    node_t(const T &v)
        : value(v), subtree_value(v), size(1), rev(false), pending(false),
          left(NULL), right(NULL), parent(NULL) {}

    inline bool is_root() const {
      return parent == NULL || (parent->left != this && parent->right != this);
    }

    inline T get_subtree_value() const {
      return pending ? join_value_with_delta(subtree_value, delta, size)
                     : subtree_value;
    }

    void push() {
      if (rev) {
        rev = false;
        std::swap(left, right);
        if (left != NULL) {
          left->rev = !left->rev;
        }
        if (right != NULL) {
          right->rev = !right->rev;
        }
      }
      if (pending) {
        value = join_value_with_delta(value, delta, 1);
        subtree_value = join_value_with_delta(subtree_value, delta, size);
        if (left != NULL) {
          left->delta = left->pending ? join_deltas(left->delta, delta) : delta;
          left->pending = true;
        }
        if (right != NULL) {
          right->delta = right->pending ? join_deltas(right->delta, delta)
                                        : delta;
          right->pending = true;
        }
        pending = false;
      }
    }

    void update() {
      size = 1;
      subtree_value = value;
      if (left != NULL) {
        subtree_value = join_values(subtree_value, left->get_subtree_value());
        size += left->size;
      }
      if (right != NULL) {
        subtree_value = join_values(subtree_value, right->get_subtree_value());
        size += right->size;
      }
    }
  };

  int num_trees;
  std::map<int, node_t*> nodes;

  static void connect(node_t *child, node_t *parent, bool is_left) {
    if (child != NULL) {
      child->parent = parent;
    }
    if (is_left) {
      parent->left = child;
    } else {
      parent->right = child;
    }
  }

  static void rotate(node_t *n) {
    node_t *parent = n->parent, *grandparent = parent->parent;
    bool parent_is_root = parent->is_root(), is_left = (n == parent->left);
    connect(is_left ? n->right : n->left, parent, is_left);
    connect(parent, n, !is_left);
    if (parent_is_root) {
      if (n != NULL) {
        n->parent = grandparent;
      }
    } else {
      connect(n, grandparent, parent == grandparent->left);
    }
    parent->update();
  }

  static void splay(node_t *n) {
    while (!n->is_root()) {
      node_t *parent = n->parent, *grandparent = parent->parent;
      if (!parent->is_root()) {
        grandparent->push();
      }
      parent->push();
      n->push();
      if (!parent->is_root()) {
        if ((n == parent->left) == (parent == grandparent->left)) {
          rotate(parent);
        } else {
          rotate(n);
        }
      }
      rotate(n);
    }
    n->push();
    n->update();
  }

  static node_t* expose(node_t *n) {
    node_t *prev = NULL;
    for (node_t *curr = n; curr != NULL; curr = curr->parent) {
      splay(curr);
      curr->left = prev;
      prev = curr;
    }
    splay(n);
    return prev;
  }

  // Helper variables.
  node_t *u, *v;

  void get_uv(int a, int b) {
    typename std::map<int, node_t*>::iterator it1, it2;
    it1 = nodes.find(a);
    it2 = nodes.find(b);
    if (it1 == nodes.end() || it2 == nodes.end()) {
      throw std::runtime_error("Queried node ID does not exist in forest.");
    }
    u = it1->second;
    v = it2->second;
  }

 public:
  link_cut_forest() : num_trees(0) {}

  ~link_cut_forest() {
    typename std::map<int, node_t*>::iterator it;
    for (it = nodes.begin(); it != nodes.end(); ++it) {
      delete it->second;
    }
  }

  int size() const {
    return nodes.size();
  }

  int trees() const {
    return num_trees;
  }

  void make_root(int i, const T &v = T()) {
    if (nodes.find(i) != nodes.end()) {
      throw std::runtime_error("Cannot make a root with an existing ID.");
    }
    node_t *n = new node_t(v);
    expose(n);
    n->rev = !n->rev;
    nodes[i] = n;
    num_trees++;
  }

  bool is_connected(int a, int b) {
    get_uv(a, b);
    if (a == b) {
      return true;
    }
    expose(u);
    expose(v);
    return u->parent != NULL;
  }

  void link(int a, int b) {
    if (is_connected(a, b)) {
      throw std::runtime_error("Cannot link nodes that are already connected.");
    }
    get_uv(a, b);
    expose(u);
    u->rev = !u->rev;
    u->parent = v;
    num_trees--;
  }

  void cut(int a, int b) {
    get_uv(a, b);
    expose(u);
    u->rev = !u->rev;
    expose(v);
    if (v->right != u || u->left != NULL) {
      throw std::runtime_error("Cannot cut edge that does not exist.");
    }
    v->right->parent = NULL;
    v->right = NULL;
    num_trees++;
  }

  T query(int a, int b) {
    if (!is_connected(a, b)) {
      throw std::runtime_error("Cannot query nodes that are not connected.");
    }
    get_uv(a, b);
    expose(u);
    u->rev = !u->rev;
    expose(v);
    return v->get_subtree_value();
  }

  void update(int a, int b, const T &d) {
    if (!is_connected(a, b)) {
      throw std::runtime_error("Cannot update nodes that are not connected.");
    }
    get_uv(a, b);
    expose(u);
    u->rev = !u->rev;
    expose(v);
    v->delta = v->pending ? join_deltas(v->delta, d) : d;
    v->pending = true;
  }
};

/*** Example Usage ***/

#include <cassert>
using namespace std;

int main() {
  link_cut_forest<int> lcf;
  lcf.make_root(0, 10);
  lcf.make_root(1, 40);
  lcf.make_root(2, 20);
  lcf.make_root(3, 10);
  lcf.make_root(4, 30);
  assert(lcf.size() == 5);
  assert(lcf.trees() == 5);
  lcf.link(0, 1);
  lcf.link(1, 2);
  lcf.link(2, 3);
  lcf.link(2, 4);
  assert(lcf.trees() == 1);

  // v=10      v=40      v=20      v=10
  //  0---------1---------2---------3
  //                      |
  //                      ----------4
  //                               v=30
  assert(lcf.query(1, 4) == 20);
  lcf.update(1, 1, 100);
  lcf.update(2, 4, 100);

  // v=10     v=100     v=100      v=10
  //  0---------1---------2---------3
  //                      |
  //                      ----------4
  //                              v=100
  assert(lcf.query(4, 4) == 100);
  assert(lcf.query(0, 4) == 10);
  assert(lcf.query(3, 4) == 10);
  lcf.cut(1, 2);

  // v=10     v=100     v=100      v=0
  //  0---------1         2---------3
  //                      |
  //                      ----------4
  //                              v=100
  assert(lcf.trees() == 2);
  assert(!lcf.is_connected(1, 2));
  assert(!lcf.is_connected(0, 4));
  assert(lcf.is_connected(2, 3));
  return 0;
}
\end{lstlisting}

\chapter{Strings}

\section{String Utilities}
\setcounter{section}{1}
\begin{lstlisting}
/*

Common string functions, many of which are substitutes for features which are
not available in standard C++, or may not be available on compilers that do not
support C++11 and later. These operations are naive implementations and often
depend on certain std::string functions that have unspecified complexity.

*/

#include <cctype>
#include <sstream>
#include <stdexcept>
#include <string>
#include <vector>
using std::string;

/*

Integer Conversion

- to_str(i) returns the string representation of integer i, much like
  std::to_string() in C++11 and later.
- to_int(s) returns the integer representation of string s, much like atoi(),
  except handling special cases of overflow by throwing an exception.
- itoa(value, &str, base) implements the non-standard C function which converts
  value into a C string, storing it into pointer str in the given base. For more
  generalized base conversion, see the math utilities section.

*/

template<class Int>
string to_str(Int i) {
  std::ostringstream oss;
  oss << i;
  return oss.str();
}

int to_int(const string &s) {
  std::istringstream iss(s);
  int res;
  if (!(iss >> res)) {
    throw std::runtime_error("to_int failed");
  }
  return res;
}

char* itoa(int value, char *str, int base = 10) {
  if (base < 2 || base > 36) {
    *str = '\0';
    return str;
  }
  char *ptr = str, *ptr1 = str, tmp_c;
  int tmp_v;
  do {
    tmp_v = value;
    value /= base;
    *ptr++ = "zyxwvutsrqponmlkjihgfedcba9876543210123456789"
             "abcdefghijklmnopqrstuvwxyz"[35 + (tmp_v - value * base)];
  } while (value);
  if (tmp_v < 0) {
    *ptr++ = '-';
  }
  for (*ptr-- = '\0'; ptr1 < ptr; *ptr1++ = tmp_c) {
    tmp_c = *ptr;
    *ptr-- = *ptr1;
  }
  return str;
}

/*

Case Conversion

- to_upper(s) returns s with all alphabetical characters converted to uppercase.
- to_lower(s) returns s with all alphabetical characters converted to lowercase.
- to_title(s) returns the title case representation of string s, where the first
  letter of every word (consecutive alphabetical characters) is capitalized.

*/

string to_upper(const string &s) {
  string res;
  for (int i = 0; i < (int)s.size(); i++) {
    res.push_back(toupper(s[i]));
  }
  return res;
}

string to_lower(const string &s) {
  string res;
  for (int i = 0; i < (int)s.size(); i++) {
    res.push_back(tolower(s[i]));
  }
  return res;
}

string to_title(const string &s) {
  string res;
  char prev = '\0';
  for (int i = 0; i < (int)s.size(); i++) {
    if (isalpha(prev)) {
      res.push_back(tolower(s[i]));
    } else {
      res.push_back(toupper(s[i]));
    }
    prev = res.back();
  }
  return res;
}

/*

Stripping

- lstrip(s) strips the left side of s in-place (that is, the input is modified)
  using the given delimiters and returns a reference to the stripped string.
- rstrip(s) strips the right side of s in-place using the given delimiters and
  returns a reference to the stripped string.
- strip(s) strips both sides of s in-place and returns a reference to the
  stripped string.

*/

string& lstrip(string &s, const string &delim = " \n\t\v\f\r") {
  size_t pos = s.find_first_not_of(delim);
  if (pos != string::npos) {
    s.erase(0, pos);
  }
  return s;
}

string& rstrip(string &s, const string &delim = " \n\t\v\f\r") {
  size_t pos = s.find_last_not_of(delim);
  if (pos != string::npos) {
    s.erase(pos);
  }
  return s;
}

string& strip(string &s, const string &delim = " \n\t\v\f\r") {
  return lstrip(rstrip(s));
}

/*

Find and Replace

- find_all(haystack, needle) returns a vector of all positions where the string
  needle appears in the string haystack.
- replace(s, old, replacement) returns a copy of s with all occurrences of the
  string old replaced with the given replacement.

*/

std::vector<int> find_all(const string &haystack, const string &needle) {
  std::vector<int> res;
  size_t pos = haystack.find(needle, 0);
  while (pos != string::npos) {
      res.push_back(pos);
      pos = haystack.find(needle, pos + 1);
  }
  return res;
}

string replace(const string &s, const string &old, const string &replacement) {
  if (old.empty()) {
    return s;
  }
  string res(s);
  size_t pos = 0;
  while ((pos = res.find(old, pos)) != string::npos) {
    res.replace(pos, old.length(), replacement);
    pos += replacement.length();
  }
  return res;
}

/*

Joining and Splitting

- join(v, delim) returns the strings in vector v concatenated, separated by the
  given delimiter.
- split(s, char delim) returns a vector of tokens of s, split on a single
  character delimiter. Note that this version will not skip empty tokens. For
  example, split("a::b", ":") returns {"a", "b"}, not {"a", "", "b"}.
- split(s, string delim) returns a vector of tokens of s, split on a set of many
  possible single character delimiters. All characters ofz delim will be removed
  from s, and the remaining token(s) of s will be added sequentially to a vector
  and returned. Unlike the first version, empty tokens are skipped. For example,
  split("a::b", ":") returns {"a", "b"}, not {"a", "", "b"}.
- explode(s, delim) returns a vector of tokens of s, split on the entire
  delimiter string delim. Unlike the split() functions above, delim is treated
  as a contiguous boundary string, not merely a set of possible boundary
  characters. This will not skip empty tokens. For example,
  explode("a::::b", "::") yields {"a", "", "b"}, not {"a", "b"}.

*/

string join(const std::vector<string> &v, const string &delim = " ") {
  string res;
  for (int i = 0; i < (int)v.size(); i++) {
    if (i > 0) {
      res += delim;
    }
    res += v[i];
  }
  return res;
}

std::vector<string> split(const string &s, char delim) {
  std::vector<string> res;
  std::stringstream ss(s);
  string curr;
  while (std::getline(ss, curr, delim)) {
    res.push_back(curr);
  }
  return res;
}

std::vector<string> split(const string &s,
                          const string &delim = " \n\t\v\f\r") {
  std::vector<string> res;
  string curr;
  for (int i = 0; i < (int)s.size(); i++) {
    if (delim.find(s[i]) == string::npos) {
      curr += s[i];
    } else if (!curr.empty()) {
      res.push_back(curr);
      curr = "";
    }
  }
  if (!curr.empty()) {
    res.push_back(curr);
  }
  return res;
}

std::vector<string> explode(const string &s, const string &delim) {
  std::vector<string> res;
  size_t last = 0, next = 0;
  while ((next = s.find(delim, last)) != string::npos) {
    res.push_back(s.substr(last, (int)next - last));
    last = next + delim.size();
  }
  res.push_back(s.substr(last));
  return res;
}

/*** Example Usage ***/

#include <cassert>
using namespace std;

int main() {
  assert(to_str(123) + "4" == "1234");
  assert(to_int("1234") == 1234);
  char buffer[50];
  assert(string(itoa(1750, buffer, 10)) == "1750");
  assert(string(itoa(1750, buffer, 16)) == "6d6");
  assert(string(itoa(1750, buffer, 2)) == "11011010110");

  assert(to_upper("Hello world") == "HELLO WORLD");
  assert(to_lower("Hello World") == "hello world");
  assert(to_title("hello world") == "Hello World");

  string s("   abc \n");
  string t = s;
  assert(lstrip(s) == "abc \n");
  assert(rstrip(s) == strip(t));

  vector<int> pos;
  pos.push_back(0);
  pos.push_back(7);
  assert(find_all("abracadabra", "ab") == pos);
  assert(replace("abcdabba", "ab", "00") == "00cd00ba");

  assert(join(split("a\nb\ncde\nf", '\n'), "|") == "a|b|cde|f");  // split v1
  assert(join(split("a::b,cde:,f", ":,"), "|") == "a|b|cde|f");  // split v2
  assert(join(explode("a..b.cde....f", ".."), "|") == "a|b.cde||f");
  return 0;
}
\end{lstlisting}

\section{Expression Parsing}
\setcounter{section}{2}
\setcounter{subsection}{0}
\subsection{String Searching (KMP)}
\begin{lstlisting}
/*

Given a single string (needle) and subsequent queries of texts (haystacks) to be
searched, determine the first positions in which the needle occurs within the
given haystacks in linear time using the Knuth-Morris-Pratt algorithm. In
comparison, std::string::find runs in quadratic time.

- kmp(needle) constructs the partial match table for a string needle that is to
  be searched for subsequently in haystack queries.
- find_in(haystack) returns the first position that needle occurs in haystack,
  or std::string::npos if it cannot be found. Note that the function can be
  modified to return all matches by simply letting the loop run and storing
  the results instead of returning early.

Time Complexity:
- O(m) per call to the constructor, where m is the length of needle.
- O(n) per call to find_in(haystack), where n is the length of haystack.

Space Complexity:
- O(m) for storage of the partial match table, where m is the length of needle.
- O(1) auxiliary space per call to find_in(haystack).

*/

#include <string>
#include <vector>
using std::string;

class kmp {
  string needle;
  std::vector<int> table;

 public:
  kmp(const string &needle) : needle(needle) {
    table.resize(needle.size());
    int i = 0, j = table[0] = -1;
    while (i < (int)needle.size()) {
      while (j >= 0 && needle[i] != needle[j]) {
        j = table[j];
      }
      i++;
      j++;
      table[i] = (needle[i] == needle[j]) ? table[j] : j;
    }
  }

  size_t find_in(const string &haystack) {
    if (needle.empty()) {
      return 0;
    }
    for (int i = 0, j = 0; j < (int)haystack.size(); ) {
      while (i >= 0 && needle[i] != haystack[j]) {
        i = table[i];
      }
      i++;
      j++;
      if (i >= (int)needle.size()) {
        return j - i;
      }
    }
    return string::npos;
  }
};

/*** Example Usage ***/

#include <cassert>

int main() {
  assert(15 == kmp("ABCDABD").find_in("ABC ABCDAB ABCDABCDABDE"));
  return 0;
}
\end{lstlisting}
\subsection{String Searching (Z Algorithm)}
\begin{lstlisting}
/*

Given a single string (needle) and a single text (haystack) to be searched,
determine the first position in which the needle occurs within the haystack in
linear time using the Z algorithm. In comparison, std::string::find runs in
quadratic time.

The find function below calls the Z algorithm on the concatenation of the needle
and the haystack, separated by a sentinel character (in this case '\0'), which
should be chosen such that it does not occur within either of the input strings.

- z_array(s) constructs the Z array for a string needle that can be used for
  string searching. The Z array on an input string s is an array z where z[i] is
  the length of the longest substring starting from s[i] which is also a prefix
  of s.
- find(needle, haystack) returns the first position that needle occurs in
  haystack, or std::string::npos if it cannot be found. Note that the function
  can be modified to return all matches by simply letting the loop run and
  storing the results instead of returning early.

Time Complexity:
- O(n) per call to z_array(s), where n is the length of s.
- O(n + m) per call to find(haystack, needle), where n is the length of haystack
  and m is the length of needle.

Space Complexity:
- O(n) auxiliary heap space for z_array(s), where n is the length of s.
- O(n + m) auxiliary heap space for find(haystack, needle) where n is the length
  of haystack and m is the length of needle.

*/

#include <algorithm>
#include <string>
#include <vector>
using std::string;

std::vector<int> z_array(const string &s) {
  std::vector<int> z(s.size());
  for (int i = 1, l = 0, r = 0; i < (int)z.size(); i++) {
    if (i <= r) {
      z[i] = std::min(r - i + 1, z[i - l]);
    }
    while (i + z[i] < (int)z.size() && s[z[i]] == s[i + z[i]]) {
      z[i]++;
    }
    if (r < i + z[i] - 1) {
      l = i;
      r = i + z[i] - 1;
    }
  }
  return z;
}

size_t find(const string &haystack, const string &needle) {
  std::vector<int> z = z_array(needle + '\0' + haystack);
  for (int i = (int)needle.size() + 1; i < (int)z.size(); i++) {
    if (z[i] == (int)needle.size()) {
      return i - (int)needle.size() - 1;
    }
  }
  return string::npos;
}

/*** Example Usage ***/

#include <cassert>

int main() {
  assert(15 == find("ABC ABCDAB ABCDABCDABDE", "ABCDABD"));
  return 0;
}
\end{lstlisting}
\subsection{String Searching (Aho-Corasick)}
\begin{lstlisting}
/*

Given a set of strings (needles) and subsequent queries of texts (haystacks)
to be searched, determine all positions in which needles occur within the given
haystacks in linear time using the Aho-Corasick algorithm.

Note that this implementation uses an ordered map for storage of the graph,
adding an additional log k factor to the time complexities of all operations,
where k is the size of the alphabet (number of distinct characters used across
the needles). It also uses an ordered set for storage of the precomputed output
tables, adding an additional log m factor to the time complexities, where m is
the number of needles. In C++11 and later, both of these containers should be
replaced by their unordered versions for constant time access, thus eliminating
the log factors from the time complexities.

- aho_corasick(needles) constructs the finite-state automaton for a set of
  needle strings that are to be searched for subsequently in haystack queries.
- find_all_in(haystack, report_match) calls the function report_match(s, pos)
  once on each occurrence of each needle that occurs in the haystack, where pos
  is the starting position in the haystack at which string s (a matched needle)
  occurs. The matches will be reported in increasing order of their ending
  positions within the haystack.

Time Complexity:
- O(m*((log m) + l*log k)) per call to the constructor, where m is the number of
  needles, l is the maximum length for any needle, and k is the size of the
  alphabet used by the needles. If unordered containers are used, then the time
  complexity reduces to O(m*l), or linear on the input size.
- O(n*(log k) + z) per call to find_all_in(haystack, report_match), where n is
  the length of haystack, k is the size of the alphabet used by the needles, and
  z is the number of matches. If unordered containers are used, then the time
  complexity reduces to O(n + z), or linear on the input size.

Space Complexity:
- O(m*l) for storage of the automaton, where where m is the number of needles
  and l is the maximum length for any needle.
- O(1) auxiliary space per call to find_all_in(haystack, report_match).

*/

#include <map>
#include <queue>
#include <set>
#include <string>
#include <vector>
using std::string;

class aho_corasick {
  std::vector<string> needles;
  std::vector<int> fail;
  std::vector<std::map<char, int> > graph;
  std::vector<std::set<int> > out;

  int next_state(int curr, char c) {
    int next = curr;
    while (graph[next].find(c) == graph[next].end()) {
      next = fail[next];
    }
    return graph[next][c];
  }

 public:
  aho_corasick(const std::vector<string> &needles) : needles(needles) {
    int total_len = 0;
    for (int i = 0; i < (int)needles.size(); i++) {
      total_len += needles[i].size();
    }
    fail.resize(total_len, -1);
    graph.resize(total_len);
    out.resize(total_len);
    int states = 1;
    std::map<char, int>::iterator it;
    for (int i = 0; i < (int)needles.size(); i++) {
      int curr = 0;
      for (int j = 0; j < (int)needles[i].size(); j++) {
        char c = needles[i][j];
        if ((it = graph[curr].find(c)) != graph[curr].end()) {
          curr = it->second;
        } else {
          curr = graph[curr][c] = states++;
        }
      }
      out[curr].insert(i);
    }
    std::queue<int> q;
    for (it = graph[0].begin(); it != graph[0].end(); ++it) {
      if (it->second != 0) {
        fail[it->second] = 0;
        q.push(it->second);
      }
    }
    while (!q.empty()) {
      int u = q.front();
      q.pop();
      for (it = graph[u].begin(); it != graph[u].end(); ++it) {
        int v = it->second, f = fail[u];
        while (graph[f].find(it->first) == graph[f].end()) {
          f = fail[f];
        }
        f = graph[f].find(it->first)->second;
        fail[v] = f;
        out[v].insert(out[f].begin(), out[f].end());
        q.push(v);
      }
    }
  }

  template<class ReportFunction>
  void find_all_in(const string &haystack, ReportFunction report_match) {
    int state = 0;
    std::set<int>::iterator it;
    for (int i = 0; i < (int)haystack.size(); i++) {
      state = next_state(state, haystack[i]);
      for (it = out[state].begin(); it != out[state].end(); ++it) {
        report_match(needles[*it], i - needles[*it].size() + 1);
      }
    }
  }
};

/*** Example Usage and Output:

Matched "a" at position 0.
Matched "ab" at position 0.
Matched "bc" at position 1.
Matched "c" at position 2.
Matched "c" at position 3.
Matched "a" at position 4.
Matched "ab" at position 4.
Matched "abccab" at position 0.

***/

#include <iostream>
using namespace std;

void report_match(const string &needle, int pos) {
  cout << "Matched \"" << needle << "\" at position " << pos << "." << endl;
}

int main() {
  vector<string> needles;
  needles.push_back("a");
  needles.push_back("ab");
  needles.push_back("bab");
  needles.push_back("bc");
  needles.push_back("bca");
  needles.push_back("c");
  needles.push_back("caa");
  needles.push_back("abccab");

  aho_corasick(needles).find_all_in("abccab", report_match);
  return 0;
}
\end{lstlisting}

\section{String Searching}
\setcounter{section}{3}
\setcounter{subsection}{0}
\subsection{Recursive Descent Parsing (Simple)}
\begin{lstlisting}
/*

Evaluate an expression in accordance to the order of operations (parentheses,
unary plus and minus signs, multiplication/division, addition/subtraction). The
following is a minimalistic recursive descent implementation using iterators.

- eval(s) returns an evaluation of the arithmetic expression s.

Time Complexity:
- O(n) per call to eval(s), where n is the length of s.

Space Complexity:
- O(n) auxiliary stack space for eval(s), where n is the length of s.

*/

#include <string>

template<class It>
int eval(It &it, int prec) {
  if (prec == 0) {
    int sign = 1, ret = 0;
    for (; *it == '-'; it++) {
      sign *= -1;
    }
    if (*it == '(') {
      ret = eval(++it, 2);
      it++;
    } else while (*it >= '0' && *it <= '9') {
      ret = 10*ret + (*(it++) - '0');
    }
    return sign*ret;
  }
  int num = eval(it, prec - 1);
  while (!((prec == 2 && *it != '+' && *it != '-') ||
           (prec == 1 && *it != '*' && *it != '/'))) {
    switch (*(it++)) {
      case '+': num += eval(it, prec - 1); break;
      case '-': num -= eval(it, prec - 1); break;
      case '*': num *= eval(it, prec - 1); break;
      case '/': num /= eval(it, prec - 1); break;
    }
  }
  return num;
}

int eval(const std::string &s) {
  std::string::iterator it = std::string(s).begin();
  return eval(it, 2);
}

/*** Example Usage ***/

#include <cassert>

int main() {
  assert(eval("1++1") == 2);
  assert(eval("1+2*3*4+3*(2+2)-100") == -63);
  return 0;
}
\end{lstlisting}
\subsection{Recursive Descent Parsing (Generic)}
\begin{lstlisting}
/*

Evaluate an expression using a generalized parser class for custom-defined
operand types, prefix unary operators, binary operators, and precedences.
Typical parentheses behavior is supported, but multiplication by juxtaposition
is not. Evaluation is performed using the recursive descent algorithm.

An arbitrary operand type is supported, with its string representation defined
by a user-specified is_operand() and eval_operand() functions. For maximum
reliability, the string representation of operands should not use characters
shared by any operator. For instance, the best practice instead of accepting
"-1" as a valid operand (since the "-" sign may conflict with the identical
binary operator), is to specify non-negative number as operands alongside the
unary operator "-".

Operators may be non-empty strings of any length, but should not contain any
parentheses or shared characters with the string representations of operands.
Ideally, operators should not be prefixes or suffices of one another, else the
tokenization process may be ambiguous. For example, if ++ and + are both
operators, then ++ may be split into either ["+", "+"] or ["++"] depending on
the lexicographical ordering of conflicting operators.

- parser(unary_op, binary_op) initializes a parser with operators specified by
  maps unary_op (of operator to function pointer) and binary_op (of operator to
  pair of function pointer and operator precedence). Operator precedences should
  be numbered upwards starting at 0 (lowest precedence, evaluated last).
- split(s) returns a vector of tokens for the expression s, split on the given
  operators during construction. Each parenthesis, operator, and operand
  satisfying is_operand() will be split into a separate token. The algorithm is
  naive, matching operators lazily in the case of overlapping operators as
  mentioned above. Under these circumstances, the parse may not always succeed.
- eval(lo, hi) returns the evaluation of a range [lo, hi) of already split-up
  expression tokens, where lo and hi must be random-access iterators.
- eval(s) returns the evaluation of expression s, after first calling split(s)
  to obtain the tokens.

Time Complexity:
- O(m) per call to the constructor, where m is the total number of operators.
- O(nmk) per call to split(s), where n is the length of s, m is the total number
  of operators defined for the parser instance, and k is the maximum length for
  any operator representation.
- O(n log m) per call to eval(lo, hi), where n is the distance between lo and hi
  and m is the total number of operators defined for the parser instance. In
  C++11 and later, std::unordered_map may be used in place of std::map for
  storing the unary_ops and binary_ops, which will eliminate the log m factor
  for a time complexity of O(n) per call.
- O(nmk + n log m) per call to eval(s), where n is the distance between lo and
  hi, and m and k are as defined previous.

Space Complexity:
- O(mk) for storage of the m operators, of maximum length k.
- O(n) auxiliary stack space for split(s), eval(lo, hi), and eval(s), where n is
  the length of the argument.

*/

#include <algorithm>
#include <cctype>
#include <functional>
#include <map>
#include <set>
#include <sstream>
#include <stdexcept>
#include <string>
#include <utility>
#include <vector>
using std::string;

// Define the custom operand type and representation below.
typedef double Operand;
typedef Operand (*UnaryOp)(Operand a);
typedef Operand (*BinaryOp)(Operand a, Operand b);

bool is_operand(const string &s) {
  int npoints = 0;
  for (int i = 0; i < (int)s.size(); i++) {
    if (s[i] == '.') {
      if (++npoints > 1) {
        return false;
      }
    } else if (!isdigit(s[i])) {
      return false;
    }
  }
  return !s.empty();
}

Operand eval_operand(const string &s) {
  Operand res;
  std::stringstream ss(s);
  ss >> res;
  return res;
}

class parser {
  typedef std::map<string, UnaryOp> unary_op_map;
  typedef std::map<string, std::pair<BinaryOp, int> > binary_op_map;
  unary_op_map unary_ops;
  binary_op_map binary_ops;
  std::set<string> ops;
  int max_precedence;

  template<class StrIt>
  Operand eval_unary(StrIt &lo, StrIt hi) {
    if (is_operand(*lo)) {
      return eval_operand(*(lo++));
    }
    unary_op_map::iterator it = unary_ops.find(*lo);
    if (it != unary_ops.end()) {
      return (it->second)(eval_unary(++lo, hi));
    }
    if (*lo != "(") {
      throw std::runtime_error("Expected \"(\" during eval.");
    }
    Operand res = eval_binary(++lo, hi, 0);
    if (*lo != ")") {
      throw std::runtime_error("Expected \")\" during eval.");
    }
    ++lo;
    return res;
  }

  template<class StrIt>
  Operand eval_binary(StrIt &lo, StrIt hi, Operand precedence) {
    if (precedence > max_precedence) {
      return eval_unary(lo, hi);
    }
    Operand v = eval_binary(lo, hi, precedence + 1);
    while (lo != hi) {
      binary_op_map::iterator it;
      it = binary_ops.find(*lo);
      if (it == binary_ops.end() || it->second.second != precedence) {
        return v;
      }
      v = (it->second.first)(v, eval_binary(++lo, hi, precedence + 1));
    }
    return v;
  }

  static string strip(string s) {
    s.erase(s.begin(), std::find_if(s.begin(), s.end(),
            std::not1(std::ptr_fun<int, int>(std::isspace))));
    s.erase(std::find_if(s.rbegin(), s.rend(),
            std::not1(std::ptr_fun<int, int>(std::isspace))).base(), s.end());
    return s;
  }

 public:
  parser(const unary_op_map &unary_ops, const binary_op_map &binary_ops)
      : unary_ops(unary_ops), binary_ops(binary_ops) {
    for (unary_op_map::const_iterator it = unary_ops.begin();
         it != unary_ops.end(); ++it) {
      ops.insert(it->first);
    }
    max_precedence = 0;
    for (binary_op_map::const_iterator it = binary_ops.begin();
         it != binary_ops.end(); ++it) {
      ops.insert(it->first);
      max_precedence = std::max(max_precedence, it->second.second);
    }
  }

  std::vector<string> split(const string &s) {
    std::vector<string> res;
    for (int i = 0; i < (int)s.size(); i++) {
      if (s[i] == ' ') {
        continue;
      }
      int next_paren = s.size();
      for (int j = i; j < (int)s.size(); j++) {
        if (s[j] == '(' || s[j] == ')') {
          next_paren = j;
          break;
        }
      }
      while (i < next_paren) {
        int found = next_paren;
        string found_op;
        for (int j = i; j < next_paren && found == next_paren; j++) {
          for (std::set<string>::iterator it = ops.begin();
               it != ops.end(); ++it) {
            if (s.substr(j, it->size()) == *it) {
              found = j;
              found_op = *it;
              break;
            }
          }
        }
        string term = strip(s.substr(i, found - i));
        if (!term.empty()) {
          res.push_back(term);
          if (!is_operand(term)) {
            throw std::runtime_error("Failed to split term: \"" + term + "\".");
          }
        }
        if (found < next_paren) {
          res.push_back(found_op);
          i = found + found_op.size();
        } else {
          i = next_paren;
        }
      }
      if (next_paren < s.size()) {
        res.push_back(string(1, s[next_paren]));
      }
    }
    return res;
  }

  template<class StrIt>
  Operand eval(StrIt lo, StrIt hi) {
    Operand res = eval_binary(lo, hi, 0);
    if (lo != hi) {
      throw std::runtime_error("Eval failed at token " + *lo + ".");
    }
    return res;
  }

  Operand eval(const string &s) {
    std::vector<string> tokens = split(s);
    return eval(tokens.begin(), tokens.end());
  }
};

/*** Example Usage ***/

#include <cassert>
#include <cmath>
using namespace std;

#define EQ(a, b) (fabs((a) - (b)) < 1e-7)

double pos(double a) { return +a; }
double neg(double a) { return -a; }
double add(double a, double b) { return a + b; }
double sub(double a, double b) { return a - b; }
double mul(double a, double b) { return a * b; }
double div(double a, double b) { return a / b; }

int main() {
  map<string, UnaryOp> unary_ops;
  unary_ops["+"] = pos;
  unary_ops["-"] = neg;

  map<string, pair<BinaryOp, int> > binary_ops;
  binary_ops["+"] = make_pair((BinaryOp)add, 0);
  binary_ops["-"] = make_pair((BinaryOp)sub, 0);
  binary_ops["*"] = make_pair((BinaryOp)mul, 1);
  binary_ops["/"] = make_pair((BinaryOp)div, 1);
  binary_ops["^"] = make_pair((BinaryOp)pow, 2);

  parser p(unary_ops, binary_ops);
  assert(EQ(p.eval("-+-((--(-+1)))"), -1));
  assert(EQ(p.eval("5*(3+3)-2-2"), 26));
  assert(EQ(p.eval("1+2*3*4+3*(+2)-100"), -69));
  assert(EQ(p.eval("3*3*3*3*3*3-2*2*2*2*2*2*2*2"), 473));
  assert(EQ(p.eval("3.14 + 3 * (7.7/9.8^32.9  )"), 3.14));
  assert(EQ(p.eval("5*(3+2)/-1*-2+(-2-2-2+3)-3-(-2)+15/2/2/2+(-2)"), 45.875));
  assert(EQ(p.eval("123456789./3/3/3*2*2*2+456/6-23/3"), 36579857.6666666667));
  assert(EQ(p.eval("10/3+10/4+10/5+10/6+10/7+10/8+10/9+10/10+15*23456"),
            351854.28968253968));
  assert(EQ(p.eval("-(5-(5-(5-(5-(5-2)))))+(3-(3-(3-(3-(3+3)))))*"
                   "(7-(7-(7-(7-(7-7+4*5)))))"), 117));
  return 0;
}
\end{lstlisting}
\subsection{Shunting Yard Parsing}
\begin{lstlisting}
/*

Evaluate an expression using a generalized parser class for custom-defined
operand types, prefix unary operators, binary operators, and precedences.
Typical parentheses behavior is supported, but multiplication by juxtaposition
is not. Evaluation is performed using the shunting yard algorithm.

An arbitrary operand type is supported, with its string representation defined
by a user-specified is_operand() and eval_operand() functions. For maximum
reliability, the string representation of operands should not use characters
shared by any operator. For instance, the best practice instead of accepting
"-1" as a valid operand (since the "-" sign may conflict with the identical
binary operator), is to specify non-negative number as operands alongside the
unary operator "-".

Operators may be non-empty strings of any length, but should not contain any
parentheses or shared characters with the string representations of operands.
Ideally, operators should not be prefixes or suffices of one another, else the
tokenization process may be ambiguous. For example, if ++ and + are both
operators, then ++ may be split into either ["+", "+"] or ["++"] depending on
the lexicographical ordering of conflicting operators.

- parser(unary_op, binary_op) initializes a parser with operators specified by
  maps unary_op (of operator to function pointer) and binary_op (of operator to
  pair of function pointer and operator precedence). Operator precedences should
  be numbered upwards starting at 0 (lowest precedence, evaluated last).
- split(s) returns a vector of tokens for the expression s, split on the given
  operators during construction. Each parenthesis, operator, and operand
  satisfying is_operand() will be split into a separate token. The algorithm is
  naive, matching operators lazily in the case of overlapping operators as
  mentioned above. Under these circumstances, the parse may not always succeed.
- eval(lo, hi) returns the evaluation of a range [lo, hi) of already split-up
  expression tokens, where lo and hi must be random-access iterators.
- eval(s) returns the evaluation of expression s, after first calling split(s)
  to obtain the tokens.

Time Complexity:
- O(m) per call to the constructor, where m is the total number of operators.
- O(nmk) per call to split(s), where n is the length of s, m is the total number
  of operators defined for the parser instance, and k is the maximum length for
  any operator representation.
- O(n log m) per call to eval(lo, hi), where n is the distance between lo and hi
  and m is the total number of operators defined for the parser instance. In
  C++11 and later, std::unordered_map may be used in place of std::map for
  storing the unary_ops and binary_ops, which will eliminate the log m factor
  for a time complexity of O(n) per call.
- O(nmk + n log m) per call to eval(s), where n is the distance between lo and
  hi, and m and k are as defined previous.

Space Complexity:
- O(mk) for storage of the m operators, of maximum length k.
- O(n) auxiliary stack space for split(s), eval(lo, hi), and eval(s), where n is
  the length of the argument.

*/

#include <algorithm>
#include <cctype>
#include <functional>
#include <map>
#include <set>
#include <sstream>
#include <stack>
#include <stdexcept>
#include <string>
#include <utility>
#include <vector>
using std::string;

// Define the custom operand type and representation below.
typedef double Operand;
typedef Operand (*UnaryOp)(Operand a);
typedef Operand (*BinaryOp)(Operand a, Operand b);

bool is_operand(const string &s) {
  int npoints = 0;
  for (int i = 0; i < (int)s.size(); i++) {
    if (s[i] == '.') {
      if (++npoints > 1) {
        return false;
      }
    } else if (!isdigit(s[i])) {
      return false;
    }
  }
  return !s.empty();
}

Operand eval_operand(const string &s) {
  Operand res;
  std::stringstream ss(s);
  ss >> res;
  return res;
}

class parser {
  typedef std::map<string, UnaryOp> unary_op_map;
  typedef std::map<string, std::pair<BinaryOp, int> > binary_op_map;
  unary_op_map unary_ops;
  binary_op_map binary_ops;
  std::set<string> ops;

  static string strip(string s) {
    s.erase(s.begin(), std::find_if(s.begin(), s.end(),
            std::not1(std::ptr_fun<int, int>(std::isspace))));
    s.erase(std::find_if(s.rbegin(), s.rend(),
            std::not1(std::ptr_fun<int, int>(std::isspace))).base(), s.end());
    return s;
  }

 public:
  parser(const unary_op_map &unary_ops, const binary_op_map &binary_ops)
      : unary_ops(unary_ops), binary_ops(binary_ops) {
    for (unary_op_map::const_iterator it = unary_ops.begin();
         it != unary_ops.end(); ++it) {
      ops.insert(it->first);
    }
    for (binary_op_map::const_iterator it = binary_ops.begin();
         it != binary_ops.end(); ++it) {
      ops.insert(it->first);
    }
  }

  std::vector<string> split(const string &s) {
    std::vector<string> res;
    for (int i = 0; i < (int)s.size(); i++) {
      if (s[i] == ' ') {
        continue;
      }
      int next_paren = s.size();
      for (int j = i; j < (int)s.size(); j++) {
        if (s[j] == '(' || s[j] == ')') {
          next_paren = j;
          break;
        }
      }
      while (i < next_paren) {
        int found = next_paren;
        string found_op;
        for (int j = i; j < next_paren && found == next_paren; j++) {
          for (std::set<string>::iterator it = ops.begin();
               it != ops.end(); ++it) {
            if (s.substr(j, it->size()) == *it) {
              found = j;
              found_op = *it;
              break;
            }
          }
        }
        string term = strip(s.substr(i, found - i));
        if (!term.empty()) {
          res.push_back(term);
          if (!is_operand(term)) {
            throw std::runtime_error("Failed to split term: \"" + term + "\".");
          }
        }
        if (found < next_paren) {
          res.push_back(found_op);
          i = found + found_op.size();
        } else {
          i = next_paren;
        }
      }
      if (next_paren < s.size()) {
        res.push_back(string(1, s[next_paren]));
      }
    }
    return res;
  }

  template<class StrIt>
  Operand eval(StrIt lo, StrIt hi) {
    std::stack<Operand> vals;
    std::stack<std::pair<string, bool> > ops;
    ops.push(std::make_pair("(", false));
    StrIt prev = hi;
    do {
      string curr = (lo == hi) ? ")" : *lo;
      if (is_operand(curr)) {
        vals.push(eval_operand(curr));
      } else if (curr == "(") {
        ops.push(std::make_pair(curr, false));
      } else if (unary_ops.find(curr) != unary_ops.end() && (prev == hi ||
                 *prev == "(" || binary_ops.find(*prev) != binary_ops.end())) {
        ops.push(std::make_pair(curr, true));
      } else {
        for (;;) {
          string op = ops.top().first;
          bool is_unary = ops.top().second;
          binary_op_map::iterator it1 = binary_ops.find(op);
          binary_op_map::iterator it2 = binary_ops.find(curr);
          if (!is_unary &&
              (it1 == binary_ops.end() ? -1 : it1->second.second) <
              (it2 == binary_ops.end() ? -1 : it2->second.second)) {
            break;
          }
          ops.pop();
          if (op == "(") {
            break;
          }
          Operand b = vals.top();
          vals.pop();
          if (is_unary) {
            unary_op_map::iterator it = unary_ops.find(op);
            if (it == unary_ops.end()) {
              throw std::runtime_error("Failed to eval unary op: " + op);
            }
            vals.push((it->second)(b));
          } else {
            Operand a = vals.top();
            vals.pop();
            if (it1 == binary_ops.end()) {
              throw std::runtime_error("Failed to eval binary op: " + op);
            }
            vals.push((it1->second.first)(a, b));
          }
        }
        if (curr != ")") {
          ops.push(std::make_pair(*lo, false));
        }
      }
      prev = lo;
    } while (lo++ != hi);
    return vals.top();
  }

  Operand eval(const string &s) {
    std::vector<string> tokens = split(s);
    return eval(tokens.begin(), tokens.end());
  }
};

/*** Example Usage ***/

#include <cassert>
#include <cmath>
using namespace std;

#define EQ(a, b) (fabs((a) - (b)) < 1e-7)

double pos(double a) { return +a; }
double neg(double a) { return -a; }
double add(double a, double b) { return a + b; }
double sub(double a, double b) { return a - b; }
double mul(double a, double b) { return a * b; }
double div(double a, double b) { return a / b; }

int main() {
  map<string, UnaryOp> unary_ops;
  unary_ops["+"] = pos;
  unary_ops["-"] = neg;

  map<string, pair<BinaryOp, int> > binary_ops;
  binary_ops["+"] = make_pair((BinaryOp)add, 0);
  binary_ops["-"] = make_pair((BinaryOp)sub, 0);
  binary_ops["*"] = make_pair((BinaryOp)mul, 1);
  binary_ops["/"] = make_pair((BinaryOp)div, 1);
  binary_ops["^"] = make_pair((BinaryOp)pow, 2);

  parser p(unary_ops, binary_ops);
  assert(EQ(p.eval("-+-((--(-+1)))"), -1));
  assert(EQ(p.eval("5*(3+3)-2-2"), 26));
  assert(EQ(p.eval("1+2*3*4+3*(+2)-100"), -69));
  assert(EQ(p.eval("3*3*3*3*3*3-2*2*2*2*2*2*2*2"), 473));
  assert(EQ(p.eval("3.14 + 3 * (7.7/9.8^32.9  )"), 3.14));
  assert(EQ(p.eval("5*(3+2)/-1*-2+(-2-2-2+3)-3-(-2)+15/2/2/2+(-2)"), 45.875));
  assert(EQ(p.eval("123456789./3/3/3*2*2*2+456/6-23/3"), 36579857.6666666667));
  assert(EQ(p.eval("10/3+10/4+10/5+10/6+10/7+10/8+10/9+10/10+15*23456"),
            351854.28968253968));
  assert(EQ(p.eval("-(5-(5-(5-(5-(5-2)))))+(3-(3-(3-(3-(3+3)))))*"
                   "(7-(7-(7-(7-(7-7+4*5)))))"), 117));
  return 0;
}
\end{lstlisting}

\section{Dynamic Programming}
\setcounter{section}{4}
\setcounter{subsection}{0}
\subsection{Longest Common Substring}
\begin{lstlisting}
/*

Given two strings, determine their longest common substring (i.e. consecutive
subsequence) using dynamic programming.

Time Complexity:
- O(n*m) per call to longest_common_substring(s1, s2), where n and m are the
  lengths of s1 and s2, respectively.

Space Complexity:
- O(min(n, m)) auxiliary heap space, where n and m are the lengths of s1 and
  s2, respectively.

*/

#include <string>
#include <vector>
using std::string;

string longest_common_substring(const string &s1, const string &s2) {
  int n = s1.size(), m = s2.size();
  if (n == 0 || m == 0) {
    return "";
  }
  if (n < m) {
    return longest_common_substring(s2, s1);
  }
  std::vector<int> curr(m), prev(m);
  int pos = 0, len = 0;
  for (int i = 0; i < n; i++) {
    for (int j = 0; j < m; j++) {
      if (s1[i] == s2[j]) {
        curr[j] = (i > 0 && j > 0) ? prev[j - 1] + 1 : 1;
        if (len < curr[j]) {
          len = curr[j];
          pos = i - curr[j] + 1;
        }
      } else {
        curr[j] = 0;
      }
    }
    curr.swap(prev);
  }
  return s1.substr(pos, len);
}

/*** Example Usage ***/

#include <cassert>

int main() {
  assert(longest_common_substring("bbbabca", "aababcd") == "babc");
  return 0;
}
\end{lstlisting}
\subsection{Longest Common Subsequence}
\begin{lstlisting}
/*

Given two strings, determine their longest common subsequence. A subsequence is
a string that can be derived from the original string by deleting some elements
without changing the order of the remaining elements (e.g. "ACE" is a
subsequence of "ABCDE", but "BAE" is not).

- longest_common_subsequence(s1, s2) returns the longest common subsequence of
  strings s1 and s2 using a classic dynamic programming approach. This
  implementation computes dp[i][j] (the length of the longest common subsequence
  for the length i prefix of s1 and the length j prefix of s2) before following
  the path backwards to construct the answer.
- hirschberg_lcs(s1, s2) returns the longest common subsequence of strings s1
  and s2 using the more memory efficient Hirschberg's algorithm.

Time Complexity:
- O(n*m) per call to longest_common_subsequence(s1, s2) as well as
  hirschberg_lcs(s1, s2), where n and m are the lengths of s1 and s2,
  respectively.

Space Complexity:
- O(n*m) auxiliary heap space for longest_common_subsequence(s1, s2), where n
  and m are the lengths of s1 and s2, respectively.
- O(log max(n, m)) auxiliary stack space and O(min(n, m)) auxiliary heap space
  for hirschberg_lcs(s1, s2), where n and m are the lengths of s1 and s2,
  respectively.

*/

#include <algorithm>
#include <iterator>
#include <string>
#include <vector>
using std::string;

string longest_common_subsequence(const string &s1, const string &s2) {
  int n = s1.size(), m = s2.size();
  std::vector<std::vector<int> > dp(n + 1, std::vector<int>(m + 1, 0));
  for (int i = 1; i <= n; i++) {
    for (int j = 1; j <= m; j++) {
      if (s1[i - 1] == s2[j - 1]) {
        dp[i][j] = dp[i - 1][j - 1] + 1;
      } else {
        dp[i][j] = std::max(dp[i][j - 1], dp[i - 1][j]);
      }
    }
  }
  string res;
  for (int i = n, j = m; i > 0 && j > 0; ) {
    if (s1[i - 1] == s2[j - 1]) {
      res += s1[i - 1];
      i--;
      j--;
    } else if (dp[i - 1][j] >= dp[i][j - 1]) {
      i--;
    } else {
      j--;
    }
  }
  std::reverse(res.begin(), res.end());
  return res;
}

template<class It>
std::vector<int> lcs_len(It lo1, It hi1, It lo2, It hi2) {
  std::vector<int> res(std::distance(lo2, hi2) + 1), prev(res);
  for (It it1 = lo1; it1 != hi1; ++it1) {
    res.swap(prev);
    int i = 0;
    for (It it2 = lo2; it2 != hi2; ++it2) {
      res[i + 1] = (*it1 == *it2) ? prev[i] + 1 : std::max(res[i], prev[i + 1]);
      i++;
    }
  }
  return res;
}

template<class It>
void hirschberg_rec(It lo1, It hi1, It lo2, It hi2, string *res) {
  if (lo1 == hi1) {
    return;
  }
  if (lo1 + 1 == hi1) {
    if (std::find(lo2, hi2, *lo1) != hi2) {
      *res += *lo1;
    }
    return;
  }
  It mid1 = lo1 + (hi1 - lo1)/2;
  std::reverse_iterator<It> rlo1(hi1), rmid1(mid1), rlo2(hi2), rhi2(lo2);
  std::vector<int> fwd = lcs_len(lo1, mid1, lo2, hi2);
  std::vector<int> rev = lcs_len(rlo1, rmid1, rlo2, rhi2);
  It mid2 = lo2;
  int maxlen = -1;
  for (int i = 0, j = (int)rev.size() - 1; i < (int)fwd.size(); i++, j--) {
    if (fwd[i] + rev[j] > maxlen) {
      maxlen = fwd[i] + rev[j];
      mid2 = lo2 + i;
    }
  }
  hirschberg_rec(lo1, mid1, lo2, mid2, res);
  hirschberg_rec(mid1, hi1, mid2, hi2, res);
}

string hirschberg_lcs(const string &s1, const string &s2) {
  if (s1.size() < s2.size()) {
    return hirschberg_lcs(s2, s1);
  }
  string res;
  hirschberg_rec(s1.begin(), s1.end(), s2.begin(), s2.end(), &res);
  return res;
}

/*** Example Usage ***/

#include <cassert>

int main() {
  assert(longest_common_subsequence("xmjyauz", "mzjawxu") == "mjau");
  assert(hirschberg_lcs("xmjyauz", "mzjawxu") == "mjau");
  return 0;
}
\end{lstlisting}
\subsection{Sequence Alignment}
\begin{lstlisting}
/*

Given two strings, determine their minimum-cost alignment. An alignment of two
strings is a transformation of both strings by inserting gap characters '_' in
some way to make the final lengths equal. The total cost of an alignment given
a gap_cost (insertion or deletion cost) and a sub_cost (substitution, i.e.
mismatch cost) is gap_cost*(the number of gaps inserted across both strings),
plus sub_cost*(the number of indices at which the two aligned strings differ).

- align_sequences(s1, s2, gap_cost, sub_cost) returns a pair of aligned strings
  for strings s1 and s2, using a classic dynamic programming approach. This
  implementation first computes dp[i][j] (the cost of aligning the length i
  prefix of s1 with the length j prefix of s2) before following the path
  backwards to construct the answer. For gap_cost = sub_cost = 1, dp[n][m] will
  be the Levenshtein edit distance, where n and m are the lengths of s1 and
  s2, respectively.
- hirschberg_align_sequences(s1, s2, gap_cost, sub_cost) returns the sequence
  alignment of strings s1 and s2 using the more memory efficient Hirschberg's
  algorithm.

Time Complexity:
- O(n*m) per call to align_sequences(s1, s2) as well as
  hirschberg_align_sequences(s1, s2), where n and m are the lengths of s1 and
  s2, respectively.

Space Complexity:
- O(n*m) auxiliary heap space for align_sequences(s1, s2), where n and m are the
  lengths of s1 and s2, respectively.
- O(log max(n, m)) auxiliary stack space and O(min(n, m)) auxiliary heap space
  for hirschberg_align_sequences(s1, s2), where n and m are the lengths of s1
  and s2, respectively.

*/

#include <algorithm>
#include <string>
#include <vector>
#include <utility>
using std::string;

std::pair<string, string> align_sequences(
    const string &s1, const string &s2, int gap_cost = 1, int sub_cost = 1) {
  int n = s1.size(), m = s2.size();
  std::vector<std::vector<int> > dp(n + 1, std::vector<int>(m + 1, 0));
  for (int i = 0; i <= n; i++) {
    dp[i][0] = i*gap_cost;
  }
  for (int j = 0; j <= m; j++) {
    dp[0][j] = j*gap_cost;
  }
  for (int i = 1; i <= n; i++) {
    for (int j = 1; j <= m; j++) {
      dp[i][j] = (s1[i - 1] == s2[j - 1]) ? dp[i - 1][j - 1] : std::min(
          dp[i - 1][j - 1] + sub_cost,
          std::min(dp[i - 1][j], dp[i][j - 1]) + gap_cost);
    }
  }
  string res1, res2;
  int i = n, j = m;
  while (i > 0 && j > 0) {
    if (s1[i - 1] == s2[j - 1] || dp[i][j] == dp[i - 1][j - 1] + sub_cost) {
      res1 += s1[--i];
      res2 += s2[--j];
    } else if (dp[i][j] == dp[i - 1][j] + gap_cost) {
      res1 += s1[--i];
      res2 += '_';
    } else if (dp[i][j] == dp[i][j - 1] + gap_cost) {
      res1 += '_';
      res2 += s2[--j];
    }
  }
  while (i > 0 || j > 0) {
    res1 += (i > 0) ? s1[--i] : '_';
    res2 += (j > 0) ? s2[--j] : '_';
  }
  std::reverse(res1.begin(), res1.end());
  std::reverse(res2.begin(), res2.end());
  return std::make_pair(res1, res2);
}

template<class It>
std::vector<int> row_cost(It lo1, It hi1, It lo2, It hi2,
                          int gap_cost, int sub_cost) {
  std::vector<int> res(std::distance(lo2, hi2) + 1), prev(res);
  for (It it1 = lo1; it1 != hi1; ++it1) {
    res.swap(prev);
    int i = 0;
    for (It it2 = lo2; it2 != hi2; ++it2) {
      res[i + 1] = (*it1 == *it2) ? prev[i] : std::min(prev[i] + sub_cost,
                                                       res[i] + gap_cost);
      i++;
    }
  }
  return res;
}

template<class It>
void hirschberg_rec(It lo1, It hi1, It lo2, It hi2,
                    string *res1, string *res2, int gap_cost, int sub_cost) {
  if (lo1 == hi1) {
    for (It it2 = lo2; it2 != hi2; ++it2) {
      *res1 += '_';
      *res2 += *it2;
    }
    return;
  }
  if (lo1 + 1 == hi1) {
    It pos = std::find(lo2, hi2, *lo1);
    bool insert = (pos == hi2) && (gap_cost*(hi2 - lo2 + 1) < sub_cost);
    if (lo2 == hi2 || insert) {
      *res1 += *lo1;
      *res2 += '_';
    }
    for (It it2 = lo2; it2 != hi2; ++it2) {
      *res1 += (pos == it2 || (!insert && it2 == lo2)) ? *lo1 : '_';
      *res2 += *it2;
    }
    return;
  }
  It mid1 = lo1 + (hi1 - lo1)/2;
  std::reverse_iterator<It> rlo1(hi1), rmid1(mid1), rlo2(hi2), rhi2(lo2);
  std::vector<int> fwd = row_cost(lo1, mid1, lo2, hi2, gap_cost, sub_cost);
  std::vector<int> rev = row_cost(rlo1, rmid1, rlo2, rhi2, gap_cost, sub_cost);
  It mid2 = lo2;
  int mincost = -1;
  for (int i = 0, j = (int)rev.size() - 1; i < (int)fwd.size(); i++, j--) {
    if (mincost < 0 || fwd[i] + rev[j] < mincost) {
      mincost = fwd[i] + rev[j];
      mid2 = lo2 + i;
    }
  }
  hirschberg_rec(lo1, mid1, lo2, mid2, res1, res2, gap_cost, sub_cost);
  hirschberg_rec(mid1, hi1, mid2, hi2, res1, res2, gap_cost, sub_cost);
}

std::pair<string, string> hirschberg_align_sequences(
    const string &s1, const string &s2, int gap_cost = 1, int sub_cost = 1) {
  if (s1.size() < s2.size()) {
    return hirschberg_align_sequences(s2, s1, gap_cost, sub_cost);
  }
  string res1, res2;
  hirschberg_rec(s1.begin(), s1.end(), s2.begin(), s2.end(), &res1, &res2,
                 gap_cost, sub_cost);
  return std::make_pair(res1, res2);
}

/*** Example Usage ***/

#include <cassert>

int main() {
  assert(align_sequences("AGGGCT", "AGGCA", 2, 3) ==
             make_pair(string("AGGGCT"), string("A_GGCA")));
  assert(hirschberg_align_sequences("AGGGCT", "AGGCA", 2, 3) ==
             make_pair(string("AGGGCT"), string("A_GGCA")));
  return 0;
}
\end{lstlisting}

\section{Suffix Array and LCP}
\setcounter{section}{5}
\setcounter{subsection}{0}
\subsection{Suffix Array and LCP (Manber-Myers)}
\begin{lstlisting}
/*

Given a string s, a suffix array is the array of the smallest starting positions
for the sorted suffices of s. That is, the i-th position of the suffix array
stores the starting position of the i-th lexicographically smallest suffix of s.
For examples, s = "cab" has the suffices "cab", "ab", and "b". When sorted, the
indices of the suffixes are "ab", "b", and "cab", so the suffix array (assuming
zero-based indices) is [1, 2, 0].

For a string s of length n the longest common prefix (LCP) array of length n - 1
stores the lengths of the longest common prefixes between all pairs of
lexicographically adjacent suffices in s. For example, "baa" has the sorted
suffices "a", "aa", and "baa", with an LCP array of [1, 0].

- suffix_array(s) constructs a suffix array from the given string s using the
  original Manber-Myers gap partitioning algorithm with a comparison-based sort.
- get_sa() returns the constructed suffix array.
- get_lcp() returns the corresponding LCP array for the suffix array.
- find(needle) returns one position that needle occurs in s (not necessarily the
  first), or std::string::npos if it cannot be found. For a needle of length m,
  this implementation uses an O(m log n) binary search, but can be optimized to
  O(m + log n) by first computing the LCP-LR array using the LCP array.

Time Complexity:
- O(n log^2 n) per call to the constructor, where n is the length of s.
- O(1) per call to get_sa().
- O(n) per call to get_lcp(), where n is the length of s.
- O(m log n) per call to find(needle), where m is the length of needle and n is
  the length of s.

Space Complexity:
- O(n) auxiliary for storage of the suffix and LCP arrays, where n is the length
  of s.
- O(n) auxiliary heap space for the constructor.
- O(1) auxiliary space for all other operations.

*/

#include <algorithm>
#include <string>
#include <utility>
#include <vector>
using std::string;

class suffix_array {
  struct comp {
    const std::vector<std::pair<int, int> > &rank;

    comp(const std::vector<std::pair<int, int> > &rank) : rank(rank) {}

    bool operator()(int i, int j) {
      return rank[i] < rank[j];
    }
  };

  string s;
  std::vector<int> sa, rank;

 public:
  suffix_array(const string &s) : s(s), sa(s.size()), rank(s.size()) {
    int n = s.size();
    for (int i = 0; i < n; i++) {
      sa[i] = i;
      rank[i] = (int)s[i];
    }
    std::vector<std::pair<int, int> > rank2(n);
    for (int gap = 1; gap < n; gap *= 2) {
      for (int i = 0; i < n; i++) {
        rank2[i] = std::make_pair(rank[i], i + gap < n ? rank[i + gap] + 1 : 0);
      }
      std::sort(sa.begin(), sa.end(), comp(rank2));
      for (int i = 0; i < n; i++) {
        rank[sa[i]] = (i > 0 && rank2[sa[i - 1]] == rank2[sa[i]])
                      ? rank[sa[i - 1]] : i;
      }
    }
  }

  std::vector<int> get_sa() {
    return sa;
  }

  std::vector<int> get_lcp() {
    int n = s.size();
    std::vector<int> lcp(n - 1);
    for (int i = 0, k = 0; i < n; i++) {
      if (rank[i] < n - 1) {
        int j = sa[rank[i] + 1];
        while (std::max(i, j) + k < n && s[i + k] == s[j + k]) {
          k++;
        }
        lcp[rank[i]] = k;
        if (k > 0) {
          k--;
        }
      }
    }
    return lcp;
  }

  size_t find(const string &needle) {
    int lo = 0, hi = (int)s.size() - 1;
    while (lo <= hi) {
      int mid = lo + (hi - lo)/2;
      int cmp = s.compare(sa[mid], needle.size(), needle);
      if (cmp < 0) {
        lo = mid + 1;
      } else if (cmp > 0) {
        hi = mid - 1;
      } else {
        return sa[mid];
      }
    }
    return string::npos;
  }
};

/*** Example Usage ***/

#include <cassert>
using namespace std;

int main() {
  suffix_array sa("banana");
  vector<int> sarr = sa.get_sa(), lcp = sa.get_lcp();
  int sarr_expected[] = {5, 3, 1, 0, 4, 2};
  int lcp_expected[] = {1, 3, 0, 0, 2};
  assert(equal(sarr.begin(), sarr.end(), sarr_expected));
  assert(equal(lcp.begin(), lcp.end(), lcp_expected));
  assert(sa.find("ana") == 1);
  assert(sa.find("x") == string::npos);
  return 0;
}
\end{lstlisting}
\subsection{Suffix Array and LCP (Counting Sort)}
\begin{lstlisting}
/*

Given a string s, a suffix array is the array of the smallest starting positions
for the sorted suffices of s. That is, the i-th position of the suffix array
stores the starting position of the i-th lexicographically smallest suffix of s.
For examples, s = "cab" has the suffices "cab", "ab", and "b". When sorted, the
indices of the suffixes are "ab", "b", and "cab", so the suffix array (assuming
zero-based indices) is [1, 2, 0].

For a string s of length n the longest common prefix (LCP) array of length n - 1
stores the lengths of the longest common prefixes between all pairs of
lexicographically adjacent suffices in s. For example, "baa" has the sorted
suffices "a", "aa", and "baa", with an LCP array of [1, 0].

- suffix_array(s) constructs a suffix array from the given string s using the
  original Manber-Myers gap partitioning algorithm with a counting sort instead
  of a comparison-based sort to reduce the running time to O(n log n).
- get_sa() returns the constructed suffix array.
- get_lcp() returns the corresponding LCP array for the suffix array.
- find(needle) returns one position that needle occurs in s (not necessarily the
  first), or std::string::npos if it cannot be found. For a needle of length m,
  this implementation uses an O(m log n) binary search, but can be optimized to
  O(m + log n) by first computing the LCP-LR array using the LCP array.

Time Complexity:
- O(n log n) per call to the constructor, where n is the length of s.
- O(1) per call to get_sa().
- O(n) per call to get_lcp(), where n is the length of s.
- O(m log n) per call to find(needle), where m is the length of needle and n is
  the length of s.

Space Complexity:
- O(n) auxiliary for storage of the suffix and LCP arrays, where n is the length
  of s.
- O(n) auxiliary heap space for the constructor.
- O(1) auxiliary space for all other operations.

*/

#include <algorithm>
#include <string>
#include <utility>
#include <vector>
using std::string;

class suffix_array {
  struct comp {
    const string &s;

    comp(const string &s) : s(s) {}

    bool operator()(int i, int j) {
      return s[i] < s[j];
    }
  };

  string s;
  std::vector<int> sa, rank;

 public:
  suffix_array(const string &s) : s(s), sa(s.size()), rank(s.size()) {
    int n = s.size();
    for (int i = 0; i < n; i++) {
      sa[i] = n - 1 - i;
      rank[i] = (int)s[i];
    }
    std::stable_sort(sa.begin(), sa.end(), comp(s));
    for (int gap = 1; gap < n; gap *= 2) {
      std::vector<int> prev_rank(rank), prev_sa(sa), cnt(n);
      for (int i = 0; i < n; i++) {
        cnt[i] = i;
      }
      for (int i = 0; i < n; i++) {
        rank[sa[i]] = (i > 0 && prev_rank[sa[i - 1]] == prev_rank[sa[i]] &&
                       sa[i - 1] + gap < n &&
                       prev_rank[sa[i - 1] + gap/2] == prev_rank[sa[i] + gap/2])
                           ? rank[sa[i - 1]] : i;
      }
      for (int i = 0; i < n; i++) {
        int s1 = prev_sa[i] - gap;
        if (s1 >= 0) {
          sa[cnt[rank[s1]]++] = s1;
        }
      }
    }
  }

  std::vector<int> get_sa() {
    return sa;
  }

  std::vector<int> get_lcp() {
    int n = s.size();
    std::vector<int> lcp(n - 1);
    for (int i = 0, k = 0; i < n; i++) {
      if (rank[i] < n - 1) {
        int j = sa[rank[i] + 1];
        while (std::max(i, j) + k < n && s[i + k] == s[j + k]) {
          k++;
        }
        lcp[rank[i]] = k;
        if (k > 0) {
          k--;
        }
      }
    }
    return lcp;
  }

  size_t find(const string &needle) {
    int lo = 0, hi = (int)s.size() - 1;
    while (lo <= hi) {
      int mid = lo + (hi - lo)/2;
      int cmp = s.compare(sa[mid], needle.size(), needle);
      if (cmp < 0) {
        lo = mid + 1;
      } else if (cmp > 0) {
        hi = mid - 1;
      } else {
        return sa[mid];
      }
    }
    return string::npos;
  }
};

/*** Example Usage ***/

#include <cassert>
using namespace std;

int main() {
  suffix_array sa("banana");
  vector<int> sarr = sa.get_sa(), lcp = sa.get_lcp();
  int sarr_expected[] = {5, 3, 1, 0, 4, 2};
  int lcp_expected[] = {1, 3, 0, 0, 2};
  assert(equal(sarr.begin(), sarr.end(), sarr_expected));
  assert(equal(lcp.begin(), lcp.end(), lcp_expected));
  assert(sa.find("ana") == 1);
  assert(sa.find("x") == string::npos);
  return 0;
}
\end{lstlisting}
\subsection{Suffix Array and LCP (Linear DC3)}
\begin{lstlisting}
/*

Given a string s, a suffix array is the array of the smallest starting positions
for the sorted suffices of s. That is, the i-th position of the suffix array
stores the starting position of the i-th lexicographically smallest suffix of s.
For examples, s = "cab" has the suffices "cab", "ab", and "b". When sorted, the
indices of the suffixes are "ab", "b", and "cab", so the suffix array (assuming
zero-based indices) is [1, 2, 0].

For a string s of length n the longest common prefix (LCP) array of length n - 1
stores the lengths of the longest common prefixes between all pairs of
lexicographically adjacent suffices in s. For example, "baa" has the sorted
suffices "a", "aa", and "baa", with an LCP array of [1, 0].

- suffix_array(s) constructs a suffix array from the given string s using the
  linear time DC3/skew algorithm by Karkkainen & Sanders (2003) with radix sort.
- get_sa() returns the constructed suffix array.
- get_lcp() returns the corresponding LCP array for the suffix array.
- find(needle) returns one position that needle occurs in s (not necessarily the
  first), or std::string::npos if it cannot be found. For a needle of length m,
  this implementation uses an O(m log n) binary search, but can be optimized to
  O(m + log n) by first computing the LCP-LR array using the LCP array.

Time Complexity:
- O(n) per call to the constructor, where n is the length of s.
- O(1) per call to get_sa().
- O(n) per call to get_lcp(), where n is the length of s.
- O(m log n) per call to find(needle), where m is the length of needle and n is
  the length of s.

Space Complexity:
- O(n) auxiliary for storage of the suffix and LCP arrays, where n is the length
  of s.
- O(n) auxiliary heap space for the constructor.
- O(1) auxiliary space for all other operations.

*/

#include <algorithm>
#include <string>
#include <utility>
#include <vector>
using std::string;

class suffix_array {
  static bool leq(int a1, int a2, int b1, int b2) {
    return (a1 < b1) || (a1 == b1 && a2 <= b2);
  }

  static bool leq(int a1, int a2, int a3, int b1, int b2, int b3) {
    return (a1 < b1) || (a1 == b1 && leq(a2, a3, b2, b3));
  }

  template<class It>
  static void radix_pass(It a, It b, It r, int n, int K) {
    std::vector<int> cnt(K + 1);
    for (int i = 0; i < n; i++) {
      cnt[r[a[i]]]++;
    }
    for (int i = 1; i <= K; i++) {
      cnt[i] += cnt[i - 1];
    }
    for (int i = n - 1; i >= 0; i--) {
      b[--cnt[r[a[i]]]] = a[i];
    }
  }

  template<class It>
  static void suffix_array_dc3(It s, It sa, int n, int K) {
    int n0 = (n + 2)/3, n1 = (n + 1)/3, n2 = n/3, n02 = n0 + n2;
    std::vector<int> s12(n02 + 3), sa12(n02 + 3), s0(n0), sa0(n0);
    s12[n02] = s12[n02 + 1] = s12[n02 + 2] = 0;
    sa12[n02] = sa12[n02 + 1] = sa12[n02 + 2] = 0;
    for (int i = 0, j = 0; i < n + n0 - n1; i++) {
      if (i % 3 != 0) {
        s12[j++] = i;
      }
    }
    radix_pass(s12.begin(), sa12.begin(), s + 2, n02, K);
    radix_pass(sa12.begin(), s12.begin(), s + 1, n02, K);
    radix_pass(s12.begin(), sa12.begin(), s, n02, K);
    int name = 0, c0 = -1, c1 = -1, c2 = -1;
    for (int i = 0; i < n02; i++) {
      if (s[sa12[i]] != c0 || s[sa12[i] + 1] != c1 || s[sa12[i] + 2] != c2) {
        name++;
        c0 = s[sa12[i]];
        c1 = s[sa12[i] + 1];
        c2 = s[sa12[i] + 2];
      }
      (sa12[i] % 3 == 1 ? s12[sa12[i]/3] : s12[sa12[i]/3 + n0]) = name;
    }
    if (name < n02) {
      suffix_array_dc3(s12.begin(), sa12.begin(), n02, name);
      for (int i = 0; i < n02; i++) {
        s12[sa12[i]] = i + 1;
      }
    } else {
      for (int i = 0; i < n02; i++) {
        sa12[s12[i] - 1] = i;
      }
    }
    for (int i = 0, j = 0; i < n02; i++) {
      if (sa12[i] < n0) {
        s0[j++] = 3*sa12[i];
      }
    }
    radix_pass(s0.begin(), sa0.begin(), s, n0, K);
    for (int p = 0, t = n0 - n1, k = 0; k < n; k++) {
      int i = (sa12[t] < n0) ? 3*sa12[t] + 1 : 3*(sa12[t] - n0) + 2, j = sa0[p];
      if (sa12[t] < n0 ? leq(s[i], s12[sa12[t] + n0],s[j], s12[j/3])
                       : leq(s[i], s[i + 1], s12[sa12[t] - n0 + 1], s[j],
                             s[j + 1], s12[j / 3 + n0])) {
        sa[k] = i;
        if (++t == n02) {
          for (k++; p < n0; p++, k++) {
            sa[k] = sa0[p];
          }
        }
      } else {
        sa[k] = j;
        if (++p == n0) {
          for (k++; t < n02; t++, k++) {
            sa[k] = (sa12[t] < n0) ? 3*sa12[t] + 1 : 3*(sa12[t] - n0) + 2;
          }
        }
      }
    }
  }

  string s;
  std::vector<int> sa;

 public:
  suffix_array(const string &s) : s(s), sa(s.size() + 1) {
    int n = s.size();
    std::vector<int> scopy(s.begin(), s.end());
    scopy.resize(n + 3);
    suffix_array_dc3(scopy.begin(), sa.begin(), n + 1, 255);
    sa.erase(sa.begin());
  }

  std::vector<int> get_sa() {
    return sa;
  }

  std::vector<int> get_lcp() {
    int n = s.size();
    std::vector<int> rank(n), lcp(n - 1);
    for (int i = 0; i < n; i++) {
      rank[sa[i]] = i;
    }
    for (int i = 0, k = 0; i < n; i++) {
      if (rank[i] < n - 1) {
        int j = sa[rank[i] + 1];
        while (std::max(i, j) + k < n && s[i + k] == s[j + k]) {
          k++;
        }
        lcp[rank[i]] = k;
        if (k > 0) {
          k--;
        }
      }
    }
    return lcp;
  }

  size_t find(const string &needle) {
    int lo = 0, hi = (int)s.size() - 1;
    while (lo <= hi) {
      int mid = lo + (hi - lo)/2;
      int cmp = s.compare(sa[mid], needle.size(), needle);
      if (cmp < 0) {
        lo = mid + 1;
      } else if (cmp > 0) {
        hi = mid - 1;
      } else {
        return sa[mid];
      }
    }
    return string::npos;
  }
};

/*** Example Usage ***/

#include <cassert>
using namespace std;

int main() {
  suffix_array sa("banana");
  vector<int> sarr = sa.get_sa(), lcp = sa.get_lcp();
  int sarr_expected[] = {5, 3, 1, 0, 4, 2};
  int lcp_expected[] = {1, 3, 0, 0, 2};
  assert(equal(sarr.begin(), sarr.end(), sarr_expected));
  assert(equal(lcp.begin(), lcp.end(), lcp_expected));
  assert(sa.find("ana") == 1);
  assert(sa.find("x") == string::npos);
  return 0;
}
\end{lstlisting}

\section{String Data Structures}
\setcounter{section}{6}
\setcounter{subsection}{0}
\subsection{Trie}
\begin{lstlisting}
/*

Maintain a map of strings to values using an ordered tree data structure. Each
node corresponds to a character, and each inserted string corresponds to a path
from the root to a node that is flagged as a terminal node.

- trie() constructs an empty map.
- size() returns the size of the map.
- empty() returns whether the map is empty.
- insert(s, v) adds an entry with string key s and value v to the map, returning
  true if a new entry was added or false if the string already exists (in which
  case the map is unchanged and the old value associated with the string key is
  preserved).
- erase(s) removes the entry with string key s from the map, returning true if
  the removal was successful or false if the string to be removed was not found.
- find(s) returns a pointer to a const value associated with string key s, or
  NULL if the key was not found.
- walk(f) calls the function f(s, v) on each entry of the map, in
  lexicographically ascending order of the string keys.

Time Complexity:
- O(n) per call to insert(s, v), erase(s), and find(s), where n is the length of
  s. Note that there is a hidden factor of log(alphabet_size) which can be
  considered constant, since char can only take on 2^CHAR_BIT values. The
  implementation may be optimized by storing the children of nodes in an
  std::unordered_map in C++11 and later, or an array if a smaller alphabet size
  is guaranteed.
- O(l) per call to walk(), where l is the total length of string keys that are
  currently in the map.
- O(1) per call to all other operations.

Space Complexity:
- O(l) for storage of the trie, where l is the total length of string keys that
  are currently in the map.
- O(n) auxiliary stack space for construction, destruction, walk(), where n is
  the maximum length of any string that has been inserted so far.
- O(n) auxiliary stack space for erase(s), where n is the length of s.
- O(1) auxiliary for all other operations.

*/

#include <cstddef>
#include <map>
#include <string>
#include <utility>
using std::string;

template<class V>
class trie {
  struct node_t {
    V value;
    bool is_terminal;
    std::map<char, node_t*> children;

    node_t() : is_terminal(false) {}
  } *root;

  typedef typename std::map<char, node_t*>::iterator cit;

  static bool erase(node_t *n, const string &s, int i) {
    if (i == (int)s.size()) {
      if (!n->is_terminal) {
        return false;
      }
      n->is_terminal = false;
      return true;
    }
    cit it = n->children.find(s[i]);
    if (it == n->children.end() || !erase(it->second, s, i + 1)) {
      return false;
    }
    if (it->second->children.empty()) {
      delete it->second;
      n->children.erase(it);
    }
    return true;
  }

  template<class KVFunction>
  static void walk(node_t *n, string &s, KVFunction f) {
    if (n->is_terminal) {
      f(s, n->value);
    }
    for (cit it = n->children.begin(); it != n->children.end(); ++it) {
      s += it->first;
      walk(it->second, s, f);
      s.pop_back();
    }
  }

  static void clean_up(node_t *n) {
    for (cit it = n->children.begin(); it != n->children.end(); ++it) {
      clean_up(it->second);
    }
    delete n;
  }

  int num_terminals;

 public:
  trie() : root(new node_t()), num_terminals(0) {}

  ~trie() {
    clean_up(root);
  }

  int size() const {
    return num_terminals;
  }

  bool empty() const {
    return num_terminals == 0;
  }

  bool insert(const string &s, const V &v) {
    node_t *n = root;
    for (int i = 0; i < (int)s.size(); i++) {
      cit it = n->children.find(s[i]);
      if (it == n->children.end()) {
        n->children[s[i]] = new node_t();
      }
      n = n->children[s[i]];
    }
    if (n->is_terminal) {
      return false;
    }
    num_terminals++;
    n->is_terminal = true;
    n->value = v;
    return true;
  }

  bool erase(const string &s) {
    if (erase(root, s, 0)) {
      num_terminals--;
      return true;
    }
    return false;
  }

  const V* find(const string &s) const {
    node_t *n = root;
    for (int i = 0; i < (int)s.size(); i++) {
      cit it = n->children.find(s[i]);
      if (it == n->children.end()) {
        return NULL;
      }
      n = it->second;
    }
    return n->is_terminal ? &(n->value) : NULL;
  }

  template<class KVFunction>
  void walk(KVFunction f) const {
    string s = "";
    walk(root, s, f);
  }
};

/*** Example Usage and Output:

("", 0)
("a", 1)
("i", 6)
("in", 7)
("inn", 8)
("tea", 3)
("ted", 4)
("ten", 5)
("to", 2)

***/

#include <cassert>
#include <iostream>
using namespace std;

void print_entry(const string &k, int v) {
  cout << "(\"" << k << "\", " << v << ")" << endl;
}

int main() {
  string s[9] = {"", "a", "to", "tea", "ted", "ten", "i", "in", "inn"};
  trie<int> t;
  assert(t.empty());
  for (int i = 0; i < 9; i++) {
    assert(t.insert(s[i], i));
  }
  t.walk(print_entry);
  assert(!t.empty());
  assert(t.size() == 9);
  assert(!t.insert(s[0], 2));
  assert(t.size() == 9);
  assert(*t.find("") == 0);
  assert(*t.find("ten") == 5);
  assert(t.erase("tea"));
  assert(t.size() == 8);
  assert(t.find("tea") == NULL);
  assert(t.erase(""));
  assert(t.find("") == NULL);
  return 0;
}
\end{lstlisting}
\subsection{Radix Tree}
\begin{lstlisting}
/*

Maintain a map of strings to values using an ordered tree data structure. Each
node corresponds to a substring of an inserted string, and each inserted string
corresponds to a path from the root to a node that is flagged as a terminal
node. Contrary to a regular trie, a radix tree is more space efficient as it
combines chains of nodes with only a single child.

- radix_tree() constructs an empty map.
- size() returns the size of the map.
- empty() returns whether the map is empty.
- insert(s, v) adds an entry with string key s and value v to the map, returning
  true if a new entry was added or false if the string already exists (in which
  case the map is unchanged and the old value associated with the string key is
  preserved).
- erase(s) removes the entry with string key s from the map, returning true if
  the removal was successful or false if the string to be removed was not found.
- find(s) returns a pointer to a const value associated with string key s, or
  NULL if the key was not found.
- walk(f) calls the function f(s, v) on each entry of the map, in
  lexicographically ascending order of the string keys.

Time Complexity:
- O(n) per call to insert(s, v), erase(s), and find(s), where n is the length of
  s. Note that there is a hidden factor of log(n) due to map lookups, which can
  be considered constant amortized. The implementation may be optimized by
  storing the children of nodes in an std::unordered_map in C++11 and later, or
  an std::vector< pair<string, node_t*> >, since the only container operations
  required are iteration over the (key, child) pairs and inserting a new pair.
  Sticking with an (ordered) std::map, we can optimize all operations by using
  map.lower_bound(), a binary tree search for a child with a shared prefix,
  instead of iteration.
- O(l) per call to walk(), where l is the total length of string keys that are
  currently in the map.
- O(1) per call to all other operations.

Space Complexity:
- O(l) for storage of the radix tree, where l is the total length of string keys
  that are currently in the map.
- O(n) auxiliary stack space for construction, destruction, walk(), where n is
  the maximum length of any string that has been inserted so far.
- O(n) auxiliary stack space for erase(s), where n is the length of s.
- O(1) auxiliary for all other operations.

*/

#include <cstddef>
#include <map>
#include <string>
#include <utility>
using std::string;

template<class V>
class radix_tree {
  struct node_t {
    V value;
    bool is_terminal;
    std::map<string, node_t*> children;

    node_t(const V &value = V(), bool is_terminal = false)
        : value(value), is_terminal(is_terminal) {}
  } *root;

  typedef typename std::map<string, node_t*>::iterator cit;

  static int lcp_len(const string &s1, const string &s2, int s2start) {
    int i = 0;
    for (int j = s2start; i < (int)s1.size() && j < (int)s2.size(); i++, j++) {
      if (s1[i] != s2[j]) {
        break;
      }
    }
    return i;
  }

  static bool insert(node_t *n, const string &s, int i, const V &v) {
    if (i == (int)s.size()) {
      if (n->is_terminal) {
        return false;
      }
      n->is_terminal = true;
      return true;
    }
    for (cit it = n->children.begin(); it != n->children.end(); ++it) {
      int len = lcp_len(it->first, s, i);
      if (len == 0) {
        continue;
      }
      if (len == (int)it->first.size()) {
        return insert(it->second, s, i + len, v);
      }
      string left = it->first.substr(0, len);
      string right = it->first.substr(len);
      node_t *tmp = new node_t();
      tmp->children[right] = it->second;
      n->children.erase(it);
      n->children[left] = tmp;
      if (len == (int)s.size() - i) {
        tmp->value = v;
        tmp->is_terminal = true;
        return true;
      }
      return insert(tmp, s, i + len, v);
    }
    n->children[s.substr(i)] = new node_t(v, true);
    return true;
  }

  static bool erase(node_t *n, const string &s, int i) {
    if (i == (int)s.size()) {
      if (!n->is_terminal) {
        return false;
      }
      n->is_terminal = false;
      return true;
    }
    for (cit it = n->children.begin(); it != n->children.end(); ++it) {
      int len = lcp_len(it->first, s, i);
      if (len == 0) {
        continue;
      }
      node_t *child = it->second;
      if (!erase(child, s, i + len)) {
        return false;
      }
      if (child->children.empty()) {
        delete child;
        n->children.erase(it);
      } else if (child->children.size() == 1) {
        node_t *grandchild = child->children.begin()->second;
        if (!child->is_terminal) {
          string merged_key(it->first + child->children.begin()->first);
          child->value = grandchild->value;
          child->is_terminal = grandchild->is_terminal;
          child->children = grandchild->children;
          delete grandchild;
          n->children.erase(it);
          n->children[merged_key] = child;
        }
      }
      return true;
    }
    return false;
  }

  template<class KVFunction>
  static void walk(node_t *n, string &s, KVFunction f) {
    if (n->is_terminal) {
      f(s, n->value);
    }
    for (cit it = n->children.begin(); it != n->children.end(); ++it) {
      s += it->first;
      walk(it->second, s, f);
      s.pop_back();
    }
  }

  static void clean_up(node_t *n) {
    for (cit it = n->children.begin(); it != n->children.end(); ++it) {
      clean_up(it->second);
    }
    delete n;
  }

  int num_terminals;

 public:
  radix_tree() : root(new node_t()), num_terminals(0) {}

  ~radix_tree() {
    clean_up(root);
  }

  int size() const {
    return num_terminals;
  }

  bool empty() const {
    return num_terminals == 0;
  }

  bool insert(const string &s, const V &v) {
    if (insert(root, s, 0, v)) {
      num_terminals++;
      return true;
    }
    return false;
  }

  bool erase(const string &s) {
    if (erase(root, s, 0)) {
      num_terminals--;
      return true;
    }
    return false;
  }

  const V* find(const string &s) const {
    node_t *n = root;
    int i = 0;
    while (i < (int)s.size()) {
      bool found = false;;
      for (cit it = n->children.begin(); it != n->children.end(); ++it) {
        if (it->first[0] == s[i]) {
          i += lcp_len(it->first, s, i);
          n = it->second;
          found = true;
          break;
        }
      }
      if (!found) {
        return NULL;
      }
    }
    return n->is_terminal ? &(n->value) : NULL;
  }

  template<class KVFunction>
  void walk(KVFunction f) const {
    string s = "";
    walk(root, s, f);
  }
};

/*** Example Usage and Output:

("", 0)
("a", 1)
("i", 6)
("in", 7)
("inn", 8)
("tea", 3)
("ted", 4)
("ten", 5)
("to", 2)

***/

#include <cassert>
#include <iostream>
using namespace std;

void print_entry(const string &k, int v) {
  cout << "(\"" << k << "\", " << v << ")" << endl;
}

int main() {
  string s[9] = {"", "a", "to", "tea", "ted", "ten", "i", "in", "inn"};
  radix_tree<int> t;
  assert(t.empty());
  for (int i = 0; i < 9; i++) {
    assert(t.insert(s[i], i));
  }
  t.walk(print_entry);
  assert(!t.empty());
  assert(t.size() == 9);
  assert(!t.insert(s[0], 2));
  assert(t.size() == 9);
  assert(t.find("") && *t.find("") == 0);
  assert(*t.find("ten") == 5);
  assert(t.erase("tea"));
  assert(t.size() == 8);
  assert(t.find("tea") == NULL);
  assert(t.erase(""));
  assert(t.find("") == NULL);
  return 0;
}
\end{lstlisting}

\chapter{Graphs}

\section{Depth-First Search}
\setcounter{section}{1}
\setcounter{subsection}{0}
\subsection{Graph Class and Depth-First Search}
\begin{lstlisting}
/*

A graph consists of a set of objects (a.k.a vertices, or nodes) and a set of
connections (a.k.a. edges) between pairs of said objects. A graph may be stored
as an adjacency list, which is a space efficient representation that is also
time-efficient for traversals.

The following class implements a simple graph using adjacency lists, along with
depth-first search and a few other applications. The constructor takes a Boolean
argument which specifies whether the instance is a directed or undirected graph.
The nodes of the graph are identified by integers indices numbered consecutively
starting from 0. The total number of nodes will automatically increase based on
the maximum node index passed to add_edge() so far.

Time Complexity:
- O(1) amortized per call to add_edge(), or O(max(n, m)) for n calls where the
  maximum node index passed as an argument is m.
- O(max(n, m)) per call for dfs(), has_cycle(), is_tree(), or is_dag(), where n
  is the number of nodes and and m is the number of edges.
- O(1) per call to all other public member functions.

Space Complexity:
- O(max(n, m)) for storage of the graph, where n is the number of nodes and m
  is the number of edges.
- O(n) auxiliary stack space for dfs(), has_cycle(), is_tree(), and is_dag().
- O(1) auxiliary for all other public member functions.

*/

#include <algorithm>
#include <vector>

class graph {
  std::vector<std::vector<int> > adj;
  bool directed;

  template<class ReportFunction>
  void dfs(int n, std::vector<bool> &visit, ReportFunction f) const {
    f(n);
    visit[n] = true;
    std::vector<int>::const_iterator it;
    for (it = adj[n].begin(); it != adj[n].end(); ++it) {
      if (!visit[*it]) {
        dfs(*it, visit, f);
      }
    }
  }

  bool has_cycle(int n, int prev, std::vector<bool> &visit,
                 std::vector<bool> &onstack) const {
    visit[n] = true;
    onstack[n] = true;
    std::vector<int>::const_iterator it;
    for (it = adj[n].begin(); it != adj[n].end(); ++it) {
      if (directed && onstack[*it]) {
        return true;
      }
      if (!directed && visit[*it] && *it != prev) {
        return true;
      }
      if (!visit[*it] && has_cycle(*it, n, visit, onstack)) {
        return true;
      }
    }
    onstack[n] = false;
    return false;
  }

 public:
  graph(bool directed = true) : directed(directed) {}

  int nodes() const {
    return (int)adj.size();
  }

  std::vector<int>& operator[](int n) {
    return adj[n];
  }

  void add_edge(int u, int v) {
    int n = adj.size();
    if (u >= n || v >= n) {
      adj.resize(std::max(u, v) + 1);
    }
    adj[u].push_back(v);
    if (!directed) {
      adj[v].push_back(u);
    }
  }

  bool is_directed() const {
    return directed;
  }

  bool has_cycle() const {
    int n = adj.size();
    std::vector<bool> visit(n, false), onstack(n, false);
    for (int i = 0; i < n; i++) {
      if (!visit[i] && has_cycle(i, -1, visit, onstack)) {
        return true;
      }
    }
    return false;
  }

  bool is_tree() const {
    return !directed && !has_cycle();
  }

  bool is_dag() const {
    return directed && !has_cycle();
  }

  template<class ReportFunction>
  void dfs(int start, ReportFunction f) const {
    std::vector<bool> visit(adj.size(), false);
    dfs(start, visit, f);
  }
};

/*** Example Usage and Output:

DFS order: 0 1 2 3 4 5 6 7 8 9 10 11

***/

#include <cassert>
#include <iostream>
using namespace std;

void print(int n) {
  cout << n << " ";
}

int main() {
  {
    graph g;
    g.add_edge(0, 1);
    g.add_edge(0, 6);
    g.add_edge(0, 7);
    g.add_edge(1, 2);
    g.add_edge(1, 5);
    g.add_edge(2, 3);
    g.add_edge(2, 4);
    g.add_edge(7, 8);
    g.add_edge(7, 11);
    g.add_edge(8, 9);
    g.add_edge(8, 10);
    cout << "DFS order: ";
    g.dfs(0, print);
    cout << endl;
    assert(g[0].size() == 3);
    assert(g.is_dag());
    assert(!g.has_cycle());
  }
  {
    graph tree(false);
    tree.add_edge(0, 1);
    tree.add_edge(0, 2);
    tree.add_edge(1, 3);
    tree.add_edge(1, 4);
    assert(tree.is_tree());
    assert(!tree.is_dag());
    tree.add_edge(2, 3);
    assert(!tree.is_tree());
  }
  return 0;
}
\end{lstlisting}
\subsection{Topological Sorting (DFS)}
\begin{lstlisting}
/*

Given a directed acyclic graph, find one of possibly many orderings of the nodes
such that for every edge from node u to v, u comes before v in the ordering.
Depth-first search is used to traverse all nodes in post-order.

toposort(nodes) takes a directed graph stored as a global adjacency list with
nodes indexed from 0 to (nodes - 1) and assigns a valid topological ordering to
the global result vector. An error is thrown if the graph contains a cycle.

Time Complexity:
- O(max(n, m)) per call to toposort(), where n is the number of nodes and m is
  the number of edges.

Space Complexity:
- O(max(n, m)) for storage of the graph, where n is the number of nodes and m
  is the number of edges.
- O(n) auxiliary stack space for toposort().

*/

#include <algorithm>
#include <stdexcept>
#include <vector>

const int MAXN = 100;
std::vector<int> adj[MAXN], res;
std::vector<bool> visit(MAXN), done(MAXN);

void dfs(int u) {
  if (visit[u]) {
    throw std::runtime_error("Not a directed acyclic graph.");
  }
  if (done[u]) {
    return;
  }
  visit[u] = true;
  for (int j = 0; j < (int)adj[u].size(); j++) {
    dfs(adj[u][j]);
  }
  visit[u] = false;
  done[u] = true;
  res.push_back(u);
}

void toposort(int nodes) {
  fill(visit.begin(), visit.end(), false);
  fill(done.begin(), done.end(), false);
  res.clear();
  for (int i = 0; i < nodes; i++) {
    if (!done[i]) {
      dfs(i);
    }
  }
  std::reverse(res.begin(), res.end());
}

/*** Example Usage and Output:

The topological order: 2 1 0 4 3 7 6 5

***/

#include <iostream>
using namespace std;

int main() {
  adj[0].push_back(3);
  adj[0].push_back(4);
  adj[1].push_back(3);
  adj[2].push_back(4);
  adj[2].push_back(7);
  adj[3].push_back(5);
  adj[3].push_back(6);
  adj[3].push_back(7);
  adj[4].push_back(6);
  toposort(8);
  cout << "The topological order:";
  for (int i = 0; i < (int)res.size(); i++) {
    cout << " " << res[i];
  }
  cout << endl;
  return 0;
}
\end{lstlisting}
\subsection{Eulerian Cycles (DFS)}
\begin{lstlisting}
/*

A Eulerian trail is a path in a graph which contains every edge exactly once. An
Eulerian cycle or circuit is an Eulerian trail which begins and ends on the same
node. A directed graph has an Eulerian cycle if and only if every node has an
in-degree equal to its out-degree, and all of its nodes with nonzero degree
belong to a single strongly connected component. An undirected graph has an
Eulerian cycle if and only if every node has even degree, and all of its nodes
with nonzero degree belong to a single connected component.

Given a graph as an adjacency list along with the starting node of the cycle,
both functions below return a vector containing all nodes reachable from the
starting node in an order which forms an Eulerian cycle. The first node of the
cycle will be repeated as the last element of the vector. All nodes of input
adjacency lists to both functions must be be between 0 and MAXN - 1, inclusive.
In addition, euler_cycle_undirected() requires that for every node v which is
found in adj[u], node u must also be found in adj[v].

Time Complexity:
- O(max(n, m)) per call to either function, where n and m are the numbers of
  nodes and edges respectively.

Space Complexity:
- O(n) auxiliary heap space for euler_cycle_directed(), where n is the number of
  nodes.
- O(n^2) auxiliary heap space for euler_cycle_undirected(), where n is the
  number of nodes. This can be reduced to O(m) auxiliary heap space on the
  number of edges if the used[][] bit matrix is replaced with an
  std::unordered_set<std::pair<int, int>>.

*/

#include <algorithm>
#include <bitset>
#include <vector>

const int MAXN = 100;

std::vector<int> euler_cycle_directed(std::vector<int> adj[], int u) {
  std::vector<int> stack, curr_edge(MAXN), res;
  stack.push_back(u);
  while (!stack.empty()) {
    u = stack.back();
    stack.pop_back();
    while (curr_edge[u] < (int)adj[u].size()) {
      stack.push_back(u);
      u = adj[u][curr_edge[u]++];
    }
    res.push_back(u);
  }
  std::reverse(res.begin(), res.end());
  return res;
}

std::vector<int> euler_cycle_undirected(std::vector<int> adj[], int u) {
  std::bitset<MAXN> used[MAXN];
  std::vector<int> stack, curr_edge(MAXN), res;
  stack.push_back(u);
  while (!stack.empty()) {
    u = stack.back();
    stack.pop_back();
    while (curr_edge[u] < (int)adj[u].size()) {
      int v = adj[u][curr_edge[u]++];
      int mn = std::min(u, v), mx = std::max(u, v);
      if (!used[mn][mx]) {
        used[mn][mx] = true;
        stack.push_back(u);
        u = v;
      }
    }
    res.push_back(u);
  }
  std::reverse(res.begin(), res.end());
  return res;
}

/*** Example Usage and Output:

Eulerian cycle from 0 (directed): 0 1 3 4 1 2 0
Eulerian cycle from 2 (undirected): 2 1 3 4 1 0 2

***/

#include <iostream>
using namespace std;

int main() {
  {
    vector<int> g[5], cycle;
    g[0].push_back(1);
    g[1].push_back(2);
    g[2].push_back(0);
    g[1].push_back(3);
    g[3].push_back(4);
    g[4].push_back(1);
    cycle = euler_cycle_directed(g, 0);
    cout << "Eulerian cycle from 0 (directed):";
    for (int i = 0; i < (int)cycle.size(); i++) {
      cout << " " << cycle[i];
    }
    cout << endl;
  }
  {
    vector<int> g[5], cycle;
    g[0].push_back(1);
    g[1].push_back(0);
    g[1].push_back(2);
    g[2].push_back(1);
    g[2].push_back(0);
    g[0].push_back(2);
    g[1].push_back(3);
    g[3].push_back(1);
    g[3].push_back(4);
    g[4].push_back(3);
    g[4].push_back(1);
    g[1].push_back(4);
    cycle = euler_cycle_undirected(g, 2);
    cout << "Eulerian cycle from 2 (undirected):";
    for (int i = 0; i < (int)cycle.size(); i++) {
      cout << " " << cycle[i];
    }
    cout << endl;
  }
  return 0;
}
\end{lstlisting}
\subsection{Unweighted Tree Centers (DFS)}
\begin{lstlisting}
/*

An unweighted tree possesses a center, centroid, and diameter. The following
functions apply to a global, pre-populated adjacency list adj[] which satisfies
the precondition that for every node v in adj[u], node u also exists in adj[v].
Nodes in adj[] must be numbered with integers between 0 (inclusive) and the
total number of nodes (exclusive), as passed in the function arguments.

- find_centers() returns a vector of either one or two tree Jordan centers. The
  Jordan center of a tree is the set of all nodes with minimum eccentricity,
  that is, the set of all nodes where the maximum distance to all other nodes in
  the tree is minimal.
- find_centroid() returns the node where all of its subtrees have a size less
  than or equal to n/2, where n is the number of nodes in the tree.
- diameter() returns the maximum distance between any two nodes in the tree,
  using a well-known double depth-first search technique.

Time Complexity:
- O(max(n, m)) per call to find_centers(), find_centroid(), and diameter(),
  where n is the number of nodes and m is the number of edges.

Space Complexity:
- O(n) auxiliary stack space for find_centers(), find_centroid(), and
  diameter(), where n is the number of nodes.

*/

#include <utility>
#include <vector>

const int MAXN = 100;
std::vector<int> adj[MAXN];

std::vector<int> find_centers(int nodes) {
  std::vector<int> leaves, degree(nodes);
  for (int i = 0; i < nodes; i++) {
    degree[i] = adj[i].size();
    if (degree[i] <= 1) {
      leaves.push_back(i);
    }
  }
  int removed = leaves.size();
  while (removed < nodes) {
    std::vector<int> nleaves;
    for (int i = 0; i < (int)leaves.size(); i++) {
      int u = leaves[i];
      for (int j = 0; j < (int)adj[u].size(); j++) {
        int v = adj[u][j];
        if (--degree[v] == 1) {
          nleaves.push_back(v);
        }
      }
    }
    leaves = nleaves;
    removed += leaves.size();
  }
  return leaves;
}

int find_centroid(int nodes, int u = 0, int p = -1) {
  int count = 1;
  bool good_center = true;
  for (int j = 0; j < (int)adj[u].size(); j++) {
    int v = adj[u][j];
    if (v == p) {
      continue;
    }
    int res = find_centroid(nodes, v, u);
    if (res >= 0) {
      return res;
    }
    int size = -res;
    good_center &= (size <= nodes / 2);
    count += size;
  }
  good_center &= (nodes - count <= nodes / 2);
  return good_center ? u : -count;
}

std::pair<int, int> dfs(int u, int p, int depth) {
  std::pair<int, int> res = std::make_pair(depth, u);
  for (int j = 0; j < (int)adj[u].size(); j++) {
    if (adj[u][j] != p) {
      res = max(res, dfs(adj[u][j], u, depth + 1));
    }
  }
  return res;
}

int diameter() {
  int furthest_node = dfs(0, -1, 0).second;
  return dfs(furthest_node, -1, 0).first;
}

/*** Example Usage ***/

#include <cassert>
using namespace std;

int main() {
  int nodes = 6;
  adj[0].push_back(1);
  adj[1].push_back(0);
  adj[1].push_back(2);
  adj[2].push_back(1);
  adj[1].push_back(4);
  adj[4].push_back(1);
  adj[3].push_back(4);
  adj[4].push_back(3);
  adj[4].push_back(5);
  adj[5].push_back(4);
  vector<int> centers = find_centers(nodes);
  assert(centers.size() == 2 && centers[0] == 1 && centers[1] == 4);
  assert(find_centroid(nodes) == 4);
  assert(diameter() == 3);
  return 0;
}
\end{lstlisting}

\section{Shortest Path}
\setcounter{section}{2}
\setcounter{subsection}{0}
\subsection{Shortest Path (BFS)}
\begin{lstlisting}
/*

Given a starting node in an unweighted, directed graph, visit every connected
node and determine the minimum distance to each such node. Optionally, output
the shortest path to a specific destination node using the shortest-path tree
from the predecessor array pred[]. bfs() applies to a global, pre-populated
adjacency list adj[] which consists of only nodes numbered with integers between
0 (inclusive) and the total number of nodes (exclusive), as passed in the
function argument.

Time Complexity:
- O(n) per call to bfs(), where n is the number of nodes.

Space Complexity:
- O(max(n, m)) for storage of the graph, where n is the number of nodes and m
  is the number of edges.
- O(n) auxiliary heap space for bfs().

*/

#include <queue>
#include <utility>
#include <vector>

const int MAXN = 100, INF = 0x3f3f3f3f;
std::vector<int> adj[MAXN];
int dist[MAXN], pred[MAXN];

void bfs(int nodes, int start) {
  std::vector<bool> visit(nodes, false);
  for (int i = 0; i < nodes; i++) {
    dist[i] = INF;
    pred[i] = -1;
  }
  std::queue<std::pair<int, int> > q;
  q.push(std::make_pair(start, 0));
  while (!q.empty()) {
    int u = q.front().first;
    int d = q.front().second;
    q.pop();
    visit[u] = true;
    for (int j = 0; j < (int)adj[u].size(); j++) {
      int v = adj[u][j];
      if (visit[v]) {
        continue;
      }
      dist[v] = d + 1;
      pred[v] = u;
      q.push(std::make_pair(v, d + 1));
    }
  }
}

/*** Example Usage and Output:

The shortest distance from 0 to 3 is 2.
Take the path: 0->1->3.

***/

#include <iostream>
using namespace std;

void print_path(int dest) {
  vector<int> path;
  for (int j = dest; pred[j] != -1; j = pred[j]) {
    path.push_back(pred[j]);
  }
  cout << "Take the path: ";
  while (!path.empty()) {
    cout << path.back() << "->";
    path.pop_back();
  }
  cout << dest << "." << endl;
}

int main() {
  int start = 0, dest = 3;
  adj[0].push_back(1);
  adj[0].push_back(3);
  adj[1].push_back(2);
  adj[1].push_back(3);
  adj[2].push_back(3);
  adj[0].push_back(3);
  bfs(4, start);
  cout << "The shortest distance from " << start << " to " << dest << " is "
       << dist[dest] << "." << endl;
  print_path(dest);
  return 0;
}
\end{lstlisting}
\subsection{Shortest Path (Dijkstra)}
\begin{lstlisting}
/*

Given a starting node in a weighted, directed graph with nonnegative weights
only, visit every connected node and determine the minimum distance to each such
node. Optionally, output the shortest path to a specific destination node using
the shortest-path tree from the predecessor array pred[]. dijkstra() applies to
a global, pre-populated adjacency list adj[] which must only consist of nodes
numbered with integers between 0 (inclusive) and the total number of nodes
(exclusive), as passed in the function argument.

Since std::priority_queue is by default a max-heap, we simulate a min-heap by
negating node distances before pushing them and negating them again after
popping them. Alternatively, the container can be declared with the following
template arguments (#include <functional> to access std::greater):
  priority_queue<pair<int, int>, vector<pair<int, int> >,
                 greater<pair<int, int> > > pq;

Dijkstra's algorithm may be modified to support negative edge weights by
allowing nodes to be re-visited (removing the visited array check in the inner
for-loop). This is known as the Shortest Path Faster Algorithm (SPFA), which has
a larger running time of O(n*m) on the number of nodes and edges respectively.
While it is as slow in the worst case as the Bellman-Ford algorithm, the SPFA
still tends to outperform in the average case.

Time Complexity:
- O(m log n) for dijkstra(), where m is the number of edges and n is the number
  of nodes.

Space Complexity:
- O(max(n, m)) for storage of the graph, where n is the number of nodes and m
  is the number of edges.
- O(n) auxiliary heap space for dijkstra().

*/

#include <queue>
#include <utility>
#include <vector>

const int MAXN = 100, INF = 0x3f3f3f3f;
std::vector<std::pair<int, int> > adj[MAXN];
int dist[MAXN], pred[MAXN];

void dijkstra(int nodes, int start) {
  std::vector<bool> visit(nodes, false);
  for (int i = 0; i < nodes; i++) {
    dist[i] = INF;
    pred[i] = -1;
  }
  dist[start] = 0;
  std::priority_queue<std::pair<int, int> > pq;
  pq.push(std::make_pair(0, start));
  while (!pq.empty()) {
    int u = pq.top().second;
    pq.pop();
    visit[u] = true;
    for (int j = 0; j < (int)adj[u].size(); j++) {
      int v = adj[u][j].first;
      if (visit[v]) {
        continue;
      }
      if (dist[v] > dist[u] + adj[u][j].second) {
        dist[v] = dist[u] + adj[u][j].second;
        pred[v] = u;
        pq.push(std::make_pair(-dist[v], v));
      }
    }
  }
}

/*** Example Usage and Output:

The shortest distance from 0 to 3 is 5.
Take the path: 0->1->2->3.

***/

#include <iostream>
using namespace std;

void print_path(int dest) {
  vector<int> path;
  for (int j = dest; pred[j] != -1; j = pred[j]) {
    path.push_back(pred[j]);
  }
  cout << "Take the path: ";
  while (!path.empty()) {
    cout << path.back() << "->";
    path.pop_back();
  }
  cout << dest << "." << endl;
}

int main() {
  int start = 0, dest = 3;
  adj[0].push_back(make_pair(1, 2));
  adj[0].push_back(make_pair(3, 8));
  adj[1].push_back(make_pair(2, 2));
  adj[1].push_back(make_pair(3, 4));
  adj[2].push_back(make_pair(3, 1));
  dijkstra(4, start);
  cout << "The shortest distance from " << start << " to " << dest << " is "
       << dist[dest] << "." << endl;
  print_path(dest);
  return 0;
}
\end{lstlisting}
\subsection{Shortest Path (Bellman-Ford)}
\begin{lstlisting}
/*

Given a starting node in a weighted, directed graph with possibly negative
weights, visit every connected node and determine the minimum distance to each
such node. Optionally, output the shortest path to a specific destination node
using the shortest-path tree from the predecessor array pred[]. bellman_ford()
applies to a global, pre-populated edge list which must only consist of nodes
numbered with integers between 0 (inclusive) and the total number of nodes
(exclusive), as passed in the function argument.

This function will also detect whether the graph contains negative-weighted
cycles, in which case there is no shortest path and an error will be thrown.

Time Complexity:
- O(n*m) per call to bellman_ford(), where n is the number of nodes and m is the
  number of edges.

Space Complexity:
- O(max(n, m)) for storage of the graph, where n is the number of nodes and m is
  the number of edges.
- O(n) auxiliary heap space for bellman_ford(), where n is the number of nodes.

*/

#include <stdexcept>
#include <vector>

struct edge { int u, v, w; };  // Edge from u to v with weight w.

const int MAXN = 100, INF = 0x3f3f3f3f;
std::vector<edge> e;
int dist[MAXN], pred[MAXN];

void bellman_ford(int nodes, int start) {
  for (int i = 0; i < nodes; i++) {
    dist[i] = INF;
    pred[i] = -1;
  }
  dist[start] = 0;
  for (int i = 0; i < nodes; i++) {
    for (int j = 0; j < (int)e.size(); j++) {
      if (dist[e[j].v] > dist[e[j].u] + e[j].w) {
        dist[e[j].v] = dist[e[j].u] + e[j].w;
        pred[e[j].v] = e[j].u;
      }
    }
  }
  // Optional: Report negative-weighted cycles.
  for (int i = 0; i < (int)e.size(); i++) {
    if (dist[e[i].v] > dist[e[i].u] + e[i].w) {
      throw std::runtime_error("Negative-weight cycle found.");
    }
  }
}

/*** Example Usage and Output:

The shortest distance from 0 to 2 is 3.
Take the path: 0->1->2.

***/

#include <iostream>
using namespace std;

void print_path(int dest) {
  vector<int> path;
  for (int j = dest; pred[j] != -1; j = pred[j]) {
    path.push_back(pred[j]);
  }
  cout << "Take the path: ";
  while (!path.empty()) {
    cout << path.back() << "->";
    path.pop_back();
  }
  cout << dest << "." << endl;
}

int main() {
  int start = 0, dest = 2;
  e.push_back((edge){0, 1, 1});
  e.push_back((edge){1, 2, 2});
  e.push_back((edge){0, 2, 5});
  bellman_ford(3, start);
  cout << "The shortest distance from " << start << " to " << dest << " is "
       << dist[dest] << "." << endl;
  print_path(dest);
  return 0;
}
\end{lstlisting}
\subsection{Shortest Path (Floyd-Warshall)}
\begin{lstlisting}
/*

Given a weighted, directed graph with possibly negative weights, determine the
minimum distance between all pairs of start and destination nodes in the graph.
Optionally, output the shortest path between two nodes using the shortest-path
tree precomputed into the parent[][] array. floyd_warshall() applies to a global
adjacency matrix dist[][], which must be initialized using initialize() and
subsequently populated with weights. After the function call, dist[u][v] will
have been modified to contain the shortest path from u to v, for all pairs of
valid nodes u and v.

This function will also detect whether the graph contains negative-weighted
cycles, in which case there is no shortest path and an error will be thrown.

Time Complexity:
- O(n^2) per call to initialize(), where n is the number of nodes.
- O(n^3) per call to floyd_warshall().

Space Complexity:
- O(n^2) for storage of the graph, where n is the number of nodes.
- O(n^2) auxiliary heap space for initialize() and floyd_warshall().

*/

#include <stdexcept>

const int MAXN = 100, INF = 0x3f3f3f3f;
int dist[MAXN][MAXN], parent[MAXN][MAXN];

void initialize(int nodes) {
  for (int i = 0; i < nodes; i++) {
    for (int j = 0; j < nodes; j++) {
      dist[i][j] = (i == j) ? 0 : INF;
      parent[i][j] = j;
    }
  }
}

void floyd_warshall(int nodes) {
  for (int k = 0; k < nodes; k++) {
    for (int i = 0; i < nodes; i++) {
      for (int j = 0; j < nodes; j++) {
        if (dist[i][j] > dist[i][k] + dist[k][j]) {
          dist[i][j] = dist[i][k] + dist[k][j];
          parent[i][j] = parent[i][k];
        }
      }
    }
  }
  // Optional: Report negative-weighted cycles.
  for (int i = 0; i < nodes; i++) {
    if (dist[i][i] < 0) {
      throw std::runtime_error("Negative-weight cycle found.");
    }
  }
}

/*** Example Usage and Output:

The shortest distance from 0 to 2 is 3.
Take the path: 0->1->2.

***/

#include <iostream>
using namespace std;

void print_path(int u, int v) {
  cout << "Take the path " << u;
  while (u != v) {
    u = parent[u][v];
    cout << "->" << u;
  }
  cout << "." << endl;
}

int main() {
  initialize(3);
  int start = 0, dest = 2;
  dist[0][1] = 1;
  dist[1][2] = 2;
  dist[0][2] = 5;
  floyd_warshall(3);
  cout << "The shortest distance from " << start << " to " << dest << " is "
       << dist[start][dest] << "." << endl;
  print_path(start, dest);
  return 0;
}
\end{lstlisting}

\section{Connectivity}
\setcounter{section}{3}
\setcounter{subsection}{0}
\subsection{Strongly Connected Components (Kosaraju)}
\begin{lstlisting}
/*

Given a directed graph, determine the strongly connected components, that is,
the set of all strongly (maximally) connected subgraphs. A subgraph is strongly
connected if there is a path between each pair of nodes. Condensing the strongly
connected components of a graph into single nodes will result in a directed
acyclic graph. kosaraju() applies to a global, pre-populated adjacency list
adj[] which must only consist of nodes numbered with integers between 0
(inclusive) and the total number of nodes (exclusive), as passed in the function
argument.

Time Complexity:
- O(max(n, m)) per call to kosaraju(), where n is the number of nodes and m is
  the number of edges.

Space Complexity:
- O(max(n, m)) auxiliary heap space for storage of the graph, where n the number
  of nodes and m is the number of edges.
- O(n) auxiliary stack space for kosaraju().

*/

#include <algorithm>
#include <vector>

const int MAXN = 100;
std::vector<int> adj[MAXN], rev[MAXN];
std::vector<bool> visit(MAXN);
std::vector<std::vector<int> > scc;

void dfs(std::vector<int> g[], std::vector<int> &res, int u) {
  visit[u] = true;
  for (int j = 0; j < (int)g[u].size(); j++) {
    if (!visit[g[u][j]]) {
      dfs(g, res, g[u][j]);
    }
  }
  res.push_back(u);
}

void kosaraju(int nodes) {
  std::fill(visit.begin(), visit.end(), false);
  std::vector<int> order;
  for (int i = 0; i < nodes; i++) {
    rev[i].clear();
    if (!visit[i]) {
      dfs(adj, order, i);
    }
  }
  std::reverse(order.begin(), order.end());
  std::fill(visit.begin(), visit.end(), false);
  for (int i = 0; i < nodes; i++) {
    for (int j = 0; j < (int)adj[i].size(); j++) {
      rev[adj[i][j]].push_back(i);
    }
  }
  scc.clear();
  for (int i = 0; i < (int)order.size(); i++) {
    if (visit[order[i]]) {
      continue;
    }
    std::vector<int> component;
    dfs(rev, component, order[i]);
    scc.push_back(component);
  }
}

/*** Example Usage and Output:

1 4 0
7 3 2
5 6

***/

#include <iostream>
using namespace std;

int main() {
  adj[0].push_back(1);
  adj[1].push_back(2);
  adj[1].push_back(4);
  adj[1].push_back(5);
  adj[2].push_back(3);
  adj[2].push_back(6);
  adj[3].push_back(2);
  adj[3].push_back(7);
  adj[4].push_back(0);
  adj[4].push_back(5);
  adj[5].push_back(6);
  adj[6].push_back(5);
  adj[7].push_back(3);
  adj[7].push_back(6);
  kosaraju(8);
  cout << "Components:" << endl;
  for (int i = 0; i < (int)scc.size(); i++) {
    for (int j = 0; j < (int)scc[i].size(); j++) {
      cout << scc[i][j] << " ";
    }
    cout << endl;
  }
  return 0;
}
\end{lstlisting}
\subsection{Strongly Connected Components (Tarjan)}
\begin{lstlisting}
/*

Given a directed graph, determine the strongly connected components. The
strongly connected components of a graph is the set of all strongly (maximally)
connected subgraphs. A subgraph is strongly connected if there is a path between
each pair of nodes. Condensing the strongly connected components of a graph into
single nodes will result in a directed acyclic graph. tarjan() applies to a
global, pre-populated adjacency list adj[] which must only consist of nodes
numbered with integers between 0 (inclusive) and the total number of nodes
(exclusive), as passed in the function argument.

Time Complexity:
- O(max(n, m)) per call to tarjan(), where n is the number of nodes and m is the
  number of edges.

Space Complexity:
- O(max(n, m)) for storage of the graph, where n the number of nodes and m is
  the number of edges.
- O(n) auxiliary stack space for tarjan().

*/

#include <algorithm>
#include <vector>

const int MAXN = 100, INF = 0x3f3f3f3f;
std::vector<int> adj[MAXN], stack;
int timer, lowlink[MAXN];
std::vector<bool> visit(MAXN);
std::vector<std::vector<int> > scc;

void dfs(int u) {
  lowlink[u] = timer++;
  visit[u] = true;
  stack.push_back(u);
  bool is_component_root = true;
  int v;
  for (int j = 0; j < (int)adj[u].size(); j++) {
    v = adj[u][j];
    if (!visit[v]) {
      dfs(v);
    }
    if (lowlink[u] > lowlink[v]) {
      lowlink[u] = lowlink[v];
      is_component_root = false;
    }
  }
  if (!is_component_root) {
    return;
  }
  std::vector<int> component;
  do {
    v = stack.back();
    visit[v] = true;
    stack.pop_back();
    lowlink[v] = INF;
    component.push_back(v);
  } while (u != v);
  scc.push_back(component);
}

void tarjan(int nodes) {
  scc.clear();
  stack.clear();
  std::fill(lowlink, lowlink + nodes, 0);
  std::fill(visit.begin(), visit.end(), false);
  timer = 0;
  for (int i = 0; i < nodes; i++) {
    if (!visit[i]) {
      dfs(i);
    }
  }
}

/*** Example Usage and Output:

Components:
5 6
7 3 2
4 1 0

***/

#include <iostream>
using namespace std;

int main() {
  adj[0].push_back(1);
  adj[1].push_back(2);
  adj[1].push_back(4);
  adj[1].push_back(5);
  adj[2].push_back(3);
  adj[2].push_back(6);
  adj[3].push_back(2);
  adj[3].push_back(7);
  adj[4].push_back(0);
  adj[4].push_back(5);
  adj[5].push_back(6);
  adj[6].push_back(5);
  adj[7].push_back(3);
  adj[7].push_back(6);
  tarjan(8);
  cout << "Components:" << endl;
  for (int i = 0; i < (int)scc.size(); i++) {
    for (int j = 0; j < (int)scc[i].size(); j++) {
      cout << scc[i][j] << " ";
    }
    cout << endl;
  }
  return 0;
}
\end{lstlisting}
\subsection{Bridges, Cut-points, and Biconnectivity}
\begin{lstlisting}
/*

Given an undirected graph, compute the following properties of the graph using
Tarjan's algorithm. tarjan() applies to a global, pre-populated adjacency list
adj[] which satisfies the precondition that for every node v in adj[u], node u
also exists in adj[v]. Nodes in adj[] must be numbered with integers between 0
(inclusive) and the total number of nodes (exclusive), as passed in the function
arguments. get_block_forest() applies to the global vector of biconnected
components bcc[] which must have already been precomputed by a call to tarjan().

A bridge is an edge such that when deleted, the number of connected components
in the graph is increased. An edge is a bridge if and only if it is not part of
any cycle.

A cut-point (i.e. cut-node, or articulation point) is any node whose removal
increases the number of connected components in the graph.

A biconnected component of a graph is a maximally biconnected subgraph. A
biconnected graph is a connected and "non-separable" graph, meaning that if any
node were to be removed, the graph will remain connected. Thus, a biconnected
graph has no articulation points.

Any connected graph decomposes into a tree of biconnected components called the
"block tree" of the graph. An unconnected graph will thus decompose into a
"block forest."

Time Complexity:
- O(max(n, m)) per call to tarjan() and get_block_forest(), where n is the
  number of nodes and m is the number of edges.

Space Complexity:
- O(max(n, m)) for storage of the graph, where n the number of nodes and m is
  the number of edges
- O(n) auxiliary stack space for tarjan().
- O(1) auxiliary stack space for get_block_forest().

*/

#include <algorithm>
#include <vector>

const int MAXN = 100;
int timer, lowlink[MAXN], tin[MAXN], comp[MAXN];
std::vector<bool> visit(MAXN);
std::vector<int> adj[MAXN], bcc_forest[MAXN];
std::vector<int> stack, cutpoints;
std::vector<std::vector<int> > bcc;
std::vector<std::pair<int, int> > bridges;

void dfs(int u, int p) {
  visit[u] = true;
  lowlink[u] = tin[u] = timer++;
  stack.push_back(u);
  int v, children = 0;
  bool cutpoint = false;
  for (int j = 0; j < (int)adj[u].size(); j++) {
    v = adj[u][j];
    if (v == p) {
      continue;
    }
    if (visit[v]) {
      lowlink[u] = std::min(lowlink[u], tin[v]);
    } else {
      dfs(v, u);
      lowlink[u] = std::min(lowlink[u], lowlink[v]);
      cutpoint |= (lowlink[v] >= tin[u]);
      if (lowlink[v] > tin[u]) {
        bridges.push_back(std::make_pair(u, v));
      }
      children++;
    }
  }
  if (p == -1) {
    cutpoint = (children >= 2);
  }
  if (cutpoint) {
    cutpoints.push_back(u);
  }
  if (lowlink[u] == tin[u]) {
    std::vector<int> component;
    do {
      v = stack.back();
      stack.pop_back();
      component.push_back(v);
    } while (u != v);
    bcc.push_back(component);
  }
}

void tarjan(int nodes) {
  bcc.clear();
  bridges.clear();
  cutpoints.clear();
  stack.clear();
  std::fill(lowlink, lowlink + nodes, 0);
  std::fill(tin, tin + nodes, 0);
  std::fill(visit.begin(), visit.end(), false);
  timer = 0;
  for (int i = 0; i < nodes; i++) {
    if (!visit[i]) {
      dfs(i, -1);
    }
  }
}

void get_block_forest(int nodes) {
  std::fill(comp, comp + nodes, 0);
  for (int i = 0; i < nodes; i++) {
    bcc_forest[i].clear();
  }
  for (int i = 0; i < (int)bcc.size(); i++) {
    for (int j = 0; j < (int)bcc[i].size(); j++) {
      comp[bcc[i][j]] = i;
    }
  }
  for (int i = 0; i < nodes; i++) {
    for (int j = 0; j < (int)adj[i].size(); j++) {
      if (comp[i] != comp[adj[i][j]]) {
        bcc_forest[comp[i]].push_back(comp[adj[i][j]]);
      }
    }
  }
}

/*** Example Usage and Output:

Cut-points: 5 1
Bridges:
1 2
5 4
3 7
Edge-Biconnected Components:
2
4
5 1 0
7
3
6
Adjacency List for Block Forest:
0 => 2
1 => 2
2 => 0 1
3 => 4
4 => 3
5 =>

***/

#include <iostream>
using namespace std;

void add_edge(int u, int v) {
  adj[u].push_back(v);
  adj[v].push_back(u);
}

int main() {
  add_edge(0, 1);
  add_edge(0, 5);
  add_edge(1, 2);
  add_edge(1, 5);
  add_edge(3, 7);
  add_edge(4, 5);
  tarjan(8);
  get_block_forest(8);
  cout << "Cut-points:";
  for (int i = 0; i < (int)cutpoints.size(); i++) {
    cout << " " << cutpoints[i];
  }
  cout << endl << "Bridges:" << endl;
  for (int i = 0; i < (int)bridges.size(); i++) {
    cout << bridges[i].first << " " << bridges[i].second << endl;
  }
  cout << "Edge-Biconnected Components:" << endl;
  for (int i = 0; i < (int)bcc.size(); i++) {
    for (int j = 0; j < (int)bcc[i].size(); j++) {
      cout << bcc[i][j] << " ";
    }
    cout << endl;
  }
  cout << "Adjacency List for Block Forest:" << endl;
  for (int i = 0; i < (int)bcc.size(); i++) {
    cout << i << " =>";
    for (int j = 0; j < (int)bcc_forest[i].size(); j++) {
      cout << " " << bcc_forest[i][j];
    }
    cout << endl;
  }
  return 0;
}
\end{lstlisting}

\section{Minimum Spanning Tree}
\setcounter{section}{4}
\setcounter{subsection}{0}
\subsection{Minimum Spanning Tree (Prim)}
\begin{lstlisting}
/*

Given a connected, undirected, weighted graph with possibly negative weights,
its minimum spanning tree is a subgraph which is a tree that connects all nodes
with a subset of its edges such that their total weight is minimized. prim()
applies to a global, pre-populated adjacency list adj[] which must only consist
of nodes numbered with integers between 0 (inclusive) and the total number of
nodes (exclusive), as passed in the function argument. If the input graph is not
connected, then this implementation will find the minimum spanning forest.

Since std::priority_queue is by default a max-heap, we simulate a min-heap by
negating node distances before pushing them and negating them again after
popping them. To modify this implementation to find the maximum spanning tree,
the two negation steps can be skipped to prioritize the max edges.

Time Complexity:
- O(m log n) per call to prim(), where m is the number of edges and n is the
  number of nodes.

Space Complexity:
- O(max(n, m)) for storage of the graph, where n the number of nodes and m is
  the number of edges.
- O(n) auxiliary heap space for prim().

*/

#include <queue>
#include <utility>
#include <vector>

const int MAXN = 100;
std::vector<std::pair<int, int> > adj[MAXN], mst;

int prim(int nodes) {
  mst.clear();
  std::vector<bool> visit(nodes);
  int total_dist = 0;
  for (int i = 0; i < nodes; i++) {
    if (visit[i]) {
      continue;
    }
    visit[i] = true;
    std::priority_queue<std::pair<int, std::pair<int, int> > > pq;
    for (int j = 0; j < (int)adj[i].size(); j++) {
      pq.push(std::make_pair(-adj[i][j].second,
                             std::make_pair(i, adj[i][j].first)));
    }
    while (!pq.empty()) {
      int u = pq.top().second.first;
      int v = pq.top().second.second;
      int w = -pq.top().first;
      pq.pop();
      if (visit[u] && !visit[v]) {
        visit[v] = true;
        if (v != i) {
          mst.push_back(std::make_pair(u, v));
          total_dist += w;
        }
        for (int j = 0; j < (int)adj[v].size(); j++) {
          pq.push(std::make_pair(-adj[v][j].second,
                                 std::make_pair(v, adj[v][j].first)));
        }
      }
    }
  }
  return total_dist;
}

/*** Example Usage and Output:

Total distance: 13
0 <-> 2
0 <-> 1
3 <-> 4
4 <-> 5
5 <-> 6

***/

#include <iostream>
using namespace std;

void add_edge(int u, int v, int w) {
  adj[u].push_back(make_pair(v, w));
  adj[v].push_back(make_pair(u, w));
}

int main() {
  add_edge(0, 1, 4);
  add_edge(1, 2, 6);
  add_edge(2, 0, 3);
  add_edge(3, 4, 1);
  add_edge(4, 5, 2);
  add_edge(5, 6, 3);
  add_edge(6, 4, 4);
  cout << "Total distance: " << prim(7) << endl;
  for (int i = 0; i < (int)mst.size(); i++) {
    cout << mst[i].first << " <-> " << mst[i].second << endl;
  }
  return 0;
}
\end{lstlisting}
\subsection{Minimum Spanning Tree (Kruskal)}
\begin{lstlisting}
/*

Given a connected, undirected, weighted graph with possibly negative weights,
its minimum spanning tree is a subgraph which is a tree that connects all nodes
with a subset of its edges such that their total weight is minimized. kruskal()
applies to a global, pre-populated adjacency list adj[] which must only consist
of nodes numbered with integers between 0 (inclusive) and the total number of
nodes (exclusive), as passed in the function argument. If the input graph is not
connected, then this implementation will find the minimum spanning forest.

Time Complexity:
- O(m log n) per call to kruskal(), where m is the number of edges and n is the
  number of nodes.

Space Complexity:
- O(max(n, m)) for storage of the graph, where n the number of nodes and m is
  the number of edges
- O(n) auxiliary stack space for kruskal().

*/

#include <algorithm>
#include <utility>
#include <vector>

const int MAXN = 100;
std::vector<std::pair<int, std::pair<int, int> > > edges;
int root[MAXN];
std::vector<std::pair<int, int> > mst;

int find_root(int x) {
  if (root[x] != x) {
    root[x] = find_root(root[x]);
  }
  return root[x];
}

int kruskal(int nodes) {
  mst.clear();
  std::sort(edges.begin(), edges.end());
  int total_dist = 0;
  for (int i = 0; i < nodes; i++) {
    root[i] = i;
  }
  for (int i = 0; i < (int)edges.size(); i++) {
    int u = find_root(edges[i].second.first);
    int v = find_root(edges[i].second.second);
    if (u != v) {
      root[u] = root[v];
      mst.push_back(edges[i].second);
      total_dist += edges[i].first;
    }
  }
  return total_dist;
}

/*** Example Usage and Output:

Total distance: 13
3 <-> 4
4 <-> 5
2 <-> 0
5 <-> 6
0 <-> 1

***/

#include <iostream>
using namespace std;

void add_edge(int u, int v, int w) {
  edges.push_back(make_pair(w, make_pair(u, v)));
}

int main() {
  add_edge(0, 1, 4);
  add_edge(1, 2, 6);
  add_edge(2, 0, 3);
  add_edge(3, 4, 1);
  add_edge(4, 5, 2);
  add_edge(5, 6, 3);
  add_edge(6, 4, 4);
  cout << "Total distance: " << kruskal(7) << endl;
  for (int i = 0; i < (int)mst.size(); i++) {
    cout << mst[i].first << " <-> " << mst[i].second << endl;
  }
  return 0;
}
\end{lstlisting}

\section{Maximum Flow}
\setcounter{section}{5}
\setcounter{subsection}{0}
\subsection{Maximum Flow (Ford-Fulkerson)}
\begin{lstlisting}
/*

Given a flow network with integer capacities, find the maximum flow from a given
source node to a given sink node. The flow of a given edge u -> v is defined as
the minimum of its capacity and the sum of the flows of all incoming edges of u.
ford_fulkerson() applies to global variables nodes, source, sink, and cap[][]
which is an adjacency matrix that will be modified by the function call.

The Ford-Fulkerson algorithm is only optimal on graphs with integer capacities,
as there exists certain real-valued flow inputs for which the algorithm never
terminates. The Edmonds-Karp algorithm is an improvement using breadth-first
search, addressing this problem.

Time Complexity:
- O(n^2*f) per call to ford_fulkerson(), where n is the number of nodes and f
  is the maximum flow.

Space Complexity:
- O(n^2) for storage of the flow network, where n is the number of nodes.
- O(n) auxiliary stack space for ford_fulkerson().

*/

#include <algorithm>
#include <vector>

const int MAXN = 100, INF = 0x3f3f3f3f;
int nodes, source, sink, cap[MAXN][MAXN];
std::vector<bool> visit(MAXN);

int dfs(int u, int f) {
  if (u == sink) {
    return f;
  }
  visit[u] = true;
  for (int v = 0; v < nodes; v++) {
    if (!visit[v] && cap[u][v] > 0) {
      int flow = dfs(v, std::min(f, cap[u][v]));
      if (flow > 0) {
        cap[u][v] -= flow;
        cap[v][u] += flow;
        return flow;
      }
    }
  }
  return 0;
}

int ford_fulkerson() {
  int max_flow = 0;
  for (;;) {
    std::fill(visit.begin(), visit.end(), false);
    int flow = dfs(source, INF);
    if (flow == 0) {
      break;
    }
    max_flow += flow;
  }
  return max_flow;
}

/*** Example Usage ***/

#include <cassert>

int main() {
  nodes = 6;
  source = 0;
  sink = 5;
  cap[0][1] = 3;
  cap[0][2] = 3;
  cap[1][2] = 2;
  cap[1][3] = 3;
  cap[2][4] = 2;
  cap[3][4] = 1;
  cap[3][5] = 2;
  cap[4][5] = 3;
  assert(ford_fulkerson() == 5);
  return 0;
}
\end{lstlisting}
\subsection{Maximum Flow (Edmonds-Karp)}
\begin{lstlisting}
/*

Given a flow network with integer capacities, find the maximum flow from a given
source node to a given sink node. The flow of a given edge u -> v is defined as
the minimum of its capacity and the sum of the flows of all incoming edges of u.
edmonds_karp() applies to a global adjacency list adj[] that will be modified by
the function call.

The Edmonds-Karp algorithm will also support real-valued flow capacities. As
such, this implementation will work as intended upon changing the appropriate
variables to doubles.

Time Complexity:
- O(min(n*m^2, m*f)) per call to edmonds_karp(), where n is the number of nodes,
  m is the number of edges, and f is the maximum flow.

Space Complexity:
- O(max(n, m)) for storage of the flow network, where n is the number of nodes
  and m is the number of edges.

*/

#include <algorithm>
#include <queue>
#include <vector>

struct edge { int u, v, rev, cap, f; };

const int MAXN = 100, INF = 0x3f3f3f3f;
std::vector<edge> adj[MAXN];

void add_edge(int u, int v, int cap) {
  adj[u].push_back((edge){u, v, (int)adj[v].size(), cap, 0});
  adj[v].push_back((edge){v, u, (int)adj[u].size() - 1, 0, 0});
}

int edmonds_karp(int nodes, int source, int sink) {
  int max_flow = 0;
  for (;;) {
    std::vector<edge*> pred(nodes, (edge*)0);
    std::queue<int> q;
    q.push(source);
    while (!q.empty() && !pred[sink]) {
      int u = q.front();
      q.pop();
      for (int j = 0; j < (int)adj[u].size(); j++) {
        edge &e = adj[u][j];
        if (!pred[e.v] && e.cap > e.f) {
          pred[e.v] = &e;
          q.push(e.v);
        }
      }
    }
    if (!pred[sink]) {
      break;
    }
    int flow = INF;
    for (int u = sink; u != source; u = pred[u]->u) {
      flow = std::min(flow, pred[u]->cap - pred[u]->f);
    }
    for (int u = sink; u != source; u = pred[u]->u) {
      pred[u]->f += flow;
      adj[pred[u]->v][pred[u]->rev].f -= flow;
    }
    max_flow += flow;
  }
  return max_flow;
}

/*** Example Usage ***/

#include <cassert>

int main() {
  add_edge(0, 1, 3);
  add_edge(0, 2, 3);
  add_edge(1, 2, 2);
  add_edge(1, 3, 3);
  add_edge(2, 4, 2);
  add_edge(3, 4, 1);
  add_edge(3, 5, 2);
  add_edge(4, 5, 3);
  assert(edmonds_karp(6, 0, 5) == 5);
  return 0;
}
\end{lstlisting}
\subsection{Maximum Flow (Dinic)}
\begin{lstlisting}
/*

Given a flow network with integer capacities, find the maximum flow from a given
source node to a given sink node. The flow of a given edge u -> v is defined as
the minimum of its capacity and the sum of the flows of all incoming edges of u.
dinic() applies to a global adjacency list adj[] that will be modified by the
function call.

Dinic's algorithm will also support real-valued flow capacities. As such, this
implementation will work as intended upon changing the appropriate variables to
doubles.

Time Complexity:
- O(n^2*m) per call to dinic(), where n is the number of nodes and m is the
  number of edges.

Space Complexity:
- O(max(n, m)) for storage of the flow network, where n is the number of nodes
  and m is the number of edges.
- O(n) auxiliary stack and heap space for dinic().

*/

#include <algorithm>
#include <queue>
#include <vector>

struct edge { int v, rev, cap, f; };

const int MAXN = 100, INF = 0x3f3f3f3f;
std::vector<edge> adj[MAXN];
int dist[MAXN], ptr[MAXN];

void add_edge(int u, int v, int cap) {
  adj[u].push_back((edge){v, (int)adj[v].size(), cap, 0});
  adj[v].push_back((edge){u, (int)adj[u].size() - 1, 0, 0});
}

bool dinic_bfs(int nodes, int source, int sink) {
  std::fill(dist, dist + nodes, -1);
  dist[source] = 0;
  std::queue<int> q;
  q.push(source);
  while (!q.empty()) {
    int u = q.front();
    q.pop();
    for (int j = 0; j < (int)adj[u].size(); j++) {
      edge &e = adj[u][j];
      if (dist[e.v] < 0 && e.f < e.cap) {
        dist[e.v] = dist[u] + 1;
        q.push(e.v);
      }
    }
  }
  return dist[sink] >= 0;
}

int dinic_dfs(int u, int f, int sink) {
  if (u == sink) {
    return f;
  }
  for (; ptr[u] < (int)adj[u].size(); ptr[u]++) {
    edge &e = adj[u][ptr[u]];
    if (dist[e.v] == dist[u] + 1 && e.f < e.cap) {
      int flow = dinic_dfs(e.v, std::min(f, e.cap - e.f), sink);
      if (flow > 0) {
        e.f += flow;
        adj[e.v][e.rev].f -= flow;
        return flow;
      }
    }
  }
  return 0;
}

int dinic(int nodes, int source, int sink) {
  int flow, max_flow = 0;
  while (dinic_bfs(nodes, source, sink)) {
    std::fill(ptr, ptr + nodes, 0);
    while ((flow = dinic_dfs(source, INF, sink)) != 0) {
      max_flow += flow;
    }
  }
  return max_flow;
}

/*** Example Usage ***/

#include <cassert>

int main() {
  add_edge(0, 1, 3);
  add_edge(0, 2, 3);
  add_edge(1, 2, 2);
  add_edge(1, 3, 3);
  add_edge(2, 4, 2);
  add_edge(3, 4, 1);
  add_edge(3, 5, 2);
  add_edge(4, 5, 3);
  assert(dinic(6, 0, 5) == 5);
  return 0;
}
\end{lstlisting}
\subsection{Maximum Flow (Push-Relabel)}
\begin{lstlisting}
/*

Given a flow network with integer capacities, find the maximum flow from a given
source node to a given sink node. The flow of a given edge u -> v is defined as
the minimum of its capacity and the sum of the flows of all incoming edges of u.
push_relabel() applies to a global adjacency matrix cap[][] and returns the
maximum flow.

Although the push-relabel algorithm is considered one of the most efficient
maximum flow algorithms, it cannot take advantage of the magnitude of the
maximum flow being less than n^3 (in which case the Ford-Fulkerson or
Edmonds-Karp algorithms may be more efficient).

Time Complexity:
- O(n^3) per call to push_relabel(), where n is the number of nodes.

Space Complexity:
- O(n^2) for storage of the flow network, where n is the number of nodes.
- O(n) auxiliary heap space for push_relabel().

*/

#include <algorithm>
#include <vector>

const int MAXN = 100, INF = 0x3f3f3f3f;
int cap[MAXN][MAXN], f[MAXN][MAXN];

int push_relabel(int nodes, int source, int sink) {
  std::vector<int> e(nodes, 0), h(nodes, 0), maxh(nodes, 0);
  for (int i = 0; i < nodes; i++) {
    std::fill(f[i], f[i] + nodes, 0);
  }
  h[source] = nodes - 1;
  for (int i = 0; i < nodes; i++) {
    f[source][i] = cap[source][i];
    f[i][source] = -f[source][i];
    e[i] = cap[source][i];
  }
  int size = 0;
  for (;;) {
    if (size == 0) {
      for (int i = 0; i < nodes; i++) {
        if (i != source && i != sink && e[i] > 0) {
          if (size != 0 && h[i] > h[maxh[0]]) {
            size = 0;
          }
          maxh[size++] = i;
        }
      }
    }
    if (size == 0) {
      break;
    }
    while (size != 0) {
      int i = maxh[size - 1];
      bool pushed = false;
      for (int j = 0; j < nodes && e[i] != 0; j++) {
        if (h[i] == h[j] + 1 && cap[i][j] - f[i][j] > 0) {
          int df = std::min(cap[i][j] - f[i][j], e[i]);
          f[i][j] += df;
          f[j][i] -= df;
          e[i] -= df;
          e[j] += df;
          if (e[i] == 0) {
            size--;
          }
          pushed = true;
        }
      }
      if (pushed) {
        continue;
      }
      h[i] = INF;
      for (int j = 0; j < nodes; j++) {
        if (h[i] > h[j] + 1 && cap[i][j] - f[i][j] > 0) {
          h[i] = h[j] + 1;
        }
      }
      if (h[i] > h[maxh[0]]) {
        size = 0;
        break;
      }
    }
  }
  int max_flow = 0;
  for (int i = 0; i < nodes; i++) {
    max_flow += f[source][i];
  }
  return max_flow;
}

/*** Example Usage ***/

#include <cassert>

int main() {
  cap[0][1] = 3;
  cap[0][2] = 3;
  cap[1][2] = 2;
  cap[1][3] = 3;
  cap[2][4] = 2;
  cap[3][4] = 1;
  cap[3][5] = 2;
  cap[4][5] = 3;
  assert(push_relabel(6, 0, 5) == 5);
  return 0;
}
\end{lstlisting}

\section{Maximum Matching}
\setcounter{section}{6}
\setcounter{subsection}{0}
\subsection{Maximum Bipartite Matching (Kuhn)}
\begin{lstlisting}
/*

Given two sets of nodes A = {0, 1, ..., n1} and B = {0, 1, ..., n2} such that
n1 < n2, as well as a set of edges E mapping nodes from set A to set B, find the
largest possible subset of E containing no edges that share the same node.
kuhn() applies to a global, pre-populated adjacency list adj[] which must only
consist of nodes numbered with integers between 0 (inclusive) and the total
number of nodes (exclusive), as passed in the function argument.

Time Complexity:
- O(m*(n1 + n2)) per call to kuhn(), where m is the number of edges.

Space Complexity:
- O(n1 + n2) auxiliary stack space for kuhn().

*/

#include <algorithm>
#include <vector>

const int MAXN = 100;
int match[MAXN];
std::vector<bool> visit(MAXN);
std::vector<int> adj[MAXN];

bool dfs(int u) {
  visit[u] = true;
  for (int j = 0; j < (int)adj[u].size(); j++) {
    int v = match[adj[u][j]];
    if (v == -1 || (!visit[v] && dfs(v))) {
      match[adj[u][j]] = u;
      return true;
    }
  }
  return false;
}

int kuhn(int n1, int n2) {
  std::fill(visit.begin(), visit.end(), false);
  std::fill(match, match + n2, -1);
  int matches = 0;
  for (int i = 0; i < n1; i++) {
    std::fill(visit.begin(), visit.begin() + n1, false);
    if (dfs(i)) {
      matches++;
    }
  }
  return matches;
}

/*** Example Usage and Output:

Matched 3 pair(s):
1 0
0 1
2 2

***/

#include <iostream>
using namespace std;

int main() {
  int n1 = 3, n2 = 4;
  adj[0].push_back(1);
  adj[1].push_back(0);
  adj[1].push_back(1);
  adj[1].push_back(2);
  adj[2].push_back(2);
  adj[2].push_back(3);
  cout << "Matched " << kuhn(n1, n2) << " pair(s):" << endl;
  for (int i = 0; i < n2; i++) {
    if (match[i] != -1) {
      cout << match[i] << " " << i << endl;
    }
  }
  return 0;
}
\end{lstlisting}
\subsection{Maximum Bipartite Matching (Hopcroft-Karp)}
\begin{lstlisting}
/*

Given two sets of nodes A = {0, 1, ..., n1} and B = {0, 1, ..., n2} such that
n1 < n2, as well as a set of edges E mapping nodes from set A to set B, find the
largest possible subset of E containing no edges that share the same node.
hopcroft_karp() applies to a global, pre-populated adjacency list adj[] which
must only consist of nodes numbered with integers between 0 (inclusive) and the
total number of nodes (exclusive), as passed in the function argument.

Time Complexity:
- O(m*sqrt(n1 + n2)) per call to hopcroft_karp(), where m is the number of
  edges.

Space Complexity:
- O(max(n, m)) for storage of the graph, where n the number of nodes and m is
  the number of edges.
- O(n1 + n2) auxiliary stack and heap space for hopcroft_karp().

*/

#include <algorithm>
#include <queue>
#include <vector>

const int MAXN = 100;
std::vector<int> adj[MAXN];
std::vector<bool> used(MAXN), visit(MAXN);
int match[MAXN], dist[MAXN];

void bfs(int n1, int n2) {
  std::fill(dist, dist + n1, -1);
  std::queue<int> q;
  for (int u = 0; u < n1; u++) {
    if (!used[u]) {
      q.push(u);
      dist[u] = 0;
    }
  }
  while (!q.empty()) {
    int u = q.front();
    q.pop();
    for (int j = 0; j < (int)adj[u].size(); j++) {
      int v = match[adj[u][j]];
      if (v >= 0 && dist[v] < 0) {
        dist[v] = dist[u] + 1;
        q.push(v);
      }
    }
  }
}

bool dfs(int u) {
  visit[u] = true;
  for (int j = 0; j < (int)adj[u].size(); j++) {
    int v = match[adj[u][j]];
    if (v < 0 || (!visit[v] && dist[v] == dist[u] + 1 && dfs(v))) {
      match[adj[u][j]] = u;
      used[u] = true;
      return true;
    }
  }
  return false;
}

int hopcroft_karp(int n1, int n2) {
  std::fill(match, match + n2, -1);
  std::fill(used.begin(), used.end(), false);
  int res = 0;
  for (;;) {
    bfs(n1, n2);
    std::fill(visit.begin(), visit.end(), false);
    int f = 0;
    for (int u = 0; u < n1; u++) {
      if (!used[u] && dfs(u)) {
        f++;
      }
    }
    if (f == 0) {
      return res;
    }
    res += f;
  }
  return res;
}

/*** Example Usage and Output:

Matched 3 pair(s):
1 0
0 1
2 2

***/

#include <iostream>
using namespace std;

int main() {
  int n1 = 3, n2 = 4;
  adj[0].push_back(1);
  adj[1].push_back(0);
  adj[1].push_back(1);
  adj[1].push_back(2);
  adj[2].push_back(2);
  adj[2].push_back(3);
  cout << "Matched " << hopcroft_karp(n1, n2) << " pair(s):" << endl;
  for (int i = 0; i < n2; i++) {
    if (match[i] != -1) {
      cout << match[i] << " " << i << endl;
    }
  }
  return 0;
}
\end{lstlisting}
\subsection{Maximum Graph Matching (Edmonds)}
\begin{lstlisting}
/*

Given a directed graph, determine a maximal subset of its edges such that no
node is shared between different edges in the resulting subset. edmonds()
applies to a global, pre-populated adjacency list adj[] which must only consist
of nodes numbered with integers between 0 (inclusive) and the total number of
nodes (exclusive), as passed in the function argument.

Time Complexity:
- O(n^3) per call to edmonds(), where n is the number of nodes.

Space Complexity:
- O(max(n, m)) for storage of the graph, where n the number of nodes and m is
  the number of edges.
- O(n) auxiliary heap space for edmonds(), where n is the number of nodes.

*/

#include <queue>
#include <vector>

const int MAXN = 100;
std::vector<int> adj[MAXN];
int p[MAXN], base[MAXN], match[MAXN];

int lca(int nodes, int u, int v) {
  std::vector<bool> used(nodes);
  for (;;) {
    u = base[u];
    used[u] = true;
    if (match[u] == -1) {
      break;
    }
    u = p[match[u]];
  }
  for (;;) {
    v = base[v];
    if (used[v]) {
      return v;
    }
    v = p[match[v]];
  }
}

void mark_path(std::vector<bool> &blossom, int u, int b, int child) {
  for (; base[u] != b; u = p[match[u]]) {
    blossom[base[u]] = true;
    blossom[base[match[u]]] = true;
    p[u] = child;
    child = match[u];
  }
}

int find_path(int nodes, int root) {
  std::vector<bool> used(nodes);
  for (int i = 0; i < nodes; ++i) {
    p[i] = -1;
    base[i] = i;
  }
  used[root] = true;
  std::queue<int> q;
  q.push(root);
  while (!q.empty()) {
    int u = q.front();
    q.pop();
    for (int j = 0; j < (int)adj[u].size(); j++) {
      int v = adj[u][j];
      if (base[u] == base[v] || match[u] == v) {
        continue;
      }
      if (v == root || (match[v] != -1 && p[match[v]] != -1)) {
        int curr_base = lca(nodes, u, v);
        std::vector<bool> blossom(nodes);
        mark_path(blossom, u, curr_base, v);
        mark_path(blossom, v, curr_base, u);
        for (int i = 0; i < nodes; i++) {
          if (blossom[base[i]]) {
            base[i] = curr_base;
            if (!used[i]) {
              used[i] = true;
              q.push(i);
            }
          }
        }
      } else if (p[v] == -1) {
        p[v] = u;
        if (match[v] == -1) {
          return v;
        }
        v = match[v];
        used[v] = true;
        q.push(v);
      }
    }
  }
  return -1;
}

int edmonds(int nodes) {
  for (int i = 0; i < nodes; i++) {
    match[i] = -1;
  }
  for (int i = 0; i < nodes; i++) {
    if (match[i] == -1) {
      int u, pu, ppu;
      for (u = find_path(nodes, i); u != -1; u = ppu) {
        pu = p[u];
        ppu = match[pu];
        match[u] = pu;
        match[pu] = u;
      }
    }
  }
  int matches = 0;
  for (int i = 0; i < nodes; i++) {
    if (match[i] != -1) {
      matches++;
    }
  }
  return matches/2;
}

/*** Example Usage and Output:

Matched 2 pair(s):
0 1
2 3

***/

#include <iostream>
using namespace std;

int main() {
  int nodes = 4;
  adj[0].push_back(1);
  adj[1].push_back(0);
  adj[1].push_back(2);
  adj[2].push_back(1);
  adj[2].push_back(3);
  adj[3].push_back(2);
  adj[3].push_back(0);
  adj[0].push_back(3);
  cout << "Matched " << edmonds(nodes) << " pair(s):" << endl;
  for (int i = 0; i < nodes; i++) {
    if (match[i] != -1 && i < match[i]) {
      cout << i << " " << match[i] << endl;
    }
  }
  return 0;
}
\end{lstlisting}

\section{Hard Problems}
\setcounter{section}{7}
\setcounter{subsection}{0}
\subsection{Maximum Clique (Bron-Kerbosch)}
\begin{lstlisting}
/*

Given an undirected graph, max_clique() returns the size of the maximum clique,
that is, the largest subset of nodes such that all pairs of nodes in the subset
are connected by an edge. max_clique_weighted() additionally uses a global array
w[] specifying a weight value for each node, returning the clique in the graph
that has maximum total weight.

Both functions apply to a global, pre-populated adjacency matrix adj[] which
must satisfy the condition that adj[u][v] is true if and only if adj[v][u] is
true, for all pairs of nodes u and v respectively between 0 (inclusive) and the
total number of nodes (exclusive) as passed in the function argument. Note that
max_clique_weighted() is an efficient implementation using bitmasks of unsigned
64-bit integers, thus requiring the number of nodes to be less than 64.

Time Complexity:
- O(3^(n/3)) per call to max_clique() and max_clique_weighted(), where n
  is the number of nodes.

Space Complexity:
- O(n^2) for storage of the graph, where n is the number of nodes.
- O(n) auxiliary stack space for max_clique() and max_clique_weighted().

*/

#include <algorithm>
#include <bitset>
#include <vector>

const int MAXN = 35;
typedef std::bitset<MAXN> bits;
typedef unsigned long long uint64;

bool adj[MAXN][MAXN];
int w[MAXN];

int rec(int nodes, bits &curr, bits &pool, bits &excl) {
  if (pool.none() && excl.none()) {
    return curr.count();
  }
  int ans = 0, u = 0;
  for (int v = 0; v < nodes; v++) {
    if (pool[v] || excl[v]) {
      u = v;
    }
  }
  for (int v = 0; v < nodes; v++) {
    if (!pool[v] || adj[u][v]) {
      continue;
    }
    bits ncurr, npool, nexcl;
    for (int i = 0; i < nodes; i++) {
      ncurr[i] = curr[i];
    }
    ncurr[v] = true;
    for (int j = 0; j < nodes; j++) {
      npool[j] = pool[j] && adj[v][j];
      nexcl[j] = excl[j] && adj[v][j];
    }
    ans = std::max(ans, rec(nodes, ncurr, npool, nexcl));
    pool[v] = false;
    excl[v] = true;
  }
  return ans;
}

int max_clique(int nodes) {
  bits curr, excl, pool;
  pool.flip();
  return rec(nodes, curr, pool, excl);
}

int rec(const std::vector<uint64> &g, uint64 curr, uint64 pool, uint64 excl) {
  if (pool == 0 && excl == 0) {
    int res = 0, u = __builtin_ctzll(curr);
    while (u < (int)g.size()) {
      res += w[u];
      u += __builtin_ctzll(curr >> (u + 1)) + 1;
    }
    return res;
  }
  if (pool == 0) {
    return -1;
  }
  int res = -1, pivot = __builtin_ctzll(pool | excl);
  uint64 z = pool & ~g[pivot];
  int u = __builtin_ctzll(z);
  while (u < (int)g.size()) {
    res = std::max(res, rec(g, curr | (1LL << u), pool & g[u], excl & g[u]));
    pool ^= 1LL << u;
    excl |= 1LL << u;
    u += __builtin_ctzll(z >> (u + 1)) + 1;
  }
  return res;
}

int max_clique_weighted(int nodes) {
  std::vector<uint64> g(nodes, 0);
  for (int i = 0; i < nodes; i++) {
    for (int j = 0; j < nodes; j++) {
      if (adj[i][j]) {
        g[i] |= 1LL << j;
      }
    }
  }
  return rec(g, 0, (1LL << nodes) - 1, 0);
}

/*** Example Usage ***/

#include <cassert>

void add_edge(int u, int v) {
  adj[u][v] = true;
  adj[v][u] = true;
}

int main() {
  add_edge(0, 1);
  add_edge(0, 2);
  add_edge(0, 3);
  add_edge(1, 2);
  add_edge(1, 3);
  add_edge(2, 3);
  add_edge(3, 4);
  add_edge(4, 2);
  w[0] = 10;
  w[1] = 20;
  w[2] = 30;
  w[3] = 40;
  w[4] = 50;
  assert(max_clique(5) == 4);
  assert(max_clique_weighted(5) == 120);
  return 0;
}
\end{lstlisting}
\subsection{Graph Coloring}
\begin{lstlisting}
/*

Given an undirected graph, assign a color to every node such that no pair of
adjacent nodes have the same color, and that the total number of colors used is
minimized. color_graph() applies to a global, pre-populated adjacency matrix
adj[][] which must satisfy the condition that adj[u][v] is true if and only if
adj[v][u] is true, for all pairs of nodes u and v respectively between 0
(inclusive) and the total number of nodes (exclusive) as passed in the function
argument.

Time Complexity:
- Exponential on the number of nodes per call to color_graph().

Space Complexity:
- O(n^2) for storage of the graph, where n is the number of nodes.
- O(n) auxiliary stack and heap space for color_graph().

*/

#include <algorithm>
#include <vector>

const int MAXN = 30;
int adj[MAXN][MAXN], min_colors, color[MAXN];
int curr[MAXN], id[MAXN + 1], degree[MAXN + 1];

void rec(int lo, int hi, int n, int used_colors) {
  if (used_colors >= min_colors) {
    return;
  }
  if (n == hi) {
    for (int i = lo; i < hi; i++) {
      color[id[i]] = curr[i];
    }
    min_colors = used_colors;
    return;
  }
  std::vector<bool> used(used_colors + 1);
  for (int i = 0; i < n; i++) {
    if (adj[id[n]][id[i]]) {
      used[curr[i]] = true;
    }
  }
  for (int i = 0; i <= used_colors; i++) {
    if (!used[i]) {
      int tmp = curr[n];
      curr[n] = i;
      rec(lo, hi, n + 1, std::max(used_colors, i + 1));
      curr[n] = tmp;
    }
  }
}

int color_graph(int nodes) {
  for (int i = 0; i <= nodes; i++) {
    id[i] = i;
    degree[i] = 0;
  }
  int res = 1, lo = 0;
  for (int hi = 1; hi <= nodes; hi++) {
    int best = hi;
    for (int i = hi; i < nodes; i++) {
      if (adj[id[hi - 1]][id[i]]) {
        degree[id[i]]++;
      }
      if (degree[id[best]] < degree[id[i]]) {
        best = i;
      }
    }
    std::swap(id[hi], id[best]);
    if (degree[id[hi]] == 0) {
      min_colors = nodes + 1;
      std::fill(curr, curr + nodes, 0);
      rec(lo, hi, lo, 0);
      lo = hi;
      res = std::max(res, min_colors);
    }
  }
  return res;
}

/*** Example Usage and Output:

Colored using 3 color(s):
Color 1: 0 3
Color 2: 1 2
Color 3: 4

***/

#include <cassert>
#include <iostream>
using namespace std;

void add_edge(int u, int v) {
  adj[u][v] = true;
  adj[v][u] = true;
}

int main() {
  add_edge(0, 1);
  add_edge(0, 4);
  add_edge(1, 3);
  add_edge(1, 4);
  add_edge(2, 3);
  add_edge(2, 4);
  add_edge(3, 4);
  int colors = color_graph(5);
  cout << "Colored using " << colors << " color(s):" << endl;
  for (int i = 0; i < colors; i++) {
    cout << "Color " << i + 1 << ":";
    for (int j = 0; j < 5; j++) {
      if (color[j] == i) {
        cout << " " << j;
      }
    }
    cout << endl;
  }
  return 0;
}
\end{lstlisting}
\subsection{Shortest Hamiltonian Cycle (TSP)}
\begin{lstlisting}
/*

Given a weighted graph, determine a cycle of minimum total distance which visits
each node exactly once and returns to the starting node. This is known as the
traveling salesman problem (TSP). Since this implementation uses bitmasks with
32-bit integers, the maximum number of nodes must be less than 32.
shortest_hamiltonian_cycle() applies to a global adjacency matrix adj[][], which
must be populated with add_edge() before the function call.

Time Complexity:
- O(2^n * n^2) per call to shortest_hamiltonian_cycle(), where n is the number
  of nodes.

Space Complexity:
- O(n^2) for storage of the graph, where n is the number of nodes.
- O(2^n * n) auxiliary heap space for shortest_hamiltonian_cycle().

*/

#include <algorithm>

const int MAXN = 20, INF = 0x3f3f3f3f;
int adj[MAXN][MAXN], dp[1 << MAXN][MAXN], order[MAXN];

void add_edge(int u, int v, int w) {
  adj[u][v] = w;
  adj[v][u] = w;  // Remove this line if the graph is directed.
}

int shortest_hamiltonian_cycle(int nodes) {
  int max_mask = (1 << nodes) - 1;
  for (int i = 0; i <= max_mask; i++) {
    std::fill(dp[i], dp[i] + nodes, INF);
  }
  dp[1][0] = 0;
  for (int mask = 1; mask <= max_mask; mask += 2) {
    for (int i = 1; i < nodes; i++) {
      if ((mask & 1 << i) != 0) {
        for (int j = 0; j < nodes; j++) {
          if ((mask & 1 << j) != 0) {
            dp[mask][i] = std::min(dp[mask][i],
                                   dp[mask ^ (1 << i)][j] + adj[j][i]);
          }
        }
      }
    }
  }
  int res = INF + INF;
  for (int i = 1; i < nodes; i++) {
    res = std::min(res, dp[max_mask][i] + adj[i][0]);
  }
  int mask = max_mask, old = 0;
  for (int i = nodes - 1; i >= 1; i--) {
    int bj = -1;
    for (int j = 1; j < nodes; j++) {
      if ((mask & 1 << j) != 0 && (bj == -1 ||
              dp[mask][bj] + adj[bj][old] > dp[mask][j] + adj[j][old])) {
        bj = j;
      }
    }
    order[i] = bj;
    mask ^= 1 << bj;
    old = bj;
  }
  return res;
}

/*** Example Usage and Output:

The shortest hamiltonian cycle has length 5.
Take the path: 0->3->2->4->1->0.

***/

#include <iostream>
using namespace std;

int main() {
  int nodes = 5;
  add_edge(0, 1, 1);
  add_edge(0, 2, 10);
  add_edge(0, 3, 1);
  add_edge(0, 4, 10);
  add_edge(1, 2, 10);
  add_edge(1, 3, 10);
  add_edge(1, 4, 1);
  add_edge(2, 3, 1);
  add_edge(2, 4, 1);
  add_edge(3, 4, 10);
  cout << "The shortest hamiltonian cycle has length "
       << shortest_hamiltonian_cycle(nodes) << "." << endl
       << "Take the path: ";
  for (int i = 0; i < nodes; i++) {
    cout << order[i] << "->";
  }
  cout << order[0] << "." << endl;
  return 0;
}
\end{lstlisting}
\subsection{Shortest Hamiltonian Path}
\begin{lstlisting}
/*

Given a weighted, directed graph, determine a path of minimum total distance
which visits each node exactly once. Unlike the traveling salesman problem, we
do not have to return to the starting vertex. Since this implementation uses
bitmasks with 32-bit integers, the maximum number of nodes must be less than 32.
shortest_hamiltonian_path() applies to a global adjacency matrix adj[][] which
must be populated before the function call.

Time Complexity:
- O(2^n * n^2) per call to shortest_hamiltonian_path(), where n is the number
  of nodes.

Space Complexity:
- O(n^2) for storage of the graph, where n is the number of nodes.
- O(2^n * n^2) auxiliary heap space for shortest_hamiltonian_path().

*/

#include <algorithm>

const int MAXN = 20, INF = 0x3f3f3f3f;
int adj[MAXN][MAXN], dp[1 << MAXN][MAXN], order[MAXN];

int shortest_hamiltonian_path(int nodes) {
  int max_mask = (1 << nodes) - 1;
  for (int i = 0; i <= max_mask; i++) {
    std::fill(dp[i], dp[i] + nodes, INF);
  }
  for (int i = 0; i < nodes; i++) {
    dp[1 << i][i] = 0;
  }
  for (int mask = 1; mask <= max_mask; mask += 2) {
    for (int i = 0; i < nodes; i++) {
      if ((mask & 1 << i) != 0) {
        for (int j = 0; j < nodes; j++) {
          if ((mask & 1 << j) != 0)
            dp[mask][i] = std::min(dp[mask][i],
                                   dp[mask ^ (1 << i)][j] + adj[j][i]);
        }
      }
    }
  }
  int res = INF + INF;
  for (int i = 1; i < nodes; i++) {
    res = std::min(res, dp[max_mask][i]);
  }
  int mask = max_mask, old = -1;
  for (int i = nodes - 1; i >= 0; i--) {
    int bj = -1;
    for (int j = 0; j < nodes; j++) {
      if ((mask & 1 << j) != 0 &&
          (bj == -1 || dp[mask][bj] + (old == -1 ? 0 : adj[bj][old]) >
                       dp[mask][j] + (old == -1 ? 0 : adj[j][old]))) {
        bj = j;
      }
    }
    order[i] = bj;
    mask ^= (1 << bj);
    old = bj;
  }
  return res;
}

/*** Example Usage and Output:

The shortest hamiltonian path has length 3.
Take the path: 0->1->2.

***/

#include <iostream>
using namespace std;

int main() {
  int nodes = 3;
  adj[0][1] = 1;
  adj[0][2] = 1;
  adj[1][0] = 7;
  adj[1][2] = 2;
  adj[2][0] = 3;
  adj[2][1] = 5;
  cout << "The shortest hamiltonian path has length "
       << shortest_hamiltonian_path(nodes) << "." << endl
       << "Take the path: " << order[0];
  for (int i = 1; i < nodes; i++) {
    cout << "->" << order[i];
  }
  cout << "." << endl;
  return 0;
}
\end{lstlisting}

\chapter{Mathematics}

\section{Math Utilities}
\setcounter{section}{1}
\begin{lstlisting}
/*

Common mathematic constants and functions, many of which are substitutes for
features which are not available in standard C++, or may not be available on
compilers that do not support C++11 and later.

Time Complexity:
- O(1) for all operations.

Space Complexity:
- O(1) auxiliary for all operations.

*/

#include <algorithm>
#include <cfloat>
#include <climits>
#include <cmath>
#include <cstdlib>
#include <limits>
#include <string>
#include <vector>

#ifndef M_PI
  const double M_PI = acos(-1.0);
#endif
#ifndef M_E
  const double M_E = exp(1.0);
#endif
const double M_PHI = (1.0 + sqrt(5.0))/2.0;
const double M_INF = std::numeric_limits<double>::infinity();
const double M_NAN = std::numeric_limits<double>::quiet_NaN();

#ifndef isnan
  #define isnan(x) ((x) != (x))
#endif

/*

Epsilon Comparisons

EQ(), NE(), LT(), GT(), LE(), and GE() relationally compares two values x and y
accounting for absolute error. For any x, the range of values considered equal
barring absolute error is [x - EPS, x + EPS]. Values outside of this range are
considered not equal (strictly less or strictly greater).

rEQ() returns whether x and y are equal barring relative error. For any x, the
range of values considered equal is [x*(1 - EPS), x*(1 + EPS)].

*/

const double EPS = 1e-9;

#define EQ(x, y) (fabs((x) - (y)) <= EPS)
#define NE(x, y) (fabs((x) - (y)) > EPS)
#define LT(x, y) ((x) < (y) - EPS)
#define GT(x, y) ((x) > (y) + EPS)
#define LE(x, y) ((x) <= (y) + EPS)
#define GE(x, y) ((x) >= (y) - EPS)
#define rEQ(x, y) (fabs((x) - (y)) <= EPS*fabs(x))

/*

Sign Functions

- sgn(x) returns -1 (if x < 0), 0 (if x == 0), or 1 (if x > 0). Unlike signbit()
  or copysign(), this does not handle the sign of NaN.
- signbit_(x) is analogous to std::signbit() in C++11 and later, returning
  whether the sign bit of the floating point number is set to true. If so, then
  x is considered "negative." Note that this works as expected on +0.0, -0.0,
  Inf, -Inf, NaN, as well as -NaN. Warning: This assumes that the sign bit is
  the leading (most significant) bit in the internal representation of the IEEE
  floating point value.
- copysign_(x, y) is analogous to std::copysign() in C++11 and later, returning
  a number with the magnitude of x but the sign of y.

*/

template<class T>
int sgn(const T &x) {
  return (T(0) < x) - (x < T(0));
}

template<class Double>
bool signbit_(Double x) {
  return (((unsigned char *)&x)[sizeof(x) - 1] >> (CHAR_BIT - 1)) & 1;
}

template<class Double>
Double copysign_(Double x, Double y) {
  return signbit_(y) ? -fabs(x) : fabs(x);
}

/*

Rounding Functions

- floor0(x) returns x rounded down, symmetrically towards zero. This function is
  analogous to trunc() in C++11 and later.
- ceil0(x) returns x rounded up, symmetrically away from zero. This function is
  analogous to round() in C++11 and later.
- round_half_up(x) returns x rounded half up, towards positive infinity.
- round_half_down(x) returns x rounded half down, towards negative infinity.
- round_half_to0(x) returns x rounded half down, symmetrically towards zero.
- round_half_from0(x) returns x rounded half up, symmetrically away from zero.
- round_half_even(x) returns x rounded half to even, using banker's rounding.
- round_half_alternate(x) returns x rounded, where ties are broken by
  alternating rounds towards positive and negative infinity.
- round_half_alternate0(x) returns x rounded, where ties are broken by
  alternating symmetric rounds towards and away from zero.
- round_half_random(x) returns x rounded, where ties are broken randomly.
- round_n_places(x, n, f) returns x rounded to n digits after the decimal, using
  the specified rounding function f(x).

*/

template<class Double>
Double floor0(const Double &x) {
  Double res = floor(fabs(x));
  return (x < 0.0) ? -res : res;
}

template<class Double>
Double ceil0(const Double &x) {
  Double res = ceil(fabs(x));
  return (x < 0.0) ? -res : res;
}

template<class Double>
Double round_half_up(const Double &x) {
  return floor(x + 0.5);
}

template<class Double>
Double round_half_down(const Double &x) {
  return ceil(x - 0.5);
}

template<class Double>
Double round_half_to0(const Double &x) {
  Double res = round_half_down(fabs(x));
  return (x < 0.0) ? -res : res;
}

template<class Double>
Double round_half_from0(const Double &x) {
  Double res = round_half_up(fabs(x));
  return (x < 0.0) ? -res : res;
}

template<class Double>
Double round_half_even(const Double &x, const Double &eps = 1e-9) {
  if (x < 0.0) {
    return -round_half_even(-x, eps);
  }
  Double ipart;
  modf(x, &ipart);
  if (x - (ipart + 0.5) < eps) {
    return (fmod(ipart, 2.0) < eps) ? ipart : ceil0(ipart + 0.5);
  }
  return round_half_from0(x);
}

template<class Double>
Double round_half_alternate(const Double &x) {
  static bool up = true;
  return (up = !up) ? round_half_up(x) : round_half_down(x);
}

template<class Double>
Double round_half_alternate0(const Double &x) {
  static bool up = true;
  return (up = !up) ? round_half_from0(x) : round_half_to0(x);
}

template<class Double>
Double round_half_random(const Double &x) {
  return (rand() % 2 == 0) ? round_half_from0(x) : round_half_to0(x);
}

template<class Double, class RoundingFunction>
Double round_n_places(const Double &x, unsigned int n, RoundingFunction f) {
  return f(x*pow(10, n)) / pow(10, n);
}

/*

Error Function

- erf_(x) returns the error encountered in integrating the normal distribution.
  Its value is 2/sqrt(pi)*(integral of e^(-t^2) dt from 0 to x). This function
  is analogous to erf(x) in C++11 and later.
- erfc_(x) returns the error function complement, that is, 1 - erf_(x). This
  function is analogous to erfc(x) in C++11 and later.

*/

#define ERF_EPS 1e-14

double erfc_(double x);

double erf_(double x) {
  if (signbit_(x)) {
    return -erf_(-x);
  }
  if (fabs(x) > 2.2) {
    return 1.0 - erfc_(x);
  }
  double sum = x, term = x, xx = x*x;
  int j = 1;
  do {
    term *= xx / j;
    sum -= term/(2*(j++) + 1);
    term *= xx / j;
    sum += term/(2*(j++) + 1);
  } while (fabs(term) > sum*ERF_EPS);
  return 2/sqrt(M_PI) * sum;
}

double erfc_(double x) {
  if (fabs(x) < 2.2) {
    return 1.0 - erf_(x);
  }
  if (signbit_(x)) {
    return 2.0 - erfc_(-x);
  }
  double a = 1, b = x, c = x, d = x*x + 0.5, q1, q2 = 0, n = 1.0, t;
  do {
    t = a*n + b*x;
    a = b;
    b = t;
    t = c*n + d*x;
    c = d;
    d = t;
    n += 0.5;
    q1 = q2;
    q2 = b / d;
  } while (fabs(q1 - q2) > q2*ERF_EPS);
  return 1/sqrt(M_PI) * exp(-x*x) * q2;
}

#undef ERF_EPS

/*

Gamma Functions

- tgamma_(x) returns the gamma function of x. Unlike the tgamma() function in
  C++11 and later, this version only supports positive x, returning NaN if x is
  less than or equal to 0.
- lgamma_(x) returns the natural logarithm of the absolute value of the gamma
  function of x. Unlike the lgamma() function in C++11 and later, this version
  only supports positive x, returning NaN if x is less than or equal to 0.

*/

double lgamma_(double x);

double tgamma_(double x) {
  if (x <= 0) {
    return M_NAN;
  }
  if (x < 1e-3) {
    return 1.0 / (x*(1.0 + 0.57721566490153286060651209*x));
  }
  if (x < 12) {
    double y = x;
    int n = 0;
    bool arg_was_less_than_one = (y < 1);
    if (arg_was_less_than_one) {
      y += 1;
    } else {
      n = (int)floor(y) - 1;
      y -= n;
    }
    static const double p[] = {
        -1.71618513886549492533811e+0, 2.47656508055759199108314e+1,
        -3.79804256470945635097577e+2, 6.29331155312818442661052e+2,
        8.66966202790413211295064e+2, -3.14512729688483675254357e+4,
        -3.61444134186911729807069e+4, 6.64561438202405440627855e+4};
    static const double q[] = {
        -3.08402300119738975254353e+1, 3.15350626979604161529144e+2,
        -1.01515636749021914166146e+3, -3.10777167157231109440444e+3,
        2.25381184209801510330112e+4, 4.75584627752788110767815e+3,
        -1.34659959864969306392456e+5, -1.15132259675553483497211e+5};
    double num = 0, den = 1, z = y - 1;
    for (int i = 0; i < 8; i++) {
      num = (num + p[i])*z;
      den = den*z + q[i];
    }
    double result = num/den + 1;
    if (arg_was_less_than_one) {
      result /= (y - 1);
    } else {
      for (int i = 0; i < n; i++) {
        result *= y++;
      }
    }
    return result;
  }
  return (x > 171.624) ? 2*DBL_MAX : exp(lgamma(x));
}

double lgamma_(double x) {
  if (x <= 0) {
    return M_NAN;
  }
  if (x < 12) {
    return log(fabs(tgamma_(x)));
  }
  static const double c[8] = {
    1.0/12, -1.0/360, 1.0/1260, -1.0/1680, 1.0/1188, -691.0/360360, 1.0/156,
    -3617.0/122400
  };
  double z = 1.0/(x*x), sum = c[7];
  for (int i = 6; i >= 0; i--) {
    sum = sum*z + c[i];
  }
  return (x - 0.5)*log(x) - x + 0.91893853320467274178032973640562 + sum/x;
}

/*

Base Conversion

- Given an integer in base a as a vector d of digits (where d[0] is the least
  significant digit), convert_base(d, a, b) returns a vector of the integer's
  digits when converted base b (again with index 0 storing the least significant
  digit). The actual value of the entire integer to be converted must be able to
  fit within an unsigned 64-bit integer for intermediate storage.
- to_base(x, b) returns the digits of the unsigned integer x in base b, where
  index 0 of the result stores the least significant digit.
- to_roman(x) returns the Roman numeral representation of the unsigned integer x
  as a C++ string.

*/

std::vector<int> convert_base(const std::vector<int> &d, int a, int b) {
  unsigned long long x = 0, power = 1;
  for (int i = 0; i < (int)d.size(); i++) {
    x += d[i]*power;
    power *= a;
  }
  int n = ceil(log(x + 1)/log(b));
  std::vector<int> res;
  for (int i = 0; i < n; i++) {
    res.push_back(x % b);
    x /= b;
  }
  return res;
}

std::vector<int> to_base(unsigned int x, int b = 10) {
  std::vector<int> res;
  while (x != 0) {
    res.push_back(x % b);
    x /= b;
  }
  return res;
}

std::string to_roman(unsigned int x) {
  static const std::string h[] =
      {"", "C", "CC", "CCC", "CD", "D", "DC", "DCC", "DCCC", "CM"};
  static const std::string t[] =
      {"", "X", "XX", "XXX", "XL", "L", "LX", "LXX", "LXXX", "XC"};
  static const std::string o[] =
      {"", "I", "II", "III", "IV", "V", "VI", "VII", "VIII", "IX"};
  std::string prefix(x / 1000, 'M');
  x %= 1000;
  return prefix + h[x/100] + t[x/10 % 10] + o[x % 10];
}

/*** Example Usage ***/

#include <cassert>
#include <iostream>

int main() {
  assert(EQ(M_PI, 3.14159265359));
  assert(EQ(M_E, 2.718281828459));
  assert(EQ(M_PHI, 1.61803398875));

  double x = -12345.6789;
  assert((-M_INF < x) && (x < M_INF));
  assert((M_INF + x == M_INF) && (M_INF - x == M_INF));
  assert((M_INF + M_INF == M_INF) && (-M_INF - M_INF == -M_INF));
  assert((M_NAN != x) && (M_NAN != M_INF) && (M_NAN != M_NAN));
  assert(!(M_NAN < x) && !(M_NAN > x) && !(M_NAN <= x) && !(M_NAN >= x));
  assert(isnan(0.0*M_INF) && isnan(0.0*-M_INF) && isnan(M_INF/-M_INF));
  assert(isnan(M_NAN) && isnan(-M_NAN) && isnan(M_INF - M_INF));

  assert(sgn(x) == -1 && sgn(0.0) == 0 && sgn(5678) == 1);
  assert(signbit_(x) && !signbit_(0.0) && signbit_(-0.0));
  assert(!signbit_(M_INF) && signbit_(-M_INF));
  assert(!signbit_(M_NAN) && signbit_(-M_NAN));
  assert(copysign(1.0, +2.0) == +1.0 && copysign(M_INF, -2.0) == -M_INF);
  assert(copysign(1.0, -2.0) == -1.0 && std::signbit(copysign(M_NAN, -2.0)));

  assert(EQ(floor0(1.5), 1.0) && EQ(ceil0(1.5), 2.0));
  assert(EQ(floor0(-1.5), -1.0) && EQ(ceil0(-1.5), -2.0));
  assert(EQ(round_half_up(+1.5), +2) && EQ(round_half_down(+1.5), +1));
  assert(EQ(round_half_up(-1.5), -1) && EQ(round_half_down(-1.5), -2));
  assert(EQ(round_half_to0(+1.5), +1) && EQ(round_half_from0(+1.5), +2));
  assert(EQ(round_half_to0(-1.5), -1) && EQ(round_half_from0(-1.5), -2));
  assert(EQ(round_half_even(+1.5), +2) && EQ(round_half_even(-1.5), -2));
  assert(NE(round_half_alternate(+1.5), round_half_alternate(+1.5)));
  assert(NE(round_half_alternate0(-1.5), round_half_alternate0(-1.5)));
  assert(EQ(round_n_places(-1.23456, 3, round_half_to0<double>), -1.235));

  assert(EQ(erf_(1.0), 0.8427007929) && EQ(erf_(-1.0), -0.8427007929));
  assert(EQ(tgamma_(0.5), 1.7724538509) && EQ(tgamma_(1.0), 1.0));
  assert(EQ(lgamma_(0.5), 0.5723649429) && EQ(lgamma_(1.0), 0.0));

  int digits[] = {6, 5, 4, 3, 2, 1};
  std::vector<int> base20 = to_base(123456, 20);
  assert(convert_base(base20, 20, 10) == std::vector<int>(digits, digits + 6));
  assert(to_roman(1234) == "MCCXXXIV");
  assert(to_roman(5678) == "MMMMMDCLXXVIII");
  return 0;
}
\end{lstlisting}

\section{Combinatorics}
\setcounter{section}{2}
\setcounter{subsection}{0}
\subsection{Combinatorial Calculations}
\begin{lstlisting}
/*

The following functions implement common operations in combinatorics. All input
arguments must be non-negative. All return values and table entries are computed
as 64-bit integers modulo an input argument m or p.

- factorial(n, m) returns n! mod m.
- factorialp(n, p) returns n! mod p, where p is prime.
- binomial_table(n, m) returns rows 0 to n of Pascal's triangle as a table t
  such that t[i][j] is equal to (i choose j) mod m.
- permute(n, k, m) returns (n permute k) mod m.
- choose(n, k, p) returns (n choose k) mod p, where p is prime.
- multichoose(n, k, p) returns (n multi-choose k) mod p, where p is prime.
- catalan(n, p) returns the nth Catalan number mod p, where p is prime.
- partitions(n, m) returns the number of partitions of n, mod m.
- partitions(n, k, m) returns the number of partitions of n into k parts, mod m.
- stirling1(n, k, m) returns the (n, k) unsigned Stirling number of the 1st kind
  mod m.
- stirling2(n, k, m) returns the (n, k) Stirling number of the 2nd kind mod m.
- eulerian1(n, k, m) returns the (n, k) Eulerian number of the 1st kind mod m,
  where n > k.
- eulerian2(n, k, m) returns the (n, k) Eulerian number of the 2nd kind mod m,
  where n > k.

Time Complexity:
- O(n) for factorial(n, m).
- O(p log n) for factorialp(n, p).
- O(n^2) for binomial_table(n, m).
- O(k) for permute(n, k, p).
- O(min(k, n - k)) for choose(n, k, p).
- O(k) for multichoose(n, k, p).
- O(n) for catalan(n, p).
- O(n^2) for partitions(n, m).
- O(n*k) for partitions(n, k, m), stirling1(n, k, m), stirling2(n, k, m),
  eulerian1(n, k, m), and eulerian2(n, k, m).

Space Complexity:
- O(n^2) auxiliary heap space for binomial_table(n, m).
- O(n*k) auxiliary heap space for partitions(n, k, m), stirling1(n, k, m),
  stirling2(n, k, m), eulerian1(n, k, m), and eulerian2(n, k, m).
- O(1) auxiliary for all other operations.

*/

#include <vector>

typedef long long int64;
typedef std::vector<std::vector<int64> > table;

int64 factorial(int n, int m = 1000000007) {
  int64 res = 1;
  for (int i = 2; i <= n; i++) {
    res = (res*i) % m;
  }
  return res % m;
}

int64 factorialp(int64 n, int64 p = 1000000007) {
  int64 res = 1;
  while (n > 1) {
    if (n / p % 2 == 1) {
      res = res*(p - 1) % p;
    }
    int max = n % p;
    for (int i = 2; i <= max; i++) {
      res = (res*i) % p;
    }
    n /= p;
  }
  return res % p;
}

table binomial_table(int n, int64 m = 1000000007) {
  table t(n + 1);
  for (int i = 0; i <= n; i++) {
    for (int j = 0; j <= i; j++) {
      if (i < 2 || j == 0 || i == j) {
        t[i].push_back(1);
      } else {
        t[i].push_back((t[i - 1][j - 1] + t[i - 1][j]) % m);
      }
    }
  }
  return t;
}

int64 permute(int n, int k, int64 m = 1000000007) {
  if (n < k) {
    return 0;
  }
  int64 res = 1;
  for (int i = 0; i < k; i++) {
    res = res*(n - i) % m;
  }
  return res % m;
}

int64 mulmod(int64 x, int64 n, int64 m) {
  int64 a = 0, b = x % m;
  for (; n > 0; n >>= 1) {
    if (n & 1) {
      a = (a + b) % m;
    }
    b = (b << 1) % m;
  }
  return a % m;
}

int64 powmod(int64 x, int64 n, int64 m) {
  int64 a = 1, b = x;
  for (; n > 0; n >>= 1) {
    if (n & 1) {
      a = mulmod(a, b, m);
    }
    b = mulmod(b, b, m);
  }
  return a % m;
}

int64 choose(int n, int k, int64 p = 1000000007) {
  if (n < k) {
    return 0;
  }
  if (k > n - k) {
    k = n - k;
  }
  int64 num = 1, den = 1;
  for (int i = 0; i < k; i++) {
    num = num*(n - i) % p;
  }
  for (int i = 1; i <= k; i++) {
    den = den*i % p;
  }
  return num*powmod(den, p - 2, p) % p;
}

int64 multichoose(int n, int k, int64 p = 1000000007) {
  return choose(n + k - 1, k, p);
}

int64 catalan(int n, int64 p = 1000000007) {
  return choose(2*n, n, p)*powmod(n + 1, p - 2, p) % p;
}

int64 partitions(int n, int64 m = 1000000007) {
  std::vector<int64> t(n + 1, 0);
  t[0] = 1;
  for (int i = 1; i <= n; i++) {
    for (int j = i; j <= n; j++) {
      t[j] = (t[j] + t[j - i]) % m;
    }
  }
  return t[n] % m;
}

int64 partitions(int n, int k, int64 m = 1000000007) {
  table t(n + 1, std::vector<int64>(k + 1, 0));
  t[0][1] = 1;
  for (int i = 1; i <= n; i++) {
    for (int j = 1, h = k < i ? k : i; j <= h; j++) {
      t[i][j] = (t[i - 1][j - 1] + t[i - j][j]) % m;
    }
  }
  return t[n][k] % m;
}

int64 stirling1(int n, int k, int64 m = 1000000007) {
  table t(n + 1, std::vector<int64>(k + 1, 0));
  t[0][0] = 1;
  for (int i = 1; i <= n; i++) {
    for (int j = 1; j <= k; j++) {
      t[i][j] = (i - 1)*t[i - 1][j] % m;
      t[i][j] = (t[i][j] + t[i - 1][j - 1]) % m;
    }
  }
  return t[n][k] % m;
}

int64 stirling2(int n, int k, int64 m = 1000000007) {
  table t(n + 1, std::vector<int64>(k + 1, 0));
  t[0][0] = 1;
  for (int i = 1; i <= n; i++) {
    for (int j = 1; j <= k; j++) {
      t[i][j] = j*t[i - 1][j] % m;
      t[i][j] = (t[i][j] + t[i - 1][j - 1]) % m;
    }
  }
  return t[n][k] % m;
}

int64 eulerian1(int n, int k, int64 m = 1000000007) {
  if (k > n - 1 - k) {
    k = n - 1 - k;
  }
  table t(n + 1, std::vector<int64>(k + 1, 1));
  for (int j = 1; j <= k; j++) {
    t[0][j] = 0;
  }
  for (int i = 1; i <= n; i++) {
    for (int j = 1; j <= k; j++) {
      t[i][j] = (i - j)*t[i - 1][j - 1] % m;
      t[i][j] = (t[i][j] + ((j + 1)*t[i - 1][j] % m)) % m;
    }
  }
  return t[n][k] % m;
}

int64 eulerian2(int n, int k, int64 m = 1000000007) {
  table t(n + 1, std::vector<int64>(k + 1, 1));
  for (int i = 1; i <= n; i++) {
    for (int j = 1; j <= k; j++) {
      if (i == j) {
        t[i][j] = 0;
      } else {
        t[i][j] = (j + 1)*t[i - 1][j] % m;
        t[i][j] = ((2*i - 1 - j)*t[i - 1][j - 1] % m + t[i][j]) % m;
      }
    }
  }
  return t[n][k] % m;
}

/*** Example Usage ***/

#include <cassert>

int main() {
  table t = binomial_table(10);
  for (int i = 0; i < (int)t.size(); i++) {
    for (int j = 0; j < (int)t[i].size(); j++) {
      assert(t[i][j] == choose(i, j));
    }
  }
  assert(factorial(10) == 3628800);
  assert(factorialp(123456) == 639390503);
  assert(permute(10, 4) == 5040);
  assert(choose(20, 7) == 77520);
  assert(multichoose(20, 7) == 657800);
  assert(catalan(10) == 16796);
  assert(partitions(4) == 5);
  assert(partitions(100, 5) == 38225);
  assert(stirling1(4, 2) == 11);
  assert(stirling2(4, 3) == 6);
  assert(eulerian1(9, 5) == 88234);
  assert(eulerian2(8, 3) == 195800);
  return 0;
}
\end{lstlisting}
\subsection{Enumerating Arrangements}
\begin{lstlisting}
/*

For the purposes of this section, we define a "size k arrangement of n" to be a
permutation of a size k subset of the integers from 0 to n - 1, for 0 <= k <= n.
There are n permute k possible arrangements, but n^k possible arrangements if
repeated values are allowed.

- next_arragement(n, k, a) tries to rearrange a[] to the next lexicographically
  greater arrangement, returning true if such an arrangement exists or false if
  the array is already in descending order (in which case a[] is unchanged). The
  input a[] must consist of exactly k distinct integers in the range [0, n).
- arrangement_by_rank(n, k, r) returns the size k arrangement of n which is
  lexicographically ranked r out of all size k arrangements of n, where r is
  a zero-based rank in the range [0, n permute k).
- rank_by_arrangement(n, k, a) returns an integer representing the zero-based
  rank of arrangement a[], which must consist of exactly k distinct integers in
  the range [0, n).
- next_arragement_with_repeats(n, k, a) tries to rearrange a[] to the next
  lexicographically greater arrangement with repeats, returning true if such an
  arrangement exists or false if the array is already in descending order (in
  which case a[] is unchanged). The input a[] must consist of exactly k (not
  necessarily distinct) integers in the range [0, n). If a[] were interpreted as
  a k digit integer in base n, this function could be thought of as incrementing
  the integer.

Time Complexity:
- O(n*k) for next_arrangement(), arrangement_by_rank(), and
  rank_by_arrangement().
- O(k) for next_arrangement_with_repeats().

Space Complexity:
- O(n) auxiliary heap space for next_arrangement(), arrangement_by_rank(), and
  rank_by_arrangement().
- O(1) auxiliary for next_arrangement_with_repeats().

*/

#include <algorithm>
#include <vector>

bool next_arrangement(int n, int k, int a[]) {
  std::vector<bool> used(n);
  for (int i = 0; i < k; i++) {
    used[a[i]] = true;
  }
  for (int i = k - 1; i >= 0; i--) {
    used[a[i]] = false;
    for (int j = a[i] + 1; j < n; j++) {
      if (!used[j]) {
        a[i++] = j;
        used[j] = true;
        for (int x = 0; i < k; x++) {
          if (!used[x]) {
            a[i++] = x;
          }
        }
        return true;
      }
    }
  }
  return false;
}

long long n_permute_k(int n, int k) {
  long long res = 1;
  for (int i = 0; i < k; i++) {
    res *= n - i;
  }
  return res;
}

std::vector<int> arrangement_by_rank(int n, int k, long long r) {
  std::vector<int> values(n), res(k);
  for (int i = 0; i < n; i++) {
    values[i] = i;
  }
  for (int i = 0; i < k; i++) {
    long long count = n_permute_k(n - 1 - i, k - 1 - i);
    int pos = r/count;
    res[i] = values[pos];
    std::copy(values.begin() + pos + 1, values.end(), values.begin() + pos);
    r %= count;
  }
  return res;
}

long long rank_by_arrangement(int n, int k, int a[]) {
  long long res = 0;
  std::vector<bool> used(n);
  for (int i = 0; i < k; i++) {
    int count = 0;
    for (int j = 0; j < a[i]; j++) {
      if (!used[j]) {
        count++;
      }
    }
    res += count*n_permute_k(n - i - 1, k - i - 1);
    used[a[i]] = true;
  }
  return res;
}

bool next_arrangement_with_repeats(int n, int k, int a[]) {
  for (int i = k - 1; i >= 0; i--) {
    if (a[i] < n - 1) {
      a[i]++;
      std::fill(a + i + 1, a + k, 0);
      return true;
    }
  }
  return false;
}

/*** Example Usage and Output:

4 permute 3 arrangements:
{0,1,2} {0,1,3} {0,2,1} {0,2,3} {0,3,1} {0,3,2} {1,0,2} {1,0,3} {1,2,0} {1,2,3}
{1,3,0} {1,3,2} {2,0,1} {2,0,3} {2,1,0} {2,1,3} {2,3,0} {2,3,1} {3,0,1} {3,0,2}
{3,1,0} {3,1,2} {3,2,0} {3,2,1}

4^2 arrangements with repeats:
{0,0} {0,1} {0,2} {0,3} {1,0} {1,1} {1,2} {1,3} {2,0} {2,1} {2,2} {2,3} {3,0}
{3,1} {3,2} {3,3}

***/

#include <cassert>
#include <iostream>
using namespace std;

template<class It>
void print_range(It lo, It hi) {
  cout << "{";
  for (; lo != hi; ++lo) {
    cout << *lo << (lo == hi - 1 ? "" : ",");
  }
  cout << "} ";
}

int main() {
  {
    int n = 4, k = 3, a[] = {0, 1, 2};
    cout << n << " permute " << k << " arrangements:" << endl;
    int count = 0;
    do {
      print_range(a, a + k);
      vector<int> b = arrangement_by_rank(n, k, count);
      assert(equal(a, a + k, b.begin()));
      assert(rank_by_arrangement(n, k, a) == count);
      count++;
    } while (next_arrangement(n, k, a));
    cout << endl;
  }
  {
    int n = 4, k = 2, a[] = {0, 0};
    cout << endl << n << "^" << k << " arrangements with repeats:" << endl;
    do {
      print_range(a, a + k);
    } while (next_arrangement_with_repeats(n, k, a));
    cout << endl;
  }
  return 0;
}
\end{lstlisting}
\subsection{Enumerating Permutations}
\begin{lstlisting}
/*

A permutation is an ordered list consisting of n (not necessarily distinct)
elements.

- next_permutation_(lo, hi) is analogous to std::next_permutation(lo, hi),
  taking two BidirectionalIterators lo and hi as a range [lo, hi) for which the
  function tries to rearrange to the next lexicographically greater permutation.
  The function returns true if such a permutation exists, or false if the range
  is already in descending order (in which case the values are unchanged). This
  implementation requires an ordering on the set of possible elements defined by
  the < operator on the iterator's value type.
- next_permutation(n, a) is analogous to next_permutation(), except that it
  takes an array a[] of size n instead of a range.
- next_permutation(x) returns the next lexicographically greater permutation of
  the binary digits of the integer x, that is, the lowest integer greater than
  x with the same number of 1-bits. This can be used to generate combinations of
  a set of n items by treating each 1 bit as whether to "take" the item at the
  corresponding position.
- permutation_by_rank(n, r) returns the permutation of the integers in the range
  [0, n) which is lexicographically ranked r, where r is a zero-based rank in
  the range [0, n!).
- rank_by_permutation(n, a) returns an integer representing the zero-based
  rank of permutation a[], which must be a permutation of the integers [0, n).
- permutation_cycles(n, a) returns the decomposition of the permutation a[] into
  cycles. A permutation cycle is a subset of a permutation whose elements are
  consecutively swapped, relative to a sorted set. For example, {3, 1, 0, 2}
  decomposes to {0, 3, 2} and {1}, meaning that starting from the sorted order
  {0, 1, 2, 3}, the 0th value is replaced by the 3rd, the 3rd by the 2nd, and
  the 2nd by the 0th (0 -> 3 -> 2 -> 0).

Time Complexity:
- O(n^2) per call to next_permutation_(lo, hi), where n is the distance between
  lo and hi.
- O(n^2) per call to next_permutation(n, a), permutation_by_rank(n, r), and
  rank_by_permutation(n, a).
- O(1) per call to next_permutation(x).
- O(n) per call to permutation_cycles().

Space Complexity:
- O(1) auxiliary for next_permutation_() and next_permutation().
- O(n) auxiliary heap space for permutation_by_rank(), rank_by_permutation(),
  and permutation_cycles().

*/

#include <algorithm>
#include <vector>

template<class It>
bool next_permutation_(It lo, It hi) {
  if (lo == hi) {
    return false;
  }
  It i = lo;
  if (++i == hi) {
    return false;
  }
  i = hi;
  --i;
  for (;;) {
    It j = i;
    if (*--i < *j) {
      It k = hi;
      while (!(*i < *--k)) {}
      std::iter_swap(i, k);
      std::reverse(j, hi);
      return true;
    }
    if (i == lo) {
      std::reverse(lo, hi);
      return false;
    }
  }
}

template<class T>
bool next_permutation(int n, T a[]) {
  for (int i = n - 2; i >= 0; i--) {
    if (a[i] < a[i + 1]) {
      for (int j = n - 1; ; j--) {
        if (a[i] < a[j]) {
          std::swap(a[i++], a[j]);
          for (j = n - 1; i < j; i++, j--) {
            std::swap(a[i], a[j]);
          }
          return true;
        }
      }
    }
  }
  return false;
}

long long next_permutation(long long x) {
  long long s = x & -x, r = x + s;
  return r | (((x ^ r) >> 2)/s);
}

std::vector<int> permutation_by_rank(int n, long long x) {
  std::vector<long long> factorial(n);
  std::vector<int> values(n), res(n);
  factorial[0] = 1;
  for (int i = 1; i < n; i++) {
    factorial[i] = i*factorial[i - 1];
  }
  for (int i = 0; i < n; i++) {
    values[i] = i;
  }
  for (int i = 0; i < n; i++) {
    int pos = x/factorial[n - 1 - i];
    res[i] = values[pos];
    std::copy(values.begin() + pos + 1, values.end(), values.begin() + pos);
    x %= factorial[n - 1 - i];
  }
  return res;
}

long long rank_by_permutation(int n, int a[]) {
  std::vector<long long> factorial(n);
  factorial[0] = 1;
  for (int i = 1; i < n; i++) {
    factorial[i] = i*factorial[i - 1];
  }
  long long res = 0;
  for (int i = 0; i < n; i++) {
    int v = a[i];
    for (int j = 0; j < i; j++) {
      if (a[j] < a[i]) {
        v--;
      }
    }
    res += v*factorial[n - 1 - i];
  }
  return res;
}

typedef std::vector<std::vector<int> > cycles;

cycles permutation_cycles(int n, int a[]) {
  std::vector<bool> visit(n);
  cycles res;
  for (int i = 0; i < n; i++) {
    if (!visit[i]) {
      int j = i;
      std::vector<int> curr;
      do {
        curr.push_back(j);
        visit[j] = true;
        j = a[j];
      } while (j != i);
      res.push_back(curr);
    }
  }
  return res;
}

/*** Example Usage and Output:

Permutations of [0, 4):
{0,1,2,3} {0,1,3,2} {0,2,1,3} {0,2,3,1} {0,3,1,2} {0,3,2,1} {1,0,2,3} {1,0,3,2}
{1,2,0,3} {1,2,3,0} {1,3,0,2} {1,3,2,0} {2,0,1,3} {2,0,3,1} {2,1,0,3} {2,1,3,0}
{2,3,0,1} {2,3,1,0} {3,0,1,2} {3,0,2,1} {3,1,0,2} {3,1,2,0} {3,2,0,1} {3,2,1,0}

Permutations of 2 zeros and 3 ones:
00111 01011 01101 01110 10011 10101 10110 11001 11010 11100

Decomposition of {3,1,0,2} into cycles:
{0,3,2} {1}

***/

#include <bitset>
#include <cassert>
#include <iostream>
using namespace std;

template<class It>
void print_range(It lo, It hi) {
  cout << "{";
  for (; lo != hi; ++lo) {
    cout << *lo << (lo == hi - 1 ? "" : ",");
  }
  cout << "} ";
}

int main() {
  {
    const int n = 4;
    int a[] = {0, 1, 2, 3}, b[n], c[n];
    for (int i = 0; i < n; i++) {
      b[i] = c[i] = a[i];
    }
    cout << "Permutations of [0, " << n << "):" << endl;
    int count = 0;
    do {
      print_range(a, a + n);
      assert(equal(b, b + n, a));
      assert(equal(c, c + n, a));
      vector<int> d = permutation_by_rank(n, count);
      assert(equal(d.begin(), d.end(), a));
      assert(rank_by_permutation(n, a) == count);
      count++;
      std::next_permutation(b, b + n);
      next_permutation(c, c + n);
    } while (next_permutation(n, a));
    cout << endl;
  }
  { // Permutations of binary digits.
    const int n = 5;
    cout << "\nPermutations of 2 zeros and 3 ones:" << endl;
    int lo = bitset<5>(string("00111")).to_ulong();
    int hi = bitset<6>(string("100011")).to_ulong();
    do {
      cout << bitset<n>(lo).to_string() << " ";
    } while ((lo = next_permutation(lo)) != hi);
    cout << endl;
  }
  { // Decomposition into cycles.
    const int n = 4;
    int a[] = {3, 1, 0, 2};
    cout << "\nDecomposition of {3,1,0,2} into cycles:" << endl;
    cycles c = permutation_cycles(n, a);
    for (int i = 0; i < (int)c.size(); i++) {
      print_range(c[i].begin(), c[i].end());
    }
    cout << endl;
  }
  return 0;
}
\end{lstlisting}
\subsection{Enumerating Combinations}
\begin{lstlisting}
/*

A combination is a subset of size k chosen from a total of n (not necessarily
distinct) elements, where order does not matter.

- next_combination(lo, mid, hi) takes random-access iterators lo, mid, and hi
  as a range [lo, hi) of n elements for which the function will rearrange such
  that the k elements in [lo, mid) becomes the next lexicographically greater
  combination. The function returns true if such a combination exists, or false
  if [lo, mid) already consists of the lexicographically greatest combination
  of the elements in [lo, hi) (in which case the values are unchanged). This
  implementation requires an ordering on the set of possible elements defined by
  the < operator on the iterator's value type.
- next_combination(n, k, a) rearranges a[] to become the next lexicographically
  greater combination of k distinct integers in the range [0, n). The array a[]
  must consist of k distinct integers in the range [0, n).
- next_combination_mask(x) interprets the bits of an integer x as a mask with
  1-bits specifying the chosen items for a combination and returns the mask of
  the next lexicographically greater combination (that is, the lowest integer
  greater than x with the same number of 1 bits). Note that this does not
  generate combinations in the same order as next_combination(), nor does it
  work if the corresponding n items are not distinct (in that case, duplicate
  combinations will be generated).
- combination_by_rank(n, k, r) returns the combination of k distinct integers in
  the range [0, n) that is lexicographically ranked r, where r is a zero-based
  rank in the range [0, n choose k).
- rank_by_combination(n, k, a) returns an integer representing the zero-based
  rank of combination a[], which must consist of k distinct integers in [0, n).
- next_combination_with_repeats(n, k, a) rearranges a[] to become the next
  lexicographically greater combination of k (not necessarily distinct) integers
  in the range [0, n). The array a[] must consist of k integers in the range
  [0, n). Note that there is a total of n multichoose k combinations if
  repetition is allowed, where n multichoose k = (n + k - 1) choose k.

Time Complexity:
- O(n) per call to next_combination(lo, hi), where n is the distance between
  lo and hi.
- O(k) per call to next_combination(n, k, a) and
  next_combination_with_repeats(n, k, a).
- O(1) per call to next_combination_mask(x).
- O(n*k) per call to combination_by_rank() and rank_by_combination().

Space Complexity:
- O(k) auxiliary heap space for combination_by_rank(n, k, r).
- O(1) auxiliary for all other operations.

*/

#include <algorithm>
#include <iterator>
#include <vector>

template<class It>
bool next_combination(It lo, It mid, It hi) {
  if (lo == mid || mid == hi) {
    return false;
  }
  It l = mid - 1, h = hi - 1;
  int len1 = 1, len2 = 1;
  while (l != lo && !(*l < *h)) {
    --l;
    ++len1;
  }
  if (l == lo && !(*l < *h)) {
    std::rotate(lo, mid, hi);
    return false;
  }
  for (; mid < h; ++len2) {
    if (!(*l < *--h)) {
      ++h;
      break;
    }
  }
  if (len1 == 1 || len2 == 1) {
    std::iter_swap(l, h);
  } else if (len1 == len2) {
    std::swap_ranges(l, mid, h);
  } else {
    std::iter_swap(l++, h++);
    int total = (--len1) + (--len2), gcd = total;
    for (int i = len1; i != 0; ) {
      std::swap(gcd %= i, i);
    }
    int skip = total/gcd - 1;
    for (int i = 0; i < gcd; i++) {
      It curr = (i < len1) ? (l + i) : (h + (i - len1));
      int k = i;
      typename std::iterator_traits<It>::value_type prev(*curr);
      for (int j = 0; j < skip; j++) {
        k = (k + len1) % total;
        It next = (k < len1) ? (l + k) : (h + (k - len1));
        *curr = *next;
        curr = next;
      }
      *curr = prev;
    }
  }
  return true;
}

bool next_combination(int n, int k, int a[]) {
  for (int i = k - 1; i >= 0; i--) {
    if (a[i] < n - k + i) {
      a[i]++;
      while (++i < k) {
        a[i] = a[i - 1] + 1;
      }
      return true;
    }
  }
  return false;
}

long long next_combination_mask(long long x) {
  long long s = x & -x, r = x + s;
  return r | (((x ^ r) >> 2)/s);
}

long long n_choose_k(long long n, long long k) {
  if (k > n - k) {
    k = n - k;
  }
  long long res = 1;
  for (int i = 0; i < k; i++) {
    res = res*(n - i)/(i + 1);
  }
  return res;
}

std::vector<int> combination_by_rank(int n, int k, long long r) {
  std::vector<int> res(k);
  int count = n;
  for (int i = 0; i < k; i++) {
    int j = 1;
    for (;; j++) {
      long long am = n_choose_k(count - j, k - 1 - i);
      if (r < am) {
        break;
      }
      r -= am;
    }
    res[i] = (i > 0) ? (res[i - 1] + j) : (j - 1);
    count -= j;
  }
  return res;
}

long long rank_by_combination(int n, int k, int a[]) {
  long long res = 0;
  int prev = -1;
  for (int i = 0; i < k; i++) {
    for (int j = prev + 1; j < a[i]; j++) {
      res += n_choose_k(n - 1 - j, k - 1 - i);
    }
    prev = a[i];
  }
  return res;
}

bool next_combination_with_repeats(int n, int k, int a[]) {
  for (int i = k - 1; i >= 0; i--) {
    if (a[i] < n - 1) {
      for (++a[i]; ++i < k; ) {
        a[i] = a[i - 1];
      }
      return true;
    }
  }
  return false;
}

/*** Example Usage and Output:

"11234" choose 3:
112 113 114 123 124 134 234

"abcde" choose 3 with masks:
abc abd acd bcd abe ace bce ade bde

5 choose 3:
{0,1,2} {0,1,3} {0,1,4} {0,2,3} {0,2,4} {0,3,4} {1,2,3} {1,2,4} {1,3,4} {2,3,4}

3 multichoose 2:
{0,0} {0,1} {0,2} {1,1} {1,2} {2,2}

***/

#include <cassert>
#include <iostream>
using namespace std;

template<class It>
void print_range(It lo, It hi) {
  cout << "{";
  for (; lo != hi; ++lo) {
    cout << *lo << (lo == hi - 1 ? "" : ",");
  }
  cout << "} ";
}

int main() {
  {
    int k = 3;
    string s = "11234";
    cout << "\"" << s << "\" choose " << k << ":" << endl;
    do {
      cout << s.substr(0, k) << " ";
    } while (next_combination(s.begin(), s.begin() + k, s.end()));
    cout << endl;
  }
  { // Unordered combinations using masks.
    int n = 5, k = 3;
    string char_set = "abcde";  // Must be distinct.
    cout << "\n\"" << char_set << "\" choose " << k << " with masks:" << endl;
    long long mask = 0, dest = 0;
    for (int i = 0; i < k; i++) {
      mask |= (1 << i);
    }
    for (int i = k - 1; i < n; i++) {
      dest |= (1 << i);
    }
    do {
      for (int i = 0; i < n; i++) {
        if ((mask >> i) & 1) {
          cout << char_set[i];
        }
      }
      cout << " ";
      mask = next_combination_mask(mask);
    } while (mask != dest);
    cout << endl;
  }
  { // Combinations of distinct integers from 0 to n - 1.
    int n = 5, k = 3, a[] = {0, 1, 2};
    cout << "\n" << n << " choose " << k << ":" << endl;
    int count = 0;
    do {
      print_range(a, a + k);
      vector<int> b = combination_by_rank(n, k, count);
      assert(equal(a, a + k, b.begin()));
      assert(rank_by_combination(n, k, a) == count);
      count++;
    } while (next_combination(n, k, a));
    cout << endl;
  }
  { // Combinations with repeats.
    int n = 3, k = 2, a[] = {0, 0};
    cout << "\n" << n << " multichoose " << k << ":" << endl;
    do {
      print_range(a, a + k);
    } while (next_combination_with_repeats(n, k, a));
    cout << endl;
  }
  return 0;
}
\end{lstlisting}
\subsection{Enumerating Partitions}
\begin{lstlisting}
/*

A partition of a natural number n is a way to write n as a sum of positive
integers where the order of the addends does not matter.

- next_partition(p) takes a reference to a vector p[] of positive integers as a
  partition of n for which the function will re-assign to become the next
  lexicographically greater partition. The function returns true if such a
  partition exists, or false if p[] already consists of the lexicographically
  greatest partition (i.e. the single integer n).
- partition_by_rank(n, r) returns the partition of n that is lexicographically
  ranked r if addends in each partition were sorted in non-increasing order,
  where r is a zero-based rank in the range [0, partitions(n)).
- rank_by_partition(p) returns an integer representing the zero-based rank of
  the partition specified by vector p[], which must consist of positive integers
  sorted in non-increasing order.
- generate_increasing_partitions(n, f) calls the function f(lo, hi) on strictly
  increasing partitions of n in lexicographically increasing order of partition,
  where lo and hi are random-access iterators to a range [lo, hi) of integers.
  Note that non-strictly increasing partitions like {1, 1, 1, 1} are skipped.

Time Complexity:
- O(n) per call to next_partition().
- O(n^2) per call to partition_by_rank(n, r) and rank_by_partition(p).
- O(p(n)) per call to generate_increasing_partitions(n, f), where p(n) is the
  number of partitions of n.

Space Complexity:
- O(1) auxiliary for next_partition().
- O(n^2) auxiliary heap space for partition_function(), partition_by_rank(), and
  rank_by_partition().
- O(n) auxiliary stack space for generate_increasing_partitions().

*/

#include <vector>

bool next_partition(std::vector<int> &p) {
  int n = p.size();
  if (n <= 1) {
    return false;
  }
  int s = p[n - 1] - 1, i = n - 2;
  p.pop_back();
  for (; i > 0 && p[i] == p[i - 1]; i--) {
    s += p[i];
    p.pop_back();
  }
  for (p[i]++; s > 0; s--) {
    p.push_back(1);
  }
  return true;
}

long long partition_function(int a, int b) {
  static std::vector<std::vector<long long> > p(
      1, std::vector<long long>(1, 1));
  if (a >= (int)p.size()) {
    int old = p.size();
    p.resize(a + 1);
    p[0].resize(a + 1);
    for (int i = 1; i <= a; i++) {
      p[i].resize(a + 1);
      for (int j = old; j <= i; j++) {
        p[i][j] = p[i - 1][j - 1] + p[i - j][j];
      }
    }
  }
  return p[a][b];
}

std::vector<int> partition_by_rank(int n, long long r) {
  std::vector<int> res;
  for (int i = n, j; i > 0; i -= j) {
    for (j = 1; ; j++) {
      long long count = partition_function(i, j);
      if (r < count) {
        break;
      }
      r -= count;
    }
    res.push_back(j);
  }
  return res;
}

long long rank_by_partition(const std::vector<int> &p) {
  long long res = 0;
  int sum = 0;
  for (int i = 0; i < (int)p.size(); i++) {
    sum += p[i];
  }
  for (int i = 0; i < (int)p.size(); i++) {
    for (int j = 0; j < p[i]; j++) {
      res += partition_function(sum, j);
    }
    sum -= p[i];
  }
  return res;
}

typedef void (*ReportFunction)(std::vector<int>::iterator,
                               std::vector<int>::iterator);

void generate_increasing_partitions(int left, int prev, int i,
                                    std::vector<int> &p, ReportFunction f) {
  if (left == 0) {
    f(p.begin(), p.begin() + i);
    return;
  }
  for (p[i] = prev + 1; p[i] <= left; p[i]++) {
    generate_increasing_partitions(left - p[i], p[i], i + 1, p, f);
  }
}

void generate_increasing_partitions(int n, ReportFunction f) {
  std::vector<int> p(n, 0);
  generate_increasing_partitions(n, 0, 0, p, f);
}

/*** Example Usage and Output:

Partitions of 4:
{1,1,1,1} {2,1,1} {2,2} {3,1} {4}

Increasing partitions of 8:
{1,2,5} {1,3,4} {1,7} {2,6} {3,5} {8}

***/

#include <cassert>
#include <iostream>
using namespace std;

template<class It>
void print_range(It lo, It hi) {
  cout << "{";
  for (; lo != hi; ++lo) {
    cout << *lo << (lo == hi - 1 ? "" : ",");
  }
  cout << "} ";
}

int main() {
  {
    int n = 4;
    vector<int> a(n, 1);
    cout << "Partitions of " << n << ":" << endl;
    int count = 0;
    do {
      print_range(a.begin(), a.end());
      vector<int> b = partition_by_rank(n, count);
      assert(equal(a.begin(), a.end(), b.begin()));
      assert(rank_by_partition(a) == count);
      count++;
    } while (next_partition(a));
    cout << endl;
  }
  {
    int n = 8;
    cout << "\nIncreasing partitions of " << n << ":" << endl;
    generate_increasing_partitions(n, print_range);
    cout << endl;
  }
  return 0;
}
\end{lstlisting}
\subsection{Enumerating Generic Combinatorial Sequences}
\begin{lstlisting}
/*

Enumerate combinatorial sequence by inheriting an abstract class. Child classes
of abstract_enumerator must implement the count() function which should return
the number of combinatorial sequences starting with the given prefix.

- to_rank(a) returns an integer representing the zero-based rank of the
  combinatorial sequence a.
- from_rank(r) returns a combinatorial sequence of integers that is
  lexicographically ranked r, where r is a zero-based rank in the range
  [0, total_count()).
- enumerate(f) calls the function f(lo, hi) on every specified combinatorial
  sequence in lexicographically increasing order, where lo and hi are two
  random-access iterators to a range [lo, hi) of integers.

Time Complexity:
- O(n^2) calls will be made to count() per call to all operations, where n is
  the length of the combinatorial sequence.

Space Complexity:
- O(n) auxiliary heap space per call to all operations.

*/

#include <vector>

class abstract_enumerator {
 protected:
  int range, length;

  abstract_enumerator(int r, int l) : range(r), length(l) {}

  virtual long long count(const std::vector<int> &prefix) {
    return 0;
  }

  std::vector<int> next(std::vector<int> &a) {
    return from_rank(to_rank(a) + 1);
  }

  long long total_count() {
    return count(std::vector<int>(0));
  }

 public:
  long long to_rank(const std::vector<int> &a) {
    long long res = 0;
    for (int i = 0; i < (int)a.size(); i++) {
      std::vector<int> prefix(a.begin(), a.end());
      prefix.resize(i + 1);
      for (prefix[i] = 0; prefix[i] < a[i]; prefix[i]++) {
        res += count(prefix);
      }
    }
    return res;
  }

  std::vector<int> from_rank(long long r) {
    std::vector<int> a(length);
    for (int i = 0; i < (int)a.size(); i++) {
      std::vector<int> prefix(a.begin(), a.end());
      prefix.resize(i + 1);
      for (prefix[i] = 0; prefix[i] < range; ++prefix[i]) {
        long long curr = count(prefix);
        if (r < curr) {
          break;
        }
        r -= curr;
      }
      a[i] = prefix[i];
    }
    return a;
  }

  void enumerate(void (*f)(std::vector<int>::iterator,
                           std::vector<int>::iterator)) {
    long long total = total_count();
    for (long long i = 0; i < total; i++) {
      std::vector<int> curr = from_rank(i);
      f(curr.begin(), curr.end());
    }
  }
};

class arrangement_enumerator : public abstract_enumerator {
 public:
  arrangement_enumerator(int n, int k) : abstract_enumerator(n, k) {}

  long long count(const std::vector<int> &prefix) {
    int n = prefix.size();
    for (int i = 0; i < n - 1; i++) {
      if (prefix[i] == prefix[n - 1]) {
        return 0;
      }
    }
    long long res = 1;
    for (int i = 0; i < length - n; i++) {
      res *= range - n - i;
    }
    return res;
  }
};

class permutation_enumerator : public arrangement_enumerator {
 public:
  permutation_enumerator(int n) : arrangement_enumerator(n, n) {}
};

class combination_enumerator : public abstract_enumerator {
  std::vector<std::vector<long long> > table;

 public:
  combination_enumerator(int n, int k)
      : abstract_enumerator(n, k), table(n + 1, std::vector<long long>(n + 1)) {
    for (int i = 0; i <= n; i++) {
      for (int j = 0; j <= i; j++) {
        table[i][j] = (j == 0) ? 1 : table[i - 1][j - 1] + table[i - 1][j];
      }
    }
  }

  long long count(const std::vector<int> &prefix) {
    int n = prefix.size();
    if (n >= 2 && prefix[n - 1] <= prefix[n - 2]) {
      return 0;
    }
    if (n == 0) {
      return table[range][length - n];
    }
    return table[range - prefix[n - 1] - 1][length - n];
  }
};

class partition_enumerator : public abstract_enumerator {
  std::vector<std::vector<long long> > table;

 public:
  partition_enumerator(int n) : abstract_enumerator(n + 1, n),
                                table(n + 1, std::vector<long long>(n + 1)) {
    std::vector<std::vector<long long> > tmp(table);
    tmp[0][0] = 1;
    for (int i = 1; i <= n; i++) {
      for (int j = 1; j <= i; j++) {
        tmp[i][j] = tmp[i - 1][j - 1] + tmp[i - j][j];
      }
    }
    for (int i = 1; i <= n; i++) {
      for (int j = 1; j <= n; j++) {
        table[i][j] = tmp[i][j] + table[i][j - 1];
      }
    }
  }

  long long count(const std::vector<int> &prefix) {
    int n = (int)prefix.size(), sum = 0;
    for (int i = 0; i < n; i++) {
      sum += prefix[i];
    }
    if (sum == range - 1) {
      return 1;
    }
    if (sum > range - 1 || (n > 0 && prefix[n - 1] == 0) ||
        (n >= 2 && prefix[n - 1] > prefix[n - 2])) {
      return 0;
    }
    if (n == 0) {
      return table[range - sum - 1][range - 1];
    }
    return table[range - sum - 1][prefix[n - 1]];
  }
};

/*** Example Usage and Output:

3 permute 2 arrangements:
{0,1} {0,2} {1,0} {1,2} {2,0} {2,1}

Permutatations of [0, 3):
{0,1,2} {0,2,1} {1,0,2} {1,2,0} {2,0,1} {2,1,0}

4 choose 3 combinations:
{0,1,2} {0,1,3} {0,2,3} {1,2,3}

Partition of 4:
{1,1,1,1} {2,1,1,0} {2,2,0,0} {3,1,0,0} {4,0,0,0}

***/

#include <iostream>
using namespace std;

template<class It>
void print_range(It lo, It hi) {
  cout << "{";
  for (; lo != hi; ++lo) {
    cout << *lo << (lo == hi - 1 ? "" : ",");
  }
  cout << "} ";
}

int main() {
  {
    cout << "3 permute 2 arrangement_enumerator:" << endl;
    arrangement_enumerator arr(3, 2);
    arr.enumerate(print_range);
    cout << endl;
  }
  {
    cout << "\nPermutatations of [0, 3):" << endl;
    permutation_enumerator perm(3);
    perm.enumerate(print_range);
    cout << endl;
  }
  {
    cout << "\n4 choose 3 combinations:" << endl;
    combination_enumerator comb(4, 3);
    comb.enumerate(print_range);
    cout << endl;
  }
  {
    cout << "\nPartition of 4:" << endl;
    partition_enumerator part(4);
    part.enumerate(print_range);
    cout << endl;
  }
  return 0;
}
\end{lstlisting}

\section{Number Theory}
\setcounter{section}{3}
\setcounter{subsection}{0}
\subsection{GCD, LCM, Mod Inverse, Chinese Remainder}
\begin{lstlisting}
/*

Common number theory operations relating to modular arithmetic.

- gcd(a, b) and gcd2(a, b) both return the greatest common division of a and b
  using the Euclidean algorithm.
- lcm(a, b) returns the lowest common multiple of a and b.
- extended_euclid(a, b) and extended_euclid2(a, b) both return a pair (x, y) of
  integers such that gcd(a, b) = a*x + b*y.
- mod(a, b) returns the value of a mod b under the true Euclidean definition of
  modulo, that is, the smallest nonnegative integer m satisfying a + b*n = m for
  some integer n. Note that this is identical to the remainder operator % in C++
  for nonnegative operands a and b, but the result will differ when an operand
  is negative.
- mod_inverse(a, m) and mod_inverse2(a, m) both return an integer x such that
  a*x = 1 (mod m), where the arguments must satisfy m > 0 and gcd(a, m) = 1.
- generate_inverse(p) returns a vector of integers where for each index i in
  the vector, i*v[i] = 1 (mod p), where the argument p is prime.
- simple_restore(n, a, p) and garner_restore(n, a, p) both return the solution x
  for the system of simultaneous congruences x = a[i] (mod p[i]) for all indices
  i in [0, n), where p[] consist of pairwise coprime integers. The solution x is
  guaranteed to be unique by the Chinese remainder theorem.

Time Complexity:
- O(log(a + b)) per call to gcd(a, b), gcd2(a, b), lcm(a, b),
  extended_euclid(a, b), extended_euclid2(a, b), mod_inverse(a, b), and
  mod_inverse2(a, b).
- O(1) for mod(a, b).
- O(p) for generate_inverse(p).
- Exponential for simple_restore(n, a, p).
- O(n^2) for garner_restore(n, a, p).

Space Complexity:
- O(log(a + b)) auxiliary stack space for gcd2(a, b), extended_euclid2(a, b),
  and mod_inverse2(a, b).
- O(p) auxiliary heap space for generate_inverse(p).
- O(n) auxiliary heap space for garner_restore(n, a, p).
- O(1) auxiliary space for all other operations.

*/

#include <cstdlib>
#include <utility>
#include <vector>

template<class Int>
Int gcd(Int a, Int b) {
  while (b != 0) {
    Int t = b;
    b = a % b;
    a = t;
  }
  return abs(a);
}

template<class Int>
Int gcd2(Int a, Int b) {
  return (b == 0) ? abs(a) : gcd(b, a % b);
}

template<class Int>
Int lcm(Int a, Int b) {
  return abs(a / gcd(a, b) * b);
}

template<class Int>
std::pair<Int, Int> extended_euclid(Int a, Int b) {
  Int x = 1, y = 0, x1 = 0, y1 = 1;
  while (b != 0) {
    Int q = a/b, prev_x1 = x1, prev_y1 = y1, prev_b = b;
    x1 = x - q*x1;
    y1 = y - q*y1;
    b = a - q*b;
    x = prev_x1;
    y = prev_y1;
    a = prev_b;
  }
  return (a > 0) ? std::make_pair(x, y) : std::make_pair(-x, -y);
}

template<class Int>
std::pair<Int, Int> extended_euclid2(Int a, Int b) {
  if (b == 0) {
    return (a > 0) ? std::make_pair(1, 0) : std::make_pair(-1, 0);
  }
  std::pair<Int, Int> r = extended_euclid2(b, a % b);
  return std::make_pair(r.second, r.first - a/b*r.second);
}

template<class Int>
Int mod(Int a, Int m) {
  Int r = a % m;
  return (r >= 0) ? r : (r + m);
}

template<class Int>
Int mod_inverse(Int a, Int m) {
  a = mod(a, m);
  return (a == 0) ? 0 : mod((1 - m*mod_inverse(m % a, a)) / a, m);
}

template<class Int>
Int mod_inverse2(Int a, Int m) {
  return mod(extended_euclid(a, m).first, m);
}

std::vector<int> generate_inverse(int p) {
  std::vector<int> res(p);
  res[1] = 1;
  for (int i = 2; i < p; i++) {
    res[i] = (p - (p / i)*res[p % i] % p) % p;
  }
  return res;
}

long long simple_restore(int n, int a[], int p[]) {
  long long res = 0, m = 1;
  for (int i = 0; i < n; i++) {
    while (res % p[i] != a[i]) {
      res += m;
    }
    m *= p[i];
  }
  return res;
}

long long garner_restore(int n, int a[], int p[]) {
  std::vector<int> x(a, a + n);
  for (int i = 0; i < n; i++) {
    for (int j = 0; j < i; j++) {
      x[i] = mod_inverse((long long)p[j], (long long)p[i])*(x[i] - x[j]);
    }
    x[i] = (x[i] % p[i] + p[i]) % p[i];
  }
  long long res = x[0], m = 1;
  for (int i = 1; i < n; i++) {
    m *= p[i - 1];
    res += x[i] * m;
  }
  return res;
}

/*** Example Usage ***/

#include <cassert>
using namespace std;

int main() {
  {
    for (int steps = 0; steps < 10000; steps++) {
      int a = rand() % 200 - 10, b = rand() % 200 - 10, g = gcd(a, b);
      assert(g == gcd2(a, b));
      if (g == 1 && b > 1) {
        int inv = mod_inverse(a, b);
        assert(inv == mod_inverse2(a, b) && mod(a*inv, b) == 1);
      }
      pair<int, int> res = extended_euclid(a, b);
      assert(res == extended_euclid2(a, b));
      assert(g == a*res.first + b*res.second);
    }
  }
  {
    int p = 17;
    std::vector<int> res = generate_inverse(p);
    for (int i = 0; i < p; i++) {
      if (i > 0) {
        assert(mod(i*res[i], p) == 1);
      }
    }
  }
  {
    int n = 3, a[] = {2, 3, 1}, m[] = {3, 4, 5};
    int x = simple_restore(n, a, m);
    assert(x == garner_restore(n, a, m));
    for (int i = 0; i < n; i++) {
      assert(mod(x, m[i]) == a[i]);
    }
    assert(x == 11);
  }
  return 0;
}
\end{lstlisting}
\subsection{Prime Generation}
\begin{lstlisting}
/*

Generate prime numbers using the Sieve of Eratosthenes.

- sieve(n) returns a vector of all the primes less than or equal to n.
- sieve(lo, hi) returns a vector of all the primes in the range [lo, hi].

Time Complexity:
- O(n log(log(n))) per call to sieve(n).
- O(sqrt(hi)*log(log(hi - lo))) per call to sieve(lo, hi).

Space Complexity:
- O(n) auxiliary heap space per call to sieve(n).
- O(hi - lo + sqrt(hi)) auxiliary heap space per call to sieve(lo, hi).

*/

#include <cmath>
#include <vector>

std::vector<int> sieve(int n) {
  std::vector<bool> prime(n + 1, true);
  int sqrtn = ceil(sqrt(n));
  for (int i = 2; i <= sqrtn; i++) {
    if (prime[i]) {
      for (int j = i*i; j <= n; j += i) {
        prime[j] = false;
      }
    }
  }
  std::vector<int> res;
  for (int i = 2; i <= n; i++) {
    if (prime[i]) {
      res.push_back(i);
    }
  }
  return res;
}

std::vector<int> sieve(int lo, int hi) {
  int sqrt_hi = ceil(sqrt(hi)), fourth_root_hi = ceil(sqrt(sqrt_hi));
  std::vector<bool> prime1(sqrt_hi + 1, true), prime2(hi - lo + 1, true);
  for (int i = 2; i <= fourth_root_hi; i++) {
    if (prime1[i]) {
      for (int j = i*i; j <= sqrt_hi; j += i) {
        prime1[j] = false;
      }
    }
  }
  for (int i = 2, n = hi - lo; i <= sqrt_hi; i++) {
    if (prime1[i]) {
      for (int j = (lo / i)*i - lo; j <= n; j += i) {
        if (j >= 0 && j + lo != i) {
          prime2[j] = false;
        }
      }
    }
  }
  std::vector<int> res;
  for (int i = (lo > 1) ? lo : 2; i <= hi; i++) {
    if (prime2[i - lo]) {
      res.push_back(i);
    }
  }
  return res;
}

/*** Example Usage and Output:

sieve(n=10000000): 0.059s
atkins(n=10000000): 0.08s
sieve([1000000000, 1005000000]): 0.034s

***/

#include <ctime>
#include <iostream>
using namespace std;

int main() {
  int pmax = 10000000;
  vector<int> p;
  time_t start;
  double delta;

  start = clock();
  p = sieve(pmax);
  delta = (double)(clock() - start)/CLOCKS_PER_SEC;
  cout << "sieve(n=" << pmax << "): " << delta << "s" << endl;

  int l = 1000000000, h = 1005000000;
  start = clock();
  p = sieve(l, h);
  delta = (double)(clock() - start)/CLOCKS_PER_SEC;
  cout << "sieve([" << l << ", " << h << "]): " << delta << "s" << endl;
  return 0;
}
\end{lstlisting}
\subsection{Primality Testing}
\begin{lstlisting}
/*

Determine whether an integer n is prime. This can be done deterministically by
testing all numbers under sqrt(n) using trial division, probabilistically using
the Miller-Rabin test, or deterministically using the Miller-Rabin test if the
maximum input is known (2^63 - 1 for the purposes here).

- is_prime(n) returns whether the integer n is prime using an optimized trial
  division technique based on the fact that all primes greater than 6 must take
  the form 6n + 1 or 6n - 1.
- is_probable_prime(n, k) returns true if the integer n is prime, or false with
  an error probability of (1/4)^k if n is composite. In other words, the result
  is guaranteed to be correct if n is prime, but could be wrong with probability
  (1/4)^k if n is composite. This implementation uses uses exponentiation by
  squaring to support all signed 64-bit integers (up to and including 2^63 - 1).
- is_prime_fast(n) returns whether the signed 64-bit integer n is prime using
  a fully deterministic version of the Miller-Rabin test.

Time Complexity:
- O(sqrt n) per call to is_prime(n).
- O(k log^3(n)) per call to is_probable_prime(n, k).
- O(log^3(n)) per call to is_prime_fast(n).

Space Complexity:
- O(1) auxiliary space for all operations.

*/

#include <cstdlib>

template<class Int>
bool is_prime(Int n) {
  if (n == 2 || n == 3) {
    return true;
  }
  if (n < 2 || n % 2 == 0 || n % 3 == 0) {
    return false;
  }
  for (Int i = 5, w = 4; i*i <= n; i += w) {
    if (n % i == 0) {
      return false;
    }
    w = 6 - w;
  }
  return true;
}

typedef unsigned long long uint64;

uint64 mulmod(uint64 x, uint64 n, uint64 m) {
  uint64 a = 0, b = x % m;
  for (; n > 0; n >>= 1) {
    if (n & 1) {
      a = (a + b) % m;
    }
    b = (b << 1) % m;
  }
  return a % m;
}

uint64 powmod(uint64 x, uint64 n, uint64 m) {
  uint64 a = 1, b = x;
  for (; n > 0; n >>= 1) {
    if (n & 1) {
      a = mulmod(a, b, m);
    }
    b = mulmod(b, b, m);
  }
  return a % m;
}

uint64 rand64u() {
  return ((uint64)(rand() & 0xf) << 60) |
         ((uint64)(rand() & 0x7fff) << 45) |
         ((uint64)(rand() & 0x7fff) << 30) |
         ((uint64)(rand() & 0x7fff) << 15) |
         ((uint64)(rand() & 0x7fff));
}

bool is_probable_prime(long long n, int k = 20) {
  if (n == 2 || n == 3) {
    return true;
  }
  if (n < 2 || n % 2 == 0 || n % 3 == 0) {
    return false;
  }
  uint64 s = n - 1, p = n - 1;
  while (!(s & 1)) {
    s >>= 1;
  }
  for (int i = 0; i < k; i++) {
    uint64 x, r = powmod(rand64u() % p + 1, s, n);
    for (x = s; x != p && r != 1 && r != p; x <<= 1) {
      r = mulmod(r, r, n);
    }
    if (r != p && !(x & 1)) {
      return false;
    }
  }
  return true;
}

bool is_prime_fast(long long n) {
  static const int np = 9, p[] = {2, 3, 5, 7, 11, 13, 17, 19, 23};
  for (int i = 0; i < np; i++) {
    if (n % p[i] == 0) {
      return n == p[i];
    }
  }
  if (n < p[np - 1]) {
    return false;
  }
  uint64 t;
  int s = 0;
  for (t = n - 1; !(t & 1); t >>= 1) {
    s++;
  }
  for (int i = 0; i < np; i++) {
    uint64 r = powmod(p[i], t, n);
    if (r == 1) {
      continue;
    }
    bool ok = false;
    for (int j = 0; j < s && !ok; j++) {
      ok |= (r == (uint64)n - 1);
      r = mulmod(r, r, n);
    }
    if (!ok) {
      return false;
    }
  }
  return true;
}

/*** Example Usage ***/

#include <cassert>

int main() {
  int len = 20;
  long long tests[] = {
    -1, 0, 1, 2, 3, 4, 5, 1000000LL, 772023803LL, 792904103LL, 813815117LL,
    834753187LL, 855718739LL, 876717799LL, 897746119LL, 2147483647LL,
    5705234089LL, 5914686649LL, 6114145249LL, 6339503641LL, 6548531929LL
  };
  for (int i = 0; i < len; i++) {
    bool p = is_prime(tests[i]);
    assert(p == is_prime_fast(tests[i]));
    assert(p == is_probable_prime(tests[i]));
  }
  return 0;
}
\end{lstlisting}
\subsection{Integer Factorization}
\begin{lstlisting}
/*

Compute the prime factorization of an integer. In the following implementations,
the prime factorization of n is represented as a sorted vector of prime integers
which together multiply to n. Note that factors are duplicated in the vector in
accordance to their multiplicity in the prime factorization of n. For 0, 1, and
prime numbers, the prime factorization is considered to be a vector consisting
of a single element - the input itself.

- prime_factorize(n) returns the prime factorization of n using trial division.
- get_divisors(n) returns a sorted vector of all (not merely prime) divisors of
  n using trial division.
- fermat(n) returns a factor of n (possibly 1 or itself) that is not necessarily
  prime. This algorithm is efficient for integers with two factors near sqrt(n),
  but is roughly as slow as trial division otherwise.
- pollards_rho_brent(n) returns a factor of n that is not necessarily prime
  using Pollard's rho algorithm with Brent's optimization. If n is prime, then n
  itself is returned. While this algorithm is non-deterministic and may fail to
  detect factors on certain runs of the same input, it can be placed in a loop
  to deterministically factor large integers, as done in prime_factorize_big().
- prime_factorize_big(n, trial_division_cutoff) returns the prime factorization
  of a 64-bit integer n using a combination of trial division, the Miller-Rabin
  primality test, and Pollard's rho algorithm. trial_division_cutoff specifies
  the largest factor to test with trial division before falling back to the rho
  algorithm. This supports 64-bit integers up to and including 2^63 - 1.

Time Complexity:
- O(sqrt n) per call to prime_factorize(n), get_divisors(n), and fermat(n).
- Unknown, but approximately O(n^(1/4)) per call to pollards_rho_brent(n) and
  prime_factorize_big(n).

Space Complexity:
- O(f) auxiliary heap space for all operations, where f is the number of factors
  returned.

*/

#include <algorithm>
#include <cmath>
#include <cstdlib>
#include <vector>

template<class Int>
std::vector<Int> prime_factorize(Int n) {
  if (n <= 3) {
    return std::vector<Int>(1, n);
  }
  std::vector<Int> res;
  for (Int i = 2; ; i++) {
    int p = 0, q = n/i, r = n - q*i;
    if (i > q || (i == q && r > 0)) {
      break;
    }
    while (r == 0) {
      p++;
      n = q;
      q = n/i;
      r = n - q*i;
    }
    for (int j = 0; j < p; j++) {
      res.push_back(i);
    }
  }
  if (n > 1) {
    res.push_back(n);
  }
  return res;
}

template<class Int>
std::vector<Int> get_divisors(Int n) {
  if (n <= 1) {
    return (n < 1) ? std::vector<Int>() : std::vector<Int>(1, 1);
  }
  std::vector<Int> res;
  for (Int i = 1; i*i <= n; i++) {
    if (n % i == 0) {
      res.push_back(i);
      if (i*i != n) {
        res.push_back(n/i);
      }
    }
  }
  std::sort(res.begin(), res.end());
  return res;
}

long long fermat(long long n) {
  if (n % 2 == 0) {
    return 2;
  }
  long long x = sqrt(n), y = 0, r = x*x - y*y - n;
  while (r != 0) {
    if (r < 0) {
      r += x + x + 1;
      x++;
    } else {
      r -= y + y + 1;
      y++;
    }
  }
  return (x == y) ? (x + y) : (x - y);
}

typedef unsigned long long uint64;

uint64 mulmod(uint64 x, uint64 n, uint64 m) {
  uint64 a = 0, b = x % m;
  for (; n > 0; n >>= 1) {
    if (n & 1) {
      a = (a + b) % m;
    }
    b = (b << 1) % m;
  }
  return a % m;
}

uint64 powmod(uint64 x, uint64 n, uint64 m) {
  uint64 a = 1, b = x;
  for (; n > 0; n >>= 1) {
    if (n & 1) {
      a = mulmod(a, b, m);
    }
    b = mulmod(b, b, m);
  }
  return a % m;
}

uint64 rand64u() {
  return ((uint64)(rand() & 0xf) << 60) |
         ((uint64)(rand() & 0x7fff) << 45) |
         ((uint64)(rand() & 0x7fff) << 30) |
         ((uint64)(rand() & 0x7fff) << 15) |
         ((uint64)(rand() & 0x7fff));
}

uint64 gcd(uint64 a, uint64 b) {
  while (b != 0) {
    uint64 t = b;
    b = a % b;
    a = t;
  }
  return a;
}

long long pollards_rho_brent(long long n) {
  if (n % 2 == 0) {
    return 2;
  }
  uint64 y = rand64u() % (n - 1) + 1;
  uint64 c = rand64u() % (n - 1) + 1;
  uint64 m = rand64u() % (n - 1) + 1;
  uint64 g = 1, r = 1, q = 1, ys = 0, x = 0;
  for (r = 1; g == 1; r <<= 1) {
    x = y;
    for (int i = 0; i < r; i++) {
      y = (mulmod(y, y, n) + c) % n;
    }
    for (long long k = 0; k < r && g == 1; k += m) {
      ys = y;
      long long lim = std::min(m, r - k);
      for (int j = 0; j < lim; j++) {
        y = (mulmod(y, y, n) + c) % n;
        q = mulmod(q, (x > y) ? (x - y) : (y - x), n);
      }
      g = gcd(q, n);
    }
  }
  if (g == n) {
    do {
      ys = (mulmod(ys, ys, n) + c) % n;
      g = gcd((x > ys) ? (x - ys) : (ys - x), n);
    } while (g <= 1);
  }
  return g;
}

bool is_prime(long long n) {
  static const int np = 9, p[] = {2, 3, 5, 7, 11, 13, 17, 19, 23};
  for (int i = 0; i < np; i++) {
    if (n % p[i] == 0) {
      return n == p[i];
    }
  }
  if (n < p[np - 1]) {
    return false;
  }
  uint64 t;
  int s = 0;
  for (t = n - 1; !(t & 1); t >>= 1) {
    s++;
  }
  for (int i = 0; i < np; i++) {
    uint64 r = powmod(p[i], t, n);
    if (r == 1) {
      continue;
    }
    bool ok = false;
    for (int j = 0; j < s && !ok; j++) {
      ok |= (r == (uint64)n - 1);
      r = mulmod(r, r, n);
    }
    if (!ok) {
      return false;
    }
  }
  return true;
}

std::vector<long long> prime_factorize_big(
    long long n, long long trial_division_cutoff = 1000000LL) {
  if (n <= 3) {
    return std::vector<long long>(1, n);
  }
  std::vector<long long> res;
  for (; n % 2 == 0; n /= 2) {
    res.push_back(2);
  }
  for (; n % 3 == 0; n /= 3) {
    res.push_back(3);
  }
  for (int i = 5, w = 4; i <= trial_division_cutoff && i*i <= n; i += w) {
    for (; n % i == 0; n /= i) {
      res.push_back(i);
    }
    w = 6 - w;
  }
  for (long long p; n > trial_division_cutoff && !is_prime(n); n /= p) {
    do {
      p = pollards_rho_brent(n);
    } while (p == n);
    res.push_back(p);
  }
  if (n != 1) {
    res.push_back(n);
  }
  sort(res.begin(), res.end());
  return res;
}

/*** Example Usage ***/

#include <cassert>
#include <set>
using namespace std;

void validate(long long n, const vector<long long> &factors) {
  if (n == 1 || is_prime(n)) {
    assert(factors == vector<long long>(1, n));
    return;
  }
  long long prod = 1;
  for (int i = 0; i < factors.size(); i++) {
    assert(is_prime(factors[i]));
    prod *= factors[i];
  }
  assert(prod == n);
}

int main() {
  { // Small tests.
    for (int i = 1; i <= 10000; i++) {
      vector<long long> v1 = prime_factorize((long long)i);
      vector<long long> v2 = prime_factorize_big(i);
      validate(i, v1);
      assert(v1 == v2);
      vector<int> d = get_divisors(i);
      set<int> s(d.begin(), d.end());
      assert(d.size() == s.size());
      for (int j = 1; j <= i; j++) {
        if (i % j == 0) {
          assert(s.count(j));
        }
      }
    }
  }
  { // Fermat works best for numbers with two factors close to each other.
    long long n = 1000003LL*100000037;
    assert(fermat(n) == 1000003);
  }
  { // Large tests.
    const int ntests = 7;
    const long long tests[] = {
      3LL*3*5*7*9949*9967*1000003,
      2LL*1000003*1000000007,
      999961LL*1000033,
      357267896789127671LL,
      2LL*2*2*2*2*2*2*3*3*3*3*5*5*7*7*11*13*17*19*23*29*31*37,
      2LL*2*2*2*2*2*2*3*3*3*3*5*5*7*7*35336848213,
      2LL*2*2*2*2*2*2*3*3*3*3*5*5*7*7*186917*186947,
    };
    for (int i = 0; i < ntests; i++) {
      validate(tests[i], prime_factorize_big(tests[i]));
    }
  }
  return 0;
}
\end{lstlisting}
\subsection{Euler's Totient Function}
\begin{lstlisting}
/*

Euler's totient function phi(n) returns the number of positive integers less
than or equal to n that are relatively prime to n. That is, phi(n) is the number
of integers k in the range [1, n] for which gcd(n, k) = 1. The computation of
phi(1..n) can be performed simultaneously, as done so by phi_table(n) which
returns a vector v such that v[i] stores phi(i) for i in the range [0, n].

Time Complexity:
- O(n log(log(n))) per call to phi(n) and phi_table(n).

Space Complexity:
- O(1) auxiliary space for phi(n).
- O(n) auxiliary heap space for phi_table(n).

*/

#include <vector>

int phi(int n) {
  int res = n;
  for (int i = 2; i*i <= n; i++) {
    if (n % i == 0) {
      while (n % i == 0) {
        n /= i;
      }
      res -= res/i;
    }
  }
  if (n > 1) {
    res -= res/n;
  }
  return res;
}

std::vector<int> phi_table(int n) {
  std::vector<int> res(n + 1);
  for (int i = 0; i <= n; i++) {
    res[i] = i;
  }
  for (int i = 1; i <= n; i++) {
    for (int j = 2*i; j <= n; j += i) {
      res[j] -= res[i];
    }
  }
  return res;
}

/*** Example Usage ***/

#include <cassert>
using namespace std;

int main() {
  assert(phi(1) == 1);
  assert(phi(9) == 6);
  assert(phi(1234567) == 1224720);
  const int n = 1000;
  vector<int> v = phi_table(n);
  for (int i = 0; i <= n; i++) {
    assert(v[i] == phi(i));
  }
  return 0;
}
\end{lstlisting}
\subsection{Binary Exponentiation}
\begin{lstlisting}
/*

Given three unsigned 64-bit integers x, n, and m, powmod() returns x raised to
the power of n (modulo m). mulmod() returns x multiplied by n (modulo m).
Despite the fact that both functions use unsigned 64-bit integers for arguments
and intermediate calculations, arguments x and n must not exceed 2^63 - 1 (the
maximum value of a signed 64-bit integer) for the result to be correctly
computed without overflow.

Binary exponentiation, also known as exponentiation by squaring, decomposes the
exponentiation into a logarithmic number of multiplications while avoiding
overflow. To further prevent overflow in the intermediate squaring computations,
multiplication is performed using a similar principle of repeated addition.

Time Complexity:
- O(log n) per call to mulmod() and powmod(), where n is the second argument.

Space Complexity:
- O(1) auxiliary.

*/

typedef unsigned long long uint64;

uint64 mulmod(uint64 x, uint64 n, uint64 m) {
  uint64 a = 0, b = x % m;
  for (; n > 0; n >>= 1) {
    if (n & 1) {
      a = (a + b) % m;
    }
    b = (b << 1) % m;
  }
  return a % m;
}

uint64 powmod(uint64 x, uint64 n, uint64 m) {
  uint64 a = 1, b = x;
  for (; n > 0; n >>= 1) {
    if (n & 1) {
      a = mulmod(a, b, m);
    }
    b = mulmod(b, b, m);
  }
  return a % m;
}

/*** Example Usage ***/

#include <cassert>

int main() {
  assert(powmod(2, 10, 1000000007) == 1024);
  assert(powmod(2, 62, 1000000) == 387904);
  assert(powmod(10001, 10001, 100000) == 10001);
  return 0;
}
\end{lstlisting}

\section{Arbitrary Precision Arithmetic}
\setcounter{section}{4}
\setcounter{subsection}{0}
\subsection{Big Integers (Simple)}
\begin{lstlisting}
/*

Perform simple arithmetic operations on arbitrary precision big integers whose
digits are internally represented as an std::string in little-endian order.

- bigint(n) constructs a big integer from a long long (default = 0).
- bigint(s) constructs a big integer from a string s, which must strictly
  consist of a sequence of numeric digits, optionally preceded by a minus sign.
- str() returns the string representation of the big integer.
- comp(a, b) returns -1, 0, or 1 depending on whether the big integers a and b
  compare less, equal, or greater, respectively.
- add(a, b) returns the sum of big integers a and b.
- sub(a, b) returns the difference of big integers a and b.
- mul(a, b) returns the product of big integers a and b.
- div(a, b) returns the quotient of big integers a and b.

Time Complexity:
- O(n) per call to the constructor, str(), comp(), add(), and sub(), where n is
  total number of digits in the argument(s) and result for each operation.
- O(n*m) per call to mul(a, b) and div(a, b) where n is the number of digits in
  a and m is the number of digits in b.

Space Complexity:
- O(n) for storage of the big integer, where n is the number of the digits.
- O(n) auxiliary heap space for str(), add(), sub(), mul(), and div(), where n
  the total number of digits in the argument(s) and result for each operation.

*/

#include <algorithm>
#include <cctype>
#include <stdexcept>
#include <string>

class bigint {
  std::string digits;
  int sign;

  void normalize() {
    size_t pos = digits.find_last_not_of('0');
    if (pos != std::string::npos) {
      digits.erase(pos + 1);
    }
    if (digits.empty()) {
      digits = "0";
    }
    if (digits.size() == 1 && digits[0] == '0') {
      sign = 1;
      return;
    }
  }

  static int comp(const std::string &a, const std::string &b,
                  int asign, int bsign) {
    if (asign != bsign) {
      return asign < bsign ? -1 : 1;
    }
    if (a.size() != b.size()) {
      return a.size() < b.size() ? -asign : asign;
    }
    for (int i = (int)a.size() - 1; i >= 0; i--) {
      if (a[i] != b[i]) {
        return a[i] < b[i] ? -asign : asign;
      }
    }
    return 0;
  }

  static bigint add(const std::string &a, const std::string &b,
                    int asign, int bsign) {
    if (asign != bsign) {
      return (asign == 1) ? sub(a, b, asign, 1) : sub(b, a, bsign, 1);
    }
    bigint res;
    res.sign = asign;
    res.digits.resize(std::max(a.size(), b.size()) + 1, '0');
    for (int i = 0, carry = 0; i < (int)res.digits.size(); i++) {
      int d = carry;
      if (i < (int)a.size()) {
        d += a[i] - '0';
      }
      if (i < (int)b.size()) {
        d += b[i] - '0';
      }
      res.digits[i] = '0' + (d % 10);
      carry = d/10;
    }
    res.normalize();
    return res;
  }

  static bigint sub(const std::string &a, const std::string &b,
                    int asign, int bsign) {
    if (asign == -1 || bsign == -1) {
      return add(a, b, asign, -bsign);
    }
    bigint res;
    if (comp(a, b, asign, bsign) < 0) {
      res = sub(b, a, bsign, asign);
      res.sign = -1;
      return res;
    }
    res.digits.assign(a.size(), '0');
    for (int i = 0, borrow = 0; i < (int)res.digits.size(); i++) {
      int d = (i < (int)b.size() ? a[i] - b[i] : a[i] - '0') - borrow;
      if (a[i] > '0') {
        borrow = 0;
      }
      if (d < 0) {
        d += 10;
        borrow = 1;
      }
      res.digits[i] = '0' + (d % 10);
    }
    res.normalize();
    return res;
  }

 public:
  bigint(long long n = 0) {
    sign = (n < 0) ? -1 : 1;
    if (n == 0) {
      digits = "0";
      return;
    }
    for (n = (n > 0) ? n : -n; n > 0; n /= 10) {
      digits += '0' + (n % 10);
    }
    normalize();
  }

  bigint(const std::string &s) {
    if (s.empty() || (s[0] == '-' && s.size() == 1)) {
      throw std::runtime_error("Invalid string format to construct bigint.");
    }
    digits.assign(s.rbegin(), s.rend());
    if (s[0] == '-') {
      sign = -1;
      digits.erase(digits.size() - 1);
    } else {
      sign = 1;
    }
    if (digits.find_first_not_of("0123456789") != std::string::npos) {
      throw std::runtime_error("Invalid string format to construct bigint.");
    }
    normalize();
  }

  std::string to_string() const {
    return (sign < 0 ? "-" : "") + std::string(digits.rbegin(), digits.rend());
  }

  friend int comp(const bigint &a, const bigint &b) {
    return comp(a.digits, b.digits, a.sign, b.sign);
  }

  friend bigint add(const bigint &a, const bigint &b) {
    return add(a.digits, b.digits, a.sign, b.sign);
  }

  friend bigint sub(const bigint &a, const bigint &b) {
    return sub(a.digits, b.digits, a.sign, b.sign);
  }

  friend bigint mul(const bigint &a, const bigint &b) {
    bigint res, row(a);
    for (int i = 0; i < (int)b.digits.size(); i++) {
      for (int j = 0; j < (b.digits[i] - '0'); j++) {
        res = add(res.digits, row.digits, res.sign, row.sign);
      }
      if (row.digits.size() > 1 || row.digits[0] != '0') {
        row.digits.insert(0, "0");
      }
    }
    res.sign = a.sign*b.sign;
    res.normalize();
    return res;
  }

  friend bigint div(const bigint &a, const bigint &b) {
    bigint res, row;
    res.digits.assign(a.digits.size(), '0');
    for (int i = (int)a.digits.size() - 1; i >= 0; i--) {
      row.digits.insert(row.digits.begin(), a.digits[i]);
      while (comp(row.digits, b.digits, row.sign, 1) > 0) {
        res.digits[i]++;
        row = sub(row.digits, b.digits, row.sign, 1);
      }
    }
    res.sign = a.sign*b.sign;
    res.normalize();
    return res;
  }
};

/*** Example Usage ***/

#include <cassert>

int main() {
  bigint a("-9899819294989142124"), b("12398124981294214");
  assert(add(a, b).to_string() == "-9887421170007847910");
  assert(sub(a, b).to_string() == "-9912217419970436338");
  assert(mul(a, b).to_string() == "-122739196911503356525379735104870536");
  assert(div(a, b).to_string() == "-798");
  assert(comp(a, b) == -1 && comp(a, a) == 0 && comp(b, a) == 1);
  return 0;
}
\end{lstlisting}
\subsection{Big Integers}
\begin{lstlisting}
/*

Perform operations on arbitrary precision big integers internally represented as
a vector of base-1000000000 digits in little-endian order. Typical arithmetic
operations involving mixed numeric primitives and strings are supported using
templates and operator overloading, as long as at least one operand is a bigint
at any given level of evaluation.

- bigint(n) constructs a big integer from a long long (default = 0).
- bigint(s) constructs a big integer from a C string or an std::string s.
- operator = is defined to copy from another big integer or to assign from an
  64-bit integer primitive.
- size() returns the number of digits in the base-10 representation.
- operators >> and << are defined to support stream-based input and output.
- v.to_string(), v.to_llong(), v.to_double(), and v.to_ldouble() return the big
  integer v converted to an std::string, long long, double, and long double
  respectively. For the latter three data types, overflow behavior is based on
  that of inputting from std::istream.
- v.abs() returns the absolute value of big integer v.
- a.comp(b) returns -1, 0, or 1 depending on whether the big integers a and b
  compare less, equal, or greater, respectively.
- operators <, >, <=, >=, ==, !=, +, -, *, /, %, ++, --, +=, -=, *=, /=, and %=
  are defined analogous to those on integer primitives. Addition, subtraction,
  and comparisons are performed using the standard linear algorithms.
  Multiplication is performed using a combination of the grade school algorithm
  (for smaller inputs) and either the Karatsuba algorithm (if the USE_FFT_MULT
  flag is set to false) or the Schonhage-Strassen algorithm (if USE_FFT_MULT is
  set to true). Division and modulo are computed simultaneously using the grade
  school method.
- a.div(b) returns a pair consisting of the quotient and remainder.
- v.pow(n) returns v raised to the power of n.
- v.sqrt() returns the integral part of the square root of big integer v.
- v.nth_root(n) returns the integral part of the n-th root of big integer v.
- rand(n) returns a random, positive big integer with n digits.

Time Complexity:
- O(n) per call to the constructors, size(), to_string(), to_llong(),
  to_double(), to_ldouble(), abs(), comp(), rand(), and all comparison and
  arithmetic operators except multiplication, division, and modulo, where n is
  total number of digits in the argument(s) and result for each operation.
- O(n*log(n)*log(log(n))) or O(n^1.585) per call to multiplication operations,
  depending on whether USE_FFT_MULT is set to true or false.
- O(n*m) per call to division and modulo operations, where n and m are the
  number of digits in the dividend and divisor, respectively.
- O(M(m) log n) per call to pow(n), where m is the length of the big integer.

Space Complexity:
- O(n) for storage of the big integer.
- O(n) auxiliary heap space for negation, addition, subtraction, multiplication,
  division, abs(), sqrt(), pow(), and nth_root().
- O(1) auxiliary space for all other operations.

*/

#include <algorithm>
#include <cmath>
#include <complex>
#include <cstdlib>
#include <cstring>
#include <iomanip>
#include <istream>
#include <ostream>
#include <sstream>
#include <stdexcept>
#include <string>
#include <utility>
#include <vector>

class bigint {
  static const int BASE = 1000000000, BASE_DIGITS = 9;
  static const bool USE_FFT_MULT = true;

  typedef std::vector<int> vint;
  typedef std::vector<long long> vll;
  typedef std::vector<std::complex<double> > vcd;

  vint digits;
  int sign;

  void normalize() {
    while (!digits.empty() && digits.back() == 0) {
      digits.pop_back();
    }
    if (digits.empty()) {
      sign = 1;
    }
  }

  void read(int n, const char *s) {
    sign = 1;
    digits.clear();
    int pos = 0;
    while (pos < n && (s[pos] == '-' || s[pos] == '+')) {
      if (s[pos] == '-') {
        sign = -sign;
      }
      pos++;
    }
    for (int i = n - 1; i >= pos; i -= BASE_DIGITS) {
      int x = 0;
      for (int j = std::max(pos, i - BASE_DIGITS + 1); j <= i; j++) {
        x = x*10 + s[j] - '0';
      }
      digits.push_back(x);
    }
    normalize();
  }

  static int comp(const vint &a, const vint &b, int asign, int bsign) {
    if (asign != bsign) {
      return asign < bsign ? -1 : 1;
    }
    if (a.size() != b.size()) {
      return a.size() < b.size() ? -asign : asign;
    }
    for (int i = (int)a.size() - 1; i >= 0; i--) {
      if (a[i] != b[i]) {
        return a[i] < b[i] ? -asign : asign;
      }
    }
    return 0;
  }

  static bigint add(const vint &a, const vint &b, int asign, int bsign) {
    if (asign != bsign) {
      return (asign == 1) ? sub(a, b, asign, 1) : sub(b, a, bsign, 1);
    }
    bigint res;
    res.digits = a;
    res.sign = asign;
    int carry = 0, size = (int)std::max(a.size(), b.size());
    for (int i = 0; i < size || carry; i++) {
      if (i == (int)res.digits.size()) {
        res.digits.push_back(0);
      }
      res.digits[i] += carry + (i < (int)b.size() ? b[i] : 0);
      carry = (res.digits[i] >= BASE) ? 1 : 0;
      if (carry) {
        res.digits[i] -= BASE;
      }
    }
    return res;
  }

  static bigint sub(const vint &a, const vint &b, int asign, int bsign) {
    if (asign == -1 || bsign == -1) {
      return add(a, b, asign, -bsign);
    }
    bigint res;
    if (comp(a, b, asign, bsign) < 0) {
      res = sub(b, a, bsign, asign);
      res.sign = -1;
      return res;
    }
    res.digits = a;
    res.sign = asign;
    for (int i = 0, borrow = 0; i < (int)a.size() || borrow; i++) {
      res.digits[i] -= borrow + (i < (int)b.size() ? b[i] : 0);
      borrow = res.digits[i] < 0;
      if (borrow) {
        res.digits[i] += BASE;
      }
    }
    res.normalize();
    return res;
  }

  static vint convert_base(const vint &digits, int l1, int l2) {
    vll p(std::max(l1, l2) + 1);
    p[0] = 1;
    for (int i = 1; i < (int)p.size(); i++) {
      p[i] = p[i - 1]*10;
    }
    vint res;
    long long curr = 0;
    for (int i = 0, curr_digits = 0; i < (int)digits.size(); i++) {
      curr += digits[i]*p[curr_digits];
      curr_digits += l1;
      while (curr_digits >= l2) {
        res.push_back((int)(curr % p[l2]));
        curr /= p[l2];
        curr_digits -= l2;
      }
    }
    res.push_back((int)curr);
    while (!res.empty() && res.back() == 0) {
      res.pop_back();
    }
    return res;
  }

  template<class It>
  static vll karatsuba(It alo, It ahi, It blo, It bhi) {
    int n = std::distance(alo, ahi), k = n/2;
    vll res(n*2);
    if (n <= 32) {
      for (int i = 0; i < n; i++) {
        for (int j = 0; j < n; j++) {
          res[i + j] += alo[i]*blo[j];
        }
      }
      return res;
    }
    vll a1b1 = karatsuba(alo, alo + k, blo, blo + k);
    vll a2b2 = karatsuba(alo + k, ahi, blo + k, bhi);
    vll a2(alo + k, ahi), b2(blo + k, bhi);
    for (int i = 0; i < k; i++) {
      a2[i] += alo[i];
      b2[i] += blo[i];
    }
    vll r = karatsuba(a2.begin(), a2.end(), b2.begin(), b2.end());
    for (int i = 0; i < (int)a1b1.size(); i++) {
      r[i] -= a1b1[i];
      res[i] += a1b1[i];
    }
    for (int i = 0; i < (int)a2b2.size(); i++) {
      r[i] -= a2b2[i];
      res[i + n] += a2b2[i];
    }
    for (int i = 0; i < (int)r.size(); i++) {
      res[i + k] += r[i];
    }
    return res;
  }

  template<class It>
  static vcd fft(It lo, It hi, bool invert = false) {
    int n = std::distance(lo, hi), k = 0, high1 = -1;
    while ((1 << k) < n) {
      k++;
    }
    std::vector<int> rev(n, 0);
    for (int i = 1; i < n; i++) {
      if (!(i & (i - 1))) {
        high1++;
      }
      rev[i] = rev[i ^ (1 << high1)];
      rev[i] |= (1 << (k - high1 - 1));
    }
    vcd roots(n), res(n);
    for (int i = 0; i < n; i++) {
      double alpha = 2*3.14159265358979323846*i/n;
      roots[i] = std::complex<double>(cos(alpha), sin(alpha));
      res[i] = *(lo + rev[i]);
    }
    for (int len = 1; len < n; len <<= 1) {
      vcd tmp(n);
      int rstep = roots.size()/(len << 1);
      for (int pdest = 0; pdest < n; pdest += len) {
        int p = pdest;
        for (int i = 0; i < len; i++) {
          std::complex<double> c = roots[i*rstep]*res[p + len];
          tmp[pdest] = res[p] + c;
          tmp[pdest + len] = res[p] - c;
          pdest++;
          p++;
        }
      }
      res.swap(tmp);
    }
    if (invert) {
      for (int i = 0; i < (int)res.size(); i++) {
        res[i] /= n;
      }
      std::reverse(res.begin() + 1, res.end());
    }
    return res;
  }

 public:
  bigint() : sign(1) {}
  bigint(int v) { *this = (long long)v; }
  bigint(long long v) { *this = v; }
  bigint(const char *s) { read(strlen(s), s); }
  bigint(const std::string &s) { read(s.size(), s.c_str()); }

  void operator=(const bigint &v) {
    sign = v.sign;
    digits = v.digits;
  }

  void operator=(long long v) {
    sign = 1;
    if (v < 0) {
      sign = -1;
      v = -v;
    }
    digits.clear();
    for (; v > 0; v /= BASE) {
      digits.push_back(v % BASE);
    }
  }

  int size() const {
    if (digits.empty()) {
      return 1;
    }
    std::ostringstream oss;
    oss << digits.back();
    return oss.str().length() + BASE_DIGITS*(digits.size() - 1);
  }

  friend std::istream& operator>>(std::istream &in, bigint &v) {
    std::string s;
    in >> s;
    v.read(s.size(), s.c_str());
    return in;
  }

  friend std::ostream& operator<<(std::ostream &out, const bigint &v) {
    if (v.sign == -1) {
      out << '-';
    }
    out << (v.digits.empty() ? 0 : v.digits.back());
    for (int i = (int)v.digits.size() - 2; i >= 0; i--) {
      out << std::setw(BASE_DIGITS) << std::setfill('0') << v.digits[i];
    }
    return out;
  }

  std::string to_string() const {
    std::ostringstream oss;
    if (sign == -1) {
      oss << '-';
    }
    oss << (digits.empty() ? 0 : digits.back());
    for (int i = (int)digits.size() - 2; i >= 0; i--) {
      oss << std::setw(BASE_DIGITS) << std::setfill('0') << digits[i];
    }
    return oss.str();
  }

  long long to_llong() const {
    long long res = 0;
    for (int i = (int)digits.size() - 1; i >= 0; i--) {
      res = res*BASE + digits[i];
    }
    return res*sign;
  }

  double to_double() const {
    std::stringstream ss(to_string());
    double res;
    ss >> res;
    return res;
  }

  long double to_ldouble() const {
    std::stringstream ss(to_string());
    long double res;
    ss >> res;
    return res;
  }

  int comp(const bigint &v) const {
    return comp(digits, v.digits, sign, v.sign);
  }

  bool operator<(const bigint &v) const { return comp(v) < 0; }
  bool operator>(const bigint &v) const { return comp(v) > 0; }
  bool operator<=(const bigint &v) const { return comp(v) <= 0; }
  bool operator>=(const bigint &v) const { return comp(v) >= 0; }
  bool operator==(const bigint &v) const { return comp(v) == 0; }
  bool operator!=(const bigint &v) const { return comp(v) != 0; }

  template<class T>
  friend bool operator<(const T &a, const bigint &b) { return bigint(a) < b; }

  template<class T>
  friend bool operator>(const T &a, const bigint &b) { return bigint(a) > b; }

  template<class T>
  friend bool operator<=(const T &a, const bigint &b) { return bigint(a) <= b; }

  template<class T>
  friend bool operator>=(const T &a, const bigint &b) { return bigint(a) >= b; }

  template<class T>
  friend bool operator==(const T &a, const bigint &b) { return bigint(a) == b; }

  template<class T>
  friend bool operator!=(const T &a, const bigint &b) { return bigint(a) != b; }

  bigint abs() const {
    bigint res(*this);
    res.sign = 1;
    return res;
  }

  bigint operator-() const {
    bigint res(*this);
    res.sign = -sign;
    return res;
  }

  bigint operator+(const bigint &v) const {
    return add(digits, v.digits, sign, v.sign);
  }

  bigint operator-(const bigint &v) const {
    return sub(digits, v.digits, sign, v.sign);
  }

  void operator*=(int v) {
    if (v < 0) {
      sign = -sign;
      v = -v;
    }
    for (int i = 0, carry = 0; i < (int)digits.size() || carry; i++) {
      if (i == (int)digits.size()) {
        digits.push_back(0);
      }
      long long curr = digits[i]*(long long)v + carry;
      carry = (int)(curr/BASE);
      digits[i] = (int)(curr % BASE);
    }
    normalize();
  }

  bigint operator*(int v) const {
    bigint res(*this);
    res *= v;
    return res;
  }

  bigint operator*(const bigint &v) const {
    static const int TEMP_BASE = 10000, TEMP_BASE_DIGITS = 4;
    vint a = convert_base(digits, BASE_DIGITS, TEMP_BASE_DIGITS);
    vint b = convert_base(v.digits, BASE_DIGITS, TEMP_BASE_DIGITS);
    int n = 1 << (33 - __builtin_clz(std::max(a.size(), b.size()) - 1));
    a.resize(n, 0);
    b.resize(n, 0);
    vll c;
    if (USE_FFT_MULT) {
      vcd at = fft(a.begin(), a.end()), bt = fft(b.begin(), b.end());
      for (int i = 0; i < n; i++) {
        at[i] *= bt[i];
      }
      at = fft(at.begin(), at.end(), true);
      c.resize(n);
      for (int i = 0; i < n; i++) {
        c[i] = at[i].real() + 0.5;
      }
    } else {
      c = karatsuba(a.begin(), a.end(), b.begin(), b.end());
    }
    bigint res;
    res.sign = sign*v.sign;
    for (int i = 0, carry = 0; i < (int)c.size(); i++) {
      long long d = c[i] + carry;
      res.digits.push_back(d % TEMP_BASE);
      carry = d/TEMP_BASE;
    }
    res.digits = convert_base(res.digits, TEMP_BASE_DIGITS, BASE_DIGITS);
    res.normalize();
    return res;
  }

  bigint& operator/=(int v) {
    if (v == 0) {
      throw std::runtime_error("Division by zero in bigint.");
    }
    if (v < 0) {
      sign = -sign;
      v = -v;
    }
    for (int i = (int)digits.size() - 1, rem = 0; i >= 0; i--) {
      long long curr = digits[i] + rem*(long long)BASE;
      digits[i] = (int)(curr/v);
      rem = (int)(curr % v);
    }
    normalize();
    return *this;
  }

  bigint operator/(int v) const {
    bigint res(*this);
    res /= v;
    return res;
  }

  int operator%(int v) const {
    if (v == 0) {
      throw std::runtime_error("Division by zero in bigint.");
    }
    if (v < 0) {
      v = -v;
    }
    int m = 0;
    for (int i = (int)digits.size() - 1; i >= 0; i--) {
      m = (digits[i] + m*(long long)BASE) % v;
    }
    return m*sign;
  }

  std::pair<bigint, bigint> div(const bigint &v) const {
    if (v == 0) {
      throw std::runtime_error("Division by zero in bigint.");
    }
    if (comp(digits, v.digits, 1, 1) < 0) {
      return std::make_pair(0, *this);
    }
    int norm = BASE/(v.digits.back() + 1);
    bigint an = abs()*norm, bn = v.abs()*norm, q, r;
    q.digits.resize(an.digits.size());
    for (int i = (int)an.digits.size() - 1; i >= 0; i--) {
      r *= BASE;
      r += an.digits[i];
      int s1 = (r.digits.size() <= bn.digits.size())
                  ? 0 : r.digits[bn.digits.size()];
      int s2 = (r.digits.size() <= bn.digits.size() - 1)
                  ? 0 : r.digits[bn.digits.size() - 1];
      int d = ((long long)s1*BASE + s2)/bn.digits.back();
      for (r -= bn*d; r < 0; r += bn) {
        d--;
      }
      q.digits[i] = d;
    }
    q.sign = sign*v.sign;
    r.sign = sign;
    q.normalize();
    r.normalize();
    return std::make_pair(q, r/norm);
  }

  bigint operator/(const bigint &v) const { return div(v).first; }
  bigint operator%(const bigint &v) const { return div(v).second; }
  bigint operator++(int) { bigint t(*this); operator++(); return t; }
  bigint operator--(int) { bigint t(*this); operator--(); return t; }
  bigint& operator++() { *this = *this + bigint(1); return *this; }
  bigint& operator--() { *this = *this - bigint(1); return *this; }
  bigint& operator+=(const bigint &v) { *this = *this + v; return *this; }
  bigint& operator-=(const bigint &v) { *this = *this - v; return *this; }
  bigint& operator*=(const bigint &v) { *this = *this * v; return *this; }
  bigint& operator/=(const bigint &v) { *this = *this / v; return *this; }
  bigint& operator%=(const bigint &v) { *this = *this % v; return *this; }

  template<class T>
  friend bigint operator+(const T &a, const bigint &b) { return bigint(a) + b; }

  template<class T>
  friend bigint operator-(const T &a, const bigint &b) { return bigint(a) - b; }

  bigint pow(int n) const {
    if (n == 0) {
      return bigint(1);
    }
    if (*this == 0 || n < 0) {
      return bigint(0);
    }
    bigint x(*this), res(1);
    for (; n != 0; n >>= 1) {
      if (n & 1) {
        res *= x;
      }
      x *= x;
    }
    return res;
  }

  bigint sqrt() const {
    if (sign == -1) {
      throw std::runtime_error("Cannot take square root of a negative number.");
    }
    bigint v(*this);
    while (v.digits.empty() || v.digits.size() % 2 == 1) {
      v.digits.push_back(0);
    }
    int n = v.digits.size();
    int ldig = (int)::sqrt((double)v.digits[n - 1]*BASE + v.digits[n - 2]);
    int norm = BASE/(ldig + 1);
    v *= norm;
    v *= norm;
    while (v.digits.empty() || v.digits.size() % 2 == 1) {
      v.digits.push_back(0);
    }
    bigint r((long long)v.digits[n - 1]*BASE + v.digits[n - 2]);
    int q = ldig = (int)::sqrt((double)v.digits[n - 1]*BASE + v.digits[n - 2]);
    bigint res;
    for (int j = n/2 - 1; j >= 0; j--) {
      for (;; q--) {
        bigint r1 = (r - (res*2*BASE + q)*q)*BASE*BASE +
            (j > 0 ? (long long)v.digits[2*j - 1]*BASE + v.digits[2*j - 2] : 0);
        if (r1 >= 0) {
          r = r1;
          break;
        }
      }
      res = res*BASE + q;
      if (j > 0) {
        int sz1 = res.digits.size(), sz2 = r.digits.size();
        int d1 = (sz1 + 2 < sz2) ? r.digits[sz1 + 2] : 0;
        int d2 = (sz1 + 1 < sz2) ? r.digits[sz1 + 1] : 0;
        int d3 = (sz1 < sz2) ? r.digits[sz1] : 0;
        q = ((long long)d1*BASE*BASE + (long long)d2*BASE + d3)/(ldig*2);
      }
    }
    res.normalize();
    return res/norm;
  }

  bigint nth_root(int n) const {
    if (sign == -1 && n % 2 == 0) {
      throw std::runtime_error("Cannot take even root of a negative number.");
    }
    if (*this == 0 || n < 0) {
      return bigint(0);
    }
    if (n >= size()) {
      int p = 1;
      while (comp(bigint(p).pow(n)) > 0) {
        p++;
      }
      return comp(bigint(p).pow(n)) < 0 ? p - 1 : p;
    }
    bigint lo(bigint(10).pow((int)ceil((double)size()/n) - 1)), hi(lo*10), mid;
    while (lo < hi) {
      mid = (lo + hi)/2;
      int cmp = comp(digits, mid.pow(n).digits, 1, 1);
      if (lo < mid && cmp > 0) {
        lo = mid;
      } else if (mid < hi && cmp < 0) {
        hi = mid;
      } else {
        return (sign == -1) ? -mid : mid;
      }
    }
    return (sign == -1) ? -(mid + 1) : (mid + 1);
  }

  static bigint rand(int n) {
    if (n == 0) {
      return bigint(0);
    }
    std::string s(1, '1' + (::rand() % 9));
    for (int i = 1; i < n; i++) {
      s += '0' + (::rand() % 10);
    }
    return bigint(s);
  }

  friend int comp(const bigint &a, const bigint &b) { return a.comp(b); }
  friend bigint abs(const bigint &v) { return v.abs(); }
  friend bigint pow(const bigint &v, int n) { return v.pow(n); }
  friend bigint sqrt(const bigint &v) { return v.sqrt(); }
  friend bigint nth_root(const bigint &v, int n) { return v.nth_root(n); }
};

/*** Example Usage ***/

#include <cassert>

int main() {
  bigint a("-9899819294989142124"), b("12398124981294214");
  assert(a + b == "-9887421170007847910");
  assert(a - b == "-9912217419970436338");
  assert(a * b == "-122739196911503356525379735104870536");
  assert(a / b == "-798");
  assert(bigint(20).pow(12345).size() == 16062);
  assert(bigint("9812985918924981892491829").nth_root(4) == 1769906);
  for (int i = -100; i <= 100; i++) {
    if (i >= 0) {
      assert(bigint(i).sqrt() == (int)sqrt(i));
    }
    for (int j = -100; j <= 100; j++) {
      assert(bigint(i) + bigint(j) == i + j);
      assert(bigint(i) - bigint(j) == i - j);
      assert(bigint(i) * bigint(j) == i * j);
      if (j != 0) {
        assert(bigint(i) / bigint(j) == i / j);
      }
      if (0 < i && i <= 10 && 0 < j && j <= 10) {
        assert(bigint(i).nth_root(j) == (long long)(pow(i, 1.0/j) + 1E-5));
        long long p = 1;
        for (int k = 0; k < j; k++) {
          p *= i;
        }
        assert(bigint(i).pow(j) == p);
      }
    }
  }
  for (int i = 0; i < 20; i++) {
    int n = rand() % 100 + 1;
    bigint a(bigint::rand(n)), s(a.sqrt()), xx(s*s), yy(s + 1);
    yy *= yy;
    assert(xx <= a && a < yy);
    bigint b(bigint::rand(rand() % n + 1) + 1), q(a/b);
    xx = q*b;
    yy = b*(q + 1);
    assert(a >= xx && a < yy);
  }
  bigint x(-6);
  assert(x.to_string() == "-6");
  assert(x.to_llong() == -6LL);
  assert(x.to_double() == -6.0);
  assert(x.to_ldouble() == -6.0);
  return 0;
}
\end{lstlisting}
\subsection{Rational Numbers}
\begin{lstlisting}
/*

Perform operations on rational numbers internally represented as two integers, a
numerator and a denominator. The template integer type must support streamed
input/output, comparisons, and arithmetic operations. Overflow is not checked
for in internal operations.

- rational(n) constructs a rational with numerator n and denominator 1.
- rational(n, d) constructs a rational with numerator n and denominator d.
- operator >> inputs a rational using the next integer from the stream as the
  numerator and 1 as the denominator.
- operator << outputs a rational as a string consisting of possibly a minus sign
  followed by the numerator, followed by a slash, followed by the denominator.
- v.to_string(), v.to_llong(), v.to_double(), and v.to_ldouble() return the big
  integer v converted to an std::string, long long, double, and long double
  respectively.
- operators <, >, <=, >=, ==, !=, +, -, *, /, %, ++, --, +=, -=, *=, /=, and %=
  are defined analogous to those on numerical primitives.

Time Complexity:
- O(log(n + d)) per call to constructor rational(n, d).
- O(1) per call to all other operations, assuming that corresponding operations
  on the template integer type are O(1) as well.

Space Complexity:
- O(1) for storage of the rational.
- O(1) auxiliary space for all operations.

*/

#include <istream>
#include <ostream>
#include <sstream>
#include <string>

template<class Int = long long>
class rational {
  Int num, den;

 public:
  rational(): num(0), den(1) {}
  rational(const Int &n) : num(n), den(1) {}

  template<class T1, class T2>
  rational(const T1 &n, const T2 &d): num(n), den(d) {
    if (den == 0) {
      throw std::runtime_error("Division by zero in rational.");
    }
    if (den < 0) {
      num = -num;
      den = -den;
    }
    Int a(num < 0 ? -num : num), b(den), tmp;
    while (a != 0 && b != 0) {
      tmp = a % b;
      a = b;
      b = tmp;
    }
    Int gcd = (b == 0) ? a : b;
    num /= gcd;
    den /= gcd;
  }

  friend std::istream& operator>>(std::istream &in, rational &r) {
    std::string s;
    in >> r.num;
    r.den = 1;
    return in;
  }

  friend std::ostream& operator<<(std::ostream &out, const rational &r) {
    out << r.num << "/" << r.den;
    return out;
  }

  std::string to_string() const {
    std::stringstream ss;
    ss << num << " " << den;
    std::string n, d;
    ss >> n >> d;
    return n + "/" + d;
  }

  long long to_llong() const {
    std::stringstream ss;
    ss << num << " " << den;
    long long n, d;
    ss >> n >> d;
    return n/d;
  }

  double to_double() const {
    std::stringstream ss;
    ss << num << " " << den;
    double n, d;
    ss >> n >> d;
    return n/d;
  }

  long double to_ldouble() const {
    long double n, d;
    std::stringstream ss;
    ss << num << " " << den;
    ss >> n >> d;
    return n/d;
  }

  bool operator<(const rational &r) const {
    return num*r.den < r.num*den;
  }

  bool operator>(const rational &r) const {
    return r.num*den < num*r.den;
  }

  bool operator<=(const rational &r) const {
    return !(r < *this);
  }

  bool operator>=(const rational &r) const {
    return !(*this < r);
  }

  bool operator==(const rational &r) const {
    return num == r.num && den == r.den;
  }

  bool operator!=(const rational &r) const {
    return num != r.num || den != r.den;
  }

  template<class T>
  friend bool operator<(const T &a, const rational &b) {
    return rational(a) < b;
  }

  template<class T>
  friend bool operator>(const T &a, const rational &b) {
    return rational(a) > b;
  }

  template<class T>
  friend bool operator<=(const T &a, const rational &b) {
    return rational(a) <= b;
  }

  template<class T>
  friend bool operator>=(const T &a, const rational &b) {
    return rational(a) >= b;
  }

  template<class T>
  friend bool operator==(const T &a, const rational &b) {
    return rational(a) == b;
  }

  template<class T>
  friend bool operator!=(const T &a, const rational &b) {
    return rational(a) != b;
  }

  rational abs() const {
    return rational(num < 0 ? -num : num, den);
  }

  friend rational abs(const rational &r) { return r.abs(); }

  rational operator+(const rational &r) const {
    return rational(num*r.den + r.num*den, den*r.den);
  }

  rational operator-(const rational &r) const {
    return rational(num*r.den - r.num*den, r.den*den);
  }

  rational operator*(const rational &r) const {
    return rational(num*r.num, r.den*den);
  }

  rational operator/(const rational &r) const {
    return rational(num*r.den, den*r.num);
  }

  rational operator%(const rational &r) const {
    return *this - r*rational(num*r.den/(r.num*den), 1);
  }

  template<class T>
  friend rational operator+(const T &a, const rational &b) {
    return rational(a) + b;
  }

  template<class T>
  friend rational operator-(const T &a, const rational &b) {
    return rational(a) - b;
  }

  template<class T>
  friend rational operator*(const T &a, const rational &b) {
    return rational(a) * b;
  }

  template<class T>
  friend rational operator/(const T &a, const rational &b) {
    return rational(a) / b;
  }

  template<class T>
  friend rational operator%(const T &a, const rational &b) {
    return rational(a) % b;
  }

  rational operator-() const { return rational(-num, den); }
  rational operator++(int) { rational t(*this); operator++(); return t; }
  rational operator--(int) { rational t(*this); operator--(); return t; }
  rational& operator++() { *this = *this + 1; return *this; }
  rational& operator--() { *this = *this - 1; return *this; }
  rational& operator+=(const rational &r) { *this = *this + r; return *this; }
  rational& operator-=(const rational &r) { *this = *this - r; return *this; }
  rational& operator*=(const rational &r) { *this = *this * r; return *this; }
  rational& operator/=(const rational &r) { *this = *this / r; return *this; }
  rational& operator%=(const rational &r) { *this = *this % r; return *this; }
};

/*** Example Usage ***/

#include <cassert>
#include <cmath>

int main() {
  #define EQ(a, b) (fabs((a) - (b)) <= 1E-9)
  typedef rational<long long> rational;

  assert(rational(-21, 1) % 2 == -1);
  rational r(rational(-53, 10) % rational(-17, 10));
  assert(EQ(r.to_ldouble(), fmod(-5.3, -1.7)));
  assert(r.to_string() == "-1/5");
  return 0;
}
\end{lstlisting}

\section{Linear Algebra}
\setcounter{section}{5}
\setcounter{subsection}{0}
\subsection{Matrix Utilities}
\begin{lstlisting}
/*

Basic matrix operations defined on a two-dimensional vector of numeric values.

- make_matrix(r, c, v) constructs and returns a matrix with r rows and c columns
  where the value at every index is initialized to v.
- make_matrix(a) returns a matrix constructed from the two dimensional array a.
- identity_matrix(n) returns the n by n identity matrix, that is, a matrix where
  a[i][j] equals 1 (if i == j), or 0 otherwise, for every i and j in [0, n).
- rows(a) returns the number of rows r in an r by c matrix a.
- columns(a) returns the number of columns c in an r by c matrix a.
- a[i][j] may be used to access or modify the entry at row i, column j of an r
  by c matrix a, for every i in [0, r) and j in [0, c).
- operators <, >, <=, >=, ==, and != defines lexicographical comparison based on
  that of std::vector.
- operators +, -, *, /, +=, -=, *=, and /= defines scalar addition, subtraction,
  multiplication, and division involving a matrix a numeric scalar value v.
- operators * and *= defines vector and matrix multiplication.
- operators ^ and ^= defines matrix exponentiation of a square matrix a by an
  integer power p.
- power_sum(a, p) returns the power sum of a square matrix a up to an integer
  power p, that is, a + a^2 + ... + a^p.
- transpose(a) returns the transpose of an r by c matrix a, that is, a new c by
  r matrix b such that a[i][j] == b[j][i] for every i in [0, r) and j in [0, c).
- transpose_in_place(a) assigns the square matrix a to its transpose, returning
  a reference to the modified argument itself.
- rotate(a, d) returns the matrix a rotated d degrees clockwise. A negative d
  specifies a counter-clockwise rotation, and d must be a multiple of 90.
- rotate_in_place(a, d) assigns the square matrix a to its rotation by d degrees
  clockwise, returning a reference to the modified argument itself. A negative d
  specifies a counter-clockwise rotation, and d must be a multiple of 90.

Time Complexity:
- O(n*m) for construction, output, comparison, and scalar arithmetic of n by m
  matrices.
- O(1) for rows(a) and columns(a).
- O(n*m) for matrix-matrix addition and subtraction of n by m matrices.
- O(n*m*log(p)) for exponentiation of an n by m matrix to power p.
- O(n*m*log^2(p)) for power sum of an n by m matrix to power p.
- O(n*m*k) for multiplication of an n by m matrix by an m by k matrix.
- O(n*m) for transpose(), transpose_in_place(), rotate(), and rotate_in_place()
  of n by m matrices.

Space Complexity:
- O(1) auxiliary space for rows(), columns(), a[i][j] access, comparison
  operators, and in-place operations.
- O(n*m*log(p)) auxiliary stack and heap space for exponentiation of an n by m
  matrix to power p, as well as the power sum of an n by m matrix up to power p.
- O(n*m) auxiliary heap space for all non-in-place operations returning an n by
  m matrix, transpose(), and rotate().

*/

#include <algorithm>
#include <cstddef>
#include <iomanip>
#include <ostream>
#include <stdexcept>
#include <vector>

typedef std::vector<std::vector<int> > matrix;

matrix make_matrix(int r, int c) {
  return matrix(r, matrix::value_type(c));
}

template<class T>
matrix make_matrix(int r, int c, const T &v) {
  return matrix(r, matrix::value_type(c, v));
}

template<class T, size_t r, size_t c>
matrix make_matrix(T (&a)[r][c]) {
  matrix res(r, matrix::value_type(c));
  for (size_t i = 0; i < r; i++) {
    for (size_t j = 0; j < c; j++) {
      res[i][j] = a[i][j];
    }
  }
  return res;
}

matrix identity_matrix(int n) {
  matrix res(n, matrix::value_type(n, 0));
  for (int i = 0; i < n; i++) {
    res[i][i] = 1;
  }
  return res;
}

int rows(const matrix &a) { return a.size(); }
int columns(const matrix &a) { return a.empty() ? 0 : a[0].size(); }

std::ostream& operator<<(std::ostream &out, const matrix &a) {
  static const int W = 10, P = 5;
  for (int i = 0; i < rows(a); i++) {
    for (int j = 0; j < columns(a); j++) {
      out << std::setw(W) << std::fixed << std::setprecision(P) << a[i][j];
    }
    out << std::endl;
  }
  return out;
}

template<class T>
matrix& operator+=(matrix &a, const T &v) {
  for (int i = 0; i < rows(a); i++) {
    for (int j = 0; j < columns(a); j++) {
      a[i][j] += v;
    }
  }
  return a;
}

template<class T>
matrix& operator-=(matrix &a, const T &v) {
  for (int i = 0; i < rows(a); i++) {
    for (int j = 0; j < columns(a); j++) {
      a[i][j] -= v;
    }
  }
  return a;
}

template<class T>
matrix& operator*=(matrix &a, const T &v) {
  for (int i = 0; i < rows(a); i++) {
    for (int j = 0; j < columns(a); j++) {
      a[i][j] *= v;
    }
  }
  return a;
}

template<class T>
matrix& operator/=(matrix &a, const T &v) {
  for (int i = 0; i < rows(a); i++) {
    for (int j = 0; j < columns(a); j++) {
      a[i][j] /= v;
    }
  }
  return a;
}

matrix& operator+=(matrix &a, const matrix &b) {
  if (rows(a) != rows(b) || columns(a) != columns(b)) {
    throw std::runtime_error("Invalid dimensions for matrix addition.");
  }
  for (int i = 0; i < rows(a); i++) {
    for (int j = 0; j < columns(a); j++) {
      a[i][j] += b[i][j];
    }
  }
  return a;
}

matrix& operator-=(matrix &a, const matrix &b) {
  if (rows(a) != rows(b) || columns(a) != columns(b)) {
    throw std::runtime_error("Invalid dimensions for matrix addition.");
  }
  for (int i = 0; i < rows(a); i++) {
    for (int j = 0; j < columns(a); j++) {
      a[i][j] -= b[i][j];
    }
  }
  return a;
}

matrix operator+(const matrix &a, const matrix &b) {
  matrix c(a);
  return c += b;
}

matrix operator-(const matrix &a, const matrix &b) {
  matrix c(a);
  return c -= b;
}

template<class T>
matrix& operator*=(matrix &a, const std::vector<T> &v) {
  if (columns(a) != (int)v.size() || v.empty()) {
    throw std::runtime_error("Invalid dimensions for matrix multiplication.");
  }
  for (int i = 0; i < rows(a); i++) {
    a[i][0] *= v[0];
    for (int j = 1; j < columns(a); j++) {
      a[i][0] += a[i][j]*v[j];
    }
  }
  for (int i = 0; i < rows(a); i++) {
    a[i].resize(1);
  }
  return a;
}

matrix operator*(const matrix &a, const matrix &b) {
  if (columns(a) != rows(b)) {
    throw std::runtime_error("Invalid dimensions for matrix multiplication.");
  }
  matrix res = make_matrix(rows(a), columns(b), 0);
  for (int i = 0; i < rows(a); i++) {
    for (int j = 0; j < columns(b); j++) {
      for (int k = 0; k < rows(b); k++) {
        res[i][j] += a[i][k]*b[k][j];
      }
    }
  }
  return res;
}

matrix& operator*=(matrix &a, const matrix &b) {
  return a = a*b;
}

template<class T>
matrix operator+(const matrix &a, const T &v) { matrix m(a); return m += v; }

template<class T>
matrix operator-(const matrix &a, const T &v) { matrix m(a); return m -= v; }

template<class T>
matrix operator*(const matrix &a, const T &v) { matrix m(a); return m *= v; }

template<class T>
matrix operator/(const matrix &a, const T &v) { matrix m(a); return m /= v; }

template<class T>
matrix operator+(const T &v, const matrix &a) { return a + v; }

template<class T>
matrix operator-(const T &v, const matrix &a) { return a - v; }

template<class T>
matrix operator*(const T &v, const matrix &a) { return a * v; }

template<class T>
matrix operator/(const T &v, const matrix &a) { return a / v; }

matrix operator^(const matrix &a, unsigned int p) {
  if (rows(a) != columns(a)) {
    throw std::runtime_error("Matrix must be square for exponentiation.");
  }
  if (p == 0) {
    return identity_matrix(rows(a));
  }
  return (p % 2 == 0) ? (a*a)^(p/2) : a*(a^(p - 1));
}

matrix operator^=(matrix &a, unsigned int p) {
  return a = a ^ p;
}

matrix power_sum(const matrix &a, unsigned int p) {
  if (rows(a) != columns(a)) {
    throw std::runtime_error("Matrix must be square for power_sum.");
  }
  if (p == 0) {
    return make_matrix(rows(a), rows(a));
  }
  return (p % 2 == 0) ? power_sum(a, p/2)*(identity_matrix(rows(a)) + (a^(p/2)))
                      : (a + a*power_sum(a, p - 1));
}

matrix transpose(const matrix &a) {
  matrix res = make_matrix(columns(a), rows(a));
  for (int i = 0; i < rows(res); i++) {
    for (int j = 0; j < columns(res); j++) {
      res[i][j] = a[j][i];
    }
  }
  return res;
}

matrix& transpose_in_place(matrix &a) {
  if (rows(a) != columns(a)) {
    throw std::runtime_error("Matrix must be square for transpose_in_place.");
  }
  for (int i = 0; i < rows(a); i++) {
    for (int j = i + 1; j < columns(a); j++) {
      std::swap(a[i][j], a[j][i]);
    }
  }
  return a;
}

matrix rotate(const matrix &a, int degrees = 90) {
  if (degrees % 90 != 0) {
    throw std::runtime_error("Rotation must be by a multiple of 90 degrees.");
  }
  if (degrees < 0) {
    degrees = 360 - ((-degrees) % 360);
  }
  matrix res;
  switch (degrees % 360) {
    case 90: {
      res = make_matrix(columns(a), rows(a));
      for (int i = 0; i < columns(a); i++) {
        for (int j = 0; j < rows(a); j++) {
          res[i][j] = a[rows(a) - j - 1][i];
        }
      }
      break;
    }
    case 180: {
      res = make_matrix(rows(a), columns(a));
      for (int i = 0; i < rows(a); i++) {
        for (int j = 0; j < columns(a); j++) {
          res[i][j] = a[rows(a) - i - 1][columns(a) - j - 1];
        }
      }
      break;
    }
    case 270: {
      res = make_matrix(columns(a), rows(a));
      for (int i = 0; i < columns(a); i++) {
        for (int j = 0; j < rows(a); j++) {
          res[i][j] = a[j][columns(a) - i - 1];
        }
      }
      break;
    }
    default: {
      res = a;
    }
  }
  return res;
}

matrix& rotate_in_place(matrix &a, int degrees = 90) {
  if (degrees % 90 != 0) {
    throw std::runtime_error("Rotation must be by a multiple of 90 degrees.");
  }
  if (degrees % 180 != 0 && rows(a) != columns(a)) {
    throw std::runtime_error("Matrix must be square for rotate_in_place.");
  }
  if (degrees < 0) {
    degrees = 360 - ((-degrees) % 360);
  }
  int n = rows(a);
  switch (degrees % 360) {
    case 90: {
      transpose_in_place(a);
      for (int i = 0; i < n; i++) {
        std::reverse(a[i].begin(), a[i].end());
      }
      break;
    }
    case 180: {
      for (int i = 0; i < columns(a); i++) {
        for (int j = 0, k = n - 1; j < k; j++, k--) {
          std::swap(a[i][j], a[i][k]);
        }
      }
      for (int j = 0; j < n; j++) {
        for (int i = 0, k = columns(a) - 1; i < k; i++, k--) {
          std::swap(a[i][j], a[k][j]);
        }
      }
      break;
    }
    case 270: {
      transpose_in_place(a);
      for (int j = 0; j < n; j++) {
        for (int i = 0, k = columns(a) - 1; i < k; i++, k--) {
          std::swap(a[i][j], a[k][j]);
        }
      }
      break;
    }
  }
  return a;
}

/*** Example Usage ***/

#include <cassert>
#include <iostream>
using namespace std;

int main() {
  int a[2][3] = {{1, 2, 3}, {4, 5, 6}};
  int a90[3][2] = {{4, 1}, {5, 2}, {6, 3}};
  int a180[2][3] = {{6, 5, 4}, {3, 2, 1}};
  int a270[3][2] = {{3, 6}, {2, 5}, {1, 4}};
  cout << make_matrix(a) << endl;
  assert(rotate(make_matrix(a), -270) == make_matrix(a90));
  assert(rotate(make_matrix(a), -180) == make_matrix(a180));
  assert(rotate(make_matrix(a), -90) == make_matrix(a270));
  assert(rotate(make_matrix(a), 0) == make_matrix(a));
  assert(rotate(make_matrix(a), 90) == make_matrix(a90));
  assert(rotate(make_matrix(a), 180) == make_matrix(a180));
  assert(rotate(make_matrix(a), 270) == make_matrix(a270));
  assert(rotate(make_matrix(a), 360) == make_matrix(a));

  int b[3][3] = {{1, 2, 3}, {4, 5, 6}, {7, 8, 9}};
  for (int d = -360; d <= 360; d += 90) {
    matrix m = make_matrix(b);
    assert(rotate_in_place(m, d) == rotate(make_matrix(b), d));
  }

  matrix m = make_matrix(5, 5, 10) + 10;
  int v[] = {1, 2, 3, 4, 5}, mv[5][1] = {{300}, {300}, {300}, {300}, {300}};
  assert(m*vector<int>(v, v + 5) == make_matrix(mv));

  m[0][0] += 5;
  assert(m[0][0] == 25 && m[1][1] == 20);
  assert(power_sum(m, 3) == m + m*m + (m^3));
  return 0;
}
\end{lstlisting}
\subsection{Row Reduction}
\begin{lstlisting}
/*

Converts a matrix to reduced row echelon form using Gaussian elimination to
solve a system of linear equations as well as compute the determinant. In
practice, this method is prone to rounding error on certain matrices. For a more
accurate algorithm for solving systems of linear equations, LU decomposition
with row partial pivoting should be used.

- row_reduce(a) assigns the matrix a to its reduced row echelon form, returning
  a reference to the modified argument itself.
- solve_system(a, b, &x) solves the system of linear equations a*x = b given an
  r by c matrix a of real values, and a length r vector b, returning 0 if there
  is one solution, -1 if there are zero solutions, or -2 if there are infinite
  solutions. If there is exactly one solution, then the vector pointed to by x
  is populated with the solution vector of length c.

Time Complexity:
- O(r^2*c) per call to row_reduce(a) and solve_system(a), where r and c are the
  number of rows and columns of a respectively.

Space Complexity:
- O(1) auxiliary for row_reduce(a).
- O(r*c) auxiliary heap space for solve_system(a).

*/

#include <cmath>
#include <cstddef>
#include <stdexcept>
#include <vector>

const double EPS = 1e-9;

template<class Matrix>
Matrix& row_reduce(Matrix &a) {
  if (a.empty()) {
    return a;
  }
  int r = a.size(), c = a[0].size(), lead = 0;
  for (int row = 0; row < r && lead < c; row++) {
    int i = row;
    while (fabs(a[i][lead]) < EPS) {
      if (++i == r) {
        i = row;
        if (++lead == c) {
          return a;
        }
      }
    }
    std::swap(a[i], a[row]);
    typename Matrix::value_type::value_type lv = a[row][lead];
    for (int j = 0; j < c; j++) {
      a[row][j] /= lv;
    }
    for (int i = 0; i < r; i++) {
      if (i != row) {
        lv = a[i][lead];
        for (int j = 0; j < c; j++) {
          a[i][j] -= lv*a[row][j];
        }
      }
    }
    for (int j = 0; j < lead; j++) {
      a[row][j] = 0;
    }
    a[row][lead++] = 1;
  }
  return a;
}

template<class Matrix, class T>
int solve_system(const Matrix &a, const std::vector<T> &b, std::vector<T> *x) {
  if (x == NULL || a.empty() || a.size() != b.size()) {
    return -1;
  }
  int r = a.size(), c = a[0].size();
  if (r < c) {
    return -2;
  }
  Matrix m(a);
  for (int i = 0; i < r; i++) {
    m[i].push_back(b[i]);
  }
  row_reduce(m);
  for (int i = 0; i < r; i++) {
    int lead = -1;
    for (int j = 0; j < c && lead < 0; j++) {
      if (fabs(m[i][j]) > EPS) {
        lead = j;
      }
    }
    if (lead < 0 && fabs(m[i][c]) > EPS) {
      return -1;
    }
    if (lead > i) {
      return -2;
    }
  }
  x->resize(c);
  for (int i = 0; i < c; i++) {
    (*x)[i] = m[i][c];
  }
  return 0;
}

/*** Example Usage ***/

#include <cassert>
using namespace std;

int main() {
  const int equations = 3, unknowns = 3;
  const int a[equations][unknowns] = {{-1, 2, 5}, {1, 0, -6}, {-4, 2, 2}};
  const int b[equations] = {3, 1, -2};
  vector<vector<double> > m(equations);
  for (int i = 0; i < equations; i++) {
    m[i].assign(a[i], a[i] + unknowns);
  }
  vector<double> x;
  assert(solve_system(m, vector<double>(b, b + equations), &x) == 0);
  for (int i = 0; i < equations; i++) {
    double sum = 0;
    for (int j = 0; j < unknowns; j++) {
      sum += a[i][j]*x[j];
    }
    assert(fabs(sum - b[i]) < EPS);
  }
  return 0;
}
\end{lstlisting}
\subsection{Determinant and Inverse}
\begin{lstlisting}
/*

Computes the determinant and inverse of a square matrix using Gaussian
elimination. The inverse of a matrix a is another matrix b such that a*b equals
the identity matrix. The inverse of a exists if and only if the determinant of a
is zero. In this case, a is called invertible or non-singular. In practice,
simple Gaussian elimination is prone to rounding error on certain matrices. For
a more accurate algorithm for solving systems of linear equations, see LU
decomposition with row partial pivoting should be.

- det_naive(a) returns the determinant of an n by n matrix a, using the classic
  divide-and-conquer algorithm by Laplace expansions.
- det(a) returns the determinant of an n by n matrix a using Gaussian
  elimination.
- invert(a) assigns the n by n matrix a to its inverse (if it exists), returning
  a reference to the modified argument itself. If a is not invertible, then its
  assigned values after the function call will be undefined (+/-Inf or +/-NaN).

Time Complexity:
- O(n!) per call to det_naive(), where n is the dimension of the matrix.
- O(n^3) per call to det() and invert() where n is the dimension of the matrix.

Space Complexity:
- O(n) auxiliary stack space and O(n!*n) auxiliary heap space for det_naive(),
  where n is the dimension of the matrix.
- O(n^2) auxiliary heap space for det() and invert().

*/

#include <cmath>
#include <map>
#include <vector>

template<class SquareMatrix>
double det_naive(const SquareMatrix &a) {
  int n = a.size();
  if (n == 1) {
    return a[0][0];
  }
  if (n == 2) {
    return a[0][0]*a[1][1] - a[0][1]*a[1][0];
  }
  double res = 0;
  SquareMatrix temp(n - 1, typename SquareMatrix::value_type(n - 1));
  for (int p = 0; p < n; p++) {
    int h = 0, k = 0;
    for (int i = 1; i < n; i++) {
      for (int j = 0; j < n; j++) {
        if (j == p) {
          continue;
        }
        temp[h][k++] = a[i][j];
        if (k == n - 1) {
          h++;
          k = 0;
        }
      }
    }
    res += (p % 2 == 0 ? 1 : -1)*a[0][p]*det_naive(temp);
  }
  return res;
}

template<class SquareMatrix>
double det(const SquareMatrix &a, double EPS = 1e-10) {
  SquareMatrix b(a);
  int n = a.size();
  double res = 1.0;
  std::vector<bool> used(n, false);
  for (int i = 0; i < n; i++) {
    int p;
    for (p = 0; p < n; p++) {
      if (!used[p] && fabs(b[p][i]) > EPS) {
        break;
      }
    }
    if (p >= n) {
      return 0;
    }
    res *= b[p][i];
    used[p] = true;
    double z = 1.0/b[p][i];
    for (int j = 0; j < n; j++) {
      b[p][j] *= z;
    }
    for (int j = 0; j < n; j++) {
      if (j != p) {
        z = b[j][i];
        for (int k = 0; k < n; k++) {
          b[j][k] -= z*b[p][k];
        }
      }
    }
  }
  return res;
}

template<class SquareMatrix>
SquareMatrix& invert(SquareMatrix &a) {
  int n = a.size();
  for (int i = 0; i < n; i++) {
    a[i].resize(2*n);
    for (int j = n; j < n*2; j++) {
      a[i][j] = (i == j - n ? 1 : 0);
    }
  }
  for (int i = 0; i < n; i++) {
    double z = a[i][i];
    for (int j = i; j < n*2; j++) {
      a[i][j] /= z;
    }
    for (int j = 0; j < n; j++) {
      if (i != j) {
        double z = a[j][i];
        for (int k = 0; k < n*2; k++) {
          a[j][k] -= z*a[i][k];
        }
      }
    }
  }
  for (int i = 0; i < n; i++) {
    a[i].erase(a[i].begin(), a[i].begin() + n);
  }
  return a;
}

/*** Example Usage ***/

#include <cassert>
using namespace std;

int main() {
  const int n = 3, a[n][n] = {{6, 1, 1}, {4, -2, 5}, {2, 8, 7}};
  vector<vector<double> > m(n), inv, res(n, vector<double>(n, 0));
  for (int i = 0; i < n; i++) {
    m[i] = vector<double>(a[i], a[i] + n);
  }
  double d = det(m);
  assert(fabs(d - det_naive(m)) < 1e-10);
  invert(inv = m);
  for (int i = 0; i < n; i++) {
    for (int j = 0; j < n; j++) {
      for (int k = 0; k < n; k++) {
        res[i][j] += a[i][k]*inv[k][j];
      }
    }
  }
  for (int i = 0; i < n; i++) {
    for (int j = 0; j < n; j++) {
      assert(fabs(res[i][j] - (i == j ? 1 : 0)) < 1e-10);
    }
  }
  return 0;
}
\end{lstlisting}
\subsection{LU Decomposition}
\begin{lstlisting}
/*

The LU decomposition of a matrix a with row-partial pivoting is a factorization
of a (after some rows are possibly permuted by a permutation matrix p) as a
product of a lower triangular matrix l and an upper triangular matrix u. This
factorization can be used to tackle many common problems in linear algebra such
as solving systems of linear equations and computing determinants. An
improvement on basic row reduction, LU decomposition by row-partial pivoting
keeps the relative magnitude of matrix values small, thus reducing the relative
error due to rounding in computed solutions.

- lu_decompose(a, &p1col) assigns the r by c matrix a to merged LU decomposition
  matrix lu, returning either 0 or 1 denoting the "sign" of the permutation
  parity (0 if the number of overall row swaps performed is even, or 1 if it is
  odd), or -1 denoting a degenerate matrix (i.e. singular for square matrices).
  The merged matrix lu has lu[i][j] = l[i][j] for i > j and lu[i][j] = u[i][j]
  for i <= j. Note that the algorithm always yields an atomic lower triangular
  matrix for which the diagonal entries l[i][i] are always equal to 1, so this
  is not explicitly stored in the resulting merged matrix. For general i and j,
  the values of the lower and upper triangular matrices should be accessed via
  the getl(lu, i, j) and getu(lu, i, j) functions. Optionally, a vector<int>
  pointer p1col may be passed to return the permutation vector p1col where
  p1col[i] stores the only column that is equal to 1 in row i of the permutation
  matrix p (all other columns in row i of p are implicitly 0). The resulting
  permutation matrix p corresponding to p1col will satisfy p*a = l*u.
- solve_system(a, b, &x) solves the system of linear equations a*x = b given an
  r by c matrix a of real values, and a length r vector b, returning 0 if there
  is one solution or -1 if there are zero or infinite solutions. If there is
  exactly one solution, then the vector pointed to by x is populated with the
  solution vector of length c.
- det(a) returns the determinant of an n by n matrix a using LU decomposition.
- invert(a) assigns the n by n matrix a to its inverse (if it exists), returning
  0 if the inversion was successful or -1 if a has no inverse.

Time Complexity:
- O(r^2*c) per call to lu_decompose(a) and solve_system(a, b), where r and c are
  the number of rows and columns respectively, in accordance to the functions'
  descriptions above.
- O(n^3) per call to det(a) and inverse(a), where n is the dimension of a.

Space Complexity:
- O(1) auxiliary for lu_decompose().
- O(n^2) for det(a) and inverse(a).
- O(r*c) auxiliary heap space for solve_system(a, b).

*/

#include <algorithm>
#include <cmath>
#include <cstddef>
#include <limits>
#include <vector>

template<class Matrix>
int lu_decompose(Matrix &a, std::vector<int> *p1col = NULL,
                 const double EPS = 1e-10) {
  int r = a.size(), c = a[0].size(), parity = 0;
  if (p1col != NULL) {
    p1col->resize(r);
    for (int i = 0; i < r; i++) {
      (*p1col)[i] = i;
    }
  }
  for (int i = 0; i < r && i < c; i++) {
    int pi = i;
    for (int k = i + 1; k < r; k++) {
      if (fabs(a[k][i]) > fabs(a[pi][i])) {
        pi = k;
      }
    }
    if (fabs(a[pi][i]) < EPS) {
      return -1;
    }
    if (pi != i) {
      if (p1col != NULL) {
        std::iter_swap(p1col->begin() + i, p1col->begin() + pi);
      }
      std::iter_swap(a.begin() + i, a.begin() + pi);
      parity = 1 - parity;
    }
    for (int j = i + 1; j < r; j++) {
      a[j][i] /= a[i][i];
      for (int k = i + 1; k < c; k++) {
        a[j][k] -= a[j][i]*a[i][k];
      }
    }
  }
  return parity;
}

template<class Matrix>
double getl(const Matrix &lu, int i, int j) {
  return i > j ? lu[i][j] : (i < j ? 0 : 1);
}

template<class Matrix>
double getu(const Matrix &lu, int i, int j) {
  return i <= j ? lu[i][j] : 0;
}

template<class Matrix, class T>
int solve_system(const Matrix &a, const std::vector<T> &b, std::vector<T> *x,
                 const double EPS = 1e-10) {
  int r = a.size(), c = a[0].size();
  if (x == NULL || a.empty() || a.size() != b.size() || r < c) {
    return -1;
  }
  x->resize(c);
  std::vector<int> p1col;
  Matrix lu;
  int status = lu_decompose(lu = a, &p1col, EPS);
  if (status < 0) {
    return status;
  }
  for (int i = 0; i < c; i++) {
    (*x)[i] = b[p1col[i]];
    for (int k = 0; k < i; k++) {
      (*x)[i] -= getl(lu, i, k)*(*x)[k];
    }
  }
  for (int i = c - 1; i >= 0; i--) {
    for (int k = i + 1; k < c; k++) {
      (*x)[i] -= getu(lu, i, k)*(*x)[k];
    }
    (*x)[i] /= getu(lu, i, i);
  }
  for (int i = 0; i < r; i++) {
    double val = 0;
    for (int j = 0; j < c; j++) {
      val += a[i][j]*(*x)[j];
    }
    if (fabs(val - b[i])/b[i] > EPS) {
      return -1;
    }
  }
  return 0;
}

template<class SquareMatrix>
double det(const SquareMatrix &a) {
  int n = a.size();
  SquareMatrix lu;
  int status = lu_decompose(lu = a);
  if (status < 0) {
    return 0;
  }
  double res = 1;
  for (int i = 0; i < n; i++) {
    res *= lu[i][i];
  }
  return status == 0 ? res : -res;
}

template<class SquareMatrix>
int invert(SquareMatrix &a) {
  int n = a.size();
  std::vector<int> p1col;
  int status = lu_decompose(a, &p1col);
  if (status < 0) {
    return status;
  }
  SquareMatrix ia(n, typename SquareMatrix::value_type(n, 0));
  for (int j = 0; j < n; j++) {
    for (int i = 0; i < n; i++) {
      if (p1col[i] == j) {
        ia[i][j] = 1.0;
      } else {
        ia[i][j] = 0.0;
      }
      for (int k = 0; k < i; k++) {
        ia[i][j] -= getl(a, i, k)*ia[k][j];
      }
    }
    for (int i = n - 1; i >= 0; i--) {
      for (int k = i + 1; k < n; k++) {
        ia[i][j] -= getu(a, i, k)*ia[k][j];
      }
      ia[i][j] /= getu(a, i, i);
    }
  }
  a.swap(ia);
  return 0;
}

/*** Example Usage ***/

#include <cassert>
using namespace std;

int main() {
  { // Solve a system.
    const int equations = 3, unknowns = 3;
    const int a[equations][unknowns] = {{-1, 2, 5}, {1, 0, -6}, {-4, 2, 2}};
    const int b[equations] = {3, 1, -2};
    vector<vector<double> > m(equations);
    for (int i = 0; i < equations; i++) {
      m[i].assign(a[i], a[i] + unknowns);
    }
    vector<double> x;
    assert(solve_system(m, vector<double>(b, b + equations), &x) == 0);
    for (int i = 0; i < equations; i++) {
      double sum = 0;
      for (int j = 0; j < unknowns; j++) {
        sum += a[i][j]*x[j];
      }
      assert(fabs(sum - b[i]) < 1e-10);
    }
  }
  { // Find the determinant.
    const int n = 3, a[n][n] = {{1, 3, 5}, {2, 4, 7}, {1, 1, 0}};
    vector<vector<double> > m(n);
    for (int i = 0; i < n; i++) {
      m[i] = vector<double>(a[i], a[i] + n);
    }
    assert(fabs(det(m) - 4) < 1e-10);
  }
  { // Find the inverse.
    const int n = 3, a[n][n] = {{6, 1, 1}, {4, -2, 5}, {2, 8, 7}};
    vector<vector<double> > m(n), res(n, vector<double>(n, 0));
    for (int i = 0; i < n; i++) {
      m[i] = vector<double>(a[i], a[i] + n);
    }
    assert(invert(m) == 0);
    for (int i = 0; i < n; i++) {
      for (int j = 0; j < n; j++) {
        for (int k = 0; k < n; k++) {
          res[i][j] += a[i][k]*m[k][j];
        }
      }
    }
    for (int i = 0; i < n; i++) {
      for (int j = 0; j < n; j++) {
        assert(fabs(res[i][j] - (i == j ? 1 : 0)) < 1e-10);
      }
    }
  }
  return 0;
}
\end{lstlisting}
\subsection{Linear Programming (Simplex)}
\begin{lstlisting}
/*

Solves a linear programming problem using Dantzig's simplex algorithm. The
canonical form of a linear programming problem is to maximize (or minimize) the
dot product c*x, subject to a*x <= b and x >= 0, where x is a vector of unknowns
to be solved, c is a vector of coefficients, a is a matrix of linear equation
coefficients, and b is a vector of boundary coefficients.

- simplex_solve(a, b, c, &x) solves the linear programming problem for an m by n
  matrix a of real values, a length m vector b, a length n vector c, returning 0
  if a solution was found or -1 if there are no solutions. If a solution is
  found, then the vector pointed to by x is populated with the solution vector
  of length n.

Time Complexity:
- Polynomial (average) on the number of equations and unknowns, but exponential
  in the worst case.

Space Complexity:
- O(m*n) auxiliary heap space.

*/

#include <cmath>
#include <limits>
#include <vector>

template<class Matrix>
int simplex_solve(const Matrix &a, const std::vector<double> &b,
                  const std::vector<double> &c, std::vector<double> *x,
                  const bool MAXIMIZE = true, const double EPS = 1e-10) {
  int m = a.size(), n = c.size();
  Matrix t(m + 2, std::vector<double>(n + 2));
  t[1][1] = 0;
  for (int j = 1; j <= n; j++) {
    t[1][j + 1] = MAXIMIZE ? c[j - 1] : -c[j - 1];
  }
  for (int i = 1; i <= m; i++) {
    for (int j = 1; j <= n; j++) {
      t[i + 1][j + 1] = -a[i - 1][j - 1];
    }
    t[i + 1][1] = b[i - 1];
  }
  for (int j = 1; j <= n; j++) {
    t[0][j + 1] = j;
  }
  for (int i = n + 1; i <= m + n; i++) {
    t[i - n + 1][0] = i;
  }
  double p1 = 0, p2 = 0;
  bool done = true;
  do {
    double mn = std::numeric_limits<double>::max(), xmax = 0, v;
    for (int j = 2; j <= n + 1; j++) {
      if (t[1][j] > 0 && t[1][j] > xmax) {
        p2 = j;
        xmax = t[1][j];
      }
    }
    for (int i = 2; i <= m + 1; i++) {
      v = fabs(t[i][1] / t[i][p2]);
      if (t[i][p2] < 0 && mn > v) {
        mn = v;
        p1 = i;
      }
    }
    std::swap(t[p1][0], t[0][p2]);
    for (int i = 1; i <= m + 1; i++) {
      if (i != p1) {
        for (int j = 1; j <= n + 1; j++) {
          if (j != p2) {
            t[i][j] -= t[p1][j]*t[i][p2] / t[p1][p2];
          }
        }
      }
    }
    t[p1][p2] = 1.0 / t[p1][p2];
    for (int j = 1; j <= n + 1; j++) {
      if (j != p2) {
        t[p1][j] *= fabs(t[p1][p2]);
      }
    }
    for (int i = 1; i <= m + 1; i++) {
      if (i != p1) {
        t[i][p2] *= t[p1][p2];
      }
    }
    for (int i = 2; i <= m + 1; i++) {
      if (t[i][1] < 0) {
        return -1;
      }
    }
    done = true;
    for (int j = 2; j <= n + 1; j++) {
      if (t[1][j] > 0) {
        done = false;
      }
    }
  } while (!done);
  x->clear();
  for (int j = 1; j <= n; j++) {
    for (int i = 2; i <= m + 1; i++) {
      if (fabs(t[i][0] - j) < EPS) {
        x->push_back(t[i][1]);
      }
    }
  }
  return 0;
}

/*** Example Usage and Output:

Solution = 33.3043 at (5.30435, 4.34783).

***/

#include <cassert>
#include <iostream>
using namespace std;

int main() {
  // Solve [x, y] that maximizes 3x + 4y, subject to x, y >= 0 and:
  //  -2x +    1y <=  0
  //   1x + 0.85y <=  9
  //   1x +    2y <= 14
  const int equations = 3, unknowns = 2;
  double a[equations][unknowns] = {{-2, 1}, {1, 0.85}, {1, 2}};
  double b[equations] = {0, 9, 14};
  double c[unknowns] = {3, 4};
  vector<vector<double>> va(equations, vector<double>(unknowns));
  vector<double> vb(b, b + equations), vc(c, c + unknowns), x;
  for (int i = 0; i < equations; i++) {
    for (int j = 0; j < unknowns; j++) {
      va[i][j] = a[i][j];
    }
  }
  assert(simplex_solve(va, vb, vc, &x) == 0);
  double maxval = 0;
  for (int i = 0; i < (int)x.size(); i++) {
    maxval += c[i]*x[i];
  }
  cout << "Solution = " << maxval << " at (" << x[0];
  for (int i = 1; i < (int)x.size(); i++) {
    cout << ", " << x[i];
  }
  cout << ")." << endl;
  return 0;
}
\end{lstlisting}

\section{Root Finding and Calculus}
\setcounter{section}{6}
\setcounter{subsection}{0}
\subsection{Root Finding (Bracketing)}
\begin{lstlisting}
/*

Finds an x in an interval [a, b] for a continuous function f such that f(x) = 0.
By the intermediate value theorem, a root must exist in [a, b] if the signs of
f(a) and f(b) differ. The answer is found with an absolute error of roughly
1/(2^n), where n is the number of iterations. Although it is possible to control
the error by looping while b - a is greater than an arbitrary epsilon, it is
simpler to let the loop run for a desired number of iterations until floating
point arithmetic break down. 100 iterations is usually sufficient, since the
search space will be reduced to 2^-100 (roughly 10^-30) times its original size.

- bisection_root(f, a, b) returns a root in an interval [a, b] for a continuous
  function f where sgn(f(a)) != sgn(f(b)), using the bisection method.
- falsi_root(f, a, b) returns a root in an interval [a, b] for a continuous
  function f where sgn(f(a)) != sgn(f(b)), using the Illinois algorithm variant
  of the false position (a.k.a. regula falsi) method.

Time Complexity:
- O(n) calls will be made to f() in bisection_root() and falsi_illinois_root(),
  where n is the number of iterations performed.

Space Complexity:
- O(1) auxiliary space for both operations.

*/

#include <stdexcept>

template<class ContinuousFunction>
double bisection_root(ContinuousFunction f, double a, double b,
                      const int ITERATIONS = 100) {
  if (a > b || f(a)*f(b) > 0) {
    throw std::runtime_error("Must give [a, b] where sgn(f(a)) != sgn(f(b)).");
  }
  double m;
  for (int i = 0; i < ITERATIONS; i++) {
    m = a + (b - a)/2;
    if (f(a)*f(m) >= 0) {
      a = m;
    } else {
      b = m;
    }
  }
  return m;
}

template<class ContinuousFunction>
double falsi_illinois_root(ContinuousFunction f, double a, double b,
                           const int ITERATIONS = 100) {
  if (a > b || f(a)*f(b) > 0) {
    throw std::runtime_error("Must give [a, b] where sgn(f(a)) != sgn(f(b)).");
  }
  double m, fm, fa = f(a), fb = f(b);
  int side = 0;
  for (int i = 0; i < ITERATIONS; i++) {
    m = (fa*b - fb*a)/(fa - fb);
    fm = f(m);
    if (fb*fm > 0) {
      b = m;
      fb = fm;
      if (side < 0) {
        fa /= 2;
      }
      side = -1;
    } else if (fa*fm > 0) {
      a = m;
      fa = fm;
      if (side > 1) {
        fb /= 2;
      }
      side = 1;
    } else {
      break;
    }
  }
  return m;
}

/*** Example Usage ***/

#include <cassert>
#include <cmath>

double f(double x) {
  return x*x - 4*sin(x);
}

int main() {
  assert(fabs(f(bisection_root(f, 1, 3))) < 1e-10);
  assert(fabs(f(falsi_illinois_root(f, 1, 3))) < 1e-10);
  return 0;
}
\end{lstlisting}
\subsection{Root Finding (Iteration)}
\begin{lstlisting}
/*

Finds an x for a continuous function f such that f(x) = 0 using iterative
approximation by an initial guess that is close to the answer. Newton's method
requires an explicit definition of the function's derivative while the secant
method starts with two initial guesses and approximates the derivative using the
secant slope from the previous iteration. For n iterations and a good initial
guess, the methods below compute approximately 2^n digits of precision, with the
secant method converging approximately 1.6 times slower than Newton's.

- newton_root(f, fprime, x0) returns a root x for a fuction f with derivative
  fprime using an initial guess x0 which should be relatively close to x.
- secant_root(f, x0, x1) returns a root x for a function f using two initial
  guesses x0 and x1 which should be relatively close to x.

Time Complexity:
- O(n) calls will be made to f() in newton_root() and secant_root(), where n is
  the number of iterations performed.

Space Complexity:
- O(1) auxiliary space for both operations.

*/

#include <cmath>
#include <stdexcept>

template<class ContinuousFunction>
double newton_root(ContinuousFunction f, ContinuousFunction fprime, double x0,
                   const double EPS = 1e-15, const int ITERATIONS = 100) {
  double x = x0, error = EPS + 1;
  for (int i = 0; error > EPS && i < ITERATIONS; i++) {
    double xnew = x - f(x)/fprime(x);
    error = fabs(xnew - x);
    x = xnew;
  }
  if (error > EPS) {
    throw std::runtime_error("Newton's method failed to converge.");
  }
  return x;
}

template<class ContinuousFunction>
double secant_root(ContinuousFunction f, double x0, double x1,
                   const double EPS = 1e-15, const int ITERATIONS = 100) {
  double xold = x0, fxold = f(x0), x = x1, error = EPS + 1;
  for (int i = 0; error > EPS && i < ITERATIONS; i++) {
    double fx = f(x);
    double xnew = x - fx*((x - xold)/(fx - fxold));
    xold = x;
    fxold = fx;
    error = fabs(xnew - x);
    x = xnew;
  }
  if (error > EPS) {
    throw std::runtime_error("Secant method failed to converge.");
  }
  return x;
}

/*** Example Usage ***/

#include <cassert>

double f(double x) {
  return x*x - 4*sin(x);
}

double fprime(double x) {
  return 2*x - 4*cos(x);
}

int main() {
  assert(fabs(f(newton_root(f, fprime, 3))) < 1e-10);
  assert(fabs(f(secant_root(f, 3, 2))) < 1e-10);
  return 0;
}
\end{lstlisting}
\subsection{Polynomial Root Finding (Differentiation)}
\begin{lstlisting}
/*

Finds every root x for a polynomial p such that p(x) = 0 by differentiation.
Each adjacent pair of local extrema is searched using the bisection method,
where local extrema are recursively found by finding the root of the derivative.

- horner_eval(p, x) evaluates the polynomial p of degree d (represented as a
  vector of size d + 1 where p[i] stores the coefficient for the x^i term) at x,
  using Horner's method.
- find_one_root(p, a, b, EPS) returns a root in the interval [a, b] for a
  polynomial p where sgn(f(a)) != sgn(f(b)), using the bisection method. If this
  precondition is not satisfied, then NaN is returned. The root is found to a
  tolerance of EPS in absolute or relative error (whichever is reached first).
- find_all_roots(p, a, b, EPS) returns a vector of all roots in the interval
  [a, b] for a polynomial p using the bisection method. The roots are found to a
  tolerance of EPS in absolute or relative error (whichever is reached first).

Time Complexity:
- O(n) per call to horner_eval(), where n is the degree of the polynomial.
- O(n log p) per call to find_one_root(), where n is the degree of the
  polynomial and p = -log10(EPS) is the number of digits of absolute or relative
  precision that is desired.
- O(n^3 log p) per call to find_all_roots(), where n is the degree of the
  polynomial and p = -log10(EPS) is the number of digits of absolute or relative
  precision that is desired.

Space Complexity:
- O(1) auxiliary space for horner_eval() and find_one_root().
- O(n^2) auxiliary heap and O(n) auxiliary stack space for find_all_roots(),
  where n is the degree of the polynomial.

*/

#include <cmath>
#include <limits>
#include <utility>
#include <vector>

double horner_eval(const std::vector<double> &p, double x) {
  double res = p.back();
  for (int i = (int)p.size() - 2; i >= 0; i--) {
    res = res*x + p[i];
  }
  return res;
}

double find_one_root(const std::vector<double> &p, double a, double b,
                     const double EPS = 1e-15) {
  double pa = horner_eval(p, a), pb = horner_eval(p, b);
  bool paneg = pa < 0, pbneg = pb < 0;
  if (paneg == pbneg) {
    return std::numeric_limits<double>::quiet_NaN();
  }
  while (b - a > EPS && a*(1 + EPS) < b && a < b*(1 + EPS)) {
    double m = a + (b - a)/2;
    if ((horner_eval(p, m) < 0) == paneg) {
      a = m;
    } else {
      b = m;
    }
  }
  return a;
}

std::vector<double> find_all_roots(const std::vector<double> &p,
                                   double a = -1e20, double b = 1e20,
                                   const double EPS = 1e-15) {
  std::vector<double> pprime;
  for (int i = 1; i < (int)p.size(); i++) {
    pprime.push_back(p[i]*i);
  }
  if (pprime.empty()) {
    return std::vector<double>();
  }
  std::vector<double> res, r = find_all_roots(pprime, a, b, EPS);
  r.push_back(b);
  for (int i = 0; i < (int)r.size(); i++) {
    double root = find_one_root(p, i == 0 ? a : r[i - 1], r[i], EPS);
    if (!std::isnan(root) && (res.empty() || root != res.back())) {
      res.push_back(root);
    }
  }
  return res;
}

/*** Example Usage ***/

#include <cassert>
using namespace std;

int main() {
  { // -1 + 2x - 6x^2 + 2x^3
    int poly[] = {-1, 2, -6, 2};
    vector<double> p(poly, poly + 4), roots = find_all_roots(p);
    assert(roots.size() == 1 && fabs(horner_eval(p, roots[0])) < 1e-10);
  }
  { // -20 + 4x + 3x^2
    int poly[] = {-20, 4, 3};
    vector<double> p(poly, poly + 3), roots = find_all_roots(p);
    assert(roots.size() == 2);
    assert(fabs(horner_eval(p, roots[0])) < 1e-10);
    assert(fabs(horner_eval(p, roots[1])) < 1e-10);
  }
  return 0;
}
\end{lstlisting}
\subsection{Polynomial Root Finding (Laguerre)}
\begin{lstlisting}
/*

Finds every complex root x for a polynomial p with complex coefficients such
that p(x) = 0 using Laguerre's method.

- horner_eval(p, x) evaluates the complex polynomial p of degree d (represented
  as a vector of size d + 1 where p[i] stores the complex coefficient for the
  x^i term) at x, using Horner's method, returning a pair where the first value
  is a vector of sub-evaluations and the second value is the final result p(x).
- find_one_root(p, x0) returns a complex root x for a polynomial p (represented
  as a vector of size d + 1 where p[i] stores the complex coefficient for the
  x^i term) using an initial guess x0 which should be relatively close to x. The
  root is found to a tolerance of EPS in absolute or relative error (whichever
  is reached first).
- find_all_roots(p) returns a vector of all complex roots for a complex
  polynomial p. The roots are found to a tolerance of EPS in absolute or
  relative error (whichever is reached first).

Time Complexity:
- O(n) per call to horner_eval(), where n is the degree of the polynomial.
- O(n log p) per call to find_one_root(), where n is the degree of the
  polynomial and p = -log10(EPS) is the number of digits of absolute or relative
  precision that is desired.
- O(n^2 log p) per call to find_all_roots(), where n is the degree of the
  polynomial and p = -log10(EPS) is the number of digits of absolute or relative
  precision that is desired.

Space Complexity:
- O(n) auxiliary heap space and O(1) auxiliary stack space for horner_eval() and
  find_one_root(), where n is the degree of the polynomial.
- O(n) auxiliary heap and O(1) auxiliary stack space per for find_one_root() and
  find_all_roots(), where n is the degree of the polynomial.

*/

#include <complex>
#include <cstdlib>
#include <vector>

typedef std::complex<double> cdouble;
typedef std::vector<cdouble> cpoly;

std::pair<cdouble, cpoly> horner_eval(const cpoly &p, const cdouble &x) {
  int n = p.size();
  cpoly b(std::max(1, n - 1));
  for (int i = n - 1; i > 0; i--) {
    b[i - 1] = p[i] + (i < n - 1 ? b[i]*x : 0);
  }
  return std::make_pair(p[0] + b[0]*x, b);
}

cpoly derivative(const cpoly &p) {
  int n = p.size();
  cpoly res(std::max(1, n - 1));
  for (int i = 1; i < n; i++) {
    res[i - 1] = p[i]*cdouble(i);
  }
  return res;
}

int comp(const cdouble &a, const cdouble &b, const double EPS = 1e-15) {
  double diff = std::abs(a) - std::abs(b);
  return (diff < -EPS) ? -1 : (diff > EPS ? 1 : 0);
}

cdouble find_one_root(const cpoly &p, const cdouble &x0,
                      const double EPS = 1e-15, const int ITERATIONS = 10000) {
  cdouble x = x0;
  int n = p.size() - 1;
  cpoly p1 = derivative(p), p2 = derivative(p1);
  for (int i = 0; i < ITERATIONS; i++) {
    cdouble y0 = horner_eval(p, x).first;
    if (comp(y0, 0, EPS) == 0) {
      break;
    }
    cdouble g = horner_eval(p1, x).first/y0;
    cdouble h = g*g - horner_eval(p2, x).first/y0;
    cdouble r = std::sqrt(cdouble(n - 1)*(h*cdouble(n) - g*g));
    cdouble d1 = g + r, d2 = g - r;
    cdouble a = cdouble(n)/(comp(d1, d2, EPS) > 0 ? d1 : d2);
    x -= a;
    if (comp(a, 0, EPS) == 0) {
      break;
    }
  }
  return x;
}

std::vector<cdouble> find_all_roots(const cpoly &p, const double EPS = 1e-15,
                                    const int ITERATIONS = 10000) {
  std::vector<cdouble> res;
  cpoly q = p;
  while (q.size() > 2) {
    cdouble z = cdouble(rand(), rand())/(double)RAND_MAX;
    z = find_one_root(p, find_one_root(q, z, EPS, ITERATIONS), EPS, ITERATIONS);
    q = horner_eval(q, z).second;
    res.push_back(z);
  }
  res.push_back(-q[0] / q[1]);
  return res;
}

/*** Example Usage and Output:

Roots of 140 - 13x - 8x^2 + x^3:
(5.00000, 0.00000)
(-4.00000, -0.00000)
(7.00000, -0.00000)
Roots of ((2 + 3i)x + 6)(x + i)(2x + (6 + 4i))(xi + 1):
(0.00000, 1.00000)
(0.00000, -1.00000)
(-0.92308, 1.38462)
(-3.00000, -2.00000)

***/

#include <cstdio>
#include <iostream>
using namespace std;

void print_roots(const vector<cdouble> &x) {
  for (int i = 0; i < (int)x.size(); i++) {
    printf("(%.5lf, %.5lf)\n", x[i].real(), x[i].imag());
  }
}

int main() {
  { // 140 - 13x - 8x^2 + x^3 = (x + 4)(x - 5)(x - 7)
    printf("Roots of 140 - 13x - 8x^2 + x^3:\n");
    cpoly p;
    p.push_back(140);
    p.push_back(-13);
    p.push_back(-8);
    p.push_back(1);
    print_roots(find_all_roots(p));
  }
  { // (-24+36i) + (-26+12i)x + (-30+40i)x^2 + (-26+12i)x^3 + (-6+4i)x^4
    // = ((2 + 3i)x + 6)(x + i)(2x + (6 + 4i))(xi + 1):
    printf("Roots of ((2 + 3i)x + 6)(x + i)(2x + (6 + 4i))(xi + 1):\n");
    cpoly p;
    p.push_back(cdouble(-24, 36));
    p.push_back(cdouble(-26, 12));
    p.push_back(cdouble(-30, 40));
    p.push_back(cdouble(-26, 12));
    p.push_back(cdouble(-6, 4));
    print_roots(find_all_roots(p));
  }
  return 0;
}
\end{lstlisting}
\subsection{Polynomial Root Finding (RPOLY)}
\begin{lstlisting}
/*

Finds every complex root x for a polynomial p with real coefficients such that
p(x) = 0 using a variant of the Jenkins-Traub algorithm known as RPOLY. This
implementation is adapted from TOMS493 (www.netlib.org/toms/) with a simple
wrapper for the C++ <complex> class.

- find_all_roots(p) returns a vector of all complex roots for a polynomial p
  with real coefficients.

Time Complexity:
- O(n) per call to find_all_roots(p), where n is the degree of the polynomial.

Space Complexity:
- O(n) auxiliary stack space, where n is the degree of the polynomial.

*/

#include <cfloat>
#include <cmath>

typedef long double LD;
const int MAXN = 105;

void divide_quadratic(int n, LD u, LD v, LD p[], LD q[], LD *a, LD *b) {
  q[0] = *b = p[0];
  q[1] = *a = -((*b)*u) + p[1];
  for (int i = 2; i < n; i++) {
    q[i] = -((*a)*u + (*b)*v) + p[i];
    *b = *a;
    *a = q[i];
  }
}

int get_flag(int n, LD a, LD b, LD *a1, LD *a3, LD *a7, LD *c, LD *d, LD *e,
             LD *f, LD *g, LD *h, LD k[], LD u, LD v, LD qk[]) {
  divide_quadratic(n, u, v, k, qk, c, d);
  if (fabsl(*c) <= 100.0*LDBL_EPSILON*fabsl(k[n - 1]) &&
      fabsl(*d) <= 100.0*LDBL_EPSILON*fabsl(k[n - 2])) {
    return 3;
  }
  *h = v*b;
  if (fabsl(*d) >= fabsl(*c)) {
    *e = a/(*d);
    *f = (*c)/(*d);
    *g = u*b;
    *a1 = (*f) * b - a;
    *a3 = (*e) * ((*g) + a) + (*h)*(b/(*d));
    *a7 = (*h) + ((*f) + u) * a;
    return 2;
  }
  *e = a/(*c);
  *f = (*d)/(*c);
  *g = (*e)*u;
  *a1 = -(a*((*d) / (*c))) + b;
  *a3 = (*e)*a + ((*g) + (*h)/(*c))*b;
  *a7 = (*g)*(*d) + (*h)*(*f) + a;
  return 1;
}

void find_polynomials(int n, int flag, LD a, LD b, LD a1, LD *a3, LD *a7,
                      LD k[], LD qk[], LD qp[]) {
  if (flag == 3) {
    k[1] = k[0] = 0.0;
    for (int i = 2; i < n; i++) {
      k[i] = qk[i - 2];
    }
    return;
  }
  if (fabsl(a1) > 10.0*LDBL_EPSILON*fabsl(flag == 1 ? b : a)) {
    *a7 /= a1;
    *a3 /= a1;
    k[0] = qp[0];
    k[1] = qp[1] - (*a7)*qp[0];
    for (int i = 2; i < n; i++) {
      k[i] = qp[i] - ((*a7)*qp[i - 1]) + (*a3)*qk[i - 2];
    }
  } else {
    k[0] = 0.0;
    k[1] = -(*a7)*qp[0];
    for (int i = 2; i < n; i++) {
      k[i] = (*a3)*qk[i - 2] - (*a7)*qp[i - 1];
    }
  }
}

void estimate_coeff(int flag, LD *uu, LD *vv, LD a, LD a1, LD a3, LD a7, LD b,
                    LD c, LD d, LD f, LD g, LD h, LD u, LD v, LD k[], int n,
                    LD p[]) {
  LD a4, a5, b1, b2, c1, c2, c3, c4, temp;
  *vv = *uu = 0.0;
  if (flag == 3) {
    return;
  }
  if (flag != 2) {
    a4 = a + u*b + h*f;
    a5 = c + (u + v*f)*d;
  } else {
    a4 = (a + g)*f + h;
    a5 = (f + u)*c + v*d;
  }
  b1 = -k[n - 1] / p[n];
  b2 = -(k[n - 2] + b1*p[n - 1]) / p[n];
  c1 = v*b2*a1;
  c2 = b1*a7;
  c3 = b1*b1*a3;
  c4 = c1 - c2 - c3;
  temp = b1*a4 - c4 + a5;
  if (temp != 0.0) {
    *uu= u - (u*(c3 + c2) + v*(b1*a1 + b2*a7)) / temp;
    *vv = v*(1.0 + c4/temp);
  }
}

void solve_quadratic(LD a, LD b1, LD c, LD *sr, LD *si, LD *lr, LD *li) {
  LD b, d, e;
  *sr = *si = *lr = *li = 0.0;
  if (a == 0) {
    *sr = (b1 != 0) ? -c / b1 : *sr;
    return;
  }
  if (c == 0) {
    *lr = -b1 / a;
    return;
  }
  b = b1 / 2.0;
  if (fabsl(b) < fabsl(c)) {
    e = (c >= 0) ? a : -a;
    e = b*(b / fabsl(c)) - e;
    d = sqrtl(fabsl(e))*sqrtl(fabsl(c));
  } else {
    e = 1.0 - (a / b)*(c / b);
    d = sqrtl(fabsl(e))*fabsl(b);
  }
  if (e >= 0) {
    d = (b >= 0) ? -d : d;
    *lr = (d - b) / a;
    *sr = (*lr != 0) ? (c / *lr / a) : *sr;
  } else {
    *lr = *sr = -b / a;
    *si = fabsl(d / a);
    *li = -(*si);
  }
}

void quadratic_iterate(int N, int * NZ, LD uu, LD vv, LD *szr, LD *szi, LD *lzr,
                       LD *lzi, LD qp[], int n, LD *a, LD *b, LD p[], LD qk[],
                       LD *a1, LD *a3, LD *a7, LD *c, LD *d, LD *e, LD *f,
                       LD *g, LD *h, LD k[]) {
  int steps = 0, flag, tried_flag = 0;
  LD ee, mp, omp = 0.0, relstp = 0.0, t, u, ui, v, vi, zm;
  *NZ = 0;
  u = uu;
  v = vv;
  do {
    solve_quadratic(1.0, u, v, szr, szi, lzr, lzi);
    if (fabsl(fabsl(*szr) - fabsl(*lzr)) > 0.01*fabsl(*lzr)) {
      break;
    }
    divide_quadratic(n, u, v, p, qp, a, b);
    mp = fabsl(-((*szr)*(*b)) + *a) + fabsl((*szi)*(*b));
    zm = sqrtl(fabsl(v));
    ee = 2.0*fabsl(qp[0]);
    t = -(*szr)*(*b);
    for (int i = 1; i < N; i++) {
      ee = ee*zm + fabsl(qp[i]);
    }
    ee = ee*zm + fabsl(*a + t);
    ee = ee*9.0 + 2.0*fabsl(t) - 7.0*(fabsl(*a + t) + zm*fabsl(*b));
    ee *= LDBL_EPSILON;
    if (mp <= 20.0*ee) {
      *NZ = 2;
      break;
    }
    if (++steps > 20) {
      break;
    }
    if (steps >= 2 && relstp <= 0.01 && mp >= omp && !tried_flag) {
      relstp = (relstp < LDBL_EPSILON) ? sqrtl(LDBL_EPSILON) : sqrtl(relstp);
      u -= u*relstp;
      v += v*relstp;
      divide_quadratic(n, u, v, p, qp, a, b);
      for (int i = 0; i < 5; i++) {
        flag = get_flag(N, *a, *b, a1, a3, a7, c, d, e, f, g, h, k, u, v, qk);
        find_polynomials(N, flag, *a, *b, *a1, a3, a7, k, qk, qp);
      }
      tried_flag = 1;
      steps = 0;
    }
    omp = mp;
    flag = get_flag(N, *a, *b, a1, a3, a7, c, d, e, f, g, h, k, u, v, qk);
    find_polynomials(N, flag, *a, *b, *a1, a3, a7, k, qk, qp);
    flag = get_flag(N, *a, *b, a1, a3, a7, c, d, e, f, g, h, k, u, v, qk);
    estimate_coeff(flag, &ui, &vi, *a, *a1, *a3, *a7, *b, *c, *d, *f, *g, *h, u,
                   v, k, N, p);
    if (vi != 0) {
      relstp = fabsl((-v + vi)/vi);
      u = ui;
      v = vi;
    }
  } while (vi != 0);
}

void real_iterate(int *flag, int *nz, LD *sss, int n, LD p[], int nn, LD qp[],
                  LD *szr, LD *szi, LD k[], LD qk[]) {
  int steps = 0;
  LD ee, kv, mp, ms, omp = 0.0, pv, s, t = 0.0;
  *flag = *nz = 0;
  for (s = *sss; ; s += t) {
    pv = p[0];
    qp[0] = pv;
    for (int i = 1; i < nn; i++) {
      qp[i] = pv = pv * s + p[i];
    }
    mp = fabsl(pv);
    ms = fabsl(s);
    ee = 0.5*fabsl(qp[0]);
    for (int i = 1; i < nn; i++) {
      ee = ee*ms + fabsl(qp[i]);
    }
    if (mp <= 20.0*LDBL_EPSILON*(2.0*ee - mp)) {
      *nz = 1;
      *szr = s;
      *szi = 0.0;
      break;
    }
    if (++steps > 10) {
      break;
    }
    if (steps >= 2 && fabsl(t) <= 0.001*fabsl(s - t) && mp > omp) {
      *flag = 1;
      *sss = s;
      break;
    }
    omp = mp;
    qk[0] = kv = k[0];
    for (int i = 1; i < n; i++) {
      qk[i] = kv = kv*s + k[i];
    }
    if (fabsl(kv) > fabsl(k[n - 1])*10.0*LDBL_EPSILON) {
      t = -pv / kv;
      k[0] = qp[0];
      for (int i = 1; i < n; i++) {
        k[i] = t*qk[i - 1] + qp[i];
      }
    } else {
      k[0] = 0.0;
      for (int i = 1; i < n; i++) {
        k[i] = qk[i - 1];
      }
    }
    kv = k[0];
    for (int i = 1; i < n; i++) {
      kv = kv*s + k[i];
    }
    t = (fabsl(k[n - 1])*10.0*LDBL_EPSILON < fabsl(kv)) ? (-pv / kv) : 0.0;
  }
}

void solve_fixedshift(int l2, int *nz, LD sr, LD v, LD k[], int n, LD p[],
                      int nn, LD qp[], LD u, LD qk[], LD svk[], LD *lzi,
                      LD *lzr, LD *szi, LD *szr) {
  int flag, _flag, __flag = 1, spass, stry, vpass, vtry;
  LD a, a1, a3, a7, b, betas, betav, c, d, e, f, g, h;
  LD oss, ots = 0.0, otv = 0.0, ovv, s, ss, ts, tss, tv, tvv, ui, vi, vv;
  *nz = 0;
  betav = betas = 0.25;
  oss = sr;
  ovv = v;
  divide_quadratic(nn, u, v, p, qp, &a, &b);
  flag = get_flag(n, a, b, &a1, &a3, &a7, &c, &d, &e, &f, &g, &h, k, u, v, qk);
  for (int j = 0; j < l2; j++) {
    _flag = 1;
    find_polynomials(n, flag, a, b, a1, &a3, &a7, k, qk, qp);
    flag = get_flag(n, a, b, &a1, &a3, &a7, &c, &d, &e, &f, &g, &h, k, u, v,
                    qk);
    estimate_coeff(flag, &ui, &vi, a, a1, a3, a7, b, c, d, f, g, h, u, v, k, n,
                   p);
    vv = vi;
    ss = k[n - 1] != 0.0 ? (-p[n] / k[n - 1]) : 0.0;
    ts = tv = 1.0;
    if (j != 0 && flag != 3) {
      tv = (vv != 0.0) ? fabsl((vv - ovv) / vv) : tv;
      ts = (ss != 0.0) ? fabsl((ss - oss) / ss) : ts;
      tvv = (tv < otv) ? tv*otv : 1.0;
      tss = (ts < ots) ? ts*ots : 1.0;
      vpass = (tvv < betav) ? 1 : 0;
      spass = (tss < betas) ? 1 : 0;
      if (spass || vpass) {
        for (int i = 0; i < n; i++) {
          svk[i] = k[i];
        }
        s = ss; stry = vtry = 0;
        for (;;) {
          if (!(_flag && spass && (!vpass || tss < tvv))) {
            quadratic_iterate(n, nz, ui, vi, szr, szi, lzr, lzi, qp, nn, &a, &b,
                              p, qk, &a1, &a3, &a7, &c, &d, &e, &f, &g, &h, k);
            if (*nz > 0) return;
            __flag = vtry = 1;
            betav *= 0.25;
            if (stry || !spass) {
              __flag = 0;
            } else {
              for (int i = 0; i < n; i++) {
                k[i] = svk[i];
              }
            }
          }
          _flag = 0;
          if (__flag != 0) {
            real_iterate(&__flag, nz, &s, n, p, nn, qp, szr, szi, k, qk);
            if (*nz > 0) {
              return;
            }
            stry = 1;
            betas *= 0.25;
            if (__flag != 0) {
              ui = -(s + s);
              vi = s * s;
              continue;
            }
          }
          for (int i = 0; i < n; i++) k[i] = svk[i];
          if (!vpass || vtry) {
            break;
          }
        }
        divide_quadratic(nn, u, v, p, qp, &a, &b);
        flag = get_flag(n, a, b, &a1, &a3, &a7, &c, &d, &e, &f, &g, &h, k, u, v,
                        qk);
      }
    }
    ovv = vv;
    oss = ss;
    otv = tv;
    ots = ts;
  }
}

void find_all_roots(int degree, LD co[], LD re[], LD im[]) {
  int j, jj, n, nm1, nn, nz, zero;
  LD k[MAXN], p[MAXN], pt[MAXN], qp[MAXN], temp[MAXN], qk[MAXN], svk[MAXN];
  LD bnd, df, dx, factor, ff, moduli_max, moduli_min, sc, x, xm;
  LD aa, bb, cc, lzi, lzr, sr, szi, szr, t, u, xx, xxx, yy;
  n = degree;
  xx = sqrtl(0.5);
  yy = -xx;
  for (j = 0; co[n] == 0; n--, j++) {
    re[j] = im[j] = 0.0;
  }
  nn = n + 1;
  for (int i = 0; i < nn; i++) p[i] = co[i];
  while (n >= 1) {
    if (n <= 2) {
      if (n < 2) {
        re[degree - 1] = -p[1] / p[0];
        im[degree - 1] = 0.0;
      } else {
        solve_quadratic(p[0], p[1], p[2], &re[degree - 2], &im[degree - 2],
                        &re[degree - 1], &im[degree - 1]);
      }
      break;
    }
    moduli_max = 0.0;
    moduli_min = LDBL_MAX;
    for (int i = 0; i < nn; i++) {
      x = fabsl(p[i]);
      if (x > moduli_max) {
        moduli_max = x;
      }
      if (x != 0 && x < moduli_min) {
        moduli_min = x;
      }
    }
    sc = LDBL_MIN / LDBL_EPSILON / moduli_min;
    if ((sc < 2 && moduli_max >= 10) || (sc > 1 && LDBL_MAX/sc >= moduli_max)) {
      sc = (sc == 0) ? LDBL_MIN : sc;
      factor = powl(2.0, logl(sc) / logl(2.0));
      if (factor != 1.0) {
        for (int i = 0; i < nn; i++) {
          p[i] *= factor;
        }
      }
    }
    for (int i = 0; i < nn; i++) {
      pt[i] = fabsl(p[i]);
    }
    pt[n] = -pt[n];
    nm1 = n - 1;
    x = expl((logl(-pt[n]) - logl(pt[0])) / (LD)n);
    if (pt[nm1] != 0) {
      xm = -pt[n] / pt[nm1];
      if (xm < x) {
        x = xm;
      }
    }
    xm = x;
    do {
      x = xm;
      xm = 0.1 * x;
      ff = pt[0];
      for (int i = 1; i < nn; i++) {
        ff = ff*xm + pt[i];
      }
    } while (ff > 0);
    dx = x;
    do {
      df = ff = pt[0];
      for (int i = 1; i < n; i++) {
        ff = x*ff + pt[i];
        df = x*df + ff;
      }
      ff = x*ff + pt[n];
      dx = ff / df;
      x -= dx;
    } while (fabsl(dx / x) > 0.005);
    bnd = x;
    for (int i = 1; i < n; i++) {
      k[i] = (LD)(n - i)*p[i] / (LD)n;
    }
    k[0] = p[0];
    aa = p[n];
    bb = p[nm1];
    zero = (k[nm1] == 0) ? 1 : 0;
    for (jj = 0; jj < 5; jj++) {
      cc = k[nm1];
      if (zero) {
        for (int i = 0; i < nm1; i++) {
          j = nm1 - i;
          k[j] = k[j - 1];
        }
        k[0] = 0;
        zero = (k[nm1] == 0) ? 1 : 0;
      } else {
        t = -aa / cc;
        for (int i = 0; i < nm1; i++) {
          j = nm1 - i;
          k[j] = t*k[j - 1] + p[j];
        }
        k[0] = p[0];
        zero = (fabsl(k[nm1]) <= fabsl(bb)*LDBL_EPSILON*10.0) ? 1 : 0;
      }
    }
    for (int i = 0; i < n; i++) {
      temp[i] = k[i];
    }
    static const LD DEG = 0.01745329251994329576923690768489L;
    for (jj = 1; jj <= 20; jj++) {
      xxx = -sinl(94*DEG)*yy + cosl(94*DEG)*xx;
      yy = sinl(94*DEG)*xx + cosl(94*DEG)*yy;
      xx = xxx;
      sr = bnd*xx;
      u = -2.0*sr;
      for (int i = 0; i < nn; i++) {
        qk[i] = svk[i] = 0.0;
      }
      solve_fixedshift(20*jj, &nz, sr, bnd, k, n, p, nn, qp, u, qk, svk, &lzi,
                       &lzr, &szi, &szr);
      if (nz != 0) {
        j = degree - n;
        re[j] = szr;
        im[j] = szi;
        nn = nn - nz;
        n = nn - 1;
        for (int i = 0; i < nn; i++) {
          p[i] = qp[i];
        }
        if (nz != 1) {
          re[j + 1] = lzr;
          im[j + 1] = lzi;
        }
        break;
      } else {
        for (int i = 0; i < n; i++) {
          k[i] = temp[i];
        }
      }
    }
    if (jj > 20) {
      break;
    }
  }
}

/*** Example Usage and Output:

Roots of -1+2x^1-6x^2+2x^3:
(0.150976, 0.403144)
(0.150976, -0.403144)
(2.69805, 0)
Roots of -20+4x^1+3x^2:
(2, 0)
(-3.33333, 0)

***/

#include <cassert>
#include <iostream>
#include <complex>
#include <vector>
using namespace std;

typedef std::complex<LD> cdouble;

vector<cdouble> find_all_roots(const vector<LD> &p) {
  int degree = p.size() - 1;
  LD c[MAXN], re[MAXN], im[MAXN];
  copy(p.rbegin(), p.rend(), c);
  find_all_roots(degree, c, re, im);
  vector<cdouble> res;
  for (int i = 0; i < (int)p.size() - 1; i++) {
    res.push_back(complex<LD>(re[i], im[i]));
  }
  return res;
}

cdouble eval(const vector<LD> &p, cdouble x) {
  cdouble res = p.back();
  for (int i = p.size() - 2; i >= 0; i--) {
    res = res*x + p[i];
  }
  return res;
}

void print_roots(const vector<LD> &p, const vector<cdouble> &x) {
  cout << "Roots of ";
  for (int i = 0; i < (int)p.size(); i++) {
    cout << showpos << (double)p[i];
    if (i > 0) {
      cout << noshowpos << "x^" << i;
    }
  }
  cout << ":" << endl;
  for (int i = 0; i < (int)x.size(); i++) {
    cout << "(" << (double)x[i].real() << ", " << (double)x[i].imag() << ")\n";
  }
}

int main() {
  { // -1 + 2x - 6x^2 + 2x^3
    int poly[] = {-1, 2, -6, 2};
    vector<LD> p(poly, poly + 4);
    print_roots(p, find_all_roots(p));
  }
  { // -20 + 4x + 3x^2
    int poly[] = {-20, 4, 3};
    vector<LD> p(poly, poly + 3);
    print_roots(p, find_all_roots(p));
  }
  return 0;
}
\end{lstlisting}
\subsection{Integration (Simpson)}
\begin{lstlisting}
/*

Computes the definite integral from a to b for a continuous function f using
Simpson's approximation: integral ~ [f(a) + 4*f((a + b)/2) + f(b)]*(b - a)/6.

- simpsons(f, a, b) returns the definite integral for a function f from a to b,
  to a tolerance of EPS in absolute error.

Time Complexity:
- O(p) per call to integrate(), where p = -log10(EPS) is the number of digits
  of absolute precision that is desired.

Space Complexity:
- O(p) auxiliary stack and O(1) auxiliary heap space, where p = -log10(EPS)
  is the number of digits of absolute precision that is desired.

*/

#include <cmath>

template<class ContinuousFunction>
double simpsons(ContinuousFunction f, double a, double b) {
  return (f(a) + 4*f((a + b)/2) + f(b))*(b - a)/6;
}

template<class ContinuousFunction>
double integrate(ContinuousFunction f, double a, double b,
                 const double EPS = 1e-15) {
  double m = (a + b) / 2;
  double am = simpsons(f, a, m);
  double mb = simpsons(f, m, b);
  double ab = simpsons(f, a, b);
  if (fabs(am + mb - ab) < EPS) {
    return ab;
  }
  return integrate(f, a, m) + integrate(f, m, b);
}

/*** Example Usage ***/

#include <cstdio>
#include <cassert>
using namespace std;

double f(double x) {
  return sin(x);
}

int main () {
  double PI = acos(-1.0);
  assert(fabs(integrate(f, 0.0, PI/2) - 1) < 1e-10);
  return 0;
}
\end{lstlisting}

\chapter{Geometry}

\section{Geometric Classes}
\setcounter{section}{1}
\setcounter{subsection}{0}
\subsection{Point}
\begin{lstlisting}
/*

A two-dimensional real-valued point class supporting epsilon comparisons.
Operations include element-wise arithmetic, norm, arg, dot product, cross
product, projection, rotation, and reflection. See also std::complex.

Time Complexity:
- O(1) per call to the constructor and all other operations.

Space Complexity:
- O(1) for storage of the point.
- O(1) auxiliary for all operations.

*/

#include <cmath>
#include <ostream>
#include <utility>

const double EPS = 1e-9;

#define EQ(a, b) (fabs((a) - (b)) <= EPS)
#define LT(a, b) ((a) < (b) - EPS)

struct point {
  double x, y;

  point() : x(0), y(0) {}
  point(double x, double y) : x(x), y(y) {}
  point(const point &p) : x(p.x), y(p.y) {}
  point(const std::pair<double, double> &p) : x(p.first), y(p.second) {}

  bool operator<(const point &p) const {
    return EQ(x, p.x) ? LT(y, p.y) : LT(x, p.x);
  }

  bool operator>(const point &p) const {
    return EQ(x, p.x) ? LT(p.y, y) : LT(p.x, x);
  }

  bool operator==(const point &p) const { return EQ(x, p.x) && EQ(y, p.y); }
  bool operator!=(const point &p) const { return !(*this == p); }
  bool operator<=(const point &p) const { return !(*this > p); }
  bool operator>=(const point &p) const { return !(*this < p); }
  point operator+(const point &p) const { return point(x + p.x, y + p.y); }
  point operator-(const point &p) const { return point(x - p.x, y - p.y); }
  point operator+(double v) const { return point(x + v, y + v); }
  point operator-(double v) const { return point(x - v, y - v); }
  point operator*(double v) const { return point(x * v, y * v); }
  point operator/(double v) const { return point(x / v, y / v); }
  point& operator+=(const point &p) { x += p.x; y += p.y; return *this; }
  point& operator-=(const point &p) { x -= p.x; y -= p.y; return *this; }
  point& operator+=(double v) { x += v; y += v; return *this; }
  point& operator-=(double v) { x -= v; y -= v; return *this; }
  point& operator*=(double v) { x *= v; y *= v; return *this; }
  point& operator/=(double v) { x /= v; y /= v; return *this; }
  friend point operator+(double v, const point &p) { return p + v; }
  friend point operator*(double v, const point &p) { return p * v; }

  double sqnorm() const { return x*x + y*y; }
  double norm() const { return sqrt(x*x + y*y); }
  double arg() const { return atan2(y, x); }
  double dot(const point &p) const { return x*p.x + y*p.y; }
  double cross(const point &p) const { return x*p.y - y*p.x; }
  double proj(const point &p) const { return dot(p) / p.norm(); }

  // Returns a proportional unit vector (p, q) = c(x, y) where p^2 + q^2 = 1.
  point normalize() const {
    return (EQ(x, 0) && EQ(y, 0)) ? point(0, 0) : (point(x, y) / norm());
  }

  // Returns (x, y) rotated 90 degrees clockwise about the origin.
  point rotate90() const { return point(-y, x); }

  // Returns (x, y) rotated t radians clockwise about the origin.
  point rotateCW(double t) const {
    return point(x*cos(t) + y*sin(t), y*cos(t) - x*sin(t));
  }

  // Returns (x, y) rotated t radians counter-clockwise about the origin.
  point rotateCCW(double t) const {
    return point(x*cos(t) - y*sin(t), x*sin(t) + y*cos(t));
  }

  // Returns (x, y) rotated t radians clockwise about point p.
  point rotateCW(const point &p, double t) const {
    return (*this - p).rotateCW(t) + p;
  }

  // Returns (x, y) rotated t radians counter-clockwise about the point p.
  point rotateCCW(const point &p, double t) const {
    return (*this - p).rotateCCW(t) + p;
  }

  // Returns (x, y) reflected across point p.
  point reflect(const point &p) const {
    return point(2*p.x - x, 2*p.y - y);
  }

  // Returns (x, y) reflected across the line containing points p and q.
  point reflect(const point &p, const point &q) const {
    if (p == q) {
      return reflect(p);
    }
    point r(*this - p), s = q - p;
    r = point(r.x*s.x + r.y*s.y, r.x*s.y - r.y*s.x) / s.sqnorm();
    r = point(r.x*s.x - r.y*s.y, r.x*s.y + r.y*s.x) + p;
    return r;
  }

  friend double sqnorm(const point &p) { return p.sqnorm(); }
  friend double norm(const point &p) { return p.norm(); }
  friend double arg(const point &p) { return p.arg(); }
  friend double dot(const point &p, const point &q) { return p.dot(q); }
  friend double cross(const point &p, const point &q) { return p.cross(q); }
  friend double proj(const point &p, const point &q) { return p.proj(q); }
  friend point normalize(const point &p) { return p.normalize(); }
  friend point rotate90(const point &p) { return p.rotate90(); }

  friend point rotateCW(const point &p, double t) {
    return p.rotateCW(t);
  }

  friend point rotateCCW(const point &p, double t) {
    return p.rotateCCW(t);
  }

  friend point rotateCW(const point &p, const point &q, double t) {
    return p.rotateCW(q, t);
  }

  friend point rotateCCW(const point &p, const point &q, double t) {
    return p.rotateCCW(q, t);
  }

  friend point reflect(const point &p, const point &q) {
    return p.reflect(q);
  }

  friend point reflect(const point &p, const point &a, const point &b) {
    return p.reflect(a, b);
  }

  friend std::ostream& operator<<(std::ostream &out, const point &p) {
    return out << "(" << (fabs(p.x) < EPS ? 0 : p.x) << ","
                      << (fabs(p.y) < EPS ? 0 : p.y) << ")";
  }
};

/*** Example Usage ***/

#include <cassert>
#define pt point

const double PI = acos(-1.0);

int main() {
  pt p(-10, 3), q;
  assert(pt(-18, 29) == p + pt(-3, 9)*6 / 2 - pt(-1, 1));
  assert(EQ(109, p.sqnorm()));
  assert(EQ(10.44030650891, p.norm()));
  assert(EQ(2.850135859112, p.arg()));
  assert(EQ(0, p.dot(pt(3, 10))));
  assert(EQ(0, p.cross(pt(10, -3))));
  assert(EQ(10, p.proj(pt(-10, 0))));
  assert(EQ(1, p.normalize().norm()));
  assert(pt(-3, -10) == p.rotate90());
  assert(pt(3, 12) == p.rotateCW(pt(1, 1), PI / 2));
  assert(pt(1, -10) == p.rotateCCW(pt(2, 2), PI / 2));
  assert(pt(10, -3) == p.reflect(pt(0, 0)));
  assert(pt(-10, -3) == p.reflect(pt(-2, 0), pt(5, 0)));
  return 0;
}
\end{lstlisting}
\subsection{Line}
\begin{lstlisting}
/*

A straight line in two dimensions supporting epsilon comparisons. The line is
represented by the form a*x + b*y + c = 0, where the coefficients are normalized
so that b is always 1 except for when the line is vertical, in which case b = 0.
Operations include checking if the line is horizontal or vertical, finding the
slope, evaluating y at some x (and vice versa), checking if a point falls on the
line, checking if another line is parallel or perpendicular, and finding the
parallel or perpendicular line through a point.

Time Complexity:
- O(1) per call to the constructor and all other operations.

Space Complexity:
- O(1) for storage of the line.
- O(1) auxiliary for all operations.

*/

#include <cmath>
#include <limits>
#include <ostream>

const double EPS = 1e-9;
const double M_NAN = std::numeric_limits<double>::quiet_NaN();

#define EQ(a, b) (fabs((a) - (b)) <= EPS)
#define LT(a, b) ((a) < (b) - EPS)

struct line {
  double a, b, c;

  line() : a(0), b(0), c(0) {}  // Invalid or uninitialized line.

  line(double a, double b, double c) {
    if (!EQ(b, 0)) {
      this->a = a / b;
      this->c = c / b;
      this->b = 1;
    } else {
      this->c = c / a;
      this->a = 1;
      this->b = 0;
    }
  }

  template<class Point>
  line(double slope, const Point &p) {
    a = -slope;
    b = 1;
    c = slope * p.x - p.y;
  }

  template<class Point>
  line(const Point &p, const Point &q) : a(0), b(0), c(0) {
    if (EQ(p.x, q.x)) {
      if (NE(p.y, q.y)) {  // Vertical line.
        a = 1;
        b = 0;
        c = -p.x;
      }  // Else, invalid line.
    } else {
      a = -(p.y - q.y) / (p.x - q.x);
      b = 1;
      c = -(a*p.x) - (b*p.y);
    }
  }

  bool operator==(const line &l) const {
    return EQ(a, l.a) && EQ(b, l.b) && EQ(c, l.c);
  }

  bool operator!=(const line &l) const {
    return !(*this == l);
  }

  // Returns whether the line is initialized and normalized.
  bool valid() const {
    if (EQ(a, 0)) {
      return !EQ(b, 0);
    }
    return EQ(b, 1) || (EQ(b, 0) && EQ(a, 1));
  }

  bool horizontal() const { return valid() && EQ(a, 0); }
  bool vertical() const { return valid() && EQ(b, 0); }
  double slope() const { return (!valid() || EQ(b, 0)) ? M_NAN : -a; }

  // Solve for x at a given y. If the line is horizontal, then either -INF, INF,
  // or NAN is returned based on whether y is below, above, or on the line.
  double x(double y) const {
    if (!valid() || EQ(a, 0)) {
      return M_NAN;  // Invalid or horizontal line.
    }
    return (-c - b*y) / a;
  }

  // Solve for y at a given x. If the line is vertical, then either -INF, INF,
  // or NAN is returned based on whether x is left of, right of, or on the line.
  double y(double x) const {
    if (!valid() || EQ(b, 0)) {
      return M_NAN;  // Invalid or vertical line.
    }
    return (-c - a*x) / b;
  }

  template<class Point>
  bool contains(const Point &p) const { return EQ(a*p.x + b*p.y + c, 0); }

  bool is_parallel(const line &l) const { return EQ(a, l.a) && EQ(b, l.b); }
  bool is_perpendicular(const line &l) const { return EQ(-a*l.a, b*l.b); }

  // Return the parallel line passing through point p.
  template<class Point>
  line parallel(const Point &p) const {
    return line(a, b, -a*p.x - b*p.y);
  }

  // Return the perpendicular line passing through point p.
  template<class Point>
  line perpendicular(const Point &p) const {
    return line(-b, a, b*p.x - a*p.y);
  }

  friend std::ostream& operator<<(std::ostream &out, const line &l) {
    return out << (fabs(l.a) < EPS ? 0 : l.a) << "x" << std::showpos
               << (fabs(l.b) < EPS ? 0 : l.b) << "y"
               << (fabs(l.c) < EPS ? 0 : l.c) << "=0" << std::noshowpos;
  }
};

/*** Example Usage ***/

#include <cassert>

struct point {
  double x, y;
  point(double x, double y) : x(x), y(y) {}
};

int main() {
  line l(2, -5, -8);
  line para = line(2, -5, -8).parallel(point(-6, -2));
  line perp = line(2, -5, -8).perpendicular(point(-6, -2));
  assert(l.is_parallel(para) && l.is_perpendicular(perp));
  assert(l.slope() == 0.4);
  assert(para == line(-0.4, 1, -0.4));  // -0.4x + y - 0.4 = 0.
  assert(perp == line(2.5, 1, 17));  // 2.5x + y + 17 = 0.
  return 0;
}
\end{lstlisting}
\subsection{Circle}
\begin{lstlisting}
/*

A circle in two dimensions supporting epsilon comparisons. The circle centered
at (h, k) is represented by the relation (x - h)^2 + (y - k)^2 = r^2, where the
radius r is normalized to a non-negative number. Operations include constructing
a circle from a line segment, constructing a circumcircle, checking if a point
falls inside the circle or on its edge, and constructing an incircle.

Time Complexity:
- O(1) per call to the constructors and all other operations.

Space Complexity:
- O(1) for storage of the circle.
- O(1) auxiliary for all operations.

*/

#include <cmath>
#include <ostream>
#include <stdexcept>
#include <utility>

const double EPS = 1e-9;

#define EQ(a, b) (fabs((a) - (b)) <= EPS)
#define LE(a, b) ((a) <= (b) + EPS)
#define LT(a, b) ((a) < (b) - EPS)

typedef std::pair<double, double> point;
#define x first
#define y second

double sqnorm(const point &a) { return a.x*a.x + a.y*a.y; }
double norm(const point &a) { return sqrt(sqnorm(a)); }

struct circle {
  double h, k, r;

  circle() : h(0), k(0), r(0) {}
  circle(double r) : h(0), k(0), r(fabs(r)) {}
  circle(const point &o, double r) : h(o.x), k(o.y), r(fabs(r)) {}
  circle(double h, double k, double r) : h(h), k(k), r(fabs(r)) {}

  // Circle with the line segment ab as a diameter.
  circle(const point &a, const point &b) {
    h = (a.x + b.x)/2.0;
    k = (a.y + b.y)/2.0;
    r = norm(point(a.x - h, a.y - k));
  }

  // Circumcircle of three points.
  circle(const point &a, const point &b, const point &c) {
    double an = sqnorm(point(b.x - c.x, b.y - c.y));
    double bn = sqnorm(point(a.x - c.x, a.y - c.y));
    double cn = sqnorm(point(a.x - b.x, a.y - b.y));
    double wa = an*(bn + cn - an);
    double wb = bn*(an + cn - bn);
    double wc = cn*(an + bn - cn);
    double w = wa + wb + wc;
    if (EQ(w, 0)) {
      throw std::runtime_error("No circumcircle from collinear points.");
    }
    h = (wa*a.x + wb*b.x + wc*c.x)/w;
    k = (wa*a.y + wb*b.y + wc*c.y)/w;
    r = norm(point(a.x - h, a.y - k));
  }

  // Circle of radius r that contains points a and b. In the general case, there
  // will be two possible circles and only one is chosen arbitrarily. However if
  // the diameter is equal to dist(a, b) = 2*r, then there is only one possible
  // center. If points a and b are identical, then there are infinite circles.
  // If the points are too far away relative to the radius, then there is no
  // possible circle. In the latter two cases, an exception is thrown.
  circle(const point &a, const point &b, double r) : r(fabs(r)) {
    if (LE(r, 0) && a == b) {  // Circle with zero area.
      h = a.x;
      k = a.y;
      return;
    }
    double d = norm(point(b.x - a.x, b.y - a.y));
    if (EQ(d, 0)) {
      throw std::runtime_error("Identical points, infinite circles.");
    }
    if (LT(r*2.0, d)) {
      throw std::runtime_error("Points too far away to make circle.");
    }
    double v = sqrt(r*r - d*d/4.0) / d;
    point m((a.x + b.x)/2.0, (a.y + b.y)/2.0);
    h = m.x + v*(a.y - b.y);
    k = m.y + v*(b.x - a.x);
    // The other answer is (h, k) = (m.x - v*(a.y - b.y), m.y - v*(b.x - a.x)).
  }

  bool operator==(const circle &c) const {
    return EQ(h, c.h) && EQ(k, c.k) && EQ(r, c.r);
  }

  bool operator!=(const circle &c) const {
    return !(*this == c);
  }

  point center() const { return point(h, k); }

  bool contains(const point &p) const {
    return LE(sqnorm(point(p.x - h, p.y - k)), r*r);
  }

  bool on_edge(const point &p) const {
    return EQ(sqnorm(point(p.x - h, p.y - k)), r*r);
  }

  friend std::ostream& operator<<(std::ostream &out, const circle &c) {
    return out << std::showpos << "(x" << -(fabs(c.h) < EPS ? 0 : c.h) << ")^2+"
                               << "(y" << -(fabs(c.k) < EPS ? 0 : c.k) << ")^2"
               << std::noshowpos << "=" << (fabs(c.r) < EPS ? 0 : c.r*c.r);
  }
};

// Returns the circle inscribed inside the triangle abc.
circle incircle(const point &a, const point &b, const point &c) {
  double al = norm(point(b.x - c.x, b.y - c.y));
  double bl = norm(point(a.x - c.x, a.y - c.y));
  double cl = norm(point(a.x - b.x, a.y - b.y));
  double l = al + bl + cl;
  point p(a.x - c.x, a.y - c.y), q(b.x - c.x, b.y - c.y);
  return EQ(l, 0) ? circle(a.x, a.y, 0)
                  : circle((al*a.x + bl*b.x + cl*c.x) / l,
                           (al*a.y + bl*b.y + cl*c.y) / l,
                           fabs(p.x*q.y - p.y*q.x) / l);
}

/*** Example Usage ***/

#include <cassert>

int main() {
  circle c(-2, 5, sqrt(10));
  assert(c == circle(point(-2, 5), sqrt(10)));
  assert(c == circle(point(1, 6), point(-5, 4)));
  assert(c == circle(point(-3, 2), point(-3, 8), point(-1, 8)));
  assert(c == incircle(point(-12, 5), point(3, 0), point(0, 9)));
  assert(c.contains(point(-2, 8)) && !c.contains(point(-2, 9)));
  assert(c.on_edge(point(-1, 2)) && !c.on_edge(point(-1.01, 2)));
  return 0;
}
\end{lstlisting}
\subsection{Triangle}
\begin{lstlisting}
/*

Common triangle calculations in two dimensions.

- triangle_area(a, b, c) returns the area of the triangle abc.
- triangle_area_sides(s1, s2, s3) returns the area of a triangle with side
  lengths s1, s2, and s3. The given lengths must be non-negative and form a
  valid triangle.
- trinagle_area_medians(m1, m2, m3) returns the area of a triangle with medians
  of lengths m1, m2, and m3. The median of a triangle is a line segment joining
  a vertex to the midpoint of the opposing edge.
- triangle_area_altitudes(h1, h2, h3) returns the area of a triangle with
  altitudes h1, h2, and h3. An altitude of a triangle is the shortest line
  between a vertex and the infinite line that is extended from its opposite
  edge.
- same_side(p1, p2, a, b) returns whether points p1 and p2 lie on the same side
  of the line containing points a and b. If one or both points lie exactly on
  the line, then the result will depend on the setting of EDGE_IS_SAME_SIDE.
- point_in_triangle(p, a, b, c) returns whether point p lies within the triangle
  abc. If the point lies on or close to an edge (by roughly EPS), then the
  result will depend on the setting of EDGE_IS_SAME_SIDE in the function above.

Time Complexity:
- O(1) for all operations.

Space Complexity:
- O(1) auxiliary for all operations.

*/

#include <algorithm>
#include <cmath>
#include <utility>

const double EPS = 1e-9;

#define EQ(a, b) (fabs((a) - (b)) <= EPS)
#define LT(a, b) ((a) < (b) - EPS)
#define GT(a, b) ((a) > (b) + EPS)
#define LE(a, b) ((a) <= (b) + EPS)
#define GE(a, b) ((a) >= (b) - EPS)

typedef std::pair<double, double> point;
#define x first
#define y second

double cross(const point &a, const point &b) { return a.x*b.y - a.y*b.x; }

double triangle_area(const point &a, const point &b, const point &c) {
  point ac(a.x - c.x, a.y - c.y), bc(b.x - c.x, b.y - c.y);
  return fabs(cross(ac, bc)) / 2.0;
}

double triangle_area_sides(double s1, double s2, double s3) {
  double s = (s1 + s2 + s3) / 2.0;
  return sqrt(s*(s - s1)*(s - s2)*(s - s3));
}

double triangle_area_medians(double m1, double m2, double m3) {
  return 4.0*triangle_area_sides(m1, m2, m3) / 3.0;
}

double triangle_area_altitudes(double h1, double h2, double h3) {
  if (EQ(h1, 0) || EQ(h2, 0) || EQ(h3, 0)) {
    return 0;
  }
  double x = h1*h1, y = h2*h2, z = h3*h3;
  double v = 2.0/(x*y) + 2.0/(x*z) + 2.0/(y*z);
  return 1.0/sqrt(v - 1.0/(x*x) - 1.0/(y*y) - 1.0/(z*z));
}

bool same_side(const point &p1, const point &p2, const point &a,
               const point &b) {
  static const bool EDGE_IS_SAME_SIDE = true;
  point ab(b.x - a.x, b.y - a.y);
  point p1a(p1.x - a.x, p1.y - a.y), p2a(p2.x - a.x, p2.y - a.y);
  double c1 = cross(ab, p1a), c2 = cross(ab, p2a);
  return EDGE_IS_SAME_SIDE ? GE(c1*c2, 0) : GT(c1*c2, 0);
}

bool point_in_triangle(const point &p, const point &a, const point &b,
                       const point &c) {
  return same_side(p, a, b, c) &&
         same_side(p, b, a, c) &&
         same_side(p, c, a, b);
}

/*** Example Usage ***/

#include <cassert>

int main() {
  assert(EQ(6, triangle_area(point(0, -1), point(4, -1), point(0, -4))));
  assert(EQ(6, triangle_area_sides(3, 4, 5)));
  assert(EQ(6, triangle_area_medians(3.605551275, 2.5, 4.272001873)));
  assert(EQ(6, triangle_area_altitudes(3, 4, 2.4)));

  assert(point_in_triangle(point(0, 0),
                           point(-1, 0), point(0, -2), point(4, 0)));
  assert(!point_in_triangle(point(0, 1),
                            point(-1, 0), point(0, -2), point(4, 0)));
  assert(point_in_triangle(point(-2.44, 0.82),
                           point(-1, 0), point(-3, 1), point(4, 0)));
  assert(!point_in_triangle(point(-2.44, 0.7),
                            point(-1, 0), point(-3, 1), point(4, 0)));
  return 0;
}
\end{lstlisting}
\subsection{Rectangle}
\begin{lstlisting}
/*

Common rectangle calculations in two dimensions.

- rectangle_area(a, b) returns the area of a rectangle with opposing vertices a
  and b.
- point_in_rectangle(p, x, y, w, h) returns whether point p lies within the
  rectangle defined by a vertex at v (x, y), a width of w, and a height of h.
  Note that negative widths and heights are supported. If the point lies on or
  close to an edge (by roughly EPS), then the result will depend on the setting
  of EDGE_IS_INSIDE.
- point_in_rectangle(p, a, b) returns whether point p lies within the rectangle
  with opposing vertices a and b. If the point lies on or close to an edge (by
  roughly EPS), then the result will depend on the setting of EDGE_IS_INSIDE.
- rectangle_intersection(a1, b1, a2, b2, &p, &q) determines the intersection
  region of the rectangle with opposing vertices a1 and b1 and the rectangle
  with opposing vertices a2 and b2. Returns -1 if the rectangles are completely
  disjoint, 0 if the rectangles partially intersect, 1 if the first rectangle is
  completely inside the second, and 2 if the second rectangle is completely
  inside the first. If there is an intersection, the opposing vertices of the
  intersection rectangle will be stored into pointers p and q if they are not
  NULL. If the intersection is a single point or line segment, then the result
  will depend on the setting of EDGE_IS_INSIDE within point_in_rectangle().

Time Complexity:
- O(1) for all operations.

Space Complexity:
- O(1) auxiliary for all operations.

*/

#include <algorithm>
#include <cmath>
#include <cstddef>
#include <utility>

const double EPS = 1e-9;

#define EQ(a, b) (fabs((a) - (b)) <= EPS)
#define LT(a, b) ((a) < (b) - EPS)
#define GT(a, b) ((a) > (b) + EPS)
#define LE(a, b) ((a) <= (b) + EPS)
#define GE(a, b) ((a) >= (b) - EPS)

typedef std::pair<double, double> point;
#define x first
#define y second

double rectangle_area(const point &a, const point &b) {
  return fabs((a.x - b.x)*(a.y - b.y));
}

bool point_in_rectangle(const point &p, const point &v, double w, double h) {
  static const bool EDGE_IS_INSIDE = true;
  if (w < 0) {
    return point_in_rectangle(p, point(v.x + w, v.y), -w, h);
  }
  if (h < 0) {
    return point_in_rectangle(p, point(v.x, v.y + h), w, -h);
  }
  return EDGE_IS_INSIDE
      ? (GE(p.x, v.x) && LE(p.x, v.x + w) && GE(p.y, v.y) && LE(p.y, v.y + h))
      : (GT(p.x, v.x) && LT(p.x, v.x + w) && GT(p.y, v.y) && LT(p.y, v.y + h));
}

bool point_in_rectangle(const point &p, const point &a, const point &b) {
  double xl = std::min(a.x, b.x), yl = std::min(a.y, b.y);
  double xh = std::max(a.x, b.x), yh = std::max(a.y, b.y);
  return point_in_rectangle(p, point(xl, yl), xh - xl, yh - yl);
}

int rectangle_intersection(const point &a1, const point &b1, const point &a2,
                           const point &b2, point *p = NULL, point *q = NULL) {
  bool a1in2 = point_in_rectangle(a1, a2, b2);
  bool b1in2 = point_in_rectangle(b1, a2, b2);
  if (a1in2 && b1in2) {
    if (p != NULL && q != NULL) {
      *p = std::min(a1, b1);
      *q = std::max(a1, b1);
    }
    return 1;  // Rectangle 1 completely inside 2.
  }
  if (!a1in2 && !b1in2) {
    if (point_in_rectangle(a2, a1, b1)) {
      if (p != NULL && q != NULL) {
        *p = std::min(a2, b2);
        *q = std::max(a2, b2);
      }
      return 2;  // Rectangle 2 completely inside 1.
    }
    return -1;  // Completely disjoint.
  }
  if (p != NULL && q != NULL) {
    if (a1in2) {
      *p = a1;
      *q = (a1 < b1) ? std::max(a2, b2) : std::min(a2, b2);
    } else {
      *p = b1;
      *q = (b1 < a1) ? std::max(a2, b2) : std::min(a2, b2);
    }
    if (*p > *q) {
      std::swap(p, q);
    }
  }
  return 0;
}

/*** Example Usage ***/

#include <cassert>

bool EQP(const point &a, const point &b) {
  return EQ(a.x, b.x) && EQ(a.y, b.y);
}

int main() {
  assert(EQ(20, rectangle_area(point(1, 1), point(5, 6))));

  assert(point_in_rectangle(point(0, -1), point(0, -3), 3, 2));
  assert(point_in_rectangle(point(2, -2), point(3, -3), -3, 2));
  assert(!point_in_rectangle(point(0, 0), point(3, -1), -3, -2));
  assert(point_in_rectangle(point(2, -2), point(3, -3), point(0, -1)));
  assert(!point_in_rectangle(point(-1, -2), point(3, -3), point(0, -1)));

  point p, q;
  assert(-1 == rectangle_intersection(point(0, 0), point(1, 1),
                                      point(2, 2), point(3, 3)));
  assert(0 == rectangle_intersection(point(1, 1), point(7, 7),
                                     point(5, 5), point(0, 0), &p, &q));
  assert(EQP(p, point(1, 1)) && EQP(q, point(5, 5)));
  assert(1 == rectangle_intersection(point(1, 1), point(0, 0),
                                     point(0, 0), point(1, 10), &p, &q));
  assert(EQP(p, point(0, 0)) && EQP(q, point(1, 1)));
  assert(2 == rectangle_intersection(point(0, 5), point(5, 7),
                                     point(1, 6), point(2, 5), &p, &q));
  assert(EQP(p, point(1, 6)) && EQP(q, point(2, 5)));

  return 0;
}
\end{lstlisting}

\section{Elementary Geometric Calculations}
\setcounter{section}{2}
\setcounter{subsection}{0}
\subsection{Angles}
\begin{lstlisting}
/*

Angle calculations in two dimensions. The constants DEG and RAD may be used as
multipliers to convert between degrees and radians. For example, if t is a value
in radians, then t*DEG is the equivalent angle in degrees.

- reduce_deg(t) takes an angle t degrees and returns an equivalent angle in the
  range [0, 360) degrees. E.g. -630 becomes 90.
- reduce_rad(t) takes an angle t radians and returns an equivalent angle in the
  range [0, 360) radians. E.g. 720.5 becomes 0.5.
- polar_point(r, t) returns a two-dimensional Cartesian point given radius r and
  angle t radians in polar coordinates (see std::polar()).
- polar_angle(p) returns the angle in radians of the line segment from (0, 0) to
  point p, relative counterclockwise to the positive x-axis.
- angle(a, o, b) returns the smallest angle in radians formed by the points a,
  o, b with vertex at point o.
- angle_between(a, b) returns the angle in radians of segment from point a to
  point b, relative counterclockwise to the positive x-axis.
- angle between(a1, b1, a2, b2) returns the smaller angle in radians between
  two lines a1*x + b1*y + c1 = 0 and a2*x + b2*y + c2 = 0, limited to [0, PI/2].
- cross(a, b, o) returns the magnitude (Euclidean norm) of the three-dimensional
  cross product between points a and b where the z-component is implicitly zero
  and the origin is implicitly shifted to point o. This operation is also equal
  to double the signed area of the triangle from these three points.
- turn(a, o, b) returns -1 if the path a->o->b forms a left turn on the plane, 0
  if the path forms a straight line segment, or 1 if it forms a right turn.

Time Complexity:
- O(1) for all operations.

Space Complexity:
- O(1) auxiliary for all operations.

*/

#include <cmath>
#include <utility>

const double EPS = 1e-9;

#define EQ(a, b) (fabs((a) - (b)) <= EPS)
#define LT(a, b) ((a) < (b) - EPS)
#define GT(a, b) ((a) > (b) + EPS)

typedef std::pair<double, double> point;
#define x first
#define y second

double sqnorm(const point &a) { return a.x*a.x + a.y*a.y; }
double norm(const point &a) { return sqrt(sqnorm(a)); }

const double PI = acos(-1.0), DEG = PI/180, RAD = 180/PI;

double reduce_deg(double t) {
  if (t < -360) {
    return reduce_deg(fmod(t, 360));
  }
  if (t < 0) {
    return t + 360;
  }
  return (t >= 360) ? fmod(t, 360) : t;
}

double reduce_rad(double t) {
  if (t < -2*PI) {
    return reduce_rad(fmod(t, 2*PI));
  }
  if (t < 0) {
    return t + 2*PI;
  }
  return (t >= 2*PI) ? fmod(t, 2*PI) : t;
}

point polar_point(double r, double t) {
  return point(r*cos(t), r*sin(t));
}

double polar_angle(const point &p) {
  double t = atan2(p.y, p.x);
  return (t < 0) ? (t + 2*PI) : t;
}

double angle(const point & a, const point & o, const point & b) {
  point u(o.x - a.x, o.y - a.y), v(o.x - b.x, o.y - b.y);
  return acos((u.x*v.x + u.y*v.y) / (norm(u)*norm(v)));
}

double angle_between(const point &a, const point &b) {
  double t = atan2(a.x*b.y - a.y*b.x, a.x*b.x + a.y*b.y);
  return (t < 0) ? (t + 2*PI) : t;
}

double angle_between(const double &a1, const double &b1,
                     const double &a2, const double &b2) {
  double t = atan2(a1*b2 - a2*b1, a1*a2 + b1*b2);
  if (t < 0) {
    t += PI;
  }
  return GT(t, PI / 2) ? (PI - t) : t;
}

double cross(const point &a, const point &b, const point &o = point(0, 0)) {
  return (a.x - o.x)*(b.y - o.y) - (a.y - o.y)*(b.x - o.x);
}

int turn(const point &a, const point &o, const point &b) {
  double c = cross(a, b, o);
  return LT(c, 0) ? -1 : (GT(c, 0) ? 1 : 0);
}

/*** Example Usage ***/

#include <cassert>

bool EQP(const point &a, const point &b) {
  return EQ(a.x, b.x) && EQ(a.y, b.y);
}

int main() {
  assert(EQ(123, reduce_deg(-8*360 + 123)));
  assert(EQ(1.2345, reduce_rad(2*PI*8 + 1.2345)));
  assert(EQP(polar_point(4, PI), point(-4, 0)));
  assert(EQP(polar_point(4, -PI/2), point(0, -4)));
  assert(EQ(45, polar_angle(point(5, 5))*RAD));
  assert(EQ(135*DEG, polar_angle(point(-4, 4))));
  assert(EQ(90*DEG, angle(point(5, 0), point(0, 5), point(-5, 0))));
  assert(EQ(225*DEG, angle_between(point(0, 5), point(5, -5))));
  assert(-1 == cross(point(0, 1), point(1, 0), point(0, 0)));
  assert(1 == turn(point(0, 1), point(0, 0), point(-5, -5)));
  return 0;
}
\end{lstlisting}
\subsection{Distances}
\begin{lstlisting}
/*

Distance calculations in two dimensions for points, lines, and line segments.

- dist(a, b) and sqdist(a, b) respectively return the distance and squared
  distance between points a and b.
- line_dist(p, a, b, c) returns the distance from point p to the line
  a*x + b*y + c = 0. If the line is invalid (i.e. a = b = 0), then -INF, INF,
  or NaN is returned based on the sign of c.
- line_dist(p, a, b) returns the distance from point p to the infinite line
  containing points a and b. If the line is invalid (i.e. a = b), then the
  distance from p to the single point is returned.
- line_dist(a1, b1, c1, a2, b2, c2) returns the distance between two lines. If
  the lines are non-parallel then the distance is considered to be 0. Otherwise,
  the distance is considered to be the perpendicular distance from any point on
  one line to the other line.
- seg_dist(p, a, b) returns the distance from point p to the line segment ab.
- seg_dist(a, b, c, d) returns the minimum distance from any point on the line
  segment ab to any point on the line segment cd. This is 0 if the segments
  touch or intersect.

Time Complexity:
- O(1) for all operations.

Space Complexity:
- O(1) auxiliary for all operations.

*/

#include <algorithm>
#include <cmath>
#include <utility>

const double EPS = 1e-9;

#define EQ(a, b) (fabs((a) - (b)) <= EPS)
#define LE(a, b) ((a) <= (b) + EPS)
#define GE(a, b) ((a) >= (b) - EPS)

typedef std::pair<double, double> point;
#define x first
#define y second

double sqnorm(const point &a) { return a.x*a.x + a.y*a.y; }
double norm(const point &a) { return sqrt(sqnorm(a)); }
double dot(const point &a, const point &b) { return a.x*b.x + a.y*b.y; }
double cross(const point &a, const point &b) { return a.x*b.y - a.y*b.x; }

double dist(const point &a, const point &b) {
  return norm(point(b.x - a.x, b.y - a.y));
}

double sqdist(const point &a, const point &b) {
  return sqnorm(point(b.x - a.x, b.y - a.y));
}

double line_dist(const point &p, double a, double b, double c) {
  return fabs(a*p.x + b*p.y + c) / sqrt(a*a + b*b);
}

double line_dist(const point &p, const point &a, const point &b) {
  if (EQ(a.x, b.x) && EQ(a.y, b.y)) {
    return dist(p, a);
  }
  double u = ((p.x - a.x)*(b.x - a.x) + (p.y - a.y)*(b.y - a.y)) / sqdist(a, b);
  return norm(point(a.x + u*(b.x - a.x) - p.x, a.y + u*(b.y - a.y) - p.y));
}

double line_dist(double a1, double b1, double c1,
                 double a2, double b2, double c2) {
  if (EQ(a1*b2, a2*b1)) {
    double factor = EQ(b1, 0) ? (a1 / a2) : (b1 / b2);
    return EQ(c1, c2*factor) ? 0
                             : fabs(c2*factor - c1) / sqrt(a1*a1 + b1*b1);
  }
  return 0;
}

double seg_dist(const point &p, const point &a, const point &b) {
  if (EQ(a.x, b.x) && EQ(a.y, b.y)) {
    return dist(p, a);
  }
  point ab(b.x - a.x, b.y - a.y), ap(p.x - a.x, p.y - a.y);
  double n = sqnorm(ab), d = dot(ab, ap);
  if (LE(d, 0) || EQ(n, 0)) {
    return norm(ap);
  }
  return GE(d, n) ? norm(point(ap.x - ab.x, ap.y - ab.y))
                  : norm(point(ap.x - ab.x*(d / n), ap.y - ab.y*(d / n)));
}

double seg_dist(const point &a, const point &b,
                const point &c, const point &d) {
  point ab(b.x - a.x, b.y - a.y);
  point ac(c.x - a.x, c.y - a.y);
  point cd(d.x - c.x, d.y - c.y);
  double c1 = cross(ab, cd), c2 = cross(ac, ab);
  if (EQ(c1, 0) && EQ(c2, 0)) {
    double t0 = dot(ac, ab) / norm(ab), t1 = t0 + dot(cd, ab) / norm(ab);
    if (LE(std::min(t0, t1), 1) && LE(0, std::max(t0, t1))) {
      return 0;
    }
  } else {
    double t = cross(ac, cd) / c1, u = c2 / c1;
    if (!EQ(c1, 0) && LE(0, t) && LE(t, 1) && LE(0, u) && LE(u, 1)) {
      return 0;
    }
  }
  return std::min(std::min(seg_dist(a, c, d), seg_dist(b, c, d)),
                  std::min(seg_dist(c, a, b), seg_dist(d, a, b)));
}

point closest_point(const point &a, const point &b, const point &p) {
  if (EQ(a.x, b.x) && EQ(a.y, b.y)) {
    return a;
  }
  point ap(p.x - a.x, p.y - a.y), ab(b.x - a.x, b.y - a.y);
  double t = dot(ap, ab) / sqnorm(ab);
  return (t <= 0) ? a : ((t >= 1) ? b : point(a.x + t*ab.x, a.y + t*ab.y));
}

/*** Example Usage ***/

#include <cassert>

int main() {
  assert(EQ(5, dist(point(-1, -1), point(2, 3))));
  assert(EQ(25, sqdist(point(-1, -1), point(2, 3))));
  assert(EQ(1.2, line_dist(point(2, 1), -4, 3, -1)));
  assert(EQ(0.8, line_dist(point(3, 3), point(-1, -1), point(2, 3))));
  assert(EQ(1.2, line_dist(point(2, 1), point(-1, -1), point(2, 3))));
  assert(EQ(0, line_dist(-4, 3, -1, 8, 6, 2)));
  assert(EQ(0.8, line_dist(-4, 3, -1, -8, 6, -10)));
  assert(EQ(1.0, seg_dist(point(3, 3), point(-1, -1), point(2, 3))));
  assert(EQ(1.2, seg_dist(point(2, 1), point(-1, -1), point(2, 3))));
  assert(EQ(0, seg_dist(point(0, 2), point(3, 3), point(-1, -1), point(2, 3))));
  assert(EQ(0.6,
            seg_dist(point(-1, 0), point(-2, 2), point(-1, -1), point(2, 3))));
  return 0;
}
\end{lstlisting}
\subsection{Line Intersection}
\begin{lstlisting}
/*

Intersection and closest point calculations in two dimensions for straight lines
and line segments.

- line_intersection(a1, b1, c1, a2, b2, c2, &p) determines whether the lines
  a1*x + b1*y + c1 = 0 and a2*x + b2*x + c2 = 0 intersects, returning -1 if
  there is no intersection because the lines are parallel, 0 if there is exactly
  one intersection (in which case the intersection point is stored into pointer
  p if it's not NULL), or 1 if there are infinite intersections because the
  lines are identical.
- line_intersection(p1, p2, p3, p4, &p) determines whether the infinite lines
  (not segments) through points p1, p2 and through points p3 and p4 intersect,
  returning -1 if there is no intersection because the lines are parallel, 0 if
  there is exactly one intersection (in which case the intersection point is
  stored into pointer p if it's not NULL), or 1 if there are infinite
  intersections because the lines are identical.
- seg_intersection(a, b, c, d, &p, &q) determines whether the line segment ab
  intersects the line segment cd, returning -1 if the segments do not intersect,
  0 if there is exactly one intersection point (in which case it is stored into
  pointer p if it's not NULL), or 1 if the intersection is another line segment
  (in which case the two endpoints are stored into pointers p and q if they are
  not NULL). If the segments are barely touching (close within EPS), then the
  result will depend on the setting of TOUCH_IS_INTERSECT.
- closest_point(a, b, c, p) returns the point on line a*x + b*y + c = 0 that is
  closest to point p. Note that the result always lies on the line through p
  which is perpendicular to the line a*x + b*y + c = 0.
- closest_point(a, b, p) returns the point on segment ab closest to point p.

Time Complexity:
- O(1) for all operations.

Space Complexity:
- O(1) auxiliary for all operations.

*/

#include <algorithm>
#include <cmath>
#include <cstddef>
#include <utility>

const double EPS = 1e-9;

#define EQ(a, b) (fabs((a) - (b)) <= EPS)
#define LT(a, b) ((a) < (b) - EPS)
#define LE(a, b) ((a) <= (b) + EPS)

typedef std::pair<double, double> point;
#define x first
#define y second

double sqnorm(const point &a) { return a.x*a.x + a.y*a.y; }
double norm(const point &a) { return sqrt(sqnorm(a)); }
double dot(const point &a, const point &b) { return a.x*b.x + a.y*b.y; }
double cross(const point &a, const point &b) { return a.x*b.y - a.y*b.x; }

int line_intersection(double a1, double b1, double c1, double a2, double b2,
                      double c2, point *p = NULL) {
  if (EQ(a1, a2) && EQ(b1, b2)) {
    return EQ(c1, c2) ? 1 : -1;
  }
  if (p != NULL) {
    p->x = (b1*c1 - b1*c2) / (a2*b1 - a1*b2);
    if (!EQ(b1, 0)) {
      p->y = -(a1*p->x + c1) / b1;
    } else {
      p->y = -(a2*p->x + c2) / b2;
    }
  }
  return 0;
}

int line_intersection(const point &p1, const point &p2,
                      const point &p3, const point &p4, point *p = NULL) {
  double a1 = p2.y - p1.y, b1 = p1.x - p2.x;
  double c1 = -(p1.x*p2.y - p2.x*p1.y);
  double a2 = p4.y - p3.y, b2 = p3.x - p4.x;
  double c2 = -(p3.x*p4.y - p4.x*p3.y);
  double x = -(c1*b2 - c2*b1), y = -(a1*c2 - a2*c1);
  double det = a1*b2 - a2*b1;
  if (EQ(det, 0)) {
    return (EQ(x, 0) && EQ(y, 0)) ? 1 : -1;
  }
  if (p != NULL) {
    *p = point(x / det, y / det);
  }
  return 0;
}

int seg_intersection(const point &a, const point &b, const point &c,
                     const point &d, point *p = NULL, point *q = NULL) {
  static const bool TOUCH_IS_INTERSECT = true;
  point ab(b.x - a.x, b.y - a.y);
  point ac(c.x - a.x, c.y - a.y);
  point cd(d.x - c.x, d.y - c.y);
  double c1 = cross(ab, cd), c2 = cross(ac, ab);
  if (EQ(c1, 0) && EQ(c2, 0)) {  // Collinear.
    double t0 = dot(ac, ab) / sqnorm(ab);
    double t1 = t0 + dot(cd, ab) / sqnorm(ab);
    double mint = std::min(t0, t1), maxt = std::max(t0, t1);
    bool overlap = TOUCH_IS_INTERSECT ? (LE(mint, 1) && LE(0, maxt))
                                      : (LT(mint, 1) && LT(0, maxt));
    if (overlap) {
      point res1 = std::max(std::min(a, b), std::min(c, d));
      point res2 = std::min(std::max(a, b), std::max(c, d));
      if (res1 == res2) {
        if (p != NULL) {
          *p = res1;
        }
        return 0;  // Collinear and meeting at an endpoint.
      }
      if (p != NULL && q != NULL) {
        *p = res1;
        *q = res2;
      }
      return 1;  // Collinear and overlapping.
    } else {
      return -1;  // Collinear and disjoint.
    }
  }
  if (EQ(c1, 0)) {
    return -1;  // Parallel and disjoint.
  }
  double t = cross(ac, cd)/c1, u = c2/c1;
  bool t_between_01 = TOUCH_IS_INTERSECT ? (LE(0, t) && LE(t, 1))
                                         : (LT(0, t) && LT(t, 1));
  bool u_between_01 = TOUCH_IS_INTERSECT ? (LE(0, u) && LE(u, 1))
                                         : (LT(0, u) && LT(u, 1));
  if (t_between_01 && u_between_01) {
    if (p != NULL) {
      *p = point(a.x + t*ab.x, a.y + t*ab.y);
    }
    return 0;  // Non-parallel with one intersection.
  }
  return -1;  // Non-parallel with no intersections.
}

point closest_point(double a, double b, double c, const point &p) {
  if (EQ(a, 0)) {
    return point(p.x, -c);  // Horizontal line.
  }
  if (EQ(b, 0)) {
    return point(-c, p.y);  // Vertical line.
  }
  point res;
  line_intersection(a, b, c, -b, a, b*p.x - a*p.y, &res);
  return res;
}

point closest_point(const point &a, const point &b, const point &p) {
  if (a == b) return a;
  point ap(p.x - a.x, p.y - a.y), ab(b.x - a.x, b.y - a.y);
  double t = dot(ap, ab) / norm(ab);
  if (t <= 0) return a;
  if (t >= 1) return b;
  return point(a.x + t * ab.x, a.y + t * ab.y);
}

/*** Example Usage ***/

#include <cassert>
#define point point

bool EQP(const point &a, const point &b) {
  return EQ(a.x, b.x) && EQ(a.y, b.y);
}

int main() {
  point p, q;

  assert(line_intersection(-1, 1, 0, 1, 1, -3, &p) == 0);
  assert(EQP(p, point(1.5, 1.5)));
  assert(line_intersection(point(0, 0), point(1, 1), point(0, 4), point(4, 0),
                           &p) == 0);
  assert(EQP(p, point(2, 2)));

  {
    #define test(a, b, c, d, e, f, g, h) seg_intersection( \
        point(a, b), point(c, d), point(e, f), point(g, h), &p, &q)

    // Intersection is a point.
    assert(0 == test(-4, 0, 4, 0, 0, -4, 0, 4) && EQP(p, point(0, 0)));
    assert(0 == test(0, 0, 10, 10, 2, 2, 16, 4) && EQP(p, point(2, 2)));
    assert(0 == test(-2, 2, -2, -2, -2, 0, 0, 0) && EQP(p, point(-2, 0)));
    assert(0 == test(0, 4, 4, 4, 4, 0, 4, 8) && EQP(p, point(4, 4)));

    // Intersection is a segment.
    assert(1 == test(10, 10, 0, 0, 2, 2, 6, 6));
    assert(EQP(p, point(2, 2)) && EQP(q, point(6, 6)));
    assert(1 == test(6, 8, 14, -2, 14, -2, 6, 8));
    assert(EQP(p, point(6, 8)) && EQP(q, point(14, -2)));

    // No intersection.
    assert(-1 == test(6, 8, 8, 10, 12, 12, 4, 4));
    assert(-1 == test(-4, 2, -8, 8, 0, 0, -4, 6));
    assert(-1 == test(4, 4, 4, 6, 0, 2, 0, 0));
    assert(-1 == test(4, 4, 6, 4, 0, 2, 0, 0));
    assert(-1 == test(-2, -2, 4, 4, 10, 10, 6, 6));
    assert(-1 == test(0, 0, 2, 2, 4, 0, 1, 4));
    assert(-1 == test(2, 2, 2, 8, 4, 4, 6, 4));
    assert(-1 == test(4, 2, 4, 4, 0, 8, 10, 0));
  }

  assert(EQP(point(2.5, 2.5), closest_point(-1, -1, 5, point(0, 0))));
  assert(EQP(point(3, 0), closest_point(1, 0, -3, point(0, 0))));
  assert(EQP(point(0, 3), closest_point(0, 1, -3, point(0, 0))));

  assert(EQP(point(3, 0),
             closest_point(point(3, 0), point(3, 3), point(0, 0))));
  assert(EQP(point(2, -1),
             closest_point(point(2, -1), point(4, -1), point(0, 0))));
  assert(EQP(point(4, -1),
             closest_point(point(2, -1), point(4, -1), point(5, 0))));
  return 0;
}
\end{lstlisting}
\subsection{Circle Intersection}
\begin{lstlisting}
/*

Circle tangent and intersection calculations in two dimensions.

- tangent(c, p, &l1, &l2) determines the line(s) tangent to circle c that passes
  through point p, returning -1 if there is no tangent line because p is
  strictly inside c, 0 if there is exactly one tangent line because p is on the
  boundary of c (in which case the line will be stored into pointer l1 if it's
  not NULL), or 1 if there are two tangent lines because p is strictly outside
  of c (in which case the lines will be stored into pointers l1 and l2 if they
  are not NULL).
- intersection(c, l, &p, &q) determines the intersection between the circle c
  and line l, returning -1 if there is no intersection, 0 if the line is one
  intersection point because the line is tangent (in which case it will be
  stored into pointer p if it's not NULL), or 1 if there are two intersection
  points because the line crosses through the circle (in which case they will be
  stored into pointers p and q if they are not NULL).
- intersection(c1, c2, &p, &q) determines the intersection points between two
  circles c1 and c2, returning -2 if circle c2 completely encloses circle c1,
  -1 if circle c1 completely encloses circle c2, 0 if the circles are completely
  disjoint, 1 if the circles are tangent with one intersection (stored in p),
  2 if the circles intersect at two points (stored in p and q), 3 if the circles
  are equal and intersect at infinite points.
- intersection_area(c1, c2) returns the intersection area of circles c1 and c2.

Time Complexity:
- O(1) for all operations.

Space Complexity:
- O(1) auxiliary for all operations.

*/

#include <algorithm>
#include <cmath>
#include <cstddef>
#include <utility>

const double EPS = 1e-9, PI = acos(-1.0);

#define EQ(a, b) (fabs((a) - (b)) <= EPS)
#define NE(a, b) (fabs((a) - (b)) > EPS)
#define LT(a, b) ((a) < (b) - EPS)
#define GT(a, b) ((a) > (b) + EPS)
#define LE(a, b) ((a) <= (b) + EPS)
#define GE(a, b) ((a) >= (b) - EPS)

typedef std::pair<double, double> point;
#define x first
#define y second

double sqnorm(const point &a) { return a.x*a.x + a.y*a.y; }
double norm(const point &a) { return sqrt(sqnorm(a)); }

struct circle {
  double h, k, r;

  circle(double h, double k, double r) {
    this->h = h;
    this->k = k;
    this->r = r;
  }
};

struct line {
  double a, b, c;

  line() : a(0), b(0), c(0) {}

  line(double a, double b, double c) {
    if (!EQ(b, 0)) {
      this->a = a / b;
      this->c = c / b;
      this->b = 1;
    } else {
      this->c = c / a;
      this->a = 1;
      this->b = 0;
    }
  }

  line(const point &p, const point &q) : a(0), b(0), c(0) {
    if (EQ(p.x, q.x)) {
      if (NE(p.y, q.y)) {  // Vertical line.
        a = 1;
        b = 0;
        c = -p.x;
      }  // Else, invalid line.
    } else {
      a = -(p.y - q.y) / (p.x - q.x);
      b = 1;
      c = -(a*p.x) - (b*p.y);
    }
  }
};

int tangent(const circle &c, const point &p, line *l1 = NULL, line *l2 = NULL) {
  point vop(p.x - c.h, p.y - c.k);
  if (EQ(sqnorm(vop), c.r*c.r)) {  // Point on an edge.
    if (l1 != 0) {  // Get perpendicular line through p.
      *l1 = line(point(c.h, c.k), p);
      *l1 = line(-l1->b, l1->a, l1->b*p.x - l1->a*p.y);
    }
    return 0;
  }
  if (LE(sqnorm(vop), c.r*c.r)) {
    return -1;  // Point inside circle, no intersection.
  }
  point q(vop.x / c.r, vop.y / c.r);
  double n = sqnorm(q), d = q.y*sqrt(sqnorm(q) - 1.0);
  point t1((q.x - d) / n, c.k), t2((q.x + d) / n, c.k);
  if (NE(q.y, 0)) {  // Common case.
    t1.y += c.r*(1.0 - t1.x*q.x) / q.y;
    t2.y += c.r*(1.0 - t2.x*q.x) / q.y;
  } else {  // Point at center horizontal, y = 0.
    d = c.r*sqrt(1.0 - t1.x*t1.x);
    t1.y += d;
    t2.y -= d;
  }
  t1.x = t1.x*c.r + c.h;
  t2.x = t2.x*c.r + c.h;
  //note: here, t1 and t2 are the two points of tangencies
  if (l1 != NULL && l2 != NULL) {
    *l1 = line(p, t1);
    *l2 = line(p, t2);
  }
  return 1;
}

int intersection(const circle &c, const line &l, point *p = NULL,
                 point *q = NULL) {
  double v = c.h*l.a + c.k*l.b + l.c;
  double aabb = l.a*l.a + l.b*l.b;
  double disc = v*v / aabb - c.r*c.r;
  if (disc > EPS) {
    return -1;
  }
  double x0 = -l.a*l.c / aabb, y0 = -l.b*v / aabb;
  if (disc > -EPS) {
    if (p != NULL) {
      *p = point(x0 + c.h, y0 + c.k);
    }
    return 0;
  }
  double k = sqrt(std::max(0.0, disc / -aabb));
  if (p != NULL && q != NULL) {
    *p = point(x0 + k*l.b + c.h, y0 - k*l.a + c.k);
    *q = point(x0 - k*l.b + c.h, y0 + k*l.a + c.k);
  }
  return 1;
}

int intersection(const circle &c1, const circle &c2, point *p = NULL,
                 point *q = NULL) {
  if (EQ(c1.h, c2.h) && EQ(c1.k, c2.k)) {
    return EQ(c1.r, c2.r) ? 3 : (c1.r > c2.r ? -1 : -2);
  }
  point d12(point(c2.h - c1.h, c2.k - c1.k));
  double d = norm(d12);
  if (GT(d, c1.r + c2.r)) {
    return 0;
  }
  if (LT(d, fabs(c1.r - c2.r))) {
    return c1.r > c2.r ? -1 : -2;
  }
  double a = (c1.r*c1.r - c2.r*c2.r + d*d) / (2*d);
  double x0 = c1.h + (d12.x*a / d), y0 = c1.k + (d12.y*a / d);
  double s = sqrt(c1.r*c1.r - a*a), rx = -d12.y*s / d, ry = d12.x*s / d;
  if (EQ(rx, 0) && EQ(ry, 0)) {
    if (p != NULL) {
      *p = point(x0, y0);
    }
    return 1;
  }
  if (p != NULL && q != NULL) {
    *p = point(x0 - rx, y0 - ry);
    *q = point(x0 + rx, y0 + ry);
  }
  return 2;
}

double intersection_area(const circle &c1, const circle &c2) {
  double r = std::min(c1.r, c2.r), R = std::max(c1.r, c2.r);
  double d = norm(point(c2.h - c1.h, c2.k - c1.k));
  if (LE(d, R - r)) {
    return PI*r*r;
  }
  if (GE(d, R + r)) {
    return 0;
  }
  return r*r*acos((d*d + r*r - R*R) / 2 / d / r) +
         R*R*acos((d*d + R*R - r*r) / 2 / d / R) -
         0.5*sqrt((-d + r + R)*(d + r - R)*(d - r + R)*(d + r + R));
}

/*** Example Usage ***/

#include <cassert>

bool EQP(const point &a, const point &b) {
  return EQ(a.x, b.x) && EQ(a.y, b.y);
}

bool EQL(const line &l1, const line &l2) {
  return EQ(l1.a, l2.a) && EQ(l1.b, l2.b) && EQ(l1.c, l2.c);
}

int main() {
  line l1, l2;
  assert(-1 == tangent(circle(0, 0, 4), point(1, 1), &l1, &l2));
  assert(0 == tangent(circle(0, 0, sqrt(2)), point(1, 1), &l1, &l2));
  assert(EQL(l1, line(-1, -1, 2)));
  assert(1 == tangent(circle(0, 0, 2), point(2, 2), &l1, &l2));
  assert(EQL(l1, line(0, -2, 4)));
  assert(EQL(l2, line(2, 0, -4)));

  point p, q;
  assert(-1 == intersection(circle(1, 1, 3), line(5, 3, -30), &p, &q));
  assert(0 == intersection(circle(1, 1, 3), line(0, 1, -4), &p, &q));
  assert(EQP(p, point(1, 4)));
  assert(1 == intersection(circle(1, 1, 3), line(0, 1, -1), &p, &q));
  assert(EQP(p, point(4, 1)));
  assert(EQP(q, point(-2, 1)));

  assert(-2 == intersection(circle(1, 1, 1), circle(0, 0, 3), &p, &q));
  assert(-1 == intersection(circle(0, 0, 3), circle(1, 1, 1), &p, &q));
  assert(0 == intersection(circle(5, 0, 4), circle(-5, 0, 4), &p, &q));
  assert(1 == intersection(circle(-5, 0, 5), circle(5, 0, 5), &p, &q));
  assert(EQP(p, point(0, 0)));
  assert(2 == intersection(circle(-0.5, 0, 1), circle(0.5, 0, 1), &p, &q));
  assert(EQP(p, point(0, -sqrt(3) / 2)));
  assert(EQP(q, point(0, sqrt(3) / 2)));

  // Each circle passes through the other's center.
  double r = 3, a = intersection_area(circle(-r/2, 0, r), circle(r/2, 0, r));
  assert(EQ(a, r*r*(2*PI / 3 - sqrt(3) / 2)));
  return 0;
}
\end{lstlisting}

\section{Intermediate Geometric Calculations}
\setcounter{section}{3}
\setcounter{subsection}{0}
\subsection{Polygon Sorting and Area}
\begin{lstlisting}
/*

Given a list of distinct points in two-dimensions, order them into a valid
polygon and determine the area.

- mean_center(lo, hi) returns the arithmetic mean of a range [lo, hi) of points,
  where lo and hi must be random-access iterators. This point is mathematically
  guaranteed to lie within the non-self-intersecting closed polygon constructed
  by sorting all other points clockwise about it. Note that this is different
  from the geometric centroid (a.k.a. barycenter) of a polygon.
- cw_comp(a, b, c) returns whether point a compares clockwise "before" point b
  when using c as a central reference point.
- cw_comp_class(c) constructs a wrapper class of cw_comp() that may be passed to
  std::sort() a range of points clockwise to produce a valid polygon.
- ccw_comp_class(c) constructs a wrapper class of cw_comp() that may be passed
  to std::sort() a range of points counter-clockwise to produce a valid polygon.
- polygon_area(lo, hi) returns the area of the polygon specified by the range
  [lo, hi) of points, where lo and hi must be BidirectionalIterators. The points
  are interpreted as a polygon based on the order given in the range. The input
  polygon does not have to be sorted using the methods above, but must be given
  in some ordering that yields a valid non-self-intersecting closed polygon.
  Optionally, the last point may be equal to the first point in the input
  without affecting the result. The area is computed using the shoelace formula.

Time Complexity:
- O(n) per call to mean_center(lo, hi) and polygon_area(lo, hi), where n is the
  distance between lo and hi.
- O(1) per call to cw_comp() and the related class comparators.

Space Complexity:
- O(1) auxiliary for all operations.

*/

#include <algorithm>
#include <cmath>
#include <stdexcept>
#include <utility>

const double EPS = 1e-9;

#define EQ(a, b) (fabs((a) - (b)) <= EPS)
#define LT(a, b) ((a) < (b) - EPS)
#define GE(a, b) ((a) >= (b) - EPS)

typedef std::pair<double, double> point;
#define x first
#define y second

double sqnorm(const point &a) { return a.x*a.x + a.y*a.y; }
double cross(const point &a, const point &b) { return a.x*b.y - a.y*b.x; }

template<class It>
point mean_center(It lo, It hi) {
  if (lo == hi) {
    throw std::runtime_error("Cannot get center of an empty range.");
  }
  double x_sum = 0, y_sum = 0, num_points = hi - lo;
  for (; lo != hi; ++lo) {
    x_sum += lo->x;
    y_sum += lo->y;
  }
  return point(x_sum / num_points, y_sum / num_points);
}

bool cw_comp(const point &a, const point &b, const point &c) {
  if (GE(a.x - c.x, 0) && LT(b.x - c.x, 0)) {
    return true;
  }
  if (LT(a.x - c.x, 0) && GE(b.x - c.x, 0)) {
    return false;
  }
  if (EQ(a.x - c.x, 0) && EQ(b.x - c.x, 0)) {
    if (GE(a.y - c.y, 0) || GE(b.y - c.y, 0)) {
      return a.y > b.y;
    }
    return b.y > a.y;
  }
  point ac(a.x - c.x, a.y - c.y), bc(b.x - c.x, b.y - c.y);
  double det = cross(ac, bc);
  if (EQ(det, 0)) {
    return sqnorm(ac) > sqnorm(bc);
  }
  return det < 0;
}

struct cw_comp_class {
  point c;
  cw_comp_class(const point &c) : c(c) {}
  bool operator()(const point &a, const point &b) const {
    return cw_comp(a, b, c);
  }
};

struct ccw_comp_class {
  point c;
  ccw_comp_class(const point &c) : c(c) {}
  bool operator()(const point &a, const point &b) const {
    return cw_comp(b, a, c);
  }
};

template<class It>
double polygon_area(It lo, It hi) {
  if (lo == hi) {
    return 0;
  }
  double area = 0;
  if (*lo != *--hi) {
    area += (lo->x - hi->x)*(lo->y + hi->y);
  }
  for (It i = hi, j = --hi; i != lo; --i, --j) {
    area += (i->x - j->x)*(i->y + j->y);
  }
  return fabs(area / 2.0);
}

/*** Example Usage ***/

#include <cassert>
#include <vector>
using namespace std;

int main() {
  // Irregular pentagon with only the vertex (1, 2) not on its convex hull.
  // The ordering here is already sorted in ccw order around their mean center,
  // though we will shuffle them to verify our sorting comparator.
  point points[] = {point(1, 3),
                    point(1, 2),
                    point(2, 1),
                    point(0, 0),
                    point(-1, 3)};
  vector<point> v(points, points + 5);
  std::random_shuffle(v.begin(), v.end());
  point c = mean_center(v.begin(), v.end());
  assert(EQ(c.x, 0.6) && EQ(c.y, 1.8));
  sort(v.begin(), v.end(), cw_comp_class(c));
  for (int i = 0; i < (int)v.size(); i++) {
    assert(v[i] == points[i]);
  }
  assert(EQ(polygon_area(v.begin(), v.end()), 5));
  return 0;
}
\end{lstlisting}
\subsection{Point-in-Polygon (Ray Casting)}
\begin{lstlisting}
/*

Given a point p and a polygon in two dimensions, determine whether p lies inside
the polygon using a ray casting algorithm.

- point_in_polygon(p, lo, hi) returns whether p lies within the polygon defined
  by the range [lo, hi) of points specifying the vertices in either clockwise
  or counter-clockwise order, where lo and hi must be random-access iterators.
  If p lies barely on an edge (within EPS), then the result will depend on the
  setting of EDGE_IS_INSIDE.

Time Complexity:
- O(n) per call to point_in_polygon(lo, hi), where n is the distance between lo
  and hi.

Space Complexity:
- O(1) auxiliary.

*/

#include <cmath>
#include <utility>

const double EPS = 1e-9;

#define EQ(a, b) (fabs((a) - (b)) <= EPS)
#define LE(a, b) ((a) <= (b) + EPS)
#define GT(a, b) ((a) > (b) + EPS)

typedef std::pair<double, double> point;
#define x first
#define y second

double cross(const point &a, const point &b, const point &o = point(0, 0)) {
  return (a.x - o.x)*(b.y - o.y) - (a.y - o.y)*(b.x - o.x);
}

template<class It>
bool point_in_polygon(const point &p, It lo, It hi) {
  static const bool EDGE_IS_INSIDE = true;
  bool ans = 0;
  for (It i = lo, j = hi - 1; i != hi; j = i++) {
    if (EQ(i->y, p.y) &&
        (EQ(i->x, p.x) ||
         (EQ(j->y, p.y) && (LE(i->x, p.x) || LE(j->x, p.x))))) {
      return EDGE_IS_INSIDE;
    }
    if (GT(i->y, p.y) != GT(j->y, p.y)) {
      double det = cross(*i, *j, p);
      if (EQ(det, 0)) {
        return EDGE_IS_INSIDE;
      }
      if (GT(det, 0) != GT(j->y, i->y)) {
        ans = !ans;
      }
    }
  }
  return ans;
}

/*** Example Usage ***/

#include <cassert>
using namespace std;

int main() {
  // Irregular trapezoid.
  point p[] = {point(-1, 3), point(1, 3), point(2, 1), point(0, 0)};
  assert(point_in_polygon(point(1, 2), p, p + 4));
  assert(point_in_polygon(point(0, 3), p, p + 4));
  assert(!point_in_polygon(point(0, 3.01), p, p + 4));
  assert(!point_in_polygon(point(2, 2), p, p + 4));
  return 0;
}
\end{lstlisting}
\subsection{Convex Hull and Diametral Pair}
\begin{lstlisting}
/*

Given a list of points in two dimensions, determine the convex hull using the
monotone chain algorithm, and the diameter of the points using the method of
rotating calipers. The convex hull is the smallest convex polygon (a polygon
such that every line crossing through it will only do so once) that contains all
of its points.

- convex_hull(lo, hi) returns the convex hull as a vector of polygon vertices in
  clockwise order, given a range [lo, hi) of points where lo and hi must be
  random-access iterators. The input range will be sorted lexicographically (by
  x, then by y) after the function call. Note that to produce the hull points in
  counter-clockwise order, replace every GE() comparison with LE(). To have the
  first point on the hull repeated as the last in the resulting vector, the
  final res.resize(k - 1) may be changed to res.resize(k).
- diametral_pair(lo, hi) returns a maximum diametral pair given a range [lo, hi)
  of points where lo and hi must be random-access iterators. The input range
  will be sorted lexicographically (by x, then by y) after the function call.

Time Complexity:
- O(n log n) per call to convex_hull(lo, hi) and diametral_pair(lo, hi), where n
  is the distance between lo and hi.

Space Complexity:
- O(n) auxiliary for storage of the convex hull in both operations.

*/

#include <algorithm>
#include <cmath>
#include <utility>
#include <vector>

const double EPS = 1e-9;

#define GE(a, b) ((a) >= (b) - EPS)

typedef std::pair<double, double> point;
#define x first
#define y second

double sqnorm(const point &a) { return a.x*a.x + a.y*a.y; }
double cross(const point &a, const point &b, const point &o = point(0, 0)) {
  return (a.x - o.x)*(b.y - o.y) - (a.y - o.y)*(b.x - o.x);
}

template<class It>
std::vector<point> convex_hull(It lo, It hi) {
  int k = 0;
  if (hi - lo <= 1) {
    return std::vector<point>(lo, hi);
  }
  std::vector<point> res(2*(int)(hi - lo));
  std::sort(lo, hi);
  for (It it = lo; it != hi; ++it) {
    while (k >= 2 && GE(cross(res[k - 1], *it, res[k - 2]), 0)) {
      k--;
    }
    res[k++] = *it;
  }
  int t = k + 1;
  for (It it = hi - 2; it != lo - 1; --it) {
    while (k >= t && GE(cross(res[k - 1], *it, res[k - 2]), 0)) {
      k--;
    }
    res[k++] = *it;
  }
  res.resize(k - 1);
  return res;
}

template<class It>
std::pair<point, point> diametral_pair(It lo, It hi) {
  std::vector<point> h = convex_hull(lo, hi);
  int m = h.size();
  if (m == 1) {
    return std::make_pair(h[0], h[0]);
  }
  if (m == 2) {
    return std::make_pair(h[0], h[1]);
  }
  int k = 1;
  while (fabs(cross(h[0], h[(k + 1) % m], h[m - 1])) >
         fabs(cross(h[0], h[k], h[m - 1]))) {
    k++;
  }
  double maxdist = 0, d;
  std::pair<point, point> res;
  for (int i = 0, j = k; i <= k && j < m; i++) {
    d = sqnorm(point(h[i].x - h[j].x, h[i].y - h[j].y));
    if (d > maxdist) {
      maxdist = d;
      res = std::make_pair(h[i], h[j]);
    }
    while (j < m && fabs(cross(h[(i + 1) % m], h[(j + 1) % m], h[i])) >
                    fabs(cross(h[(i + 1) % m], h[j], h[i]))) {
      d = sqnorm(point(h[i].x - h[(j + 1) % m].x, h[i].y - h[(j + 1) % m].y));
      if (d > maxdist) {
        maxdist = d;
        res = std::make_pair(h[i], h[(j + 1) % m]);
      }
      j++;
    }
  }
  return res;
}

/*** Example Usage ***/

#include <cassert>
using namespace std;

int main() {
  { // Irregular pentagon with only the vertex (1, 2) not on the hull.
    vector<point> v;
    v.push_back(point(1, 3));
    v.push_back(point(1, 2));
    v.push_back(point(2, 1));
    v.push_back(point(0, 0));
    v.push_back(point(-1, 3));
    std::random_shuffle(v.begin(), v.end());
    vector<point> h;
    h.push_back(point(-1, 3));
    h.push_back(point(1, 3));
    h.push_back(point(2, 1));
    h.push_back(point(0, 0));
    assert(convex_hull(v.begin(), v.end()) == h);
  }
  {
    vector<point> v;
    v.push_back(point(0, 0));
    v.push_back(point(3, 0));
    v.push_back(point(0, 3));
    v.push_back(point(1, 1));
    v.push_back(point(4, 4));
    pair<point, point> res = diametral_pair(v.begin(), v.end());
    assert(res.first == point(0, 0));
    assert(res.second == point(4, 4));
  }
  return 0;
}
\end{lstlisting}
\subsection{Minimum Enclosing Circle}
\begin{lstlisting}
/*

Given a list of points in two dimensions, find the circle with smallest area
which contains all the given points using a randomized algorithm.

- minimum_enclosing_circle(lo, hi) returns the minimum enclosing circle given a
  range [lo, hi) of points, where lo and hi must be random-access iterators. The
  input range will be shuffled after the function call, though this is only to
  avoid the worst-case running time and is not necessary for correctness.

Time Complexity:
- O(n) on average per call to minimum_enclosing_circle(lo, hi), where n is the
  distance between lo and hi.

Space Complexity:
- O(1) auxiliary.

*/

#include <algorithm>
#include <cmath>
#include <stdexcept>
#include <utility>

const double EPS = 1e-9;

#define EQ(a, b) (fabs((a) - (b)) <= EPS)
#define LE(a, b) ((a) <= (b) + EPS)

typedef std::pair<double, double> point;
#define x first
#define y second

double sqnorm(const point &a) { return a.x*a.x + a.y*a.y; }
double norm(const point &a) { return sqrt(sqnorm(a)); }

struct circle {
  double h, k, r;

  circle() : h(0), k(0), r(0) {}
  circle(double h, double k, double r) : h(h), k(k), r(fabs(r)) {}

  // Circle with the line segment ab as a diameter.
  circle(const point &a, const point &b) {
    h = (a.x + b.x)/2.0;
    k = (a.y + b.y)/2.0;
    r = norm(point(a.x - h, a.y - k));
  }

  // Circumcircle of three points.
  circle(const point &a, const point &b, const point &c) {
    double an = sqnorm(point(b.x - c.x, b.y - c.y));
    double bn = sqnorm(point(a.x - c.x, a.y - c.y));
    double cn = sqnorm(point(a.x - b.x, a.y - b.y));
    double wa = an*(bn + cn - an);
    double wb = bn*(an + cn - bn);
    double wc = cn*(an + bn - cn);
    double w = wa + wb + wc;
    if (EQ(w, 0)) {
      throw std::runtime_error("No circumcircle from collinear points.");
    }
    h = (wa*a.x + wb*b.x + wc*c.x)/w;
    k = (wa*a.y + wb*b.y + wc*c.y)/w;
    r = norm(point(a.x - h, a.y - k));
  }

  bool contains(const point &p) const {
    return LE(sqnorm(point(p.x - h, p.y - k)), r*r);
  }
};

template<class It>
circle minimum_enclosing_circle(It lo, It hi) {
  if (lo == hi) {
    return circle(0, 0, 0);
  }
  if (lo + 1 == hi) {
    return circle(lo->x, lo->y, 0);
  }
  std::random_shuffle(lo, hi);
  circle res(*lo, *(lo + 1));
  for (It i = lo + 2; i != hi; ++i) {
    if (res.contains(*i)) {
      continue;
    }
    res = circle(*lo, *i);
    for (It j = lo + 1; j != i; ++j) {
      if (res.contains(*j)) {
        continue;
      }
      res = circle(*i, *j);
      for (It k = lo; k != j; ++k) {
        if (!res.contains(*k)) {
          res = circle(*i, *j, *k);
        }
      }
    }
  }
  return res;
}

/*** Example Usage ***/

#include <cassert>
#include <vector>
using namespace std;

int main() {
  vector<point> v;
  v.push_back(point(0, 0));
  v.push_back(point(0, 1));
  v.push_back(point(1, 0));
  v.push_back(point(1, 1));
  circle res = minimum_enclosing_circle(v.begin(), v.end());
  assert(EQ(res.h, 0.5) && EQ(res.k, 0.5) && EQ(res.r, 1/sqrt(2)));
  return 0;
}
\end{lstlisting}
\subsection{Closest Pair}
\begin{lstlisting}
/*

Given a list of points in two dimensions, find the closest pair among them using
a divide and conquer algorithm.

- closest_pair(lo, hi, &res) returns the minimum Euclidean distance between any
  two pair of points in the range [lo, hi), where lo and hi must be
  random-access iterators. The input range will be sorted lexicographically (by
  x, then by y) after the function call. If there is an answer, the closest pair
  will be stored into pointer *res.

Time Complexity:
- O(n log^2 n) per call to closest_pair(lo, hi, &res), where n is the distance
  between lo and hi.

Space Complexity:
- O(n log^2 n) auxiliary stack space for closest_pair(lo, hi, &res), where n is
  the distance between lo and hi.

*/

#include <algorithm>
#include <cmath>
#include <limits>
#include <utility>

const double EPS = 1e-9;

#define EQ(a, b) (fabs((a) - (b)) <= EPS)
#define LT(a, b) ((a) < (b) - EPS)

typedef std::pair<double, double> point;
#define x first
#define y second

double sqnorm(const point &a) { return a.x*a.x + a.y*a.y; }
double norm(const point &a) { return sqrt(sqnorm(a)); }
bool cmp_x(const point &a, const point &b) { return LT(a.x, b.x); }
bool cmp_y(const point &a, const point &b) { return LT(a.y, b.y); }

template<class It>
double closest_pair(It lo, It hi, std::pair<point, point> *res = NULL,
                    double mindist = std::numeric_limits<double>::max(),
                    bool sort_x = true) {
  if (lo == hi) {
    return std::numeric_limits<double>::max();
  }
  if (sort_x) {
    std::sort(lo, hi, cmp_x);
  }
  It mid = lo + (hi - lo)/2;
  double midx = mid->x;
  double d1 = closest_pair(lo, mid, res, mindist, false);
  mindist = std::min(mindist, d1);
  double d2 = closest_pair(mid + 1, hi, res, mindist, false);
  mindist = std::min(mindist, d2);
  std::sort(lo, hi, cmp_y);
  int size = 0;
  It t[hi - lo];
  for (It it = lo; it != hi; ++it) {
    if (fabs(it->x - midx) < mindist) {
      t[size++] = it;
    }
  }
  for (int i = 0; i < size; i++) {
    for (int j = i + 1; j < size; j++) {
      point a(*t[i]), b(*t[j]);
      if (b.y - a.y >= mindist) {
        break;
      }
      double dist = norm(point(a.x - b.x, a.y - b.y));
      if (mindist > dist) {
        mindist = dist;
        if (res) {
          *res = std::make_pair(a, b);
        }
      }
    }
  }
  return mindist;
}

/*** Example Usage ***/

#include <cassert>
#include <vector>
using namespace std;

int main() {
  vector<point> v;
  v.push_back(point(2, 3));
  v.push_back(point(12, 30));
  v.push_back(point(40, 50));
  v.push_back(point(5, 1));
  v.push_back(point(12, 10));
  v.push_back(point(3, 4));
  pair<point, point> res;
  assert(EQ(closest_pair(v.begin(), v.end(), &res), sqrt(2)));
  assert(res.first == point(2, 3));
  assert(res.second == point(3, 4));
  return 0;
}
\end{lstlisting}
\subsection{Segment Intersection Finding}
\begin{lstlisting}
/*

Given a list of line segments in two dimensions, determine whether any pair of
segments intersect using a sweep line algorithm.

- find_intersection(lo, hi, &res1, &res2) returns whether any pair of segments
  intersect given a range [lo, hi) of segments, where lo and hi are
  random-access iterators. If there an intersection is found, then one such pair
  of segments will be stored into pointers res1 and res2. If some segments are
  barely touching (close within EPS), then the result will depend on the setting
  of TOUCH_IS_INTERSECT.

Time Complexity:
- O(n log n) per call to find_intersection(lo, hi, &res1, &res2), where n is
  the distance between lo and hi.

Space Complexity:
- O(n) auxiliary heap space for find_intersection(lo, hi, &res1, &res2), where n
  is the distance between lo and hi.

*/

#include <algorithm>
#include <cmath>
#include <set>
#include <utility>

const double EPS = 1e-9;

#define EQ(a, b) (fabs((a) - (b)) <= EPS)
#define LT(a, b) ((a) < (b) - EPS)
#define LE(a, b) ((a) <= (b) + EPS)

typedef std::pair<double, double> point;
#define x first
#define y second

double sqnorm(const point &a) { return a.x*a.x + a.y*a.y; }
double norm(const point &a) { return sqrt(sqnorm(a)); }
double dot(const point &a, const point &b) { return a.x*b.x + a.y*b.y; }
double cross(const point &a, const point &b, const point &o = point(0, 0)) {
  return (a.x - o.x)*(b.y - o.y) - (a.y - o.y)*(b.x - o.x);
}

int seg_intersection(const point &a, const point &b, const point &c,
                     const point &d, point *p = NULL, point *q = NULL) {
  static const bool TOUCH_IS_INTERSECT = true;
  point ab(b.x - a.x, b.y - a.y);
  point ac(c.x - a.x, c.y - a.y);
  point cd(d.x - c.x, d.y - c.y);
  double c1 = cross(ab, cd), c2 = cross(ac, ab);
  if (EQ(c1, 0) && EQ(c2, 0)) {  // Collinear.
    double t0 = dot(ac, ab) / sqnorm(ab);
    double t1 = t0 + dot(cd, ab) / sqnorm(ab);
    double mint = std::min(t0, t1), maxt = std::max(t0, t1);
    bool overlap = TOUCH_IS_INTERSECT ? (LE(mint, 1) && LE(0, maxt))
                                      : (LT(mint, 1) && LT(0, maxt));
    if (overlap) {
      point res1 = std::max(std::min(a, b), std::min(c, d));
      point res2 = std::min(std::max(a, b), std::max(c, d));
      if (res1 == res2) {
        if (p != NULL) {
          *p = res1;
        }
        return 0;  // Collinear and meeting at an endpoint.
      }
      if (p != NULL && q != NULL) {
        *p = res1;
        *q = res2;
      }
      return 1;  // Collinear and overlapping.
    } else {
      return -1;  // Collinear and disjoint.
    }
  }
  if (EQ(c1, 0)) {
    return -1;  // Parallel and disjoint.
  }
  double t = cross(ac, cd)/c1, u = c2/c1;
  bool t_between_01 = TOUCH_IS_INTERSECT ? (LE(0, t) && LE(t, 1))
                                         : (LT(0, t) && LT(t, 1));
  bool u_between_01 = TOUCH_IS_INTERSECT ? (LE(0, u) && LE(u, 1))
                                         : (LT(0, u) && LT(u, 1));
  if (t_between_01 && u_between_01) {
    if (p != NULL) {
      *p = point(a.x + t*ab.x, a.y + t*ab.y);
    }
    return 0;  // Non-parallel with one intersection.
  }
  return -1;  // Non-parallel with no intersections.
}

struct segment {
  point p, q;

  segment() {}
  segment(const point &p, const point &q) : p(min(p, q)), q(max(p, q)) {}

  bool operator<(const segment &rhs) const {
    if (p.x < rhs.p.x) {
      double c = cross(q, rhs.p, p);
      if (c != 0) {
        return c > 0;
      }
    } else if (rhs.p.x < p.x) {
      double c = cross(rhs.q, q, rhs.p);
      if (c != 0) {
        return c < 0;
      }
    }
    return p.y < rhs.p.y;
  }
};

template<class SegIt>
struct event {
  point p;
  int type;
  SegIt seg;

  event() {}
  event(const point &p, int type, SegIt seg) : p(p), type(type), seg(seg) {}

  bool operator<(const event &rhs) const {
    if (p.x != rhs.p.x) {
      return p.x < rhs.p.x;
    }
    if (type != rhs.type) {
      return rhs.type < type;
    }
    return p.y < rhs.p.y;
  }
};

bool intersect(const segment &s1, const segment &s2) {
  return seg_intersection(s1.p, s1.q, s2.p, s2.q) >= 0;
}

template<class It>
bool find_intersection(It lo, It hi, segment *res1, segment *res2) {
  int cnt = 0;
  event<It> e[2*(int)(hi - lo)];
  for (It it = lo; it != hi; ++it) {
    if (it->p > it->q) {
      std::swap(it->p, it->q);
    }
    e[cnt++] = event<It>(it->p, 1, it);
    e[cnt++] = event<It>(it->q, -1, it);
  }
  std::sort(e, e + cnt);
  std::set<segment> s;
  std::set<segment>::iterator it, next, prev;
  for (int i = 0; i < cnt; i++) {
    It seg = e[i].seg;
    if (e[i].type == 1) {
      it = s.lower_bound(*seg);
      if (it != s.end() && intersect(*it, *seg)) {
        *res1 = *it;
        *res2 = *seg;
        return true;
      }
      if (it != s.begin() && intersect(*--it, *seg)) {
        *res1 = *it;
        *res2 = *seg;
        return true;
      }
      s.insert(*seg);
    } else {
      it = s.lower_bound(*seg);
      next = prev = it;
      prev = it;
      if (it != s.begin() && it != --s.end()) {
        if (intersect(*(++next), *(--prev))) {
          *res1 = *next;
          *res2 = *prev;
          return true;
        }
      }
      s.erase(it);
    }
  }
  return false;
}

/*** Example Usage ***/

#include <vector>
using namespace std;

int main() {
  vector<segment> v;
  v.push_back(segment(point(0, 0), point(2, 2)));
  v.push_back(segment(point(3, 0), point(0, -1)));
  v.push_back(segment(point(0, 2), point(2, -2)));
  v.push_back(segment(point(0, 3), point(9, 0)));
  segment res1, res2;
  assert(find_intersection(v.begin(), v.end(), &res1, &res2));
  assert(res1.p == point(0, 0) && res1.q == point(2, 2));
  assert(res2.p == point(0, 2) && res2.q == point(2, -2));
  return 0;
}
\end{lstlisting}

\section{Advanced Geometric Computations}
\setcounter{section}{4}
\setcounter{subsection}{0}
\subsection{Convex Polygon Cut}
\begin{lstlisting}
/*

Given a convex polygon (a polygon such that every line crossing through it will
only do so once) in two dimensions, and two points specifying an infinite line,
cut off the right part of the polygon, and return the resulting left part.

- convex_cut(lo, hi, p, q) returns the points of the left side of a polygon, in
  clockwise order, after it has been cut by the line containing points p and q.
  The original convex polygon is given by the range [lo, hi) of points in
  clockwise order, where lo and hi must be random-access iterators.

Time Complexity:
- O(n) per call to convex_cut(lo, hi, p, q), where n is the distance between lo
  and hi.

Space Complexity:
- O(n) auxiliary for storage of the resulting convex cut.

*/

#include <cmath>
#include <cstddef>
#include <utility>
#include <vector>

const double EPS = 1e-9;

#define EQ(a, b) (fabs((a) - (b)) <= EPS)
#define LT(a, b) ((a) < (b) - EPS)
#define GT(a, b) ((a) > (b) + EPS)

typedef std::pair<double, double> point;
#define x first
#define y second

double cross(const point &a, const point &b, const point &o = point(0, 0)) {
  return (a.x - o.x)*(b.y - o.y) - (a.y - o.y)*(b.x - o.x);
}

int turn(const point &a, const point &o, const point &b) {
  double c = cross(a, b, o);
  return LT(c, 0) ? -1 : (GT(c, 0) ? 1 : 0);
}

int line_intersection(const point &p1, const point &p2,
                      const point &p3, const point &p4, point *p = NULL) {
  double a1 = p2.y - p1.y, b1 = p1.x - p2.x;
  double c1 = -(p1.x*p2.y - p2.x*p1.y);
  double a2 = p4.y - p3.y, b2 = p3.x - p4.x;
  double c2 = -(p3.x*p4.y - p4.x*p3.y);
  double x = -(c1*b2 - c2*b1), y = -(a1*c2 - a2*c1);
  double det = a1*b2 - a2*b1;
  if (EQ(det, 0)) {
    return (EQ(x, 0) && EQ(y, 0)) ? 1 : -1;
  }
  if (p != NULL) {
    *p = point(x / det, y / det);
  }
  return 0;
}

template<class It>
std::vector<point> convex_cut(It lo, It hi, const point &p, const point &q) {
  if (EQ(p.x, q.x) && EQ(p.y, q.y)) {
    throw std::runtime_error("Cannot cut using line from identical points.");
  }
  std::vector<point> res;
  for (It i = lo, j = hi - 1; i != hi; j = i++) {
    int d1 = turn(q, p, *j), d2 = turn(q, p, *i);
    if (d1 >= 0) {
      res.push_back(*j);
    }
    if (d1*d2 < 0) {
      point r;
      line_intersection(p, q, *j, *i, &r);
      res.push_back(r);
    }
  }
  return res;
}

/*** Example Usage ***/

#include <cassert>
using namespace std;

int main() {
  {
    vector<point> v;
    v.push_back(point(1, 3));
    v.push_back(point(2, 2));
    v.push_back(point(2, 1));
    v.push_back(point(0, 0));
    v.push_back(point(-1, 3));
    // Cut using the vertical line through (0, 0).
    vector<point> c;
    c.push_back(point(-1, 3));
    c.push_back(point(0, 3));
    c.push_back(point(0, 0));
    assert(convex_cut(v.begin(), v.end(), point(0, 0), point(0, 1)) == c);
  }
  { // On a non-convex input, the result may be multiple disjoint polygons!
    vector<point> v;
    v.push_back(point(0, 0));
    v.push_back(point(2, 2));
    v.push_back(point(0, 4));
    v.push_back(point(3, 4));
    v.push_back(point(3, 0));
    vector<point> c;
    c.push_back(point(1, 0));
    c.push_back(point(0, 0));
    c.push_back(point(1, 1));
    c.push_back(point(1, 3));
    c.push_back(point(0, 4));
    c.push_back(point(1, 4));
    assert(convex_cut(v.begin(), v.end(), point(1, 0), point(1, 4)) == c);
  }
  return 0;
}
\end{lstlisting}
\subsection{Polygon Intersection and Union}
\begin{lstlisting}
/*

Given two polygons, determine the areas of their intersection and union using a
sweep line algorithm and the inclusion-exclusion principle.

- intersection_area(lo1, hi1, lo2, hi2) returns the intersection area of two
  polygons respectively specified by two ranges [lo1, hi1) and [lo2, hi2) of
  vertices in clockwise order, where lo1, hi1, lo2, and hi2 must be
  random-access iterators.
- union_area(lo1, hi1, lo2, hi2) returns the union area of two polygons
  respectively specified by two ranges [lo1, hi1) and [lo2, hi2) of vertices in
  clockwise order, where lo1, hi1, lo2, and hi2 must be random-access iterators.

Time Complexity:
- O(n^2 log n) per call to intersection_area(lo1, hi1, lo2, hi2) and
  union_area(lo1, hi1, lo2, hi2) where n is the sum of distances between lo1 and
  hi1 and lo2 and hi2 respectively.

Space Complexity:
- O(n) auxiliary heap space for intersection_area(lo1, hi1, lo2, hi2) and
  union_area(lo1, hi1, lo2, hi2), where n is the sum of distances between lo1
  and hi1 and lo2 and hi2 respectively.

*/

#include <algorithm>
#include <cmath>
#include <set>
#include <utility>
#include <vector>

const double EPS = 1e-9;

#define EQ(a, b) (fabs((a) - (b)) <= EPS)
#define LT(a, b) ((a) < (b) - EPS)
#define LE(a, b) ((a) <= (b) + EPS)

typedef std::pair<double, double> point;
#define x first
#define y second

double sqnorm(const point &a) { return a.x*a.x + a.y*a.y; }
double dot(const point &a, const point &b) { return a.x*b.x + a.y*b.y; }
double cross(const point &a, const point &b, const point &o = point(0, 0)) {
  return (a.x - o.x)*(b.y - o.y) - (a.y - o.y)*(b.x - o.x);
}

int seg_intersection(const point &a, const point &b, const point &c,
                     const point &d, point *p = NULL, point *q = NULL) {
  static const bool TOUCH_IS_INTERSECT = true;
  point ab(b.x - a.x, b.y - a.y);
  point ac(c.x - a.x, c.y - a.y);
  point cd(d.x - c.x, d.y - c.y);
  double c1 = cross(ab, cd), c2 = cross(ac, ab);
  if (EQ(c1, 0) && EQ(c2, 0)) {  // Collinear.
    double t0 = dot(ac, ab) / sqnorm(ab);
    double t1 = t0 + dot(cd, ab) / sqnorm(ab);
    double mint = std::min(t0, t1), maxt = std::max(t0, t1);
    bool overlap = TOUCH_IS_INTERSECT ? (LE(mint, 1) && LE(0, maxt))
                                      : (LT(mint, 1) && LT(0, maxt));
    if (overlap) {
      point res1 = std::max(std::min(a, b), std::min(c, d));
      point res2 = std::min(std::max(a, b), std::max(c, d));
      if (res1 == res2) {
        if (p != NULL) {
          *p = res1;
        }
        return 0;  // Collinear and meeting at an endpoint.
      }
      if (p != NULL && q != NULL) {
        *p = res1;
        *q = res2;
      }
      return 1;  // Collinear and overlapping.
    } else {
      return -1;  // Collinear and disjoint.
    }
  }
  if (EQ(c1, 0)) {
    return -1;  // Parallel and disjoint.
  }
  double t = cross(ac, cd)/c1, u = c2/c1;
  bool t_between_01 = TOUCH_IS_INTERSECT ? (LE(0, t) && LE(t, 1))
                                         : (LT(0, t) && LT(t, 1));
  bool u_between_01 = TOUCH_IS_INTERSECT ? (LE(0, u) && LE(u, 1))
                                         : (LT(0, u) && LT(u, 1));
  if (t_between_01 && u_between_01) {
    if (p != NULL) {
      *p = point(a.x + t*ab.x, a.y + t*ab.y);
    }
    return 0;  // Non-parallel with one intersection.
  }
  return -1;  // Non-parallel with no intersections.
}

int line_intersection(const point &p1, const point &p2,
                      const point &p3, const point &p4, point *p = NULL) {
  double a1 = p2.y - p1.y, b1 = p1.x - p2.x;
  double c1 = -(p1.x*p2.y - p2.x*p1.y);
  double a2 = p4.y - p3.y, b2 = p3.x - p4.x;
  double c2 = -(p3.x*p4.y - p4.x*p3.y);
  double x = -(c1*b2 - c2*b1), y = -(a1*c2 - a2*c1);
  double det = a1*b2 - a2*b1;
  if (EQ(det, 0)) {
    return (EQ(x, 0) && EQ(y, 0)) ? 1 : -1;
  }
  if (p != NULL) {
    *p = point(x / det, y / det);
  }
  return 0;
}

struct event {
  double y;
  int mask_delta;

  event(double y = 0, int mask_delta = 0) {
    this->y = y;
    this->mask_delta = mask_delta;
  }

  bool operator<(const event &e) const {
    if (y != e.y) {
      return y < e.y;
    }
    return mask_delta < e.mask_delta;
  }
};

template<class It>
double intersection_area(It lo1, It hi1, It lo2, It hi2) {
  It plo[2] = {lo1, lo2}, phi[] = {hi1, hi2};
  std::set<double> xs;
  for (It i1 = lo1; i1 != hi1; ++i1) {
    xs.insert(i1->x);
  }
  for (It i2 = lo2; i2 != hi2; ++i2) {
    xs.insert(i2->x);
  }
  for (It i1 = lo1, j1 = hi1 - 1; i1 != hi1; j1 = i1++) {
    for (It i2 = lo2, j2 = hi2 - 1; i2 != hi2; j2 = i2++) {
      point p;
      if (seg_intersection(*i1, *j1, *i2, *j2, &p) == 0) {
        xs.insert(p.x);
      }
    }
  }
  std::vector<double> xsa(xs.begin(), xs.end());
  double res = 0;
  for (int k = 0; k < (int)xsa.size() - 1; k++) {
    double x = (xsa[k] + xsa[k + 1])/2;
    point sweep0(x, 0), sweep1(x, 1);
    std::vector<event> events;
    for (int poly = 0; poly < 2; poly++) {
      It lo = plo[poly], hi = phi[poly];
      double area = 0;
      for (It i = lo, j = hi - 1; i != hi; j = i++) {
        area += (j->x - i->x)*(j->y + i->y);
      }
      for (It j = lo, i = hi - 1; j != hi; i = j++) {
        point p;
        if (line_intersection(*j, *i, sweep0, sweep1, &p) == 0) {
          double y = p.y, x0 = i->x, x1 = j->x;
          int sgn_area = (area < 0 ? -1 : (area > 0 ? 1 : 0));
          if (x0 < x && x1 > x) {
            events.push_back(event(y, sgn_area*(1 << poly)));
          } else if (x0 > x && x1 < x) {
            events.push_back(event(y, -sgn_area*(1 << poly)));
          }
        }
      }
    }
    std::sort(events.begin(), events.end());
    double a = 0;
    int mask = 0;
    for (int j = 0; j < (int)events.size(); j++) {
      if (mask == 3) {
        a += events[j].y - events[j - 1].y;
      }
      mask += events[j].mask_delta;
    }
    res += a*(xsa[k + 1] - xsa[k]);
  }
  return res;
}

template<class It>
double polygon_area(It lo, It hi) {
  if (lo == hi) {
    return 0;
  }
  double area = 0;
  if (*lo != *--hi) {
    area += (lo->x - hi->x)*(lo->y + hi->y);
  }
  for (It i = hi, j = --hi; i != lo; --i, --j) {
    area += (i->x - j->x)*(i->y + j->y);
  }
  return fabs(area / 2.0);
}

template<class It>
double union_area(It lo1, It hi1, It lo2, It hi2) {
  return polygon_area(lo1, hi1) + polygon_area(lo2, hi2) -
         intersection_area(lo1, hi1, lo2, hi2);
}

/*** Example Usage ***/

#include <cassert>
using namespace std;

int main() {
  vector<point> p, s;
  // Irregular pentagon a triangle of area 1.5 overlapping quadrant 2.
  p.push_back(point(1, 3));
  p.push_back(point(1, 2));
  p.push_back(point(2, 1));
  p.push_back(point(0, 0));
  p.push_back(point(-1, 3));
  // Square of area 12.5 in quadrant 2.
  s.push_back(point(0, 0));
  s.push_back(point(0, 3));
  s.push_back(point(-3, 3));
  s.push_back(point(-3, 0));
  assert(EQ(1.5, intersection_area(p.begin(), p.end(), s.begin(), s.end())));
  assert(EQ(12.5, union_area(p.begin(), p.end(), s.begin(), s.end())));
  return 0;
}
\end{lstlisting}
\subsection{Delaunay Triangulation (Simple)}
\begin{lstlisting}
/*

Given a set P of two dimensional points, the Delaunay triangulation of P is a
set of non-overlapping triangles that covers the entire convex hull of P such
that no point in P lies within the circumcircle of any of the resulting
triangles. For any point p in the convex hull of P (but not necessarily in P),
the nearest point is guaranteed to be a vertex of the enclosing triangle from
the triangulation.

The triangulation may not exist (e.g. for a set of collinear points), or may not
be unique if it does exists. The following program assumes its existence and
produces one such valid result using a simple algorithm which encases each
triangle in a circle and rejecting the triangle if another point in the
tessellation is within the generalized circle.

- delaunay_triangulation(lo, hi) returns a Delaunay triangulation for the input
  range [lo, hi) of points, where lo and hi must be random-access iterators, or
  an empty vector if a triangulation does not exist.

Time Complexity:
- O(n^4) per call to delaunay_triangulation(lo, hi), where n is the distance
  between lo and hi.

Space Complexity:
- O(n) auxiliary heap space for storage of the Delaunay triangulation.

*/

#include <algorithm>
#include <cmath>
#include <utility>
#include <vector>

const double EPS = 1e-9;

#define EQ(a, b) (fabs((a) - (b)) <= EPS)
#define LT(a, b) ((a) < (b) - EPS)
#define LE(a, b) ((a) <= (b) + EPS)

typedef std::pair<double, double> point;
#define x first
#define y second

double sqnorm(const point &a) { return a.x*a.x + a.y*a.y; }
double dot(const point &a, const point &b) { return a.x*b.x + a.y*b.y; }
double cross(const point &a, const point &b, const point &o = point(0, 0)) {
  return (a.x - o.x)*(b.y - o.y) - (a.y - o.y)*(b.x - o.x);
}

int seg_intersection(const point &a, const point &b, const point &c,
                     const point &d, point *p = NULL, point *q = NULL) {
  static const bool TOUCH_IS_INTERSECT = false;  // false is important!
  point ab(b.x - a.x, b.y - a.y);
  point ac(c.x - a.x, c.y - a.y);
  point cd(d.x - c.x, d.y - c.y);
  double c1 = cross(ab, cd), c2 = cross(ac, ab);
  if (EQ(c1, 0) && EQ(c2, 0)) {  // Collinear.
    double t0 = dot(ac, ab) / sqnorm(ab);
    double t1 = t0 + dot(cd, ab) / sqnorm(ab);
    double mint = std::min(t0, t1), maxt = std::max(t0, t1);
    bool overlap = TOUCH_IS_INTERSECT ? (LE(mint, 1) && LE(0, maxt))
                                      : (LT(mint, 1) && LT(0, maxt));
    if (overlap) {
      point res1 = std::max(std::min(a, b), std::min(c, d));
      point res2 = std::min(std::max(a, b), std::max(c, d));
      if (res1 == res2) {
        if (p != NULL) {
          *p = res1;
        }
        return 0;  // Collinear and meeting at an endpoint.
      }
      if (p != NULL && q != NULL) {
        *p = res1;
        *q = res2;
      }
      return 1;  // Collinear and overlapping.
    } else {
      return -1;  // Collinear and disjoint.
    }
  }
  if (EQ(c1, 0)) {
    return -1;  // Parallel and disjoint.
  }
  double t = cross(ac, cd)/c1, u = c2/c1;
  bool t_between_01 = TOUCH_IS_INTERSECT ? (LE(0, t) && LE(t, 1))
                                         : (LT(0, t) && LT(t, 1));
  bool u_between_01 = TOUCH_IS_INTERSECT ? (LE(0, u) && LE(u, 1))
                                         : (LT(0, u) && LT(u, 1));
  if (t_between_01 && u_between_01) {
    if (p != NULL) {
      *p = point(a.x + t*ab.x, a.y + t*ab.y);
    }
    return 0;  // Non-parallel with one intersection.
  }
  return -1;  // Non-parallel with no intersections.
}

struct triangle {
  point a, b, c;

  triangle(const point &a, const point &b, const point &c) : a(a), b(b), c(c) {}

  bool operator==(const triangle &t) const {
    return EQ(a.x, t.a.x) && EQ(a.y, t.a.y) &&
           EQ(b.x, t.b.x) && EQ(b.y, t.b.y) &&
           EQ(c.x, t.c.x) && EQ(c.y, t.c.y);
  }
};

template<class It>
std::vector<triangle> delaunay_triangulation(It lo, It hi) {
  int n = hi - lo;
  std::vector<double> x, y, z;
  for (It it = lo; it != hi; ++it) {
    x.push_back(it->x);
    y.push_back(it->y);
    z.push_back(sqnorm(*it));
  }
  std::vector<triangle> res;
  for (int i = 0; i < n - 2; i++) {
    for (int j = i + 1; j < n; j++) {
      for (int k = i + 1; k < n; k++) {
        if (j == k) {
          continue;
        }
        double nx = (y[j] - y[i])*(z[k] - z[i]) - (y[k] - y[i])*(z[j] - z[i]);
        double ny = (x[k] - x[i])*(z[j] - z[i]) - (x[j] - x[i])*(z[k] - z[i]);
        double nz = (x[j] - x[i])*(y[k] - y[i]) - (x[k] - x[i])*(y[j] - y[i]);
        if (LE(0, nz)) {
          continue;
        }
        point s1[] = {lo[i], lo[j], lo[k], lo[i]};
        for (int m = 0; m < n; m++) {
          if (nx*(x[m] - x[i]) + ny*(y[m] - y[i]) + nz*(z[m] - z[i]) > 0) {
            goto skip;
          }
        }
        // Handle four points on a circle.
        for (int t = 0; t < (int)res.size(); t++) {
          point s2[] = {res[t].a, res[t].b, res[t].c, res[t].a};
          for (int u = 0; u < 3; u++) {
            for (int v = 0; v < 3; v++) {
              if (seg_intersection(s1[u], s1[u + 1], s2[v], s2[v + 1]) == 0) {
                goto skip;
              }
            }
          }
        }
        res.push_back(triangle(lo[i], lo[j], lo[k]));
        skip:;
      }
    }
  }
  return res;
}

/*** Example Usage ***/

#include <cassert>
using namespace std;

int main() {
  vector<point> v;
  v.push_back(point(1, 3));
  v.push_back(point(1, 2));
  v.push_back(point(2, 1));
  v.push_back(point(0, 0));
  v.push_back(point(-1, 3));
  vector<triangle> t;
  t.push_back(triangle(point(1, 3), point(1, 2), point(-1, 3)));
  t.push_back(triangle(point(1, 3), point(2, 1), point(1, 2)));
  t.push_back(triangle(point(1, 2), point(2, 1), point(0, 0)));
  t.push_back(triangle(point(1, 2), point(0, 0), point(-1, 3)));
  assert(delaunay_triangulation(v.begin(), v.end()) == t);
  return 0;
}
\end{lstlisting}
\subsection{Delaunay Triangulation (Fast)}
\begin{lstlisting}
/*

Given a set P of two dimensional points, the Delaunay triangulation of P is a
set of non-overlapping triangles that covers the entire convex hull of P such
that no point in P lies within the circumcircle of any of the resulting
triangles. For any point p in the convex hull of P (but not necessarily in P),
the nearest point is guaranteed to be a vertex of the enclosing triangle from
the triangulation.

The triangulation may not exist (e.g. for a set of collinear points), or may not
be unique if it does exists. The following program assumes its existence and
produces one such valid result using TABLE_DELAUNAY, a divide and conquer
algorithm with linear merging. Its fully documented version along with debugging
messages for the current asserts() may be found at the following link:
http://people.sc.fsu.edu/~jburkardt/f_src/table_delaunay/table_delaunay.html

- delaunay_triangulation(lo, hi) returns a Delaunay triangulation for the input
  range [lo, hi) of points, where lo and hi must be random-access iterators, or
  an empty vector if a triangulation does not exist.

Time Complexity:
- O(n log n) per call to delaunay_triangulation(lo, hi), where n is the distance
  between lo and hi.

Space Complexity:
- O(n) auxiliary heap space for storage of the Delaunay triangulation.

*/

#include <algorithm>
#include <cassert>
#include <cmath>
#include <cstddef>
#include <limits>
#include <utility>
#include <vector>

int wrap(int ival, int ilo, int ihi) {
  int jlo = std::min(ilo, ihi), jhi = std::max(ilo, ihi);
  int wide = jhi + 1 - jlo, res = jlo;
  if (wide != 1)  {
    assert(wide != 0);
    int tmp = (ival - jlo) % wide;
    if (tmp < 0) {
      res += std::abs(wide);
    }
    res += tmp;
  }
  return res;
}

void permute(int n, double a[][2], int p[]) {
  for (int istart = 1; istart <= n; istart++) {
    if (p[istart - 1] < 0) {
      continue;
    }
    if (p[istart - 1] == istart) {
      p[istart - 1] = -p[istart - 1];
      continue;
    }
    double tmp0 = a[istart - 1][0], tmp1 = a[istart - 1][1];
    int iget = istart;
    for (;;) {
      int iput = iget;
      iget = p[iget - 1];
      p[iput - 1] = -p[iput - 1];
      assert(!(iget < 1 || n < iget));
      if (iget == istart) {
        a[iput - 1][0] = tmp0;
        a[iput - 1][1] = tmp1;
        break;
      }
      a[iput - 1][0] = a[iget - 1][0];
      a[iput - 1][1] = a[iget - 1][1];
    }
  }
  for (int i = 0; i < n; i++) {
    p[i] = -p[i];
  }
}

int* sort_heap(int n, double a[][2]) {
  double aval[2];
  int i, ir, j, l, idxt;
  int *idx;
  if (n < 1) {
    return NULL;
  }
  if (n == 1) {
    idx = new int[1];
    idx[0] = 1;
    return idx;
  }
  idx = new int[n];
  for (int i = 0; i < n; i++) {
    idx[i] = i + 1;
  }
  l = n/2 + 1;
  ir = n;
  for (;;) {
    if (1 < l) {
      l--;
      idxt = idx[l - 1];
      aval[0] = a[idxt - 1][0];
      aval[1] = a[idxt - 1][1];
    } else {
      idxt = idx[ir - 1];
      aval[0] = a[idxt - 1][0];
      aval[1] = a[idxt - 1][1];
      idx[ir - 1] = idx[0];
      if (--ir == 1) {
        idx[0] = idxt;
        break;
      }
    }
    i = l;
    j = 2*l;
    while (j <= ir) {
      if (j < ir && (a[idx[j - 1] - 1][0] <  a[idx[j] - 1][0] ||
                    (a[idx[j - 1] - 1][0] == a[idx[j] - 1][0] &&
                     a[idx[j - 1] - 1][1] <  a[idx[j] - 1][1]))) {
        j++;
      }
      if ( aval[0]  < a[idx[j - 1] - 1][0] ||
          (aval[0] == a[idx[j - 1] - 1][0] &&
           aval[1]  < a[idx[j - 1] - 1][1])) {
        idx[i - 1] = idx[j - 1];
        i = j;
        j *= 2;
      } else {
        j = ir + 1;
      }
    }
    idx[i - 1] = idxt;
  }
  return idx;
}

int lrline(double xu, double yu, double xv1, double yv1,
          double xv2, double yv2, double dv) {
  static const double tol = 1e-7;
  double dx = xv2 - xv1, dy = yv2 - yv1;
  double dxu = xu - xv1, dyu = yu - yv1;
  double t = dy*dxu - dx*dyu + dv*sqrt(dx*dx + dy*dy);
  double tolabs = tol*std::max(std::max(fabs(dx), fabs(dy)),
                      std::max(fabs(dxu), std::max(fabs(dyu), fabs(dv))));
  return tolabs < t ? 1 : (-tolabs <= t ? 0 : -1);
}

void vbedg(double x, double y, int point_num, double point_xy[][2],
           int tri_num, int tri_nodes[][3], int tri_neigh[][3],
           int *ltri, int *ledg, int *rtri, int *redg) {
  int a, b;
  double ax, ay, bx, by;
  bool done;
  int e, l, t;
  if (*ltri == 0) {
    done = false;
    *ltri = *rtri;
    *ledg = *redg;
  } else {
    done = true;
  }
  for (;;) {
    l = -tri_neigh[*rtri - 1][*redg - 1];
    t = l / 3;
    e = l % 3 + 1;
    a = tri_nodes[t - 1][e - 1];
    if (e <= 2) {
      b = tri_nodes[t - 1][e];
    } else {
      b = tri_nodes[t - 1][0];
    }
    ax = point_xy[a - 1][0];
    ay = point_xy[a - 1][1];
    bx = point_xy[b - 1][0];
    by = point_xy[b - 1][1];
    if (lrline(x, y, ax, ay, bx, by, 0.0) <= 0) {
      break;
    }
    *rtri = t;
    *redg = e;
  }
  if (done) {
    return;
  }
  t = *ltri;
  e = *ledg;
  for (;;) {
    b = tri_nodes[t - 1][e - 1];
    e = wrap(e - 1, 1, 3);
    while (0 < tri_neigh[t - 1][e - 1]) {
      t = tri_neigh[t - 1][e - 1];
      if (tri_nodes[t - 1][0] == b) {
        e = 3;
      } else if (tri_nodes[t - 1][1] == b) {
        e = 1;
      } else {
        e = 2;
      }
    }
    a = tri_nodes[t - 1][e - 1];
    ax = point_xy[a - 1][0];
    ay = point_xy[a - 1][1];
    bx = point_xy[b - 1][0];
    by = point_xy[b - 1][1];
    if (lrline(x, y, ax, ay, bx, by, 0.0) <= 0) {
      break;
    }
  }
  *ltri = t;
  *ledg = e;
  return;
}

int diaedg(double x0, double y0, double x1, double y1,
           double x2, double y2, double x3, double y3) {
  double ca, cb, s, tol, tola, tolb;
  int value;
  tol = 100.0*std::numeric_limits<double>::epsilon();
  double dx10 = x1 - x0, dy10 = y1 - y0;
  double dx12 = x1 - x2, dy12 = y1 - y2;
  double dx30 = x3 - x0, dy30 = y3 - y0;
  double dx32 = x3 - x2, dy32 = y3 - y2;
  tola = tol*std::max(std::max(fabs(dx10), fabs(dy10)),
                      std::max(fabs(dx30), fabs(dy30)));
  tolb = tol*std::max(std::max(fabs(dx12), fabs(dy12)),
                      std::max(fabs(dx32), fabs(dy32)));
  ca = dx10*dx30 + dy10*dy30;
  cb = dx12*dx32 + dy12*dy32;
  if (tola < ca && tolb < cb) {
    value = -1;
  } else if (ca < -tola && cb < -tolb) {
    value = 1;
  } else {
    tola = std::max(tola, tolb);
    s = (dx10*dy30 - dx30*dy10)*cb + (dx32*dy12 - dx12*dy32)*ca;
    if (tola < s) {
      value = -1;
    } else if (s < -tola) {
      value = 1;
    } else {
      value = 0;
    }
  }
  return value;
}

int swapec(int i, int *top, int *btri, int *bedg, int point_num,
           double point_xy[][2], int tri_num, int tri_nodes[][3],
           int tri_neigh[][3], int stack[]) {
  int a, b, c, e, ee, em1, ep1, f, fm1, fp1, l, r, s, swap, t, tt, u;
  double x = point_xy[i - 1][0], y = point_xy[i - 1][1];
  for (;;) {
    if (*top <= 0) {
      break;
    }
    t = stack[*top - 1];
    *top -= 1;
    if (tri_nodes[t - 1][0] == i) {
      e = 2;
      b = tri_nodes[t - 1][2];
    } else if (tri_nodes[t - 1][1] == i) {
      e = 3;
      b = tri_nodes[t - 1][0];
    } else {
      e = 1;
      b = tri_nodes[t - 1][1];
    }
    a = tri_nodes[t - 1][e - 1];
    u = tri_neigh[t - 1][e - 1];
    if (tri_neigh[u - 1][0] == t) {
      f = 1;
      c = tri_nodes[u - 1][2];
    } else if (tri_neigh[u - 1][1] == t) {
      f = 2;
      c = tri_nodes[u - 1][0];
    } else {
      f = 3;
      c = tri_nodes[u - 1][1];
    }
    swap = diaedg(x, y, point_xy[a - 1][0], point_xy[a - 1][1],
                        point_xy[c - 1][0], point_xy[c - 1][1],
                        point_xy[b - 1][0], point_xy[b - 1][1]);
    if (swap == 1) {
      em1 = wrap(e - 1, 1, 3);
      ep1 = wrap(e + 1, 1, 3);
      fm1 = wrap(f - 1, 1, 3);
      fp1 = wrap(f + 1, 1, 3);
      tri_nodes[t - 1][ep1 - 1] = c;
      tri_nodes[u - 1][fp1 - 1] = i;
      r = tri_neigh[t - 1][ep1 - 1];
      s = tri_neigh[u - 1][fp1 - 1];
      tri_neigh[t - 1][ep1 - 1] = u;
      tri_neigh[u - 1][fp1 - 1] = t;
      tri_neigh[t - 1][e - 1] = s;
      tri_neigh[u - 1][f - 1] = r;
      if (0 < tri_neigh[u - 1][fm1 - 1]) {
        *top += 1;
        stack[*top - 1] = u;
      }
      if (0 < s) {
        if (tri_neigh[s - 1][0] == u) {
          tri_neigh[s - 1][0] = t;
        } else if (tri_neigh[s - 1][1] == u) {
          tri_neigh[s - 1][1] = t;
        } else {
          tri_neigh[s - 1][2] = t;
        }
        *top += 1;
        if (point_num < *top) {
          return 8;
        }
        stack[*top - 1] = t;
      } else {
        if (u == *btri && fp1 == *bedg) {
          *btri = t;
          *bedg = e;
        }
        l = - (3*t + e - 1);
        tt = t;
        ee = em1;
        while (0 < tri_neigh[tt - 1][ee - 1]) {
          tt = tri_neigh[tt - 1][ee - 1];
          if (tri_nodes[tt - 1][0] == a) {
            ee = 3;
          } else if (tri_nodes[tt - 1][1] == a) {
            ee = 1;
          } else {
            ee = 2;
          }
        }
        tri_neigh[tt - 1][ee - 1] = l;
      }
      if (0 < r) {
        if (tri_neigh[r - 1][0] == t) {
          tri_neigh[r - 1][0] = u;
        } else if (tri_neigh[r - 1][1] == t) {
          tri_neigh[r - 1][1] = u;
        } else {
          tri_neigh[r - 1][2] = u;
        }
      } else {
        if (t == *btri && ep1 == *bedg) {
          *btri = u;
          *bedg = f;
        }
        l = -(3*u + f - 1);
        tt = u;
        ee = fm1;
        while (0 < tri_neigh[tt - 1][ee - 1]) {
          tt = tri_neigh[tt - 1][ee - 1];
          if (tri_nodes[tt - 1][0] == b) {
            ee = 3;
          } else if (tri_nodes[tt - 1][1] == b) {
            ee = 1;
          } else {
            ee = 2;
          }
        }
        tri_neigh[tt - 1][ee - 1] = l;
      }
    }
  }
  return 0;
}

void perm_inv(int n, int p[]) {
  int i, i0, i1, i2;
  assert(n > 0);
  for (i = 1; i <= n; i++) {
    i1 = p[i - 1];
    while (i < i1) {
      i2 = p[i1 - 1];
      p[i1 - 1] = -i2;
      i1 = i2;
    }
    p[i - 1] = -p[i - 1];
  }
  for (i = 1; i <= n; i++) {
    i1 = -p[i - 1];
    if (0 <= i1) {
      i0 = i;
      for (;;) {
        i2 = p[i1 - 1];
        p[i1 - 1] = i0;
        if (i2 < 0) {
          break;
        }
        i0 = i1;
        i1 = i2;
      }
    }
  }
}

int dtris2(int point_num, double point_xy[][2],
           int tri_nodes[][3], int tri_neigh[][3]) {
  double cmax;
  int e, error;
  int i, j, k, l, m, m1, m2, n;
  int ledg, lr, ltri, redg, rtri, t, top;
  double tol;
  int *stack = new int[point_num];
  tol = 100.0*std::numeric_limits<double>::epsilon();
  int *idx = sort_heap(point_num, point_xy);
  permute(point_num, point_xy, idx);
  m1 = 0;
  for (i = 1; i < point_num; i++) {
    m = m1;
    m1 = i;
    k = -1;
    for (j = 0; j <= 1; j++) {
      cmax = std::max(fabs(point_xy[m][j]), fabs(point_xy[m1][j]));
      if (tol*(cmax + 1.0) < fabs(point_xy[m][j] - point_xy[m1][j])) {
        k = j;
        break;
      }
    }
    assert(k != -1);
  }
  m1 = 1;
  m2 = 2;
  j = 3;
  for (;;) {
    assert(point_num >= j);
    m = j;
    lr = lrline(point_xy[m - 1][0], point_xy[m - 1][1],
                point_xy[m1 - 1][0], point_xy[m1 - 1][1],
                point_xy[m2 - 1][0], point_xy[m2 - 1][1], 0.0);
    if (lr != 0) {
      break;
    }
    j++;
  }
  int tri_num = j - 2;
  if (lr == -1) {
    tri_nodes[0][0] = m1;
    tri_nodes[0][1] = m2;
    tri_nodes[0][2] = m;
    tri_neigh[0][2] = -3;
    for (i = 2; i <= tri_num; i++) {
      m1 = m2;
      m2 = i + 1;
      tri_nodes[i - 1][0] = m1;
      tri_nodes[i - 1][1] = m2;
      tri_nodes[i - 1][2] = m;
      tri_neigh[i - 1][0] = -3*i;
      tri_neigh[i - 1][1] = i;
      tri_neigh[i - 1][2] = i - 1;
    }
    tri_neigh[tri_num - 1][0] = -3*tri_num - 1;
    tri_neigh[tri_num - 1][1] = -5;
    ledg = 2;
    ltri = tri_num;
  } else {
    tri_nodes[0][0] = m2;
    tri_nodes[0][1] = m1;
    tri_nodes[0][2] = m;
    tri_neigh[0][0] = -4;
    for (i = 2; i <= tri_num; i++) {
      m1 = m2;
      m2 = i+1;
      tri_nodes[i - 1][0] = m2;
      tri_nodes[i - 1][1] = m1;
      tri_nodes[i - 1][2] = m;
      tri_neigh[i - 2][2] = i;
      tri_neigh[i - 1][0] = -3*i - 3;
      tri_neigh[i - 1][1] = i - 1;
    }
    tri_neigh[tri_num - 1][2] = -3*(tri_num);
    tri_neigh[0][1] = -3*(tri_num) - 2;
    ledg = 2;
    ltri = 1;
  }
  top = 0;
  for (i = j + 1; i <= point_num; i++) {
    m = i;
    m1 = tri_nodes[ltri - 1][ledg - 1];
    if (ledg <= 2) {
      m2 = tri_nodes[ltri - 1][ledg];
    } else {
      m2 = tri_nodes[ltri - 1][0];
    }
    lr = lrline(point_xy[m - 1][0], point_xy[m - 1][1],
                point_xy[m1 - 1][0], point_xy[m1 - 1][1],
                point_xy[m2 - 1][0], point_xy[m2 - 1][1], 0.0);
    if (0 < lr) {
      rtri = ltri;
      redg = ledg;
      ltri = 0;
    } else {
      l = -tri_neigh[ltri - 1][ledg - 1];
      rtri = l / 3;
      redg = (l % 3) + 1;
    }
    vbedg(point_xy[m - 1][0], point_xy[m - 1][1],
          point_num, point_xy, tri_num, tri_nodes, tri_neigh,
          &ltri, &ledg, &rtri, &redg);
    n = tri_num + 1;
    l = -tri_neigh[ltri - 1][ledg - 1];
    for (;;) {
      t = l / 3;
      e = (l % 3) + 1;
      l = -tri_neigh[t - 1][e - 1];
      m2 = tri_nodes[t - 1][e - 1];
      if (e <= 2) {
        m1 = tri_nodes[t - 1][e];
      } else {
        m1 = tri_nodes[t - 1][0];
      }
      tri_num++;
      tri_neigh[t - 1][e - 1] = tri_num;
      tri_nodes[tri_num - 1][0] = m1;
      tri_nodes[tri_num - 1][1] = m2;
      tri_nodes[tri_num - 1][2] = m;
      tri_neigh[tri_num - 1][0] = t;
      tri_neigh[tri_num - 1][1] = tri_num - 1;
      tri_neigh[tri_num - 1][2] = tri_num + 1;
      top++;
      assert(point_num >= top);
      stack[top - 1] = tri_num;
      if (t == rtri && e == redg) {
        break;
      }
    }
    tri_neigh[ltri - 1][ledg - 1] = -3*n - 1;
    tri_neigh[n - 1][1] = -3*tri_num - 2;
    tri_neigh[tri_num - 1][2] = -l;
    ltri = n;
    ledg = 2;
    error = swapec(m, &top, &ltri, &ledg, point_num, point_xy,
                   tri_num, tri_nodes, tri_neigh, stack);
    assert(error == 0);
  }
  for (i = 0; i < 3; i++) {
    for (j = 0; j < tri_num; j++) {
      tri_nodes[j][i] = idx[tri_nodes[j][i] - 1];
    }
  }
  perm_inv(point_num, idx);
  permute(point_num, point_xy, idx);
  delete[] idx;
  delete[] stack;
  return tri_num;
}

/*** Wrapper ***/

const double EPS = 1e-9;
#define EQ(a, b) (fabs((a) - (b)) <= EPS)

typedef std::pair<double, double> point;
#define x first
#define y second

struct triangle {
  point a, b, c;

  triangle(const point &a, const point &b, const point &c) : a(a), b(b), c(c) {}

  bool operator==(const triangle &t) const {
    return EQ(a.x, t.a.x) && EQ(a.y, t.a.y) &&
           EQ(b.x, t.b.x) && EQ(b.y, t.b.y) &&
           EQ(c.x, t.c.x) && EQ(c.y, t.c.y);
  }
};

template<class It>
std::vector<triangle> delaunay_triangulation(It lo, It hi) {
  int n = hi - lo;
  double points[n][2];
  int tri_nodes[3*n][3], tri_neigh[3*n][3];
  int curr = 0;
  for (It it = lo; it != hi; ++curr, ++it) {
    points[curr][0] = it->x;
    points[curr][1] = it->y;
  }
  int m = dtris2(n, points, tri_nodes, tri_neigh);
  std::vector<triangle> res;
  for (int i = 0; i < m; i++) {
    res.push_back(triangle(lo[tri_nodes[i][0] - 1],
                           lo[tri_nodes[i][1] - 1],
                           lo[tri_nodes[i][2] - 1]));
  }
  return res;
}

/*** Example Usage ***/

#include <cassert>
using namespace std;

int main() {
  vector<point> v;
  v.push_back(point(1, 3));
  v.push_back(point(1, 2));
  v.push_back(point(2, 1));
  v.push_back(point(0, 0));
  v.push_back(point(-1, 3));
  vector<triangle> t;
  t.push_back(triangle(point(-1, 3), point(0, 0), point(1, 2)));
  t.push_back(triangle(point(-1, 3), point(1, 2), point(1, 3)));
  t.push_back(triangle(point(1, 2), point(0, 0), point(2, 1)));
  t.push_back(triangle(point(1, 3), point(1, 2), point(2, 1)));
  assert(delaunay_triangulation(v.begin(), v.end()) == t);
  return 0;
}
\end{lstlisting}


\backmatter

\printindex

\end{document}
